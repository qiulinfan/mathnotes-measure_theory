\chapter*{Homework 3: on Lebesgue-Stieljes measures(30/40)}

\begin{center}
\textit{None of the following questions will be graded. Do them, but do not hand them in}.
\end{center}


\section*{Fun facts about increasing functions}
    Let $F\colon\mathbb{R}\to\mathbb{R}$ be an increasing function, that is, $F(x)\le F(y)$ whenever $x\le y$. 
    \begin{itemize}
    \item[(a)]
      Prove that the following limits exist (and make sure you understand the definitions):
      \begin{itemize}
      \item[(i)]
        $F(a-):=\lim_{x\to a-}F(x)\in\mathbb{R}$ and $F(a+):=\lim_{x\to a+}F(x)\in\mathbb{R}$ for $a\in\mathbb{R}$;
      \item[(ii)]
        $F(\infty):=\lim_{x\to\infty}F(x)\in(-\infty,\infty]$;
      \item[(iii)]
        $F(-\infty):=\lim_{x\to-\infty}F(x)\in[-\infty,\infty)$.
      \end{itemize}
    \item[(b)]
      Fix any $a\in\mathbb{R}$.
      \begin{itemize}
      \item[(i)]
        Prove that $F(a-)\le F(a)\le F(a+)$;
      \item[(ii)]
        Prove that $F$ is continuous at $a$ iff $F(a-)=F(a+)$.
      \end{itemize}
      We say that a function is \emph{left continuous} if $F(a-)=F(a)$ for every $a\in\mathbb{R}$. It is \emph{right-continuous} if instead $F(a+)=F(a)$ for every $a\in\mathbb{R}$.
    \item[(c)]
      If $X$ is a metric space (or, more generally, a topological space), then a function $f\colon X\to\mathbb{R}$ is \emph{upper semicontinuous} if the set $\{x\in X\mid f(x)<a\}$ is open for every $a\in\mathbb{R}$. It is \emph{lower semicontinuous} if instead the set $\{x\in X\mid f(x)>a\}$ is open for every $a\in\mathbb{R}$.
      Prove that our function $F\colon\mathbb{R}\to\mathbb{R}$ is right-continuous (resp. left continuous) iff it is upper semicontinuous (resp.\ lower semicontinuous). Give an example showing that this is no longer true if $F$ is not assumed increasing.
  \item[(d)]
    Prove that the following are equivalent:
    \begin{itemize}
    \item[(i)]
      $F$ is surjective;
    \item[(ii)]
      $F$ is continuous, $F(\infty)=\infty$, and $F(-\infty)=-\infty$.
    \end{itemize}
  \item[(e)]
    Let $A\subset\mathbb{R}$ be the set of points where $F$ fails to be continuous. Prove that $A$ is a countable  (i.e. empty, finite, or countably infinite) set. \textit{Hint}: prove that for any integers $m,n\ge 1$, the set of points $x\in[-m,m]$ where $F(x+)-F(x-)\ge 1/n$ is finite.
  \end{itemize}

\section*{Locally finite measures}
  If $X$ is a metric space (or, more generally, a topological space), then a Borel measure $\mu$ on $X$ is said to be \emph{locally finite} if $\mu(K)<\infty$ for every compact set $K\subset X$. Now let $\mu$ be a Borel measure on $\mathbb{R}$, that is $\mu\colon\mathcal{B}(\mathbb{R})\to[0,\infty]$ satisfies $\mu(\emptyset)=0$ and is countably additive.
  \begin{itemize}
  \item[(a)]
    Prove that the following are equivalent:
    \begin{itemize}
    \item[(i)]
      $\mu$ is locally finite;
    \item[(ii)]
      $\mu([-N,N])<\infty$ for every $N\ge0$;
    \item[(iii)]
      $\mu(I)<\infty$ for every bounded interval $I$.
    \end{itemize}
  \item[(b)]
    Prove that if $\mu$ is locally finite, then $\mu$ is $\sigma$-finite. Is the converse true? Give a proof or a counterexample.
  \end{itemize}
  
\section*{Basic formulas for LS measures}
  Let $F\colon\mathbb{R}\to\mathbb{R}$ be a distribution function, and $\mu=\mu_F$ the associated Lebesgue--Stieltjes measure. From its definition using h-intervals, it follows that $\mu((a,b])=F(b)-F(a)$ for $-\infty<a<b<\infty$. Using this property together with basic general properties of ($\sigma$-finite) measures, we proved in class that $\mu((a,b))=F(b-)-F(a)$ for $\infty<a\le b<\infty$. Using a similar strategy, prove the following:
  \begin{itemize}
  \item[(a)]
    $\mu([a,b])=F(b)-F(a-)$ for $-\infty<a\le b<\infty$;
  \item[(b)]
    $\mu([a,b))=F(b-)-F(a-)$ for $-\infty<a\le b<\infty$;
  \item[(c)]
    $\mu(\{a\}))=F(a)-F(a-)$ for $-\infty<a<\infty$;
  \item[(d)]
    $\mu((-\infty,b])=F(b)-F(-\infty)$ for $-\infty<b<\infty$;
  \item[(e)]
    $\mu((-\infty,b))=F(b-)-F(-\infty)$ for $-\infty<b<\infty$;
  \item[(e)]
    $\mu((a,\infty))=F(\infty)-F(a)$ for $-\infty<a<\infty$;
  \item[(f)]
    $\mu([a,\infty))=F(\infty)-F(a-)$ for $-\infty<a<\infty$;
  \item[(g)]
    $\mu([-\infty,\infty))=F(\infty)-F(-\infty)$.
  \end{itemize}

\section*{Vitali sets}
  For $x, y\in [-1,1]$, write $x\sim y$ iff $x-y\in \mathbb{Q}$. 
  \begin{itemize}
  \item[(a)]
    Show that $\sim$ is an equivalence relation, i.e. show that (i) $x\sim x$, (ii) $x\sim y$ implies $y\sim x$, (iii) if $x\sim y$ and $y\sim z$, then $x\sim z$.
  \item[(b)]
    The set $[-1,1]$ is partitioned into equivalence classes. Let $V\subset[-1,1]$ be a set containing exactly one element from each equivalence class. (Here, we use the Axiom of choice.) We call $V$ a \emph{Vitali set}. 
    Let $\{r_1, r_2,\dots\} =[-2,2]\cap \mathbb{Q}$. Define $V_i= r_i+V=\{r_i+x\mid x\in V\}$.
    Prove that the sets $V_1, V_2, \dots$ are mutually disjoint, and that 
    \[
      [-1, 1]\subset \bigcup_{i=1}^\infty V_i \subset [-3,3].
    \]
  \end{itemize}

\section*{Vitali sets, season 2}
  Let $V\subset[-1,1]$ be a Vitali set (see above).
  \begin{itemize}
  \item[(a)]
    Using the translation invariance of Lebesgue measure, prove $V$ is not Lebesgue
    measurable.
  \item[(b)]
    Prove that if $E$ is a Lebesgue measurable set and satisfies $E\subset V$, then $m(E)=0$. 
  \item[(c)]
    Using the technique in~(a), prove the following statement: if $A\subset\mathbb{R}$ is any Lebesgue measurable set with $m(A)>0$, then $A$ contains a set which is not Lebesgue measurable. 
  \end{itemize}

\section*{The middle-thirds Cantor set}
  Let $C$ be the middle-thirds Cantor set, defined as
  \[
    C:=\bigcap_{n=1}^\infty C_n,
  \]
 where
\[
  C_n:=\bigcup_{a_1, \dots, a_n\in \{0,2\}} \big[\sum_{i=1}^n \frac{a_i}{3^i}, \sum_{i=1}^{n} \frac{a_i}{3^i} +\frac{1}{3^n} \big]
\]
\begin{itemize}
\item[(a)]
  Set $C_0=[0,1]$. Show that $C_n\subset C_{n-1}$ for all $n\ge1$.
  Also prove that $C_n$ is the union of $2^n$ \emph{disjoint} closed intervals, that the set  $U_n:=C_{n-1}\setminus C_n$ is the union of the middle thirds open intervals of the disjoint closed intervals of $C_{n-1}$, and that
\[
  U_n=\bigcup_{a_1, \cdots, a_{n-1}\in \{0,2\}} \big( \sum_{i=1}^{n-1} \frac{a_i}{3^i} + \frac{1}{3^n}, \sum_{i=1}^{n-1} \frac{a_i}{3^i} +\frac{2}{3^n} \big).
\]
(We interpret this as the interval $(1/3,2/3)$ when $n=1$.)
  Thus, $C$ is the set obtained by removing successive middle thirds of the remaining disjoint closed intervals starting with $[0,1]$. Sketch the first few sets $C_n$ and $U_n$.
\item[(b)]
  Show that $C$ is a compact set, and that $m(C)=0$, where $m$ denotes Lebesgue measure. Also show that $C$ does not contain any non-empty open interval $(a,b)$. 
\item[(c)]
  Show that $C$ equals the set of numbers $x\in [0,1]$ which have a base-3 expansion of the form
  $x=0.a_1a_2a_3\cdots$ where $a_i$ is either $0$ or $2$, i.e.\
  \[
    C=\{ \sum_{i=1}^\infty \frac{a_i}{3^i} \mid \text{ $a_i\in \{0,2\}$ for all $i\in \mathbb{N}$} \}.
  \]
  (Note: A point may have two base-3 expansions such as $1/3=0.1000\ldots=0.0222\ldots$; this number is in $C$ since one of the expansions is of the desired form.) 
\item[(d)]
  Show that $\frac14, \frac9{13}\in C$ but $\frac{5}{36}\notin C$.
\end{itemize}

\section*{The Devil's Staircase: an increasing function build on Cantor set}

  Let $C$ be the middle-thirds Cantor set, and define $F\colon C\to [0,1]$ by
  \begin{equation}\label{eq:Cantorfunctiondefn}
    F(x)= \sum_{i=1}^\infty \frac{a_i/2}{2^i}
  \end{equation}
  for $x=\sum_{i=1}^\infty \frac{a_i}{3^i}$, $a_i\in \{0, 2\}$. 
  \begin{itemize}
  \item[(a)]
    Prove that $F$ is an increasing function, and that $F(C)=[0,1]$.
  \item[(b)]
    Suppose that $x,y\in C$ and $x<y$. Prove that $F(x)=F(y)$ iff $x$ and $y$ are the endpoints of a removed open interval, that is, one of the $2^{n-1}$ disjoint open intervals whose union equals $U_n=C_{n-1}\setminus C_n$ for some $n\ge1$.
  \item[(c)]
    Prove that $F\colon C\to[0,1]$ extends uniquely to a continuous function which is constant on all the intervals in $U_n$, $n\ge 1$. Sketch the graph of $F$.
    \textit{Hint}: to prove continuity, it suffices to show that $F([0,1])=[0,1]$ (Why?)
  \item[(d)]
    Prove that $F'(x)=0$ for a.e.\ $x$. In other words, there exists a set $E\subset[0,1]$ such that $m(E)=0$, and such that $\lim_{h\to 0}(F(x+h)-F(x))/h=0$ for $x\in[0,1]\setminus E$.
  \end{itemize}
  (Remark 1: because of~(c) and~(d), the graph of $F$ is called the \emph{Devil's Staircase}; it is horizontal almost everywhere, and has no vertical jumps, but nevertheless climbs upwards.)

  (Remark 2: the fact that $F(C)=[0,1]$ implies that $C$ has the same cardinality as $[0,1]$, in particular the Cantor set is uncountable.) 


\begin{center}
\textit{Some of the following questions will be graded. Do them, and do hand them in}.
\end{center}


\section*{fun facts about distribution functions}
  \begin{itemize}
  \item[(a)]  Let $A\subset\mathbb{R}$ be a countable set. Exhibit a distribution function $F$ that is discontinuous at every point in $A$, but continuous everywhere else. Justify your answer. \textit{Hint}: play around with the Heaviside function.
  \item[(b)]Let $F\colon\mathbb{R}\to\mathbb{R}$ be an increasing function. Prove that there exists a unique distribution function $G$ such that $G(x)=F(x)$ for all points $x$ where $F$ is continuous.
    \textit{Hint}: there is a simple formula for $G$ in terms of $F$.
  \end{itemize}

\begin{solution} \textbf{of (a):}\\
We list \( A = \{a_n\}_{n=1}^{\infty} \) as a sequence to label its elements. Define:
\[
F(x) = \sum_{n=1}^{\infty} \frac{1}{2^n} H(x - a_n)
\]
where \( H(x) \) is the Heaviside function: \(
H(x) =
\begin{cases}
0, & x < 0, \\
1, & x \geq 0.
\end{cases}
\).\\
\textbf{Claim 1.1 $F$ is non-decreasing}.\\
Proof: Suppose $y > x \in \mathbb{R}$, then $H(y-a_n) \geq H(x-a_n)$ for each $n \in \mathbb{N}$, so we have $F(y) \geq F(x)$.\\\\
\textbf{Claim 1.2 $F$ is right continuous but not left continuous (thus discontinuous) at every $a_n$.}\\
Proof: Let $\epsilon > 0$.\\
We take $N \in \mathbb{N}$ s.t. $\sum_{k\geq N, n\in \mathbb{N}} \frac{1}{2^k} < \epsilon$.\\
Then we take $\delta>0$ such that $a_1, a_2,\cdots, a_N \not \in (a_n, a_n+\delta)$.(This can be done since there are only finite points here)\\
Thus $\forall y \in (a_n, a_n+\delta)$, we have $|F(y) - F(a_n)| < \epsilon$, since $F(y) < F(a_n) + \sum_{k\geq N, n\in \mathbb{N}} \frac{1}{2^k}$.
Since $\epsilon$ is arbitrary, this finishes the proof that $F$ is right continuous at $a_n$.\\
Also, $\forall y < a_n$, we have $F(y) < F(a_n) - \frac{1}{2^n}$, which means that $|F(y) -F(a_n)|>\frac{1}{2^n}$ for any $y$ on the left, so $F$ is not left continuous at $a_n$.\\\\
\textbf{Claim 1.3: $F$ is continuous at every $x \in \mathbb{R} \setminus A$.}\\
Proof: This is similar to the proof in Claim 1.2. \\
Fix $x \in \mathbb{R} \setminus A$. Let  $\epsilon > 0$.\\
We take $N \in \mathbb{N}$ s.t. $\sum_{k\geq N, n\in \mathbb{N}} \frac{1}{2^k} < \epsilon$.\\
Then we take $\delta>0$ such that $a_1, a_2,\cdots, a_N \not \in (a_n-\delta, a_n+\delta)$. This can be done since there are only finite points here.\\
Thus $\forall y \in (a_n-\delta, a_n+\delta)$, we have $|F(y) - F(a_n)| <\sum_{k\geq N, n\in \mathbb{N}} \frac{1}{2^k} <  \epsilon$.\\\
Since $\epsilon$ is arbitrary, this finishes the proof that $F$ is continuous at $x$.\\\\

By claim 1.1, 1.2, 1.3, we have proved that $F$ is a distribution function that is discontinuous at every point of $A$ but continuous elsewhere.\\\\
\end{solution}

\begin{proof} \textbf{of (b):}\\
Given an increasing function \( F: \mathbb{R} \to \mathbb{R} \), we define $G:\mathbb{R} \to \mathbb{R}$ by:
\[
G(x) = \lim_{y \to x^+} F(y).
\]
We will show that this is the unique distribution function $G$ such that $G(x)=F(x)$ for all points $x$ where $F$ is continuous.\\
\textbf{Incresing:} Since \( F \) is increasing, for any \( x < y \) we have \( F(x) \le F(y) \). Thus, for any \( x < y \) we have:
\[
G(x) = \lim_{z \to x^+} F(z) \le \lim_{z \to y^+} F(z) = G(y).
\]
Thus, \( G \) is increasing.\\
\textbf{Right-continuity:} Since $F$ is an increasing function, it can only have jump discontinuities, and the right limit exists for all $X$. By construction, $G$ is right-ctn.\\
Above finishes the proof that $G$ is a distribution function.\\
\textbf{Agree with $F$ at ctn point:} $G(x) = F(x)$ where $F$ is continuous at $x$, since $G(x) = \lim_{y\to x^+} F(y) = F(x) $ there.\\

It remains to show that it is unique.\\
Suppose $R$ is another such function. It suffices to show: $R $ agrees with $G$ on discontinuous points of $F$.\\
Since $R,G$ are right continuous, their right limit must exist at each point. Therefore, let $x$ be an arbitrary point where $F$ is discontinuous at $x$, \textbf{it suffices to show that there is a sequence $\{x_n\}$ approaching $x$, such that $\lim_n G(x_n)  = \lim_n R(x_n)$.}\\
Since $F$ is increasing, the points where $F$ is disctn is at most countable. Therefore\textbf{ the points where $F$ is ctn, denote it as $C$, is dense in $\mathbb{R}$.} Thus we can pick a sequence $\{x_n\}$ in $C$ approaching $x$, then $G(x_n) = R(x_n) = F(x_n)$ for each $n$, impling that $\lim_n G(x_n)  = \lim_n R(x_n)$. This finishes the proof of uniqueness.

\end{proof}



\begin{comment}
Claim 1: 任何 $F: \mathbb{R} \to \mathbb{R}$ 如果是 increasing 的, 它的不连续点一定可数. 从而连续点一定稠密. 
假设 R(x) 是一个在连续点上和 F(x) 相同的 distribution function. Take arbitrary $y$ where F is not ctn on $Y$.
\end{comment}




\section*{Finding intervals}
  Let $E\subset\mathbb{R}$ be a Lebesgue measurable subset with $m(E)>0$. Prove that for every $\alpha\in(0,1)$ there exists an (nonempty) bounded open interval $I$ such that $m(E\cap I)\ge\alpha m(I)$.
  \textit{Hint}: first reduce to the case when $E$ is bounded, then use outer regularity.
\begin{proof}
Let $\alpha \in (0,1)$ be arbitrary and fix it.\\
We first consider the case that $E$ is bounded. By outer regularity of Lebesgue measure, there exists an open set \( G \) such that

\[
E \subset G \quad \text{and} \quad m(G) - m(E) \leq  (1/\alpha-1) \; m(E)
\]
since $\alpha \in (0,1)$.
Then we have:
$$
m(G) \leq  1 / \alpha \; m(E)
$$
Note that in $\mathbb{R}$, an open set is just a countable disjoint union of open intervals. We write:
$$
G = \bigsqcup_{i\in \mathbb{N}} I_i
$$
Since $E \subset G$, we have:
$$
E = \bigsqcup_{i\in \mathbb{N}}  (I_i \cap E)
$$
Thus 
$$
m(E) = \sum_{i\in \mathbb{N}}  m(I_i \cap E) \geq \alpha \sum_{i\in \mathbb{N}} m(I_i)
$$
So \textbf{there must exist some $i$ such that $m(I_i \cap E) \geq \alpha \, m(I_i)$, otherwise contradicting with the ineq above.}\\
This finishes the proof of the bounded case.\\
The we consider the case when $E$ is unbounded. 
We can write
\[
E = \bigsqcup_{n\in \bZ} (E \cap (n, n+1]) 
\]
where each $E_n :=E \cap (n, n+1]$ is bounded.\\
We apply the case where $E$ is bounded, confirming that there is some interval $I$ such that $m(E_1\cap I)\ge\alpha m(I)$. By monotonicity of measure, we have $m(E \cap I) \geq m(E_1 \cap I) \geq \alpha m(I)$.
\end{proof}




\section*{So many differences}
  Let $E\subset\mathbb{R}$ be a Lebesgue measurable subset with $m(E)>0$.
  \begin{itemize}
  \item[(a)]
    Prove that the set
    \[
      E-E:=\{x-y\mid x,y\in E\}\subset\mathbb{R}
    \]
    contains a nonempty open interval centered at the
    origin. \textit{Hint}: use the previous exercise with $\alpha$
    large enough, together with the translation invariance of Lebesgue
    measure.
  \item[(b)]
    Prove that there exists $\epsilon>0$ such that $E\times E\subset\mathbb{R}^2$ intersects every line $y=x+t$ with $|t|<\epsilon$.
  \item[(c)]
    Let $C\subset\mathbb{R}$ be the middle-third Cantor set (so $m(C)=0$). Does $C-C$ contain a nonempty open interval centered at the origin?
  \end{itemize}  
\begin{proof}
\textbf{of (a): }\pic[0.65]{assets/hw3-image-20250131212105730.png}
\pic[0.7]{assets/hw3-image-20250131212120558.png}
\end{proof}

\begin{proof}
    \textbf{of (b):}\\
    Consider taking $\epsilon$ as the one in (a) where the interval contained in $E-E$ is $(-\epsilon, \epsilon)$, then the box $(-\epsilon, \epsilon) \times (-\epsilon, \epsilon)$ is contained in $E\times E$. It trivially follows that $E\times E\subset\mathbb{R}^2$ intersects every line $y=x+t$ with $|t|<\epsilon$, since the intercept of this line with $y$-axis is below $\epsilon$ and above $-\epsilon$.
    \pic[0.25]{assets/hw3-image-20250131212337897.png}
\end{proof}

\begin{solution}
        \textbf{of (c):} $C-C$ contain a nonempty open interval centered at the origin, and we will prove that one such interval is $(-1,1)$.
\begin{proof}
Recall the balanced ternary representation of $[-1/2, 1/2]$: $\forall x\in [-1/2, 1/2]$, there is a seq of $(a_n)_{n\in\mathbb{N}} $ in $\{-1,0,1\}$ s,t,
\[
x=\sum_{n=1}^\infty \frac{a_n}{3^n},\qquad a_n\in\{-1,0,1\},
\]
Thus every $x\in [-1,1]$ can be halved, ternary expanded and then doubled to recover:
\begin{align}
    x=2\sum_{n=1}^\infty \frac{a_n}{3^n} &= \sum_{n=1}^\infty \frac{2a_n}{3^n},\qquad a_n\in\{-1,0,1\} \\
    &= \sum_{n=1}^\infty \frac{b_n}{3^n},\qquad b_n\in\{-2,0,2\} 
\end{align}

And by the problem "The middle-thirds Cantor set", we learned that \[
    C=\{ \sum_{i=1}^\infty \frac{a_i}{3^i} \mid \text{ $a_i\in \{0,2\}$ for all $i\in \mathbb{N}$} \}.
  \]
Therefore we can write every number $x \in [-1, 1]$ into a difference of two $x,y \in C$, i.e. an element of $C-C$:
\begin{align}
    x &= \sum_{n=1}^\infty \frac{b_n}{3^n},\qquad b_n\in\{-2,0,2\} \\
    &= \sum_{n=1}^\infty \frac{p_n-q_n}{3^n},\qquad p_n,q_n\in\{-2,0,2\}\\
    &= \sum_{n=1}^\infty \frac{p_n}{3^n} - \sum_{n=1}^\infty \frac{q_n}{3^n},\qquad p_n,q_n\in\{-2,0,2\}
\end{align}
since each series converges independently. Here we let $p_n = 2, q_n = 0$ if $b_n = 2$; $p_n = 0, q_n = 2$ if $b_n = -2$, $p_n = 0, q_n = 0$ if $b_n = 0$.\\
Thus $x \in C-C$, so $[-1,1] \subset C-C$.
\end{proof}

\end{solution}


\section*{a holey set}
  Let $(x_n)_1^\infty$ be a countable dense sequence in $(0,1)$. For each $t>0$, consider the set
  \[
    A_t:=[0,1]\setminus\bigcup_{n=1}^\infty(x_n-2^{-n}t,x_n+2^{-n}t).
  \]
  \begin{itemize}
  \item[(a)]
  Prove that $A_t$ is a compact (possibly empty) subset of $\mathbb{R}$. Also prove that $A_t$ has empty interior, that is, $A_t$ contains no nonempty open set.
\item[(b)]
  Prove that $t\mapsto m(A_t)$ is continuous.
\item[(c)]
  Prove that there exists $t>0$ such that $m(A_t)=597/2025$.
\end{itemize}
\begin{proof}
   \textbf{ of a:}
   \pic[0.7]{assets/hw3-image-20250131221847429.png}
\end{proof}
\begin{proof}
   \textbf{ of b:}
Define for each $n\in \mathbb{N}$
$$
I_n(t) = (x_n - 2^{-n}t, x_n + 2^{-n}t)
$$
Then $$A_t = [0,1] \setminus \bigcup_{n=1}^\infty I_n(t)$$So
$$
m(A_t) = m([0,1]) - m(\bigcup_{n=1}^\infty I_n(t)) = 1 - m(\bigcup_{n=1}^\infty I_n(t))
$$

\textbf{Thus it suffices to show $t \mapsto m(\bigcup_{n=1}^\infty I_n(t))$ is continuous.}

Let $\epsilon >0$.
Let $t >0$.\\
We consider $p \in (t, t+\epsilon/2)$:

By set inclusion relation and measure's property, we have:
\begin{align}
    m(\bigcup_{n=1}^\infty I_n(p)) -    m(\bigcup_{n=1}^\infty I_n(t)) &=  m(\bigcup_{n=1}^\infty I_n(p) \setminus \bigcup_{n=1}^\infty I_n(t))
\end{align}
Since 
$$
(\bigcup_{n=1}^\infty I_n(p)) \setminus (\bigcup_{n=1}^\infty I_n(t)) \subset \bigcup_{n=1}^\infty (I_n(p) \setminus I_n(t))
$$
We have:
\begin{align}
      m(\bigcup_{n=1}^\infty I_n(p)) -    m(\bigcup_{n=1}^\infty I_n(t))  &\leq m(\bigcup_{n=1}^\infty (I_n(p) \setminus I_n(t)))\\
      & \leq \sum_{n=1}^\infty (m(I_n(p)) - m(I_n(t)))\\
      &= \sum_{n=1}^\infty 2 \cdot 2^{-n}(p-t) \\
      & = 2(p-t) \\
      & \leq \epsilon
\end{align}

Similarly for $p \in (t-\epsilon/2,t)$, we get the same bound. This finishes the proof pf (b).
\end{proof}
\begin{proof}
 \textbf{   of c:\\}
 We use the same notation of $I_n(t)$ as in (b). We have:
 $$
 m(\bigcup_{n=1}^\infty I_n(t) )  \leq  \sum_{n=1}^\infty m(I_n(t)) = \sum_{n=1}^\infty m(I_n(t)) =  \sum_{n=1}^\infty2\cdot 2^{-n} t = 2t.
 $$
 So by choosing $t:= 1/6$, we have $m(A_t) = 1 -  m(\bigcup_{n=1}^\infty I_n(t) ) \geq 2/3$.
 And by choosing $t:= 4$, $I_1(t)$ covers an interval of length $4$, so $A_t = \varnothing$, $m(A_t) = 0$.
 By intermediate value theorem, there exists some $t \in(1/6, 4)$ such that $m(A_t)=597/2025$.
\end{proof}


\section*{a Cantor measure}
  (A Cantor measure.)
  Let $E\subset \mathbb{R}$ be a nonempty compact set with the following property: for every $x\in E$ and every $\epsilon>0$, the set $(x-\epsilon,x)\cup(x,x+\epsilon)$ has nonempty intersection with both $E$ and $E^c$. Prove that there exists a Borel measure $\mu$ on $\mathbb{R}$ with the following properties:
  \begin{itemize}
  \item[(i)]
    if $I\subset \mathbb{R}$ is a nonempty open interval, then $\mu(I)>0$ iff $I\cap E\ne\emptyset$.
  \item[(ii)]
    $\mu(\{x\})=0$ for all $x\in\mathbb{R}$; 
  \item[(iii)]
    $\mu(\mathbb{R})=0.597597597\dots$.
  \end{itemize}
  \textit{Hint}: set $\mu=\mu_F$, where $F$ is a distribution function whose graph is similar to the Devil's staircase above.
 
\begin{proof}
   Write $T:=0.597597597\dots$\\
    Since $E$ is compact, $E^c$ is open.  Also, since $E$ is compact, it takes min and max element.\\
    Thus we consider $A:= E^c \cap (\min E, \max E)$, this is an open set. We know any open set in $\mathbb{R}$ is a countable disjoint union of open intervals, so $A:=E^c \cap (\min E, \max E)  = \bigsqcup_{n=1}^\infty I_n$ for some disjoint intervals $I_1 = (a_1,b_1), I_2=(a_2,b_2), \cdots$ .\\\\
Now we construct a function $G: A \rightarrow [0,T)$ by sending $G(x) = \sum_{b_i \leq a_N} \frac{T}{2^n}$, for $x \in I_N$.\\
This is an \textbf{increasing step function} since, each $I_n$ is disjoint and on a fixed interval $I_N$, the number of $b_i$ that its $a_N$ surpasses is constant. And suppose $y>x$ is on $I_M$, we have must $G(y) \geq G(x)$ because he number of $b_i$ that $a_M$ surpasses is at least at many as that $a_N$ surpasses. \\
And for each $x \in A$, we have \textbf{$G(x) < T$}, by geometric series\\
Then we construct $F$ out of $G$, define:
$$
F:= \begin{cases}
    0, \quad \quad x \leq \min E\\
    \inf {\{G(y) \mid y \geq x, y\in A} \}, \quad \quad x \in (\min E, \max E)\\
    T, \quad \quad x \geq \max E\\
    
\end{cases}
$$
\textbf{$F$ is increasing:} It is constant on $(-\infty, \min E) \cup (\max E, \infty)$ and is the infimum of $G(y)$ with $y\geq x$ on $(\min E, \max E)$. Since $G$ is increasing, $F$ is also increasing.\\
\textbf{$F$ is right continuous:} It suffices to prove the right-continuity of $F$ on $x \in E^c \cap (\min E, \max E)$. \\
Fix $x_0 \in E^c \cap (\min E, \max E)$.\\
Let $\epsilon > 0$. \\
Let $k \in \mathbb{N}$ such that $\epsilon > \frac{T}{2^k+1}$.\\
We define for each $y$, $S_y := \{b_i \mid x_0 \leq  b_i  \leq y\}$ as the set of all $b_i$ (right endpoint of $I_k$) that is witin $x_0$ and $y$. Note that $I_z \subset I_y$ for all $y > z$.\\
Consider $y_1 := \min (\{b_1, \cdots, b_k\} \setminus S_x)$.\\
Then for all $y \in (x_0,y_1)$, we have:
$$
F(y)  \leq F(x_0)  + \sum_{i=k}^\infty \frac{T}{2^i} \leq F(x_0) + \epsilon
$$
By defining $\delta : = y_1 - x_0$, we have shown the right continuity of $F$.\\
(By dual reason, we can prove that $F$ is left continuous. So $F$ is actually continuous.)
Above, we have shown that $F$ is a distribution function.\\\\
Now let $\mu_F$ be the Lebesgue-Stieljes measure associated with $F$. We will prove for the three properties above:\\
\begin{itemize}
    \item[(i)] Let $\{x\}$ be a singleton set in $\mathbb{R}$, for each $n\in \mathbb{N}$, we can construct an h-intervals seq of covering of $\{x\}$ by $(x-1/n,x]$ as the first covering set and $\varnothing$ as all other covering sets.\\
Then by the definition of $\mu_F$, we have:
$$
\mu_F(\{x\}) = \inf_{n\in\mathbb{N}} (F(x) - F(x-1/n))
$$
By continuity, it shows that $\mu_F(\{x\}) = 0$.
\item[(ii)] $$ \mu_F(\mathbb{R}) = \lim_{x \to \infty} F(x)  -  \lim_{x \to -\infty}  F(x)  = T-0 = T = 0.597597597\cdots$$

\item[(iii)] Let $I = (a,b)$ be a nonempty open interval.\\
Suppose $\mu(I)>0$, then $F(b) - F(a) > 0$, so by definition of $G$, some must be at least two different intervals $I_{n_1}$, $I_{n_2}$ in $A$ such that for some $x,y \in (a,b)$, $x\in I_{n_1}$ and $y \in I_{n_2}$, thus $\exists$ some $e \in E$ such that $e \in (x,y)$. Thus $I\cap E\ne\emptyset$.\\
Suppose $I\cap E\ne\emptyset$. Let $e \in E \cap I$. Since $\forall \epsilon >0$, $(x-\epsilon,x) \cup (x,x+\epsilon)$ has nonempty intersection with both $E$ and $E^c$, $e$ has some open neighborhood $B_\epsilon(e) \subset I$, intersecting two different $I_N$, $I_M \subset A$. Take $n \in I_N$, $m \in I_m$. Then $F(m) - F(n) = G(m) - G(n) > 0$, so $\mu_F(E) \geq F(m) -F(n) > 0$ by monotinicity of measure.\\
This finishes the proof.


\end{itemize}

\end{proof}



`