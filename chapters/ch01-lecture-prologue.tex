\chapter*{Prologue}
\noindent In this Prologue, we will only state important prerequisites that we should know without proving them, for the sake of time saving.
\section{set theory}
\begin{definition}{$\limsup$ and $\liminf$ of of a sequence of sets}
$$
\limsup E_n = \intsec_{k=1}^\infty \bigcup_{n=k}^\infty E_n
$$
$$
\liminf E_n = \bigcup_{k=1}^\infty \intsec_{n=k}^\infty E_n
$$
\end{definition}
limsup represent the elements that occur in infinitely many $E_n$; and liminf represent the elements that occur in all but finitely many $E_n$.


    
\begin{definition}{Cartesian product}
Let $A$ be some index set, define
    $$
    \prod_{\alpha \in A} X_\alpha := \{ f: A\rar \bigcup_{\alpha \in A} X_\alpha \mid f(\alpha) \in X_\alpha \forall \alpha   \}
    $$
\end{definition}
Notice that if all $X_\alpha = Y$ for a fixed $Y$, then we have $\prod_{\alpha \in A} X_\alpha = Y^A$.

\section{ordering}
\begin{definition}{partial ordering, linear ordering}
    A partial ordering on a nonempty set $X$ is a relation $R$ that is 
    \begin{enumerate}
        \item transitive: $xRy, yRz \implies xRz$
        \item reflexitive: $xRx \; \forall x$
        \item antisymmetric: $xRy,yRx \implies x=y$
    \end{enumerate}
    A linear ordering $R$ is a partial ordering that also satisfies:
    \begin{enumerate}[resume]
        \item strongly connected: $\forall x,y \in X$, either $xRy$ or $yRx$.
    \end{enumerate}
\end{definition}



\begin{axiom}{The Hausdorff Maximal Principle}\label{The Hausdorff Maximal Principle}
Every partially ordered set has a maximal linearly ordered subset. (i.e. there is a linearly ordered subset s.t. no other subset properly including it is linearly ordered.)
\end{axiom}

\begin{definition}{maximal/minimal element}
    A maximal element $x \in X$ means that no other element is greater than it. (notice: \textbf{It is possible that there exists another element that cannot be compared with it. The definition only says they cannot be greater than it.})\\
    Dually can define minimal element.
\end{definition}
\begin{axiom}{Zorn's Lemma} \label{Zorn's Lemma}
If $X$ is a partially ordered set and every linearly ordered subset has an upper bound, then $X$ has a maximal element.
\end{axiom}
\begin{remark}
    Zorn's Lemma and the Hausdorff Maximal Principle can imply each other.
\end{remark}

\begin{definition}{well ordering}
If $X$ is linearly ordered and every nonempty subset has a \textbf{unique minimal element}, then $X$ is said to be \textbf{well ordered}.
\end{definition}


\begin{lemma}{well ordering principle}
    Every nonempty set can be well ordered.
\end{lemma}

\begin{corollary}{Axiom of choice}
    The Cartesian product of a nonempty collection of nonempty sets is nonempty.
\end{corollary}

\section{cardinality}
\begin{proposition}
    Let $X$ be any set. We must have $\card(\cP(X)) > \card(X)$
\end{proposition}

\begin{definition}
 $$\mathfrak{c} := \card(\mathbb{R})$$
\end{definition}

\begin{proposition}
    \begin{equation}
        \mathcal{A}rd(\cP(\bN)) = \mathfrak{c}
    \end{equation}
\end{proposition}
\begin{proof}
    Using base-2 expansion, can create a bijective function from $\cP(\bN)$ to $[0,1]$.
\end{proof}


\begin{proposition}
    Given $f: X \rar [0,\infty)$, $A:= \{ f(x)>0 \}$, if $A$ is uncountable, then $\sum_{x \in X}f(x) = \infty$
\end{proposition}


\section{topology in metric space}
\begin{definition}{dense, nowhere dense, separable}
We say $E \sub X$ is dense in $X$ if $\overline{E} = X$.\\
We say $E \sub X$ is nowhere dense in $X$ if 
$\overline{E} = \emptyset $.\\
We say $X$ is separable if it has a countable dense subset.
\end{definition}


\begin{proposition}
    TFAE for metric spaces:
    \begin{enumerate}
        \item $x \in \overline{E}$.
        \item Every open ball centered at $x$ has nonempty intersection with $E$.
        \item There is a seq in $E$ converging to $x$.
    \end{enumerate}
\end{proposition}


\begin{definition}{Cauchy, complete}
    A sequence in metric space $X$ is said to be Cauchy if $d(x_n,x_m) \rar 0$ as $n,m \rar \infty$.\\
    A subset $E\sub E$ is said to be complete if every Cauchy sequence in $E$ converges.
\end{definition}

\begin{proposition}
    Complete is a stronger condition than closed, in any metric space. And a closed subset of a complete metric space is complete.
\end{proposition}

\begin{theorem}
    TFAE for metric spaces:
    \begin{enumerate}
        \item $E$ is complete and totally bounded.
        \item $E$ is compact.
        \item $E$ is sequentially compact. (Every sequence in $E$ has a subseq converging to some point in $E$.)
    \end{enumerate}
\end{theorem}