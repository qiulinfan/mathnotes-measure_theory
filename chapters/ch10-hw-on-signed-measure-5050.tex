\chapter{on signed measure (50/50)}

\section{Three real Banach spaces and a fake one}
  \begin{itemize}
  \item[(a)]  Let 
    \[
      \ell^\infty_0 := \{ a= (a_1, a_2, \cdots)\mid a_i\in \mathbb{R}, \lim_{n\to \infty} a_n =0\}.
    \]Prove that $(\ell^\infty_0, \| \cdot\|_\infty)$, where $\|a\|_\infty=\sup_n|a_n|$, is a Banach space.
  \item[(b)]   Let
    \[
      C^0_b(\mathbb{R}) := \{ f: \mathbb{R} \to \mathbb{R} \mid \text{$f$ is continuous and bounded}\}.
    \]
    Prove that $(C^0_b(\mathbb{R}), \| \cdot\|_\infty)$, where $\|f\|_\infty=\sup_{x\in \mathbb{R}}|f(x)|$, is a Banach space.
  \item[(c)]   Let
    \[
      C^0_0(\mathbb{R}) := \{ f: \mathbb{R}\to  \mathbb{R} \mid \text{$f$ is continuous, $\lim_{x\to\pm\infty} f(x) =0$}\}.
    \]
    Prove that $(C^0_0(\mathbb{R}), \| \cdot\|_\infty)$, where $\|f\|_\infty=\sup_{x\in\mathbb{R}}|f(x)|$, is a Banach space.
    \item[(d)] Recall that 
    \[
      C^0_c(\mathbb{R}) = \{ f\colon \mathbb{R} \to \mathbb{R} \mid \text{$f$ is continuous and $f=0$ outside a bounded set}\}.
    \]
    Show that $(C^0_c(\mathbb{R}), \| \cdot\|_\infty)$, where $\|f\|_\infty=\sup_{x\in \mathbb{R}}|f(x)|$, is not a Banach space.
\end{itemize}



\begin{proof}
 \textbf{of (a): }  Since we showed in class that \[
\ell^\infty =L^\infty (\mathbb{N}, \mathcal{P}(\mathbb{N}), \mu_{counting} )
\] and $L^\infty$ spaces are Banach, $\ell^\infty$ is Banach.\\
Thus it suffices to show that $\ell^\infty_0$ is closed in $\ell^\infty$, since a closed subset of a complete metric space is complete.\\
Let \((a^{(k)})_{k=1}^\infty\) be a sequence in \(\ell^\infty_0\) converging in norm to \(a \in \ell^\infty\), i.e.,
\[
\|a^{(k)} - a\|_\infty \to 0
\]
Let $\varepsilon > 0$.\\
Since $\|a^{(k)} - a\|_\infty \to 0$, there exists \(K\) such that for all \(k \geq K\),
\[
   \| a^{(k)} - a||  =  \sup_n |a_n^{(k)} - a_n| < \frac{\varepsilon}{2}  
\]
This implies that \[
\forall n, |a_n^{(K)} - a_n| < \varepsilon
\]Since \(a^{(K)} \in \ell^\infty_0\), \(a_n^{(K)} \to 0\) as \(n \to \infty\). Thus there exists $N \in \mathbb{N}$ s.t. for all $n\geq N$, \[
|a_n^{(K)} | \leq \frac{\varepsilon}{2}
\]
Then for all $n\geq N$, we have:
\[
|a_n| \leq |a_n - a_n^{(K)}| + |a_n^{(K)}| < \varepsilon
\]
This shows that \[
\lim_{n\to \infty} |a_n  | <\epsilon
\]
Since $\varepsilon>0$ is arbitrary, this implies \[\lim_{  n \to \infty} a_n = 0\]
Hence \(a \in \ell^\infty_0\). So \(\ell^\infty_0\) is closed in \(\ell^\infty\), thus itself Banach.
\end{proof}

\begin{proof}
    \textbf{of (b):} 
Let \((f_n)_{n\in \mathbb{N}}\) be a Cauchy seq in \((C^0_b(\mathbb{R}), \|\cdot\|_\infty)\), then
\[\forall \varepsilon > 0, \exists N \in \mathbb{N}\;\; s.t. \|f_n - f_m\|_\infty  =
\sup_{x \in \mathbb{R}} |f_n(x) - f_m(x)| < \varepsilon
\]In particular, for each fixed \(x \in \mathbb{R}\), \((f_n(x))_{n\in\mathbb{N}}\) is a Cauchy sequence in \(\mathbb{R}\), hence converges (since \(\mathbb{R}\) is complete). So we can define the pointwise limit: \[
f(x) := \lim_{n \to \infty} f_n(x)
\]
\textbf{Claim 1: $f_n\to f$ in $\| \cdot\|_\infty$.}\\
Let $\varepsilon> 0$.\\
Since \((f_n)\) is Cauchy in \(\|\cdot\|_\infty\), there exists \(N\) such that: \[
\|f_n - f_m\|_\infty < \varepsilon, \quad \forall n, m \geq N
\]
Fix \(m \geq N\), and let \(n \to \infty\). For each \(x\), we get: \[
|f_n(x) - f_m(x)| < \varepsilon\;\;  \forall n\implies\lim_{n\to \infty}|f_n(x) - f_m(x)|  =  |f(x) - f_m(x)|\leq \varepsilon
\]
Since this is true for each $x\in \mathbb{R}$, we obtain: \[
\|f - f_m\|_\infty \leq \varepsilon, \quad \text{for all } m \geq N
\]
Since $\varepsilon>0$ is arbitrary, this shows that \[
\lim_{n\to \infty} \| f- f_n\|_{\infty} = 0
\]
\textbf{Claim 2: $f\in (C^0_b(\mathbb{R}), \|\cdot\|_\infty)$.}\\
Since \(\lim_{n\to \infty} \| f- f_n\|_{\infty} = 0\), it also implies that the convergence is uniform.\\
We know the uniform limit of continuous functions is continuous, so $f$ is continuous. It remains to show \(f\) is bounded, and this directly follows from the uniform convergence. We take $\varepsilon = 1$. We have proved that there exists $N$ s.t. for all $m\geq N$, 
\[
\|f - f_m\|_\infty \leq 1
\]
Thus \[
\sup_{x\in \mathbb{R}} |f(x) | \leq \sup_{x\in \mathbb{R}} |f_N(x) |  + 1
\]
Since \(f_n \in C^0_b(\mathbb{R})\)), it is bounded, thus \[
\sup_{x\in \mathbb{R}} |f(x) |  < \infty
\]showing that the limit function is bounded. This finishes the proof that \(f \in C^0_b(\mathbb{R})\).
Thus, every Cauchy seq in \((C^0_b(\mathbb{R}), \|\cdot\|_\infty)\) converges in \((C^0_b(\mathbb{R}), \|\cdot\|_\infty)\), i.e. it is Banach.
\end{proof}


\begin{proof}
    \textbf{of (c):}
    Let \((f_n)_{n\in \mathbb{N}}\) be a Cauchy seq in \((C^0_0(\mathbb{R}), \|\cdot\|_\infty)\), then for each fixed \(x \in \mathbb{R}\), \((f_n(x))_{n\in\mathbb{N}}\) is a Cauchy sequence in \(\mathbb{R}\), so for the same reason as (b), we can define the pointwise limit: \[
f(x) := \lim_{n \to \infty} f_n(x)
\]And for the same reason as (b), we get  \[
f_n \to f \text{ in }\|\cdot\|_\infty
\]
which also implies that the pointwise convergence is uniform. Since each $f_n$ is continuous, the uniform limit $f$ is continuous.\\
Thus it suffices to show that $\lim_{x\to\pm\infty} f(x) =0$.\\
Let \(\epsilon > 0\). Since \(f_n \to f\) uniformly, there exists \(N\) such that for all \(n \ge N\), \(\|f_n - f\|_\infty < \epsilon/2\). Also, since \(f_N \in C^0_0(\mathbb{R})\), there exists \(M > 0\) such that \(|f_N(x)| < \epsilon/2\) for all \(|x| > M\).\\
Then for \(|x| > M\), \[
|f(x)| \le |f(x) - f_N(x)| + |f_N(x)| < \epsilon/2+ \epsilon/2 < \epsilon
\]
So \(\lim_{x\to\pm\infty} f(x) = 0\), i.e., \(f \in C^0_0(\mathbb{R})\). Thus, every Cauchy seq in \((C^0_0(\mathbb{R}), \|\cdot\|_\infty)\) converges in \((C^0_0(\mathbb{R}), \|\cdot\|_\infty)\), i.e. it is Banach.
\end{proof}


\begin{proof}
    of (d): 
We consider a continuous (smooth actually) function \(\phi : \mathbb{R} \to \mathbb{R}\) with \(\operatorname{supp}(\phi) =  [0,2]\) (here we take the closure):\[
\phi(x)
:=\begin{cases}
\exp\!\Bigl(\!-\frac{1}{x\,(2 - x)}\Bigr), 
& 0 < x < 2,\\
0, & \text{otherwise}
\end{cases}
\]
\pic[0.35]{assets/hw9-Screenshot 2025-03-27 at 20.28.30.png}
This function reaches its maximum at $x = 1$, \[
\|\phi \|_\infty = \frac{1}{e}
\]
For each integer \(n \ge 1\), define \[
   \phi_n(x) = \phi\!\bigl(x - n\bigr)
\]
Then each \(\phi_n\) is also continuous, and \(\operatorname{supp}(\phi_n) =[n,n+2]\).\\
Consider  the sequence $(S_N)_1^\infty$, defined as: \[
   S_N(x) := \sum_{n=1}^{N} 2^{-n}\,\phi_n(x)
\]
Then each \(S_N \in C^0_c(\mathbb{R})\), since finite sum of continuous functions is also continuous, and $\operatorname{supp} (S_N) = [1,N+2]$, thus each $S_N \in C^0_c(\mathbb{R})$.\\
\textbf{Claim: $(S_N)_1^\infty$ is Cauchy in the sup norm.}\\
This is because for each (WLOG)  \(M > N\in \mathbb{N} \),  \begin{align*}
      \|S_M - S_N\|_{\infty}  &= 
    \Bigl\|\sum_{n=N+1}^{M} \frac{1}{2^n}\phi_n\Bigr\|_{\infty} \\ &\leq  
    \sum_{n=N+1}^{M} \frac{1}{2^n} \|\phi\|_{\infty} \\&\leq  
    \sum_{n=N+1}^{\infty} \frac{1}{2^n} \|\phi\|_{\infty}\\& =  \sum_{n=N+1}^{\infty} \frac{1}{2^n e}  = \frac{1}{2^N e} \overset{N\to \infty}{\longrightarrow} 0
\end{align*}
Thus for arbitrary $\varepsilon >0$, exists $K\in\mathbb{N}$ s.t. for all $M,N \geq K$, $  \|S_M - S_N\|_{\infty}  < \varepsilon$. And by same reason as (b), (c), $(S_N)_1^\infty$ converges by $\| \cdot \|_\infty$ into its pointwise limit: \[
  S(x) := \sum_{n=1}^{\infty} 2^{-n}\,\phi_n(x)
\]
But $S(x)$ does not have compact support, $\operatorname{supp}(S) = [0,\infty)$. So \(   S \notin C^0_c(\mathbb{R})\). This serves as a counterexample showing that \(C^0_c(\mathbb{R})\) is not Banach.
\end{proof}



 \section{$\nu^+(E), \nu^-(E),|\nu|(E)$ 的formula from original $\nu$}
  Let $\nu$ be a signed measure on $(X,\mathcal{A})$, and $E\in\mathcal{A}$. Prove the following statements:
  \begin{itemize}
  \item[(i)]$\nu^+(E)= \sup\{ \nu(F)\mid: F\in \mathcal{A}, F\subset E\}$, and   $\nu^-(E)= -  \inf\{ \nu(F)\mid  F\in \mathcal{A}, F\subset E\}$; 
  \item[(ii)]   $|\nu|(E)= \sup\{ \sum_{i=1}^N |\nu(E_i)|\mid N\in\mathbb{N}, \, E= \bigcup_{i=1}^N E_i \text{ disjoint union}\}$;
  \item[(iii)]   $|\nu|(E)\ge|\nu(E)|$. In the case $\nu$ finite, it achieves equality iff $E$ is positive or negative for $\nu$.
%    \sup\{ \sum_{i=1}^N |\nu(E_i)|\mid N\in\N, \, E= \bigcup_{i=1}^N E_i \text{ disjoint union}\}$. 
  \end{itemize}

\begin{proof}
    \textbf{of (i):} By the Hahn decomposition theorem, we can take a Hahn decomposition \(X = P \sqcup N\) where \[
  \nu(A) \ge 0 \quad\text{for all }A\subset P,\qquad
  \nu(B) \le 0 \quad\text{for all }B\subset N
  \]
Fix $E\in\mathcal{A}$. By Jordan decomposition we have \[
  \nu^+(E) = \nu(E\cap P)
  \]Fix \(F\subset E\), we have:\[
  F = (F \cap P ) \sqcup (F \cap N)
  \]
  Since $\nu (F\cap N) \leq 0$, we have:  \[
  \nu(F) \le \nu\bigl(F\cap P\bigr) \;\le\; \nu\bigl(E\cap P\bigr) = 
  \nu^+(E)
  \]Since $F$ is arbitrary, this shows:  \[
  \sup\{\nu(F) \mid F \subset E\} \leq 
  \nu^+(E)
  \]
On the other hand, taking \(F = E\cap P\subset E\), we get \[
  \nu(F) =\nu\bigl(E\cap P\bigr) =  \nu^+(E)
  \]
  Hence \[
  \sup\{\nu(F) \mid F \subset E\} \geq 
  \nu^+(E)
  \]
  Combining both inequalities gives \[\nu^+(E) =  \sup\{\nu(F)\mid F\subset E\}\]
Similarly,  since $\nu (F\cap P) \geq 0$ and $\nu(F) = \nu (F\cap  P) + \nu\bigl(F\cap N\bigr)$, we have  $  \nu(F) \geq \nu\bigl(F\cap N\bigr) $. And Since $\nu \bigl (E\cap N\bigr)  =  \nu(F \cap N) +  \nu((E \setminus F) \cap N) $ with $ \nu((E \setminus F) \cap N)\leq 0$, we get $\nu\bigl(F\cap N\bigr) \ge \nu\bigl (E\cap N\bigr)$. \\
Putting it together:
\[
  \nu(F) \geq \nu\bigl(F\cap N\bigr) \ge \nu\bigl (E\cap N\bigr) = 
 -  \nu^-(E)
  \]Since $F$ is arbitrary, this shows:  \[
  \inf\{\nu(F) \mid F \subset E\} \geq  -  \nu^-(E)
  \]
On the other hand, taking \(F = E\cap N\subset E\), we get \[
  \nu(F) = \nu\bigl(E\cap N\bigr) = -   \nu^-(E)
  \]
  Hence \[
  \inf\{\nu(F) \mid F \subset E\} \leq  -  \nu^-(E)
  \]
  Combining both inequalities gives \[\nu^-(E) =  - \inf\{\nu(F)\mid F\subset E\}\]

\end{proof}
\begin{proof}
    \textbf{of (ii):}
Let $E\in \mathcal{A}$.
By def of total variation measure,  \[
  |\nu|(E) = 
  \nu^+(E)+\nu^-(E)
  \]
One direction of the equality is easy. Take a Hahn decomposition \(X = P \sqcup N\) where \[
  \nu(A) \ge 0 \quad\text{for all }A\subset P,\qquad
  \nu(B) \le 0 \quad\text{for all }B\subset N
  \]
Then by Jordan decomposition, we have:\[
  \nu^+(E) = \nu(E\cap P),\quad \nu^-(E) =  - \nu(E\cap N)
  \]
  So by taking $E_1 : = E \cap P$, $E_2 : = E \cap N$, we have: \[
    |\nu|(E) = 
  \nu^+(E)+\nu^-(E)  = \nu(E_1)  + \nu(E_2)
  \]This shows that \[
|\nu|(E) \leq 
\sup\bigl\{\sum|\nu(E_i)|\bigr\}
\]
  And for the other direction, for any disjoint measurable partition \(E = \bigcup_{i=1}^N E_i\), we have   \[
  |\nu(E_i)| = 
  \bigl|\nu^+(E_i) - \nu^-(E_i)\bigr| \leq 
  \nu^+(E_i) + \nu^-(E_i) = 
  |\nu|(E_i)
  \]
  Therefore \[  \sum_{i=1}^N \bigl|\nu(E_i)\bigr| \leq 
  \sum_{i=1}^N \bigl|\nu|(E_i) = 
  |\nu|\Bigl(\bigcup_{i=1}^N E_i\Bigr) = 
  |\nu|(E)
  \]
  since \(\lvert \nu\rvert\) is a p.m. and the \(E_i\)'s are disjoint. Thus \[
  \sup\Bigl\{\sum_{i=1}^N |\nu(E_i)|\Bigr\} \leq |\nu|(E)  \]
Combining the two inequalities gives
\[
|\nu|(E) = 
\sup\Bigl\{\sum_{i=1}^N \bigl|\nu(E_i)\bigr|\Bigm\vert\;N\in\mathbb{N},E=\bigcup_{i=1}^N E_i \text{ disjoint}\Bigr\}
\]
proving the statement.
\end{proof}

\begin{proof}
    \textbf{of (iii):}
    Let $E \in \mathcal{A}$.
The ineq \(|\nu|(E)\geq |\nu(E)|\) follows from triangular ineq on $\mathbb{R}$:
\[
|\nu(E)| = 
\bigl|\nu^+(E) - \nu^-(E)\bigr| \leq 
\nu^+(E) + \nu^-(E) = 
|\nu|(E)
\]
Now we assume \(\nu\) is finite (i.e.\ \(\lvert \nu\rvert(X) < \infty\)). The equality condition \(|\nu(E)| =  |\nu| (E)\) is detailedly:
\[
\bigl|\nu^+(E) - \nu^-(E)\bigr| = 
\nu^+(E) + \nu^-(E)
\]
Since \(\lvert \nu\rvert(X) < \infty\), $\nu^+(E) <\infty$ and $\nu^-(E) <\infty $.\\
Case 1: $\nu^+(E) \geq \nu^-(E)$, then \begin{align*}
    \bigl|\nu^+(E) - \nu^-(E)\bigr| = 
\nu^+(E) + \nu^-(E)& \iff \nu^+(E) - \nu^-(E) = 
\nu^+(E) + \nu^-(E) \\
&\iff -\nu^-(E) = \nu^-(E) \\
&\iff \nu^-(E) = 0 \\
& \iff E \subset P 
\end{align*}
Case 2: $\nu^+(E) < \nu^-(E)$, then  \begin{align*}
    \bigl|\nu^+(E) - \nu^-(E)\bigr| = 
\nu^+(E) + \nu^-(E)& \iff \nu^-(E)  - \nu^+(E) = 
\nu^+(E) + \nu^-(E) \\
&\iff -\nu^+(E) = \nu^+(E) \\
&\iff \nu^+(E) = 0 \\
& \iff E \subset N
\end{align*}
Therefore the equality condition implies that $E$ must be positive or negative for $\nu$; and in converse, if $E$ is neither positive nor negative set, in either case it implies \(|\nu(E)| \not=  |\nu| (E)\), thus when $\nu$ finite, \(|\nu(E)| =  |\nu| (E)\) iff $E$ is positive or negative for $\nu$.
\end{proof}






\section{Signed integrals}
  Let $\nu$ be a signed measure on $(X, \mathcal{A})$.
  \begin{itemize}
  \item[(i)]Prove that $\int g \, d |\nu|= \int g \, d \nu^+ + \int g \, d\nu^-$ for $g\in L^+(|\nu|)$ or $g\in L^1(|\nu|)$.
  \item[(ii)]Define $L^1(\nu)= L^1(\nu^+)\cap L^1(\nu^-)$. Prove that $L^1(\nu)=L^1(|\nu|)$. %(Hint: One first needs to show that $\int g \dd |\nu|= \int g \dd \nu^+ + \int g \dd\nu^-$ for measurable functions $g$.) 
  \item[(iii)] Define $\int f \, d\nu  = \int f\, d\nu^+ - \int f \, d\nu^-$ for $f\in L^1(\nu)$. 
    Prove that if $f\in L^1(\nu)$, then 
    \begin{equation*}
      \left| \int f\, d\nu \right| \le \int |f| \, d |\nu|
    \end{equation*}
  \item[(iv)]Suppose that $\nu$ is a finite measure (i.e. $\nu^{\pm}(X)<\infty$.) Prove that if $E\in \mathcal{A}$, then 
    \[
      |\nu|(E)=\sup \left\{ \left| \int_E f \,d\nu \right|\mid  \|f\|_\infty \le 1 \right\}.
    \]
  \end{itemize}

\begin{proof}
    \textbf{of (i)}:
    Take a Hahn decomposition $X = P \sqcup N $.\\
    Then by Jordan decomposition, \[
    \nu^+ (E) = \nu(E \cap P), \quad     \nu^- (E) = -\nu(E \cap N),\quad  \forall E \subset X
    \]
    and therefore $P$ is null set of $\nu^-$ and $N$ is null set of $\nu^+$. So on $P$,  $|\nu| = \nu^+ + \nu^-  = \nu^+$; on $N$,  $|\nu| = \nu^+ + \nu^-  = \nu^-$
    Thus, suppose $g\in L^+(|\nu|)$,  \begin{align*}
        \int g \, d |\nu|= \int _X g\, d|\nu|   & =  \int_P g\, d |\nu| + \int_N g\, d |\nu|\quad \text{since } X = P \sqcup N \\
        & = \int_P g\, d \nu^+ + \int_N g\, d \nu^-  \quad \text{since } |\nu| = \nu^+,\nu^- \text{ on }P,N    \\
        & = \int g\, d \nu^+ + \int g\, d \nu^- \quad \text{since }N,P \text{ is null for } \nu^+, \nu^-
    \end{align*}
    Suppose $g\in L^1(|\nu|)$, then  \begin{align*}
        \int g \, d |\nu|= \int _X g\, d|\nu|   & =   \int_X g^+ \, d|\nu| -  \int_X g^-\, d|\nu| \quad \text{by def}  \\
        & =\bigg( \int_P g^+\, d \nu^+ + \int_N g^+\, d \nu^-\bigg) -  \bigg( \int_P g^-\, d \nu^+ + \int_N g^-\, d \nu^-\bigg)\quad \text{since } X = P \sqcup N \\
        & = ( \int_P g^+\, d \nu^+  -  \int_P g^-\, d \nu^+\bigg) + \bigg( \int_N g^+\, d \nu^-- \int_N g^-\, d \nu^-\bigg) \\
        & = \int_P g \, d\nu^+  +  \int_N g \, d\nu^- \quad \text{since $g\in L^1(|\nu|)$} \\
        & =\int g \, d\nu^+  +  \int g \, d\nu^- \quad \text{since }N,P \text{ is null for } \nu^+, \nu^-
    \end{align*}
    This finishes the proof.
\end{proof}

\begin{proof}
    \textbf{of (ii):}
    WTS: $L^1(\nu^+)\cap L^1(\nu^-)=L^1(|\nu|)$.\\
(\(\Rightarrow\)): Suppose \(f \in L^1(|\nu|)\), i.e. \(\int |f| \, d|\nu| < \infty\).  \\
Let $\phi$ be arbitrary positive-valued simple function: \[
 \phi = \sum_{j=1}^n a_j \chi_{E_j}
\]
then \[
\int \phi \, d |\nu| =  \sum_{i=1}^n a_j |\nu| (E_j)
\]
Since $\nu^-(E_j),\nu^+ (E_j) \leq   \nu^+ (E_j) + \nu^-(E_j) =   |\nu|  (E_j)$ for each $j$, we have \[
\int \phi \, d \nu^+ ,\int \phi \, d \nu^- \leq    \int \phi \, d |\nu| 
\]
Since $\phi$ is arbitrary, we have
$$\int |f| \,d \nu^+ =    \sup \{   \int \phi \,d \nu^+ : 0\leq \phi \leq |f|, \phi  \text{ simple} \}  \leq   \sup \{   \int \phi \,d |\nu| : 0\leq \phi \leq |f|, \phi  \text{ simple} \} =  \int |f| \,d |\nu|$$
Same for $\nu^-$. This shows that \[
\int |f| \, d\nu^+ , \int |f| \, d\nu^-  \leq \int |f| \, d|\nu|  < \infty
\]
i.e. $f\in L^1(\nu^+)$ and $f\in L^1(\nu^-)$, so $f\in L^1(\nu^+)\cap L^1(\nu^-)$.\\
Thus $$L^1(|\nu|) \subset   L^1(\nu^+)\cap L^1(\nu^-) $$
(\(\Leftarrow\)): Suppose \(f \in L^1(\nu^+) \cap L^1(\nu^-)\), i.e. \[
\int |f| \, d\nu^+ < \infty, \quad \int |f| \, d\nu^- < \infty
\]
Since $|f| $ is non-negative and measurable, we have $|f| \in L^+(|\nu|)$. Thus by (i) we have:
\[
\int |f| \, d|\nu| = \int |f| \, d\nu^+ + \int |f| \, d\nu^- < \infty
\]
So \(f \in L^1(|\nu|)\).\\
This shows that:  $$   L^1(\nu^+)\cap L^1(\nu^-) \subset L^1(|\nu|) $$
Combining both direction, we finished the proof that: \[
  L^1(\nu^+)\cap L^1(\nu^-)  =  L^1(|\nu|) 
\]

\end{proof}


\begin{proof}
    \textbf{of (iii):}
Suppose $f\in L^1(\nu)$, then
\begin{align*}
    \left| \int f \, d\nu \right| &= \left| \int f \, d\nu^+ - \int f \, d\nu^- \right| \quad\text{by def}\\
&\le \left| \int f \, d\nu^+ \right| + \left| \int f \, d\nu^- \right|\quad \text{by tri ineq}\\
&\le \int |f| \, d\nu^+ + \int |f| \, d\nu^-\quad \text{by property of $L^1$ integration }\\
& = \int |f| \, d|\nu| \quad\text{ from (i)}
\end{align*}
Therefore, \[
\left| \int f \, d\nu \right| \le \int |f| \, d|\nu|
\]
\end{proof}



\begin{proof}
    \textbf{of (iv):}
Suppose that $\nu$ is a finite measure (i.e. $\nu^{\pm}(X)<\infty$), let $E\in \mathcal{A}$.\\
We denote: \[
S := \sup \left\{ \left| \int_E f \, d\nu \right| \,\middle|\, \|f\|_\infty \le 1 \right\}
\]
\textbf{First we show \( S \le |\nu|(E) \):}\\
For any bounded measurable \(f\) with \(\|f\|_\infty \le 1\), \begin{align*}
    \left| \int_E f \, d\nu \right| &\le \int_E |f| \, d|\nu| \quad\text{by (iii)}\\
    &\leq \int_{E} 1 \, d|\nu| \quad\text{by linearity of integration}\\
    & = |\nu| (E)
\end{align*}
So by taking the supremum over such \(f\), we get:
\[
S \le |\nu|(E)
\]
\textbf{Next we will show \( |\nu|(E) \le S \):}\\
We take a Hahn decomposition, getting $X  = P \sqcup N$ where \[
\nu^+ (B) = \nu(P \cup B) \geq 0,\nu^-(B) = -\nu(P \cup B) \leq 0, \;\;\text{for all } B \subset X
\]
Then
\[
|\nu|(E) = \nu^+(E) + \nu^-(E) = \nu(E\cap P) - \nu(E \cap N)
\]
Now define:
\[
f := \chi_P - \chi_{N}
\]
Then \(f\) is measurable since $P,N$ are measurable. And \(\|f\|_\infty \le 1\) since $f(x)\in \{ -1,1\}\,\forall x\in X$
Compute:
\[
\int_E f \, d\nu = \int_{E \cap P} 1 \, d\nu - \int_{E\cap N} 1 \, d\nu
= \nu(E \cap P) - \nu(E \cap N ) = \nu^+(E) + \nu^-(E) =  |\nu|(E)
\]
Thus \[
|\nu|(E) = \left| \int_E f \, d\nu \right| \le S
\]
Combining both inequalities, we get: \[
|\nu|(E) = S
\]
\end{proof}


\section{finite signed measures on $(X,\mathcal{A})$ 是一个 NVM}
  Let $(X, \mathcal{A})$ be a measurable space. 
  \begin{itemize}
  \item[(a)]Let $\lambda$, $\mu$ be finite \emph{positive} measures on $(X, \mathcal{A})$. 
    Let $\nu=\lambda-\mu$. Prove that   \begin{equation*}
      \nu^+(E)\le \lambda(E), \qquad \nu^-(E)\le \mu(E), \qquad |\nu|(E)\le \lambda(E)+\mu(E)
  \end{equation*}
  for every $E\in \mathcal{A}$.
\item[(b)]Let $\nu$ and $\kappa$ be finite \emph{signed} measures on $(X, \mathcal{A})$ (i.e.  $\nu(E),\kappa(E)\in\mathbb{R}$ for all $E\in\mathcal{A}$).  Show that 
\[
  |\nu+\kappa|(E) \le |\nu|(E)+|\kappa|(E) 
\]
for every $E\in \mathcal{A}$. 
\item[(c)] Let $\mathcal{M}$ be the collection of finite signed measure $\nu$ on $(X,\mathcal{A})$. 
  For $\nu\in \mathcal{M}$, define   \[
    \| \nu\|= |\nu|(X)
  \]
  Prove that $\|\cdot\|$ is a norm on $\mathcal{M}$ with an appropriate definition of the sum of two signed measures and the multiplication of a signed measure by a (real) scalar. 
\item[(d)] Suppose $(X,\mathcal{A})=(\mathbb{R},\mathcal{B}(\mathbb{R}))$. Compute $\|\delta_x-\delta_y\|$ for $x,y\in\mathbb{R}$.
\end{itemize}

\textit{Remark}: the norm on $\mathcal{M}$ is called the \emph{the total variation norm}.

% Consider $(\R, \cB[\R])$ and the Dirac measures $\delta_a$ for each $a\in \R$ defined by $\delta_a(E)=1$ if $a\in E$ and $\delta_a(E)=0$ if $a\notin E$. What is the distance $\|\delta_a-\delta_b\|$ for $a\neq b$?) 

\begin{proof}
    \textbf{of (a):}\\
Recall in problem 2 we get:
\[
\nu^+(E) = \sup\{ \nu(F) : F \subset E, F \in \mathcal{A} \}, \quad 
\nu^-(E) = -\inf\{ \nu(F) : F \subset E, F \in \mathcal{A} \}
\]
\textbf{Claim 1: $\nu^+(E)\le \lambda(E)$.}\\
Let \(F \subset E\), \(F \in \mathcal{A}\). Then:
\[
\nu(F) = \lambda(F) - \mu(F) \le \lambda(F) \le \lambda(E)
\]
since \(F \subset E\) and \(\lambda\) is positive.  
Taking the sup over all such \(F\), we get
\[
\nu^+(E) = \sup_{F \subset E} \nu(F) \le \lambda(E)
\]\textbf{Claim 2: $ \nu^-(E)\le \mu(E)$.}\\
Similarly as Claim 1, for any \(F \subset E\), since $\lambda$ and $\mu$ are p.m., we have
\[
\nu(F) = \lambda(F) - \mu(F) \ge -\mu(F) \ge -\mu(E)
\implies -\nu(F) \le \mu(E)
\]
Taking the inf over $F \subset E$, we get
\[
\nu^-(E) = -\inf_{F \subset E} \nu(F) \le \mu(E)
\]
\textbf{Claim 3: $|\nu|(E)\le \lambda(E)+\mu(E)$.}\\
This is just combining the two ineqs:  \[
|\nu|(E) = \nu^+(E) + \nu^-(E) \le \lambda(E) + \mu(E)
\]
\end{proof}

\begin{proof}
    \textbf{of (b):} \\
Let $E\in \mathcal{A}$. WTS: $|\nu+\kappa|(E) \le |\nu|(E)+|\kappa|(E) $.\\
Recall in problem 2 we showed that for a signed measure \(\sigma\) and a measurable set \(E\) , we have:
\[
|\sigma|(E) = \sup \left\{ \sum_{i=1}^n |\sigma(E_i)| :E= \bigsqcup_{i=1}^N E_i\right\}
\]
Let \(\{E_i\}_{i=1}^n\) be any finite measurable partition of \(E\). Then for each $E_i$:
\[
|(\nu + \kappa)(E_i)| = |\nu(E_i) + \kappa(E_i)| \le |\nu(E_i)| + |\kappa(E_i)|
\quad \text{(by tri ineq on \(\mathbb{R}\))}
\]
Summing over the partition, we have:\[
\sum_{i=1}^n |(\nu + \kappa)(E_i)| \le \sum_{i=1}^n |\nu(E_i)| + \sum_{i=1}^n |\kappa(E_i)|
\]
Now take the supremum over all such partitions of \(E\):
\begin{align*}
    |\nu + \kappa|(E) &=\sup \left\{ \sum_{i=1}^n |(\nu + \kappa)(E_i)|
 : E= \bigsqcup_{i=1}^N E_i \right\} \\
    & \leq \sup \left\{ \sum_{i=1}^n |\nu(E_i)| +  \sum_{i=1}^n |\kappa(E_i)|
 : E= \bigsqcup_{i=1}^N E_i \right\}\\
 &\leq \sup \left\{ \sum_{i=1}^n |\nu(E_i)| 
 : E= \bigsqcup_{i=1}^N E_i \right\} + \sup \left\{ \sum_{i=1}^n |\kappa(E_i)|
 : E= \bigsqcup_{i=1}^N E_i \right\}\\
 &= |\nu|(E)+|\kappa|(E) 
\end{align*}
Since measurable $E$ is arbitrary, this finishes the proof.
\end{proof}


\begin{proof}
   \textbf{ of (c)}:
\[
\mathcal{M} : = \{ \text{all finite signed measures on }(X, \mathcal{A})\}
\]
and for $\nu \in \mathcal{M}$, we define:  \[
\|\nu\| := |\nu|(X)
\]
WTS: $\| \cdot \|$ is a norm on $\mathcal{M}$.\\
\begin{enumerate}
    \item \textbf{Positive Definiteness}:\\Let $\nu \in \mathcal{M}$. Since $|\nu|$ is a positive measure, \( \|\nu\| =   |\nu| (X) \geq 0\).\\Since $|\nu|$ is a positive measure, \( \|\nu\| =   |\nu| (X) \geq 0\).\\
Suppose $ |\nu|(X) = 0 $, then $X$ is a $|\nu|$-null set, so $|\nu|(E) = 0$ for all $E \in \mathcal{A}$. Thus $\nu = 0$.\\
And suppose $\nu = 0 $, then $|\nu | = 0$ also, so $|\nu|(X) = 0$.\\
Thus, \(\|\nu\| = 0\) iff \(\nu = 0\). This finishes the proof of positive definiteness.
\item \textbf{Absolute Homogeneity}: \\Since for any measurable set \(E\):
\begin{align*}
    |a \nu|(E) &= \sup \left\{ \sum_{i=1}^n |(a \nu)(E_i)| : E= \bigsqcup_{i=1}^N E_i\right\}\\
&= \sup \left\{ \sum_{i=1}^n |a||\nu(E_i)|: E= \bigsqcup_{i=1}^N E_i \right\}\\
& = |a| \sup \left\{ \sum_{i=1}^n |\nu(E_i)|: E= \bigsqcup_{i=1}^N E_i \right\}\\
&= |a| \cdot |\nu|(E)
\end{align*}
We have: \[
\|a \nu\| = |a \nu|(X) = |a| \cdot |\nu|(X) = |a| \cdot \|\nu\|
\]
finishing the proof of absolute homogeneity.
\item \textbf{Triangle Inequality}:\\
Recall we just proved in (b) that for any measurable $E$:\[
  |\nu+\kappa|(E) \le |\nu|(E)+|\kappa|(E) 
\]Thus \[
\|\nu + \kappa\| = |\nu + \kappa|(X) \le |\nu|(X) + |\kappa|(X)=  \|\nu\| + \|\kappa\|
\]
finishing the proof of triangle inequality.\\
\end{enumerate}
So we can conclude that \(\|\nu\| := |\nu|(X) \text{ defines a norm on } \mathcal{M}\), with the standard definitions of addition and scalar multiplication of signed measures.
\end{proof}



\begin{proof}
    \textbf{of (d)}\\
    Suppose $(X,\mathcal{A})=(\mathbb{R},\mathcal{B}(\mathbb{R}))$. Compute $\|\delta_x-\delta_y\|$ for $x,y\in\mathbb{R}$.

    
Recall def: For any Borel set \(A \subset \mathbb{R}\), \[
\delta_x(A) = \begin{cases}
1 & \text{if } x \in A \\
0 & \text{otherwise}
\end{cases}
\]
So we define the signed measure \(\nu := \delta_x - \delta_y\) as:
\[
\nu(A) = \delta_x(A) - \delta_y(A)
\]
If \(x = y\), then \(\delta_x = \delta_y \), then \(\nu = 0\), so \( \| \nu \|= 0  \). This is the trivial case.
if \(x \ne y\): We first compute the Jordan decomposition.\\
We know that $\nu^+(E)= \sup\{ \nu(F)\mid: F\in \mathcal{A}, F\subset E\}$, and  $\nu^-(E)= -  \inf\{ \nu(F)\mid  F\in \mathcal{A}, F\subset E\}$.
For any $E \ni x$, we have \[
\nu^+ (E)  = \nu  (\{x \}) = 1
\]
In other cases, we have: \[
\nu^+ (E)  = \nu  (E \setminus \{y \}) = 0 
\]
For any $E\ni y$, we have  \[
\nu^- (y)  = -\nu  (\{y \}) = 1
\]
In other cases, we have: \[
\nu^- (E)  = -\nu  (E \setminus \{x \}) = 0 
\]
And we thus discover that: \[
\nu^+ = \delta_x, \quad \nu^- = \delta_y
\]
So \[
\|\nu \| = |\nu| (\mathbb{R}) = \delta_x(\mathbb{R}) + \delta_y(\mathbb{R}) = 1 + 1 = 2
\]
Thus we can conclude that \[\|\nu \| = 
\begin{cases}
    2 & \text{if } x \not= y\\
0 & \text{otherwise}
\end{cases}
\]
\end{proof}



\section{and more: finite signed measures on $(X,\mathcal{A})$ 组成一个 real Banach space}
  Prove that the normed vector space $\mathcal{M}$ in the previous problem is in fact a Banach space.
\begin{proof}
In problem 4 we have shown that on $(\mathcal{M}, \|\cdot\|)$ is a normed vector space, where \[
\mathcal{M} : = \{ \text{all finite signed measures on }(X, \mathcal{A})\}
\]
and  \[
\|\nu\| := |\nu|(X)
\]
Now we prove that the NVM $(\mathcal{M}, \|\cdot\|)$ is complete, i.e. it is a Banach space.\\
 Let $(\nu_n)$ be a Cauchy sequence in $\mathcal{M}$. We have
\[
|\nu_n(B) - \nu_m(B)| = |(\nu_n - \nu_m)(B)| \le \|\nu_n - \nu_m\| \quad \text{for all } B \in \mathcal{A}
\]
In particular, $(\nu_n(B))_n$ is a Cauchy sequence for all $B \in \mathcal{A}$. For each $B \in \mathcal{A}$, this is a Cauchy seq in $\mathbb{R}$, thus converges. So we can get: \[
\nu(B) := \lim_n \nu_n(B)
\]
as the pointwise limit (by a point we mean a set).\\
\textbf{Claim 1: $\nu \in \mathcal{M}$}.\\
Since for all $n$, $\nu_n(\varnothing) = 0$, we have:
\[
\nu(\varnothing) := \lim_n \nu_n(\varnothing) = 0
\]
For a countable disjoint union of measurable sets \(E = \bigsqcup_{i=1}^\infty E_i\), \[
 \lim_n   \nu_n(E) = \lim_n \sum_i \nu_n(E_i)
\]is the limit of a finite sum of numerical sequences in $\mathbb{R}$. So we can exchange the order of taking limit and sum. Then we get: \[
  \nu(E) = \lim_n \nu_n(E) = \lim_n \sum_i  \nu_n(E_i) = \sum_i \lim_n \nu_n(E_i) = \sum_i \nu(E_i)
  \]
And notice, for each measurable set $B\in \mathcal{A}$, \textbf{since $(\nu_n(B))_n$ is a Cauchy sequence in $\mathbb{R}$, it is bounded}, thus does not admit $\infty, -\infty$ values. verifying that $\nu$ \textbf{is a valid signed measure.}\\
Also, this means that taking Hahn Decomposition $X = P \sqcup N$ by $\nu$, we have \[
\nu^+ (X) = \nu(P ) ,\quad \nu^- (X) = -\nu(N) 
\]
Since $\nu(P), \nu(N)$ are bounded, we have:
Thus \[
|\nu| (X) =  \nu^+ (X) + \nu^- (X)  < \infty
\]
This verifies that $\nu$ is a finite s.m.\\
\textbf{ Claim 2:  $\nu_n \to \nu$ in $\|\cdot\|$.}
Fix $\varepsilon > 0$. There exists $N$ such that $\|\nu_n - \nu_m\| < \varepsilon / 2$ for all $m, n \ge N$. Thus for all $n\geq N$ we have:
\[
|(\nu_n - \nu)(B)| = \lim_{m} |(\nu_n - \nu_m)(B)| \le \varepsilon / 2, \quad \forall B \in \mathcal{A},\ \forall n \ge N
\]Notice that
\[
\nu^+(B) = \sup\{\nu(C) \mid C \in \mathcal{A},\ C \subset B\}
\quad
\]and \[
\nu^-(B) =  - \inf\{\nu(C) \mid C \in \mathcal{A},\ C \subset B\}  =
\sup\{-\nu(C) \mid C \in \mathcal{A},\ C \subset B\}
\]
It follows that
\[
(\nu_n - \nu)^+(X) = \sup\{(\nu_n - \nu)(B) \mid B \in \mathcal{A}\} \le \varepsilon / 2, \quad \forall n \ge N
\]
Similarly,
\[
(\nu_n - \nu)^-(X) = \sup\{-(\nu_n - \nu)(B) \mid B \in \mathcal{A}\} \le \varepsilon /2,\quad \forall n \ge N
\]
Thus \[
|\nu_n - \nu | (X)  = (\nu_n - \nu)^+(X) + (\nu_n - \nu)^-(X) \leq \varepsilon
\]
This holds for all $n\geq N$. And since $\varepsilon > 0$ is arbitrary, this proves that \[
\lim_{n\to\infty} \|\nu_n - \nu\| = 0
\]
As a result, $\nu_n \to \nu$ in $\|\cdot\|$, completeing the proof.
\end{proof}



\vspace*{10mm}
\begin{center}
  \textit{Nur f\"ur Verr\"uckte}
\end{center}
(It's \textbf{really} not necessary to attempt these problems. Do not, under any circumstances, hand them in!)
Does there exist a signed Borel measure $\nu$ on $\mathbb{R}$ with the property that  for every $\alpha\in \mathbb{R}$ there exists a Borel set $E\subset\mathbb{R}$ with $\nu(E)=\alpha$.