\chapter{inequilities on $L^p$ spaces [Fol 6.1]}
对应 Folland 6.1(2).\\
我们将证明 Hölder's ineq 以及它的 corollary Minkowski's ineq, 从而证明: $L^p$ 是一个 normed VS, 并且是一个 Banach space (这里 $1\leq p <\infty$, 但是 later we will also prove $L^\infty$ 也是 Banach space).\\
这两个不等式非常重要.

\section{Hölder's ineq}

\begin{theorem}{Hölder's ineq}
    Consider conjugate pair: $p,q \in [1,\infty)$ s.t.  \[
    \frac{1}{p} + \frac{1}{q} = 1
    \]
则对于任意两个 measurable function $f,g:X \to \mathbb{C}$,  一定有: \[
 \| fg\|_1 \leq \| f\|_p \cdot \|g\|_q 
 \]特别地, 如果 $f \in L^p(\mu)$, $g\in L^q(\mu)$, 则 $fg \in L^1(\mu)$, 并且 equality holds iff \[
   \|g\|_q^q |f|^p = \|f\|_p^p |g|^q \quad \mu\text{-a.e.}
 \]
\end{theorem}
\begin{remark}
For $(p,q) = (2,2)$, this is \textbf{Cauchy-Swartz ineq}: \[
 \| fg\|_1 \leq \| f\|_2 \cdot \|g\|_2
\]
即: \[
 \bigg(\int  f\,\overline{g}   \bigg) \leq \int |fg| \leq  \sqrt{\bigg(\int |f|^2 \bigg)\bigg(\int |g|^2 \bigg)}
\]
\end{remark}
\begin{proof}
Trivial Case 1: 如果 $\|f\|_p = 0$ (或者$\|g\|_q = 0$  ), then $f$ is zero $\mu$-almost everywhere, and the product $fg$ is zero $\mu$-almost everywhere, 于是两边都是 $0$, ineq trivially true.\\
Trivial Case 2: 如果 $\|f\|_p = \infty$ or $\|g\|_q = \infty$, 则右边 infinite, ineq trivially true.
因而我们只需要考虑 $\|f\|_p$ and $\|g\|_q$ are in $(0, \infty)$ 的情况就好了.\\

Main case: 我们需要一个 Lemma: 
\begin{lemma}{Young's inequality for products}
Whenever $p, q \in (1, \infty)$ with $\frac{1}{p} + \frac{1}{q} = 1$, 都有
\begin{equation}
    ab \leq \frac{a^p}{p} + \frac{b^q}{q} ,\quad \forall a,b\geq 0
\end{equation}
where equality is achieved if and only if $a^p = b^q$.\\
另一个等价形式是: \[
a^{\lambda} b^{1-\lambda} \leq \lambda a + (1-\lambda)b,\quad \forall a,b\geq 0
\]
\end{lemma}
\begin{proof}
    \textbf{of Lemma:}\\
    $b=0$ 则 trivial case. 因而 setting $t : =\frac{a}{b}$, reduced to show: \[
    t^\lambda \leq \lambda t  + (1-\lambda)
    \]
  with eq iff $t=1$. 这是显然的, 因为 by Calculus, $t^\lambda -\lambda t$ 是 strictly increasing for $t<1$, strictly decreasing for $t>1$ 的, max 在  $t=1$, 正好是 $1-\lambda$.\end{proof}
使用 Young's inequality for products 得到:
\begin{equation}
    \frac{|f(x)|}{\|f\|_p} \frac{|g(x)|}{\|g\|_q} \leq \frac{|f(x)|^p}{p \|f\|_p^p} + \frac{|g(x)|^q}{q \|g\|_q^q}, \quad x \in X
\end{equation}

Integrating both sides gives
\begin{equation}
    \frac{\|fg\|_1}{\|f\|_p \|g\|_q} \leq \frac{\|f\|_p^p}{p \|f\|_p^p} + \frac{\|g\|_q^q}{q \|g\|_q^q} = \frac{1}{p} + \frac{1}{q} = 1,
\end{equation}
which proves the claim.\\
Integration 的 equality holds iff point equality holds a.e., 并且, by Young's inequality for products, 上面的 equality holds iff \[   \|g\|_q^q |f|^p = \|f\|_p^p |g|^q \quad \mu\text{-a.e.}\]
\end{proof}
\begin{remark}
1. 显然, 根据我们的证明过程可知: \textbf{Hölder's ineq also holds on any measurable subset $S \subset X$}: \[
\int_S |fg| \leq  \bigg(\int_S |f|^p \bigg)^{\frac{1}{p}}\bigg(\int_S |g|^q \bigg)^{\frac{1}{q}}
\]
2. 这里的满足 $\frac{1}{p} + \frac{1}{q} = 1$ 的 $p,q$ 我们称之为: \textbf{Hölder conjugate}, 并称它们互为对方的 \textbf{conjugate exponent}.\\
3. 左边实际上是两个正值函数的 inner product, 相当于把一个投影到另一个上;  \\
几何直观: Hölder's ineq 在退化为 Cauchy-Swartz 时表示, 两个函数/向量的内积一定小于等于长度积; 而 Hölder's ineq 更广义: 表示它们的内积一定小于它们取任意相互 conjugate 的 norm 长度的积.
并且 sooner 我们会学到: 对作为 Hölder conjugates 的 $p,q$, $L^p$ 和 $L^q$ 互为 dual space, 从而 Hölder ineq 表示的是就是 norm 与其 dual norm 之间的 maximal inner product 控制关系.
\end{remark}


\begin{remark}
    Hölder's ineq 有一个 generalization:
    对于任意 $0<s<\infty$ and $0<p_1,\dots, p_n< \infty$ such that 
\[
  \frac1{p_1}+\frac1{p_2}+\dots+\frac1{p_n}=\frac1{s};
\]
都有
\[
  \| f_1f_2\cdots f_n\|_s\le \|f_1\|_{p_1}\|f_2\|_{p_2}\cdots \|f_n\|_{p_n}.
\]

    This generalization will be proved in hw8.
\end{remark}



\subsection{Minkowski's ineq: tri ineq on $L^p$, 确认 $\|\cdot\|_p$-norm 是 $L^p$ 上的 valid norm }
Minkowski's ineq 即 $L^p$ space 上的 tri ineq.
\begin{corollary}{Mincowski's ineq}
对于任意 $1\leq p < \infty$, 都有: \[
\|f +g \| \leq \|f\|_p  + \| g\|_p
\]
\end{corollary}
\begin{proof}
显然, 对于任意 $x$ 都有: \[
    |f + g|^p \leq \bigg( |f| + |g|\bigg) |f+g|^{p-1} 
    \]
    因而: \begin{align*}
           \int |f + g|^p &\leq \int |f| \cdot |f+g|^{p-1} \; + \; \int |g| \cdot |f+g|^{p-1}
    \end{align*}
    我们定义 $$h(x):= |f(x)+g(x)|^{p-1}$$于是 \begin{align*}
          \int  |f + g|^p &\leq \int |fh| + \int |gh| \\
            &\leq \|f\|_p \| h\|_q + \|g\|_p \| h\|_q \\
            &= \bigg(\|f\|_p + \|g\|_p \bigg)  \bigg(\int |f+g|^{(p-1)q}\bigg )^{1/q}
    \end{align*}
其中 $q$ 是 $p$ 的 Hölder conjugate.  这里的 punchline is actually: 由于 \[
q : = \frac{p}{p-1}
\] actually, \[
(p-1)q = p
\]
因而: 
\begin{align*}
        \int  |f + g|^p  &\leq \bigg(\|f\|_p + \|g\|_p \bigg)  \bigg(\int |f+g|^{(p-1)q}\bigg )^{1/q}\\ &= \bigg(\|f\|_p + \|g\|_p \bigg)  \bigg(\int |f+g|^{p}\bigg )^{1/q}\\
        & = \bigg(\|f\|_p + \|g\|_p \bigg)  \bigg(\int |f+g|^{p}\bigg )^{1-1/p}
\end{align*}
两边同时除以 $\big(\int |f+g|^{p}\big )^{1-1/p}$  得到: \[
\bigg(  \int  |f + g|^p \bigg)^{1/p} =: \| f+g\|_p \leq \|f\|_p + \|g\|_p
\]
从而得证.
\end{proof}
\begin{remark}
    这里的技巧是: 把一个 $p$ 次方的函数拆成一个 $1$ 次方的函数和一个 $p-1$ 次方的函数, 并且使用Hölder, 这样就得到了一个 1 次的函数的 $p$-norm 和另一个 $p-1$ 次的函数的 $q$ norm, 但是注意 $q(p-1) = p$, 因而这个函数就变成了 \[
    \bigg( \int |\phi|^p \bigg)^{1/q}
    \]的形式. 并且注意到: \[
    \frac{ \displaystyle \int |\phi|^p }{    \bigg( \displaystyle\int |\phi|^p  \bigg)^{1/q}} = \bigg( \displaystyle\int |\phi|^p  \bigg)^{1/p} = \|\phi\|_p
    \]
\end{remark}
\begin{remark}
    Minkowski 不等式证明的是 \(1\leq p < \infty\) 时的 $p$-norm 的三角不等式. 但是对于 $0<p<1$, 它并不成立. 因为这个时候 $p-1< 0$, 我们刚才的证明不作效.\\
    直观的证明: 在 $p\geq 1$ 的时候, $|x|^p$ 是一个 strictly convex 的函数; 而在 $0< p<1$ 的时候, $|x|^p$ 则是一个 strictly concave 的函数.\\
因而我们运用 strictly concave 的性质: \[
|a+b|^p > |a|^p + |b|^p
\]
再由积分可得到反例. (比如取 indicator function 进行积分)
\end{remark}

\section{properties of $L^p$ spaces ($1\leq p < \infty$)}
\subsection{$L^p$ ($1\leq p < \infty$) is Banach}
\begin{theorem}{$L^p$ space ($1\leq p < \infty$) is Banach}
    $L^p$ ($1\leq p < \infty$) is Banach.
\end{theorem}
\begin{proof}
    By last lec 的定理: 一个 NVS 是 Banach 的等价条件是任意 abs conv series 都 conv. 因而我们证明这一点即可.\\
Suppose $f_n \in L^p$ for each $n$, 并且这个 series abs conv, 即: \[
B := \sum_{k=1}^\infty \| f_k \|_p < \infty
\]
我们 define: \[
g(x) : = \sum_{k=1}^\infty f_k(x),\quad g_n(x) : = \sum_{k=1}^n f_k(x)
\]
我们 WTS: \[
\lim_{n\to\infty} g_n = g
\]
in $p$-norm induced metric sense, 即, for some $f \in L^p$, 有 \[
\lim_{n\to\infty} \big\|  g - g_n \big \|_p = 0
\]
我们 Set: \[
G_n := \sum_{k=1}^n |f_k|,\quad G:= \sum_{k=1}^\infty |f_k|
\]
这个函数以及函数列的定义是为了使用 DCT, 作 donimating function 用. \\
By measurable function 的 limit behavior, 有 \[
G_n,G\in L^+
\]
并且 \[
\|G_n\|_p \leq \sum_{k=1}^n \|f_k\|_p \leq B
\]
由于 $G_n \nearrow G$, by MCT 有 \[
\int G^p = \lim_{n\to \infty} \int G_n^p \leq B^p < \infty
\] 由于 $G\in L^p$, 有 $$G(x) < \infty \quad a.e.$$ 于是: \[
g(x) : = \sum_{k=1}^\infty f_k(x) <\infty \quad a.e.
\]
又 $|g_n|,|g|\leq G, g_n \to g$, 可得到: \[
|g_n - g|^p \leq 2^p G^p \in L^1
\] 因而 by DCT 可以得到: \[
\lim _n\int |g_n - g|^p    = 0
\]
从而\[
  \lim_{n\to\infty} \big\|  g - g_n \big \|_p = \bigg( \lim _n\int |g_n - g|^p   \bigg)^{1/p} = 0
\]
\end{proof}
\begin{remark}
1. 我们说一个 function seq converge to 一个 function 指的是 in the sense of distance, 而这里就是 metric induced by norm, 即\textbf{它们的差的 $L_p$ norm converge to $0$.}\\
2. 注意, 我们 recall: $f_k \to f$ a.e. 并不说明 $f_k\to f$ in $L^1$, 因为每个点 converge 的速度不一样. 当然, 对 $L^p$ 也同理.\\
3. \textbf{虽然 a.e. convergence 不能推出 $L^p$ convergence, 但是配合 DCT, 则可以推出.} \textbf{DCT 是我们证明 $L^p$ convergence 的关键.}\\
4. 要证明 \[
  \lim_{n\to\infty} \big\|  g - g_n \big \|_p = 0
\]完全可以忽略积分外的 $1/p$ 次方. 其实只需要证明 \[
\lim _n\int |g_n - g|^p    = 0 
\] 就可以了. 证明 $L^p$ convergence, 比起 $L^1$ convergence 略困难的地方就是被积函数变得更大了.
\end{remark}


\subsection{Criterion for $L^p$ convergence: 逐点 a.e. conv $+$ $L^p$ 积分值 conv}
我们刚才 mention: DCT 对于 function seq $L^p$ convergence 的证明有很大作用. 这里我们就提供一个 DCT 推出的 $L^p$ convergence 的判断准则: 
\begin{theorem}{Criterion for $L^p$ convergence}
    if $f_n\to f$ a.e. and $\|f_n\|_p\to\|f\|_p$, then $\|f_n-f\|_p\to0$. 
\end{theorem}
即 \[
\text{a.e. conv } + L^p\text{  norm conv} \implies L^p conv 
\]
但是 converse 并不成立. 反例是 typewriter function.
\begin{proof}
    In Hw 8.
\end{proof}



\subsection{dense subsets of $L^p$, and specially $L^p(\mathbb{R},m)$ }
\begin{proposition}
    对于任意 $1\leq p < \infty$, the set of $\{$simple functions$\}$, is dense in $L^p$.\\
    即: \[
    \{f:X \to \mathbb{C}\mid f=\sum_1^n  a_j \chi_{E_j},\mu(E_j)<\infty \}
    \]是 $L^p$ 的 dense subset.
\end{proposition}
\begin{remark}
    我们已经 proved this for $L^1$, 而其实这个 density 推广至 $L^p$ 也成立.
\end{remark}
\begin{proof}
对 $f$ 使用 simple function seq 逼近, 使用 $2^p |f|^p$ 作为 dominating function of $|f_k-f|^p$; 而后使用 DCT 得证.
\end{proof}

\begin{theorem}{$C_c^0(\mathbb{R}^n)$ is dense in $L^p(\mathbb{R},m)$ for $1\leq p < \infty$}
$C_c^0(\mathbb{R}^n)$ is dense in $L^p(\mathbb{R},m)$ for $1\leq p < \infty$
\end{theorem}
\begin{proof}
    exercise. Similar to the proof for $L^1$, 只需要使用加入 $p$ power 的 function 作为 dominating function 即可.
\end{proof}