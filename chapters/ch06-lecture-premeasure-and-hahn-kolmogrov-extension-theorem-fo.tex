\chapter{premeasure and Hahn-Kolmogrov extension Theorem [Fol 1.4, finished]}
我们发现: 有些子集簇上的 "length" 很明显, 并且也符合 measure 的定义, 但是这个子集簇却并不构成一个 $\sigma$-algebra. 比如:
\begin{example}
    $\{ \text{all half-open, half-closed intervals}\} \sub \mathbb{R}$ 上, 以 interval 的 length 作为 measure, 很显然符合 measure function 的定义, 但是 $\{ \text{all half-open, half-closed intervals}\} \sub \mathbb{R}$ 并不是一个 $\sigma$-algebra, 因为它可以通过 ctbl union 出 open interval, 并不在这个子集簇中. 不过, 这是一个 algebra.\\
\end{example}
因此, 我们想要一个方法来 \textbf{extend a "measure" function on an algebra, to a measure on a $\sigma$-algebra.}

\begin{definition}{premeasure}
给定 $\cP(X)$ 上的一个 \textbf{algebra} $\mathcal{A}_0$, 我们称 \(\mu_0: \mathcal{A}_0 \rar [0,+\infty]\)  为一个 premeasure, if
\begin{enumerate}
    \item \(\mu_0(\varnothing)  = 0\)
    \item \(\mu_0\) ctbl disjoint additive in $\mathcal{A}_0$
\end{enumerate}
\end{definition}

\begin{remark}
premeasure 就是定义在 algebra instead of $\sigma$-algebra 上的 measure. 显然, 通过和 measure 相同的方式可证明, premeasure 在 $\mathcal{A}_0$ 上是 \textbf{monotone 以及 ctbl subadditive 的.  }  
\end{remark}


\section{induce outer measure out of a premeasure: preserving $\mu_0$ on $\mathcal{A}_0$}
\begin{proposition}
\label{construct outer measure out of a premeasure}
    Any premeasure can induce an outer measure:
    \begin{equation}
        \mu^*(E) = \inf \{  \sum_{i=1}^\infty \mu_0(A_i) \mid A_i \in \mathcal{A}_0, E \sub \bigcup_{i=1}^\infty A_i   \}
    \end{equation}
    并且, we have:
    \begin{equation}
        \mu^*|_{\mathcal{A}_0} = \mu_0
    \end{equation}
    并且 \textbf{every set in $\mathcal{A}_0$ is $\mu^*$-measurable.}
\end{proposition}
\begin{proof}
    \textbf{这个 outer measure 的 construction directly follows from} \ref{construct outer measure out of a "elementary length function"}.\\
    \noindent \textbf{Proof that $\mu^*$ restricted to $\mathcal{A}_0$ is $\mu_0$}: 令 $E \in \mathcal{A}_0$, 假设 $E \sub \bigcup_{i=1}^\infty A_i$, 我们令 $B_n := E \cap (A_n \setminus \bigcup_{i=1}^{n-1} A_i)$, 即把 covering intersecting $E$ 变成 disjoint covering $(B_n)$, 从而由 $\mu_0$ 的 ctbl disjoint additivity 可得, 这一个新 covering 的 measure sum $\sum_{i=1}^\infty \mu_0(B_i) := \mu_0(E)$. 并且由于 $\mathcal{A}_0$ 是一个 algebra, 这些 $B_n$ 也在 $\mathcal{A}_0$ 里面, 从而它满足 monotonicty, then $\mu_0(E) = \sum_{i=1}^\infty \mu_0(B_i) \leq \sum_{i=1}^\infty \mu_0(A_i) $\\
    \noindent \textbf{Proof that every set in $\mathcal{A}_0$ is $\mu^*$-measurable}: Fix $A \in \mathcal{A}_0$, 我们取任意 $E \sub X$.
    Let $\epsilon > 0$, by def of the outer measure, 存在一个 seq $\{ B_i\}_{i=1}^\infty \sub \mathcal{A}_0$, 使得 $E \sub \bigcup_{i=1}^\infty B_i$ 并且 $\sum_{i=1}^\infty \mu_0(B_i) \leq \mu^*(E) + \epsilon$. 有 disjoint additivity of $\mu_0$ 可得, $\sum_{i=1}^\infty \mu_0(B_i) = \sum_{i=1}^\infty \mu_0(B_i \cap A) + \sum_{i=1}^\infty \mu_0(B_i\cap A^c)$. 从而 $\mu^*(E) \geq \mu^*(E \cap A) + \mu^*(E\cap A^c)$, 得证. (实际上这是个 trivial argument, 通过$\epsilon$ argument 来严格证明.)
\end{proof}
\begin{remark}
    这一 simple proposition 表明的是, $\mu_0$ induce 出的 outer measure 在 $\mathcal{A}_0$ 上 \textbf{presearve $\mu_0$ 的 measure 与 measurability.}
\end{remark}






\section{Hahn-Kolmogrov Theorem}
\begin{definition}{$\sigma$-finite measure}
Let $(X,\mathcal{M}, \mu)$ be a measure space.\\
如果 $\mu(X) < \infty$, 则称 $\mu$ 是 finite 的.\\
如果存在一个 sequence $(E_i)$ in $\mathcal{M}$ 使得 $\bigcup_{i} E_i = X$ 并且每个 $\mu(E_i) < \infty$, 则称 $\mu$ 是 $\sigma$-finite 的.
\end{definition}
\begin{remark}
一个 finite measure 说明 $\mathcal{M}$ 中的所有集合的 measure 都 finite.
\end{remark}



\begin{theorem}{Hahn-Kolmogrov Theorem}
\label{Hahn-Kolmogrov Theorem}
给定一个 premeasure $\mu_0$ on algebra $\mathcal{M}_0$ of $X$, 以及其 induced outer measure $\mu*$, 我们令 
$$
\mathcal{M} := <\mathcal{M}_0>
$$
表示 $\sigma$-algebra generated by the algebra $\mathcal{M}_0$.\\
并令
$$
\mu := \mu^* |_\mathcal{M}
$$
then we have:
\begin{enumerate}
    \item $(X,\mathcal{M}_0, \mu_0)$ extends to $(X,\mathcal{M},\mu)$\\
    即: $\mu  |_{\mathcal{M}_0} = \mu_0$
    \item $\mu | _\mathcal{M}$ 是 \textbf{the largest extension of $\mu_0$ to $\mathcal{M}$} (即: 对于任意其他的 $\mathcal{M}$ 上的 measure $\nu$ that extends $\mu_0$ to $\mathcal{M}$, 都有 $\nu(E) \leq \mu(E)$ for all $E \in \mathcal{M}$);\\
    并且 \textbf{if $\mu_0$ is $\sigma$-finite}, 则 $\mu$ 是 \textbf{the unique extension} of $\mu_0$ to $\mathcal{M}$.
\end{enumerate}
\end{theorem}
\begin{proof}
\textbf{Proof of $(X,\mathcal{A}_0, \mu_0)$ extends to $(X,\mathcal{M},\mu)$:}\\
这个 Statement directly follows from \ref{Carathéodory's Theorem}(Carathéodory's Theorem) 以及上一个 proposition \ref{construct outer measure out of a premeasure}. \\
\noindent 1. 我们首先用 $\mu_0$ induce 出 $\mu^*$, 再 restrict $\mu^*$ to $ \mathcal{M}^* :=\{ \text{all } \mu^* \text{-measurable sets}  \}$, 得到一个 $\sigma$-algebra $\mathcal{M}^*$.\\
\noindent 注意此时: 由上一个 proposition \ref{construct outer measure out of a premeasure} 可得 $\mathcal{M}_0$ 中所有集合都是 $\mu^*$-measurable 的, thus $M_0 \sub \mathcal{M}^*$, 由于 $\mathcal{M}^*$ 是一个 $\sigma$-algebra, 由 \ref{inclusion properties of generated sigma-algebra} 可得: $\mathcal{M} := <\mathcal{M}_0> \sub \mathcal{M}^*$. \\
\noindent 2. 由 Carathéodory's Theorem 可以得到: $\mu^* | _{\mathcal{M}^*}$ 是一个 measure, 从而 $\mu :=\mu^* |_{\mathcal{M}}$ 也是一个 measure(等于把 $\mu^* | _{\mathcal{M}^*}$ 限制在了一个更小的 sub-$\sigma$-algebra 上).\\
\noindent\textbf{(Note: this is a trivial fact that if $M^*$ is a $\sigma$-algebra and $M \subset M^*$is also a $\sigma$-algebra, then $\mu |_{M}$ is a measure if given that $\mu$ is a $\sigma$-algebra on $M^*$)}\\\\
\noindent \textbf{Proof of $\mu$ being the largest extension of $\mu_0$ to $\mathcal{M}$:}
\noindent 假设 $\nu$ 是一个 $\mathcal{M}$ 上的 $\sigma$-algebra s.t. $\nu|_{\mathcal{M}_0} = \mu_0 $.\\
\noindent Let $E \sub \mathcal{M}$. (WTS: $\nu(E) \leq \mu(E)$, 即$\nu(E) \leq \mu^*(E)$ .)\\
\noindent 由外测度 \(\mu^*\) 的定义, 对于任意 \(\epsilon>0\), 存在一列集合 \(\{A_i\}_{i=1}^\infty \subset \mathcal{A}_0\) 满足
\[
E\subset \bigcup_{i=1}^\infty A_i \quad \text{且} \quad \sum_{i=1}^\infty \mu_0(A_i) \le \mu^*(E)+\epsilon.
\]
由于 \(\nu\) 在 \(\mathcal{A}_0\) 上和 \(\mu_0\) 一致,即
\[
\nu(A_i) = \mu_0(A_i) \quad \forall i,
\]
因此,
\[
\sum_{i=1}^\infty \nu(A_i) = \sum_{i=1}^\infty \mu_0(A_i) \le \mu^*(E)+\epsilon
\]
利用 \(\nu\) 的 additivity 和 monotoncity 得
\[
\nu(E) \le \nu\Bigl(\bigcup_{i=1}^\infty A_i\Bigr) \le \sum_{i=1}^\infty \nu(A_i) = \sum_{i=1}^\infty \mu_0(A_i) \le \mu^*(E)+\epsilon
\]

由于 \(\epsilon\) arbitrary, 得到
\[
\nu(E) \le \mu^*(E)
\]


\noindent (证明思路: 在 $\mathcal{M}$ 上 $\mu$ 就等于 $\mu_0$ induce 的外测度, 对于其他的 extended measure, 其作用在一个集合上的测度一定小于等于任意的 $\mathcal{M}_0$ covering 的 premeasure 和, 而我们可以通过控制这个 covering 的测度和与它的外测度的差距(since inf), 从而使得这个测度小于等它的外测度加一个无限小的 $\epsilon$, 从而得证.) \\\\

\noindent \textbf{Proof of $\mu$ being the unique extension of $\mu_0$ to $\mathcal{M}$, provided that $\mu_0$ is $\sigma$-finite}:\\
\noindent (recall $\mu_0$ is $\sigma$-finite 即 $\mu_0(X) < \infty$) It remains to show that $\nu(E) \geq \mu^*(E)$.

\noindent Continuing 上一个 proof, we have:
$$
\mu^*(E) \leq \mu^*(\bigcup_{i=1}^\infty A_i) = \nu(\bigcup_{i=1}^\infty A_i) = \nu(E) + \nu(\bigcup_{i=1}^\infty A_i \setminus E)
$$
$$
\leq \nu(E) + \mu^*(\bigcup_{i=1}^\infty A_i \setminus E)
$$
我们只要 controling $\mu^*(\bigcup_{i=1}^\infty A_i \setminus E) = \mu^*(\bigcup_{i=1}^\infty A_i ) - \mu^*(E) = \epsilon $ 逼近 0, 即可得到反向的不等式关系.\\
\noindent (证明思路: 我们证明了 $\nu(E) \leq \mu^*(E)$ 之后, 注意到 covering set 和 $E$ 之间的差集的 $\nu$-measure 自然也小于等于这个差集的 $\mu^*$-measure, which can approximate 0.)
\\\\

\end{proof}


\begin{remark}
\noindent 1. 我们首先容易定义 $X$ 上的一个 algebra $\mathcal{M}_0$ 和一个 algebra 上的 premeasure $\mu_0$; \\\\
    \noindent 2. 然后用 inf of covering sum 来 induce 出一个 $\cP(X)$ 上的 outer measure $\mu^*$, 而后我们限制 $\mu^*$ 到 $\mu^*|_{\mathcal{M}^*}$ (where $\mathcal{M}^*$ 表示所有的 $\mu^*$-measurable sets), by Carathéodory's theorem 这就 induce 出了一个 complete measure. \\\\
    \noindent 3. 我们可以再取 $\mathcal{M}^*$ 的一个 sub $\sigma$-algebra $\mathcal{M} := <\mathcal{M}_0>$, 限制在这个集合上的 $\mu^*|_{\mathcal{M}}$ 自然也是一个 measure, 并且是 $\mathcal{M}_0$ extend 到 $\mathcal{M}$ 上的 lartest measure. By Hahn-Kolmogrov Thm, 这个 measure 如果是 $\sigma$-finite 的则是 $\mathcal{M}_0$ extend 到 $\mathcal{M}$ 上的 unique measure.\\
    \noindent (Notice: \textbf{自然地, $(X, \mathcal{M}^*, \mu^* |_{\mathcal{M}^*})$ 是 $(X, \mathcal{M}, \mu^*|_{\mathcal{M}})$ 的一个 completion.})
    
\end{remark}