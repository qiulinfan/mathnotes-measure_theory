\chapter{on differentiaion (50/50)}

\begin{center}
\textit{None of the following questions will be graded. Do them, but do not hand them in}.
\end{center}



\section{Completion of $(X\times Y, \mathcal{A}\otimes \mathcal{B}, \mu\times \nu)$ = Completion of $(X\times Y, \bar{\mathcal{A}}\otimes \bar{\mathcal{B}}, \bar{\mu}\times \bar{\nu})$}
Let $(X, \mathcal{A}, \mu)$ and $(Y, \mathcal{B}, \nu)$ be measure spaces. Let $(X, \bar{\mathcal{A}}, \bar{\mu})$ and $(Y, \bar{\mathcal{B}}, \bar{\nu})$ be their completions, respectively. 
Then, the completion of $(X\times Y, \mathcal{A}\otimes \mathcal{B}, \mu\times \nu)$ is same as the completion of $(X\times Y, \bar{\mathcal{A}}\otimes \bar{\mathcal{B}}, \bar{\mu}\times \bar{\nu})$.






\section{Modified HL maximal inequality ($\geq$ instead of $>$)}
Prove that there is a constant $C_n>0$ that only depends on $n$ such that for every $f\in L^1(\mathbb{R}^n)$ and $\alpha>0$,
\[
m(\{x\in \mathbb{R}^n\mid Hf(x)\ge \alpha\}) \le \frac{C_n}{\alpha} \int_{\mathbb{R}^n} |f(x)|\;d x
\]
(Remark: We had $Hf(x)> \alpha$ for the HL maximal inequality. Here we have $Hf(x)\ge \alpha$.) 





\section{density of a mble set at a point: $D_E(x)=1$ for a.e. $x\in E$, $0$ for a.e. $x\in E^c$}
  For a Lebesgue measurable subset $E$ of $\mathbb{R}^n$, the \emph{density of $E$ at $x$} is defined as 
  \[
    D_E(x)= \lim_{r\to 0} \frac{m(E\cap B(x,r))}{m(B(x,r)}
  \]
  provided that the limit exists.
  Prove that $D_E(x)=1$ for a.e. $x\in E$ and $D_E(x)=0$ for a.e. $x\in E^c$. 
  \textit{Hint}: ask Lebesgue.







\newpage
\vspace*{5mm}
\begin{center}
\textit{Some of the following questions will be graded. Do them, and do hand them in}.
\end{center}

\section{An identity: $\int_0^\infty e^{-2sx}\frac{\sin^2x}{x}\;d x=\frac14\log(1+s^{-2})$}
  Prove that $\int_0^\infty e^{-2sx}\frac{\sin^2x}{x}\;d x=\frac14\log(1+s^{-2})$ for $s>0$ by integrating the function $e^{-2sx}\sin(2xy)$ with respect to $x$ and $y$ over suitable regions.
\begin{proof}
For fixed $x> 0$, by FTC we have:   \[\sin^2 (x)  = \int_0^x \sin(2t)\,dt\]
We do change of variable \(t = xy\). This is a valid diffeomorphism mapping $y\in (0,1)$ to $t \in (0,x)$.\\
Then by change of variable theorem we have: \[\int_{(0,x)} \sin(2t)\,dt = \int_{(0,1)}  x\sin(2xy)\,dy\]Thus \[
\frac{\sin^2 x}{x} = 
\int_0^1 \sin(2xy)\,dy
\]
Then we get: \[
\int_0^\infty e^{-2sx}\frac{\sin^2x}{x}\;d x = \int_0^\infty e^{-2sx}\bigg[\int_0^1 \sin(2xy)\,dy\bigg]\;d x
\]
Consider the function \[
f(x,y) := e^{-2sx}\sin(2xy),\quad (x,y) \in (0,\infty) \times (0,1)
\]
$f$ is a composition of continuous functions, thus continuous. Note that it is also in $L^1( (0,\infty) \times (0,1))$ since $|f(x,y)|$ is bounded by $g (x,y):= e^{-2sx}$, which is $L^1$ on the same domain (its integral is $\frac{1}{2s}$), then by DCT, $f \in L^1( (0,\infty) \times (0,1))$.\\
Thus we can apply Fubini's theorem to switch the order of integration: \begin{align}
    \int_0^\infty e^{-2sx}\bigg[\int_0^1 \sin(2xy)\,dy\bigg]\;d x &= \int_{(0,\infty) \times (0,1)} e^{-2sx}\,\sin(2xy)\,d (x\times y) \\
    &= \int_0^1 
\Bigl(\int_0^\infty e^{-2 s x} \sin(2xy)\,dx\Bigr)dy
\end{align}
Recall back in Calculus we use integration by part to get: \[
\int_0^\infty e^{-a x}\,\sin(b x)\,dx= 
\frac{b}{a^2 + b^2}
\]
for \(a>0\).  In our case, \(a = 2s\) and \(b = 2y\).
Thus \[
\int_0^\infty e^{-2 s x}\,\sin(2xy)\,dx =
\frac{2y}{(2s)^2 + (2y)^2} = 
\frac{y}{2\,(s^2 + y^2)}
\]
Therefore we here get \begin{align}
    \int_0^\infty e^{-2sx}\frac{\sin^2x}{x}\;d x  & =\int_0^1 
\Bigl(\int_0^\infty e^{-2 s x} \sin(2xy)\,dx\Bigr)dy\\
    &= \int_0^1 \frac{y}{2\,(s^2 + y^2)} \, dy \\
    & = \frac{1}{2 } \int_0^1 \frac{y}{s^2 + y^2} \, dy 
\end{align}
By Calculus we have (by chain rule): \[
\int_0^1 \frac{y}{s^2 + y^2}\,dy  =     \bigg[   \frac{1}{2}\log \bigl(s^2 + y^2\bigr)\bigg]_0^1  = \frac{1}{2}
\log\Bigl(\frac{s^2 + 1}{s^2}\Bigr) = 
\frac{1}{2}\log \bigl(1 + \tfrac1{s^2}\bigr)
\]
Thus we conclude: 
\begin{align}
    \int_0^\infty e^{-2sx}\frac{\sin^2x}{x}\;d x  &= \frac{1}{2 } \int_0^1 \frac{y}{s^2 + y^2} \, dy\\ &= \frac{1}{2} \cdot \frac{1}{2}\log \bigl(1 + \tfrac1{s^2}\bigr) \\
    &= \frac{1}{4}\log \bigl(1 + \tfrac1{s^2}\bigr)
\end{align}
as desired. 
\end{proof}



\section{$E\in\mathcal{A}\otimes\mathcal{A} \implies $diagonal of $E \in \mathcal{A}$}
  \begin{itemize}
  \item[(a)]Prove that if $E\in\mathcal{A}\otimes\mathcal{A}$, then  \[
      \{ x\in X: (x,x)\in E\} \in \mathcal{A}
    \]
  \item[(b)]Using this fact, find an example of a subset $E\subset \mathbb{R}\times \mathbb{R}$ such that $E_x\in \mathcal{L}(\mathbb{R})$ for all $x\in \mathbb{R}$ and $E^y\in \mathcal{L}(\mathbb{R})$ for all $y\in \mathbb{R}$, but $E\notin \mathcal{L}(\mathbb{R})\otimes  \mathcal{L}(\mathbb{R})$.
    \textit{Hint}: ask Vitali.
  \end{itemize}

\begin{proof}
    \textbf{of (a):}\\
We consider the map: \begin{align}
    \phi: X &\to X \times X  \\
     x & \mapsto (x,x)
\end{align}
Then it suffices to show that $\phi$ is $(\mathcal{A},\mathcal{A}\otimes \mathcal{A})$-measurable. Since if so, then for each $E\in\mathcal{A}\otimes\mathcal{A}$, $\phi^{-1}(E)  =  \{x\in X : (x,x)\in E\}\in \mathcal{A}$, which is exactly what we want.\\
Let $A \times B \in  \mathcal{A}\otimes \mathcal{A} $ be a measurable rectangle, we discover that:
$$\phi^{-1}(A \times B)  =  \{x\in X : x \in A , x \in B\} = A \cap B \in \mathcal{A}$$
\pic[0.3]{assets/hw7-image-20250314182209267.png}
We first prove a lemma:
\begin{lemma}
    Suppose $f: X\to Y\times Z$ is a function from a measurable space $(X,\mathcal{A})$ to a product measure space $(Y\times Z, \mathcal{B}_1 \otimes \mathcal{B}_2)$.\\
  Claim: If $f^{-1}(B_1\times B_2)\in\mathcal{A}$ for each measurable rectangle $B_1 \times B_2 \in \mathcal{B}_1 \otimes \mathcal{B}_2$, then $f$ is an $(\mathcal{A}, \mathcal{B}_1 \otimes \mathcal{B}_2)$-measurable function.
\end{lemma}
\begin{proof}
    \textbf{of Lemma:}\\
     Since $f^{-1}(B\times C) \in \mathcal{A}$ for each measurable rectangle $B_1 \times B_2 \in \mathcal{B}_1 \otimes \mathcal{B}_2$, the preimage of any countable disjoint unions of measurable rectangles, is also in $\mathcal{A}$, since $\mathcal{A}$ is an $\sigma$-algebra.\\
     We want to show: $f^{-1}(E) \in\mathcal{A}$ for any $E \in  \mathcal{B}_1 \otimes \mathcal{B}_2$. It is equivalent to show that \[
\mathcal{B}_1 \otimes \mathcal{B}_2 \subset   \mathcal{C} : = \{ E \in Y\times Z  : \phi^{-1} (E) \in \mathcal{A}        \}
\]
Note that, it suffices to show that: $\mathcal{C}$ is an $\sigma$-algebra. This is because we have shown $$\{\text{all disjoint unions of measurable rectangles in }   Y\times Z\} \subset \mathcal{C}$$, and this is an algebra generating $\mathcal{B}_1 \otimes \mathcal{B}_2$. Thus, if $\mathcal{C}$ is an $\sigma$-algebra, we must have \(
   \mathcal{B}_1 \otimes \mathcal{B}_2 \subset \mathcal{C}
   \). \\
   And since $\{\text{all disjoint unions of measurable rectangles in }   Y\times Z\}$ is an algebra, it suffices to show that $\mathcal{C}$ is a monotone class, by the monotone class lemma.\\
   
Suppose \(E_1 \subseteq E_2 \subseteq \cdots\) with each \(E_n \in \mathcal{C}\), i.e. $\phi^{-1}(E_n) \in \mathcal{A}$.
Since \(\{E_n\}\) is increasing, we hve
\[
  \phi^{-1}(E_1) \;\subseteq\; \phi^{-1}(E_2) \;\subseteq\; \cdots \;\subseteq\; \phi^{-1}(E_n) \;\subseteq\; \cdots
\]
Since \(\mathcal{A}\) is an \(\sigma\)-algebra, we have \[
  \phi^{-1}\Bigl(\bigcup_{n=1}^\infty E_n\Bigr) = 
  \bigcup_{n=1}^\infty \phi^{-1}(E_n) \in\mathcal{A}
\]Thus \[
   \bigcup_{n=1}^\infty E_n \;\in\; \mathcal{C}
\]
This is dually true for decreasing intersection, \textbf{finishing the proof that $\mathcal{C}$ is a monotone class thus $\sigma$-algebra,} \textbf{thus proving the lemma.}\\
\end{proof}
After we proved the Lemma, we return to the original statement, concluding that $\phi$ is $(\mathcal{A},\mathcal{A}\otimes \mathcal{A})$-measurable, thus finishing the proof: if $E\in\mathcal{A}\otimes\mathcal{A}$, then  \[
      \{ x\in X: (x,x)\in E\} \in \mathcal{A}
    \]
\end{proof}



\begin{solution}
    \textbf{of (b):}\\
Take a Vitali set $V \subset \mathbb{R}$, and consider:
\[
  E := \{(x,y) \in \mathbb{R}^2 : x \neq y\}
         \;\cup\;
         \{(x,x) : x \in V\}.
\]
\pic[0.3]{assets/hw7-image-20250314185531013.png}
Then for any fixed $x \in \mathbb{R}$, we have: \[
  E_x  = \bigl\{y : (x,y)\in E\bigr\} = 
    \begin{cases}
      \mathbb{R}, & x \in V \\
      \mathbb{R}\setminus\{x\}, & x \notin V
    \end{cases}
\]
And for any fixed $y \in \mathbb{R}$, we have: \[
  E^y =  \bigl\{x : (x,y)\in E\bigr\} = 
    \begin{cases}
      \mathbb{R}, & y \in V \\
      \mathbb{R}\setminus\{y\}, & y \notin V
    \end{cases}
\]Thus $E_x\in \mathcal{L}(\mathbb{R})$ for all $x\in \mathbb{R}$ and $E^y\in \mathcal{L}(\mathbb{R})$ for all $y\in \mathbb{R}$.\\
However, we have $E\notin \mathcal{L}(\mathbb{R})\otimes  \mathcal{L}(\mathbb{R})$, since by (a) we have proved that if $E\in \mathcal{L}(\mathbb{R})\otimes  \mathcal{L}(\mathbb{R})$, then \[
  V =     \{ x\in \mathbb{R}: (x,x)\in E\} \in \mathcal{L}(\mathbb{R})
    \]
But it contradicts with the fact that $V$ is not Lebesgue measurable.\\
Thus $E$ satisfies our requirements.\\
(This happends since, as shown in class, the product measure space of two complete measure space is not necesarily complete. Here, the diagonal is a null set in $\mathbb{R}^2$ and thus our Vitali portion is a subnull set, but $\mathcal{L}(\mathbb{R})\otimes \mathcal{L}(\mathbb{R})$ is not complete (its completion is $\mathcal{L}(\mathbb{R}^2)$.)
\end{solution}



\section{Too dense: $m(E\cap I)\le \alpha m(I) $ for all $I$ $\implies m(E)=0$ for mble $E$}
  Prove that if $E\subset \mathcal{L}(\mathbb{R})$ is a Lebesgue measurable subset such that  \[
    m(E\cap I)\le 0.123m(I) 
  \]
  for all open intervals $I\subset  \mathcal{L}(\mathbb{R})$, then $m(E)=0$. 
\begin{proof}
    Since $E$ is Lebesgue measurable, $m(E) = m^*(E)$.\\
    Let $\epsilon > 0$.\\
    Then by definition of outer mesure, we can pick open intervals seq $\{I_k\}_{k=1}^\infty$ covering $E$ s.t. \[
   m(E) >  \sum_{k=1}^\infty m(I_k )    -\epsilon
    \] Since $E \subset \bigcup_k I_k$, we have  \begin{align}
        E & = (\bigcup_k I_k)  \cap E \\
        & = \bigcup_k (I_k \cap E)\\
    \end{align}
    Thus \begin{align}
        m(E) = m(\bigcup_k (I_k \cap E)) &\leq \sum_k m(I_k \cap E)\quad\text{by ctbl subadditivity }\\
        &\leq 0.123 \sum_k \,m(I_k)  \quad \text{by our requirement}
    \end{align}
    Thus we have: 
    \begin{align}
            \sum_k  m(I_k )    -\epsilon &<  0.123 \sum_k m(I_k)\\
            0.877 \sum_k m(I_k) &< \epsilon\\
            \sum_k m(I_k) &< \frac{\epsilon}{0.877}
    \end{align}
  Thus   \[
  m(E) \leq   \sum_k  m(I_k )   <  \frac{\epsilon}{0.877}
    \]
    Since $\epsilon > 0$ is arbitrary, this proves that \[
    m(E)  = 0
    \]
\end{proof}




\section{给定任意 $0<\alpha <1$, prescribe 出一个在 $0$ 处 density 为 $\alpha/2$ 的集合}
  Let $0<\alpha <1$.
    Find an example of a Lebesgue measurable subset $E$ of $[0,\infty)\subset  \mathcal{L}(\mathbb{R})$ whose density at $0$ is $\alpha/2$. 
    \textit{Hint}: Consider $E=\bigcup_{n=1}^\infty I_n$. where $I_n=(x_n,x_n+\delta_n)$ are disjoint small intervals accumulating at $0$.
\begin{proof}
Consider take \[
E := \bigcup_{n=1}^{\infty}
\bigr( \frac{1}{n},
 \frac{1}{n}+ \frac{\alpha}{n(n-1)}\bigr)
\] as the union of a countable sequence of intervals drawing near $0$. \\
Notice: There intervals are \textbf{mutually disjoint}, since \[
\frac{1}{n-1} - \frac{1}{n}  =\frac{1}{n(n-1)}  > \frac{\alpha}{n(n-1)} 
\]
we thus have for $n \geq 2$, \[
\frac{1}{n} +
 \frac{\alpha}{n(n-1)} < \frac{1}{n-1}
\]
We use $x_n: = \frac{1}{n}$; $I_n := \bigl(x_{n},\,x_{n} + \delta_{n}\bigr)$ to denote each component interval; $J_n: = (x_n, x_{n-1})$ to denote the open interval where $I_n $ is located at; and $\delta_n := \frac{\alpha}{n(n-1)}$ to denote the length of each interval. Note that for each $n$, \[
\delta_n = \alpha (\frac{1}{n-1} - \frac{1}{n})  = \alpha(x_{n-1} - x_n)  = \alpha J_n
\]
\pic[0.4]{assets/hw7-image-20250314233130795.png}
Now we show that this set has Lebesgue density $\frac{\alpha}{2}$ at $0$ below.\\
Let $r>0$ (WLOG $r<1$), then we have  \[
     \frac{1}{n+1} < r \;\le\; \frac{1}{n} \quad \text{ for some } n \in \mathbb{N}
   \]
Then for each \(k \ge n+2\), we have \(\frac{1}{k} < \frac{1}{n+1} < r\). Hence $I_k$ is \textbf{entirely contained} in \((0,r)\):
\[ \bigcup_{k=n+2}^\infty I_k   \subseteq E \cap (-r,r) \]
We know that by telescoping,    \[
     \sum_{k = n+2}^\infty \frac{1}{k(k-1)} =
     \left(\frac{1}{n+1} - \frac{1}{n+2}\right)
     + \left(\frac{1}{n+2} - \frac{1}{n+3}\right)
     + \cdots = 
     \frac{1}{n+1}
   \]
Multiplying this by \(\tfrac{\alpha}{2}\) gives: \[
     \sum_{k = n+2}^\infty \frac{\alpha}{k(k-1)} = 
     \frac{\alpha}{n+1}
   \]
Thus by monotonicity of measure: \[
  m\bigl(E \cap (-r,r)\bigr) \geq   \frac{\alpha}{n+1}
\]
And for each $k \leq n$, $I_k$ exceeds $(0,r)$ on the right, thus we get dually: \[
  m\bigl(E \cap (-r,r)\bigr) \leq   \frac{\alpha}{n-1}
\] And we have: \[
\frac{2}{n+1} \leq m(-r,r) \leq \frac{2}{n}
\]since $\frac{1}{n+1}\leq r \leq \frac{1}{n}$.\\
Therefore we get: \[
   \frac{\frac{\alpha}{n+1}}{\frac{2}{n}}\leq   \frac{m\bigl(E \cap (-r,r)\bigr)}{m((-r,r))} \leq  \frac{\frac{\alpha}{n-1}}{\frac{2}{n+1}}
\]
Further simplify: \[
  \frac{n}{n+1}\cdot  \frac{\alpha}{2} \leq   \frac{m\bigl(E \cap (-r,r)\bigr)}{m((-r,r))} \leq    \frac{n+1}{n-1}\cdot  \frac{\alpha}{2}
\]
As $r\to 0^+$, we must have $n\to \infty$, and we know \[
\lim_{n\to \infty} \frac{n}{n+1} \cdot  \frac{\alpha}{2} = \lim_{n\to \infty} \frac{n+1}{n-1} \cdot  \frac{\alpha}{2}= \frac{\alpha}{2}
\]
Thus by \textbf{Squeeze Theorem}, we have: \[
     \lim_{r\to 0^+}  \frac{m\bigl(E \cap (-r,r)\bigr)}{m((-r,r))}= 
     \frac{\alpha}{2}
   \]
Hence by def, \(E\) indeed has Lebesgue density \(\alpha/2\) at \(0\).\\
(My note: The key point here is that, the harmonic seq shrinks very slowly in proportion as $n$ grows, $J_n$ almost have same length as $J_{n+1}$ for large $n$, thus $m(J_n) / m(\cup_{k> N}J_k) = 0$ as we knows, so that whether $r$ lies in $I_n$ or $J_n \setminus I_n$ does not quite matter. \\
On the other hand, the counterexample in class, using the geometric sequence as build block of $J_n$, fails since the length of $J_n$ is too much compared to $\cup_{k\geq n} J_k$, actually $m(J_n) = m(\cup_{k> n} J_k)$, thus whether $r$ lies in $I_n $ or $J_n \setminus I_n$ makes a lot difference, making the density at $0$ undefined.)
\end{proof}




\section{Seqs of complex numbers: $\ell^1\subsetneq\bigcap_{1<p<\infty}\ell^p$ and $\bigcup_{1<p<\infty}\ell^p\subsetneq\ell^\infty$}
  \begin{itemize}
  \item[(a)]
    Prove that $\ell^1\subsetneq\bigcap_{1<p<\infty}\ell^p$.
  \item[(b)]
    Prove that $\bigcup_{1<p<\infty}\ell^p\subsetneq\ell^\infty$.
  \end{itemize}
    
\begin{proof}
    \textbf{of (a):}\\
We first want to show: for any \(1 < p < \infty\), we have: \[
\ell^1 \subseteq \ell^p
\]
Fix $p>1$.\\
Let \((x_n) \in \ell^1\). By definition, 
\[
  \sum_{n=1}^{\infty} |x_n| < \infty
\]
We need to show that \(\sum_{n=1}^\infty |x_n|^p < \infty\).\\
\textbf{Claim: There are at most finitely many $n\in \mathbb{N}$ s.t. $|x_n| \geq 1$}.\\
Proof of Claim: Suppose for contradiction that there are inifinitely many $n\in \mathbb{N}$ s.t. $|x_n| \geq 1$, say, all terms in the subseqence $\{x_{n_j}\}_{j=1}^\infty$ has $|x_{n_j}|\geq 1$. Then \[
\sum_{n=1}^{\infty} |x_n| \geq \sum_{j=1}^{\infty} |x_{n_j}|\geq \sum_{j=1}^{\infty} 1 = \infty
\]which contradicts with \((x_n) \in \ell^1\).\\
Thus, suppose only on the finite terms $\{x_{n_j}\}_{j=1}^N$ we have $|x_{n_j}|\geq 1$ (WLOG $N\geq 1$). Then
\[
\sum_{n=1}^{\infty} |x_n| =\sum_{j=1}^{N} |x_{n_j}| + \sum_{n \not = n_j \text{ for any }j} |x_n|
\]
Since for $n$ s.t. n $\not = n_j \text{ for any subseq index }j$, we have $|x_n| <1$, for these indexes we have: \[
|x_n|^p < |x_n| \quad \text{for any } p >1
\] Thus we have \[
\sum_{n \not = n_j \text{ for any }j} |x_n|^p  < \sum_{n \not = n_j \text{ for any }j} |x_n| < \infty
\]
And also, \[
\sum_{j=1}^{N} |x_{n_j}|^p < \infty \quad \text{ since only have finite terms}
\]
Thus \[
\sum_{n=1}^{\infty} |x_n|^p =\sum_{j=1}^{N} |x_{n_j}|^p + \sum_{n \not = n_j \text{ for any }j} |x_n|^p < \infty
\]
Thus \[
\ell^1 \subseteq \ell^p
\]
Since $p>1$ is arbitrary, this proves that
\[
   \ell^1 \subseteq \bigcap_{1<p<\infty}\ell^p
\]
To show the strictness of the inclusion, we consider the \textbf{harmonic series} \(\sum_{n=1}^\infty \frac{1}{n}\). We know that it diverges and for any \(p>1\), the \textbf{\(p\)-series} \(\sum_{n=1}^\infty \frac{1}{n^p}\) (absolutely for sure) converges, thus  \(\bigl(\tfrac{1}{n}\bigr) \notin \ell^1\) but \(\bigl(\tfrac{1}{n}\bigr) \in \ell^p\) for every \(p>1\), showing that \[
   \ell^1 \not= \bigcap_{1<p<\infty}\ell^p
\]This finishes the proof that 
\[
   \ell^1 \subsetneq \bigcap_{1<p<\infty}\ell^p
\]
\end{proof}


    
\begin{proof}
    \textbf{of (b):}\\
Fix $p>1$.\\
Suppose sequence \(( x_n) \) belongs \(\ell^p\), then \[
\sum_{n=1}^{\infty} |x_n|^p < \infty
\]
This implies that \( x_n \to 0 \) as \( n \to \infty \), because if it did not, there would be infinitely many terms where \( |x_n| \) is bounded away from zero, leading to divergence of the sum.\\
Suppose for contradiction that \[
\sup_{n} |x_n| = \infty
\]Then there are infinitely many terms $n$ s.t. $|x_n| > 1$, since otherwise, exists some $N$ s.t. all $|x_n| \leq 1$ for $n\geq N$, then $\sup |x_n|\leq \max(1, \max_{1\leq n \leq N-1} |x_n| )< \infty$.\\
Suppose for the subseq $\{x_{n_j}\}_{j=1}^\infty$ we have $|x_{n_j}| > 1$. Thus \[
\sum_{n=1}^{\infty} |x_n|^p  \geq \sum_{j=1}^{\infty} |x_{n_j}|^p > \sum_{j=1}^{\infty} 1^p = \infty
\]which contradicts with \(\sum_{n=1}^{\infty} |x_n|^p < \infty\). Therefore we have: \[
\sup_{n} |x_n| < \infty
\]
This shows that \[
\ell^p \subseteq \ell^\infty
\] Since $p>1$ is arbitrary, this proves that \[
\bigcup_{1<p<\infty} \ell^p \subseteq \ell^\infty
\]Now we show the inclusion is strict. Consider the sequence \( x_n = 1 \) for all \( n \). Clearly, \(( x_n)\in \ell^\infty \) because it is bounded. However, \( x_n \notin \ell^p \) for any \( p > 1 \):
\[
\sum_{n=1}^{\infty} |1|^p = \sum_{n=1}^{\infty} 1 = \infty
\]
This shows \[
\bigcup_{1<p<\infty} \ell^p \not =  \ell^\infty
\]Thus we have \[
\bigcup_{1<p<\infty} \ell^p \subsetneq \ell^\infty
\]
\end{proof}




\vspace*{10mm}

\begin{center}
  \textit{Nur f\"ur Verr\"uckte}
\end{center}
(It's \textbf{really} not necessary to attempt these problems. Do not, under any circumstances, hand them in!) 





\section{Prescribing a Lebesgue density, Season 2}
  Let $0<\alpha <1$ and $n\ge 1$. 
    Find an example of a Lebesgue measurable subset $E$ of $\mathcal{L}(\mathbb{R})^n$ whose density at $0$ is $\alpha$.
    \textit{Hint}: think spherically.