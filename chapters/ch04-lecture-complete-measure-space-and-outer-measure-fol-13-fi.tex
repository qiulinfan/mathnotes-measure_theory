\chapter{complete measure space and outer measure [Fol 1.3, finished; 1.4]}

\begin{definition}{nul set, subnull set, almost everywhere}
对于 measure space $(X, \mathcal{M}, \mu)$
\begin{enumerate}
    \item 我们称 $A \in \mathcal{M}$ 为一个 \textbf{null set}, 如果 $\mu(A) = 0$;
    \item 我们称 $B \sub \mathcal{M}$ 为一个 \textbf{subnull set}, 如果存在某个 null set $A$ containing it.
    \item 我们称一个 statement about $X$ 是 \textbf{almost everywhere (a.e.)} 的, 如果这个 statement 除了在某个 null set 上之外, 在 $X$ 上处处成立.
\end{enumerate}
\end{definition}


\begin{definition}{complete measure space}
    我们称 $(X,\mathcal{M}, \mu)$ 是一个 complete measure space, 如果它其中的任意 subnull set 都是 null set. (即它 measurable)
\end{definition}
\begin{remark}
    我们知道, 根据 measure 的 monotonicity, subnull set 的 measure, 如果存在, 一定是 $\leq$ 它所在的 null set 的, 即一定 $=0$. 所以 complete measure space 的实际意思是: 这个 measure space 里, 任意 null set 的所有子集都是 measurable 的, 即所有足够小的集合都在这个 $\sigma$-algebra 里.
\end{remark}




\begin{example} 一个 not complete 的 measure space 的例子:
$$
X = \{1,2\}, \mathcal{M} = {\emptyset, X}, \mu(\forall) = 0.
$$
这个例子中, $\{1\}, \{2\}$ 这两个集合不是 measurable 的, 但是却是 nullset (全集) 的子集.
\end{example}



\begin{theorem}{every measure space can be completed}
    Suppose $(X, \mathcal{M},\mu)$ is a measure space.\\
    Let 
    \[
    \cN := \{\text{all null sets in }   \mathcal{M} \}
    \]
    Claim:
    \[
    \ol{M}  := \{   E\cup F \mid E \in \mathcal{M}, F \sub N \text{ for some } N \in \cN \}
    \]
    is a $\sigma$-algebra, 并且在 $\ol{\mathcal{M}}$ 上存在一个 unique 的 extension $\ol{\mu}$ of $\mu$.
\end{theorem}
\begin{proof}
    这一部分的 proof 以及 remark 在 hw2. 这里, $\overline{M}$ 称为 \textbf{completion of $\mathcal{M}$ with respect to $\mu$}, 以及 $\overline{\mu}$ 称为 \textbf{completion of $\mu$.}
\end{proof}




\section{outer measure}
\begin{definition}{outer measure}
    An outer measure on $X$ is a function $\mu^*: \mathcal{P}(X) \rightarrow {[0,\infty)}$ such that
    \begin{enumerate}
        \item $\mu(\varnothing) = 0$
        \item monotone ($A \subset  B \implies \mu^*(A) \leq \mu^*(B)$)
        \item countable subadditive ($\mu^*(\bigcup_{i=1}^\infty E_i)  \leq \sum_{i=1}^\infty \mu^*(E_i)$)
    \end{enumerate}
\end{definition}
\begin{remark}
    我们对比 measure 和 outer measure 的定义:
    measure 的条件比 outer measure 强在:
    \begin{enumerate}
        \item measure 是定义在一个严格的 $\sigma$-algebra 上的, 而 outer measure 则是定义在整个幂集上的. 
        \item measure 要求 disjoint countable additivity, outer measure 并不要求
    \end{enumerate}
\end{remark}

在这两个条件的缩减下, 我们规定 outer measure 具有 monotonicity 和 countable subadditivity. 注意: measure 本身也有这个性质, 这是 measure 的 countable additivity 的推论. \\
outer measure 的意义在于, 我们的 measure 只定义在 $\sigma$-algebra 上, 而我们想要给每个子集都赋予一个近似于测度的东西. 

\section{induce outer measure out of a "elementary length function"}
\begin{theorem}{construct outer measure out of an "elementary  length function" }\label{construct outer measure out of a "elementary length function"}
    另 $\cE \sub \cP(X)$ 为一个包含 $\varnothing, X$ 的集合, 并定义 $\rho: \cE \rar [0,\infty)$ 为一个满足 $\rho(\varnothing) = 0$ 的函数, 则
\[
\mu^*(A )  = \inf \{    \sum_{i=1}^\infty \rho(E_i) \mid E_i \in \cE \text{ for each i and }  A \sub \bigcup_{i=1}^\infty E_i    \}
\]
is an outer measure.
\end{theorem}
\begin{proof}
\begin{enumerate}
    \item 取所有 $E_j = \varnothing$, 得到 $\mu^*(\varnothing) = 0$
    \item monotonicity 显然, 因为如果 $A \sub B$, 那么 $A$ 取 inf 的这个集合是包含于 $B$ 的, 因而取到的 inf 是小于等于的.
    \item  证明 ctbl subadditivity, 我们使用经典的 $\epsilon  / 2^i$ argument. 这个 statement 直观上是显然的, 因为对一个 seq of sets, 每一个里面都有一个 seq of covering, 那么这个 seq of seq of covering 总体也是这个 seq union 的一个  covering. 不过我们不能这么说, 因为这里有一个 inf 操作的换序. 所以我们令 $\epsilon >0$, 对于每个 $A_i$ 的 covering $(E_{i,k})_{k\in\bN}$, 我们令 $\sum_k \rho(E_{i,k}) \leq \mu^*(A_i) + \epsilon / 2^i$,  最后可以得到 $\mu^*(\bigcup_i A_i) \leq \sum_i \mu^*(A_i)$. 由于 $\epsilon$ arbitrary, 得证.
\end{enumerate}
\end{proof}
\begin{example}
    我们取 $\cE$ 为 $\mathbb{R}$ 上所有的 intervals, 并取 $\rho $ 为 interval 的 length, 就得到了一个外测度. (也就是 Lebesgue outer measure)
\end{example}