\chapter*{Homework 2: on Carathéodory's and Hahn-Holmogrov Thm(40/40)}

\begin{center}
\textit{None of the following questions will be graded. Do them, but do not hand them in}.
\end{center}

\section*{The Borel--Cantelli Lemma}
Let \((X, \mathcal{A}, \mu)\) be a measure space. Let \(A_i \in \mathcal{A}\) for \(i \in \mathbb{N}\), and suppose that
\[
\sum_{i=1}^\infty \mu(A_i) < \infty.
\]
(a) Prove that \(\mu(\limsup_i A_i) = 0\), where
\[
\limsup_i A_i = \{x \in X \mid x \in A_i \text{ for infinitely many } i\}.
\]
(By the way, why is \(\limsup_i A_i\) measurable?)

(b) Conversely, is it true that if \(A_i \in \mathcal{A}\) for \(i \in \mathbb{N}\), and \(\mu(\limsup_i A_i) = 0\), then \(\sum_i \mu(A_i) < \infty\)? Provide a proof or a counterexample.
(Wrong)
\begin{remark}
\begin{theorem}{Borel--Cantelli Lemma}
    一个 measure 和有限的 set seq, 其 $\lim \sup$ (出现 infinitely many times 的元素) 是零测的.
\end{theorem}
其实这是 trivial 的, 因为如果出现 infinitely many times 的元素不是零测的, say $\mu(\lim\sup_i A_i) := k>0$, 那么有 infinitely many 个 $A_i$ 的测度大于等于 $k$, 那么 \(\sum_{i=1}^\infty \mu(A_i) > k \times \infty\) 就一定不是有限的了.\\
其 application: 一个 prob space 中, a ctbl seq of 事件发生的概率的和如果收敛, 那么它们包含的任何事件发生无穷多次的概率为 0,意味着事件至多发生有限次 (almost surely).
\end{remark}


\section*{The Completion of a Measure Space}
Let \((X, \mathcal{A}, \mu)\) be a measure space, and set
\[
\overline{\mathcal{A}} := \{E \cup F \mid E \in \mathcal{A} \text{ and } F \text{ is a } \mu\text{-subnull set}\}.
\]

(a) Prove that \(\overline{\mathcal{A}}\) is a \(\sigma\)-algebra.
(b) Define \(\overline{\mu}(A) := \mu(E)\) if \(A = E \cup F \in \overline{\mathcal{A}}\). Prove that \(\overline{\mu}\) is a well-defined measure on \(\overline{\mathcal{A}}\).
(c) Prove that \(\overline{\mu}\) extends \(\mu\) (i.e., \(\overline{\mu}(A) = \mu(A)\) if \(A \in \mathcal{A}\)).
(d) Prove that\textbf{ \(\overline{\mu}\) is the unique extension of \(\mu\) to \((X, \overline{\mathcal{A}})\)}. In other words, prove that if \(\mu'\) is another measure on \((X, \overline{\mathcal{A}})\) that extends \(\mu\), then \(\mu' = \overline{\mu}\).
(e) Prove that \textbf{\(\overline{\mu}\) is complete}.
(f) Suppose \((X, \mathcal{A}', \mu')\) is another complete measure space that extends \((X, \mathcal{A}, \mu)\) (i.e., \(\mathcal{A} \subset \mathcal{A}'\) and \(\mu'|_{\mathcal{A}} = \mu\)). Show that \textbf{\(\overline{\mathcal{A}} \subset \mathcal{A}'\) and \(\mu'|_{\overline{\mathcal{A}}} = \overline{\mu}\).} 
\textbf{Hint:} Start by reading Theorem 1.9 in Folland.
\begin{proof}
    略.(嘻嘻)
\end{proof}


\section*{The Hahn--Kolmogorov Extension as a Completion}
Let \((X, \mathcal{A}_0, \mu_0)\) be a \(\sigma\)-finite measure pre-measure space, and \((X, \mathcal{A}, \mu)\) its Hahn–Kolmogorov extension. Prove that \((X, \mathcal{A}, \mu)\) is the completion of its restriction to the \(\sigma\)-algebra \(\langle \mathcal{A}_0 \rangle\) generated by \(\mathcal{A}_0\).
\begin{proof}
    Proved in lec notes.
\end{proof}



\newpage
\begin{center}
\textit{Some of the following questions will be graded. Do them, and do hand them in}.
\end{center}


\section*{\(\mu(\emptyset) = 0\) 的定义并非 redundant}
Let \((X, \mathcal{A})\) be a measurable space. Is the condition \(\mu(\emptyset) = 0\) in the definition of a measure on \((X, \mathcal{A})\) redundant? In other words, if \(\mu : \mathcal{A} \to [0, \infty]\) is a function such that
\[
\mu\left(\bigcup_{i=1}^\infty A_i\right) = \sum_{i=1}^\infty \mu(A_i),
\]
for any disjoint subsets \(A_i \in \mathcal{A}\), \(i \in \mathbb{N}\), does it follow that \(\mu(\emptyset) = 0\)? If not, what can you say?
\begin{proof}
    It does not follow.\\
    Counterexample: Consider $\mu(E) = \infty \;\; \forall E \in \mathcal{A}$.\\
    This measure satisfies the countably disjoint additivity condition, since for every disjoint sequence of sets in $\mathcal{A}$, $\mu\left(\bigcup_{i=1}^\infty A_i\right) = \sum_{i=1}^\infty \mu(A_i) = \infty$ has infinite measure. 
\end{proof}


\section*{measurable set seq 的 limit 也 measurable (且如果 seq tail $\sigma$-finite $\implies$ limit commute )}
Let \((X, \mathcal{A}, \mu)\) be a measure space, and let \(A_i \in \mathcal{A}, i \in \mathbb{N}\). Assume that the sets \(A_i\) converge to the set \(A \subset X\) in the sense that:
- If \(x \in A\), then \(x \in A_i\) for all but finitely many \(i\);
- If \(x \notin A\), then \(x \notin A_i\) for all but finitely many \(i\).

\noindent \textbf{(a) Prove that \(A\) is measurable, that is, \(A \in \mathcal{A}\).}
\begin{proof}
    Deduing from the conditions:
    If \(x \in A\), then \(x \in A_i\) for all but finitely many \(i\); $\implies $ $A \subset \liminf A_i$
    If \(x \notin A\), then \(x \notin A_i\) for all but finitely many \(i\). $\implies $ if \(x \in A_i\) for all but finitely many \(i\) then $x\in A$ $\implies$  $\limsup A_i \subset A$
    Thus 
    \begin{equation}
    \limsup A_i \subset A \subset \liminf A_i    
    \end{equation}
    
\noindent   \textbf{Claim1: For any sequence of sets $(A_i)_{i \in \mathbb{N}}$, we have 
$$\liminf A_i \subset \limsup A_i$$} 
Proof of Claim 1: Follows trivially from the definition, since \(x \notin A_i\) for all but finitely many \(i\) $\implies$ \(x \notin A_i\) for infinitely many \(i\).\\\\
Combining claim (1) with (2.1) we have
\begin{equation}
\limsup A_i  =  A  = \liminf A_i    
\end{equation}
\noindent \textbf{Claim 2: For any sequence of sets $(A_i)_{i\in \mathbb{N}}$ in a $\sigma$-algebra, $\liminf_i Ai$ and $\limsup_i A_i$ is also in the $\sigma$-algebra.}\\
Proof of Claim 2: This follows from the def and fact that union and intersection of a countable sequence sets in a $\sigma$-algebra is also in this $\sigma$-algebra. We have
\pic[0.8]{assets/hw2(1).jpeg}    
This finishes the proof of Claim 2.\\
\noindent Combining claim 2 with (2.2), $A \in \mathcal{A}$, this finishes the proof.\\
\end{proof}

\noindent \textbf{(b)  Prove that if there exists \(n \geq 1\) such that \(\mu\left(\bigcup_{i=n}^\infty A_i\right) < \infty\), then \(\mu(A) = \lim_i \mu(A_i)\).}
\begin{proof}
\pic[0.7]{assets/hw2(2).png}
\pic[0.7]{assets/hw2(3).png}
    
\end{proof}


\noindent \textbf{(c) Give an example showing that the condition in (b) is necessary}.
\begin{solution}
\noindent  Let $\mu$ be the Lebesgue measure defined on $\mathcal{B}(\mathbb{R})$. We set for each $i\in \mathbb{N}$ that
\[
   A_i =(i,i+1)
\]
\noindent Since this is an interval, it is Lebesgue measurable. Note that no element of any $A_i$ show up infinitely many times in the sequence. So 
$$
\liminf A_i = \limsup A_i = \varnothing
$$
\noindent So $A = \varnothing$, we have $\lim_i \mu(A_i) = 0$.
\noindent But we have $\lim_i \mu(A_i)  = 1$ since it is true for every $i$.\\
\noindent In this case, $\mu(\bigcup_{i=1}^\infty A_i) = \infty$, which causes (b) to fail.
\end{solution}


\noindent \textbf{Hint:} In analysis, it is often fruitful to use \(\limsup\) and \(\liminf\) to study limits.\\

\section*{measure space of two elements}
\noindent Let \(X\) be a set with two elements, for example, \(X = \{O, Q\}\).\\
\noindent \textbf{(a) Find all \(\sigma\)-algebras on \(X\).}
\begin{solution}
    \begin{enumerate}
    \item trivial \(\sigma\)-algebra:\[
     \mathcal{A}_1 :=\{\varnothing,\; X\}
   \]
   \item power set:
   \[
     \mathcal{A}_2 := \mathcal{P}(X)
     \;=\;\{\varnothing,\;\{O\},\;\{Q\},\;X\}
   \]
\end{enumerate}
These are the only $\sigma$-algebras on $X$.\\\\
\end{solution}

\noindent \textbf{(b) Let \(\mathcal{A}\) be a \(\sigma\)-algebra on \(X\), and \(\mu\) a measure on \((X, \mathcal{A})\). Is \(\mu\) necessarily complete? Provide a proof or a counterexample.}
\begin{solution}
    It is not necessarily complete.\\
    Cosider the trivial \(\sigma\)-algebra:\(\mathcal{A}_1 :=\{\varnothing,\; X\}\), and set $\mu$ as that $\mu(\varnothing) = \mu(X) = 0$. This makes $X$ a null set, so $\{O\}, \{Q\}$ are subnull sets, but they are not measurable by $\mu$.\\\\
\end{solution}
\noindent \textbf{(c) Find all outer measures \(\mu^*\) on \(X\). For each outer measure on \(X\), find the \(\sigma\)-algebra of \(\mu^*\)-measurable sets (see Carathéodory’s theorem).}
\begin{solution}
    Suppose \(\mu^*\) is an outer measure on \(X\). Since $\mathcal{P}(X)$  only has four elements: \(\varnothing\), \(\{O\}\), \(\{Q\}\), \(X\); and the outer measure of $\varnothing$ is 0, so we first parametrize \(\mu^*\) by:
\[
a :=\mu^*(\{O\}),\quad
b :=\mu^*(\{Q\}),\quad
c := \mu^*(X).
\]
\end{solution}
\noindent Then $\mu^*$ is well-defined iff it satisfies:
\begin{enumerate}
    \item $a,b \leq c$
    \item \(c = \mu^*(\{O\}\cup\{Q\}) \;\le\; \mu^*(\{O\}) + \mu^*(\{Q\}) \;=\; a + b.\)
\end{enumerate}
\noindent Any $(a,b,c) \in [0,\infty]^3$ satisfying
\[
\max(a,b) \;\le\; c \;\le\; a+b,
\]
can make $\mu^*$ a well-defined outer measure on \(X\).\\
\noindent Therefore
\[
S :=\{\text{all } \sigma \text{-algebra on } X\} = \{ \mu^* : \mathcal{P}(X) \to [0,\infty]\mid \max(\mu^*(\{O\}), \mu^*(\{Q\})) \leq \mu^*(X) \leq \mu^*(\{O\})+ \mu^*(\{Q\})\} 
\]
\\
Now we specify the \(\sigma\)-algebra of \(\mu^*\)-measurable sets for each $\mu^* \in S$.\\
By Carathéodory's criterion, a set \(E\subset X\) is \(\mu^*\)-measurable iff for all \(A\subset X\),
\[
\mu^*(A)
\;=\;
\mu^*(A\cap E)\;+\;\mu^*(A\cap E^c).
\]
Note that \(\varnothing\), \( X\) are always measurable since for any \(A \subset X\), \(A\cap \varnothing=\varnothing,\;A\cap ( \varnothing)^c = A\); and \(A\cap X=A,\;A\cap (X)^c=\varnothing\).  So it suffices to check for $\{O\}, \{Q\}$. We first check for $\{O\}$. $\{O\}$ is \(\mu^*\)-measurable iff \(
  \mu^*(A) \;=\;
  \mu^*(A\cap \{O\}) \;+\;\mu^*(A\cap \{O\}^c)
  \) for any choice of $A$.
There are only four possibilities for \(A\): \(\varnothing\), \(\{O\}\), \(\{Q\}\), \(X\).
\begin{enumerate}
    \item   If \(A = \varnothing\), both sides are 0, always stands.
    \item If \(A = \{O\}\), then \(\mu^*(\{O\}) + \mu^*(\varnothing)=a+0=a\), always stands.
    \item If \(A = \{Q\}\), then \(\mu^*(\varnothing)+\mu^*(\{Q\})=0+b=b\), always stands.
    \item If \(A = X\), then \(\mu^*(X)=c =\mu^*(\{O\})+\mu^*(\{Q\})=a+b\).
\end{enumerate}
\noindent Therefore $\{O\}$ is \(\mu^*\)-measurable iff $c =a+b$.  For the same reasoning, \(\{Q\}\) is \(\mu^*\)-measurable iff \(c = a+b\).\\
Thus we can conclude that:
\begin{enumerate}
    \item If \(c = a+b\), $\{   \mu^* \text{-measurable sets}   \} =  \mathcal{P}(X)$.
    \item otherwise, $\{   \mu^* \text{-measurable sets}   \} =  \{\varnothing, X\}$.
\end{enumerate}
\noindent \textbf{(d) Find an example of a collection \(\mathcal{E}\) of subsets of \(X\) with \(\emptyset, X \in \mathcal{E}\) and a function \(\rho : \mathcal{E} \to [0, \infty]\) with \(\rho(\emptyset) = 0\) such that \(\mathcal{E} \not\subset \mathcal{A}\), where \(\mathcal{A}\) is the Carathéodory \(\sigma\)-algebra for the outer measure \(\mu^*\) induced by \((\mathcal{E}, \rho)\).}
\begin{solution}
\noindent Consider \(\mathcal{E}=\{\varnothing,X,\{O\}\}\), with $\rho$ such that \(\rho(\emptyset)=0\), \(\rho(X)=1\), \(\rho(\{O\})=1\). \\ \noindent The outer measure \(\mu^*\) induced by $\mu^*$ is: \(\mu^*(X)=1\), \(\mu^*(\{O\})=1\), \(\mu^*(\{Q\})=1\). (the inf of length sum of sets covering $\{Q\}$ is 1, by taking $\{X\}$ as the covering.)\\
\noindent Since \(c \not = a+b\), by (4), the Carathéodory \(\sigma\)-algebra by $\mu^*$ by $\cE$ is \(\{\varnothing,X\}\), so \(\mathcal{E}\not\subset \mathcal{A}\).\\\\
\end{solution}
\noindent \textbf{Remark:} The Hahn–Kolmogorov theorem states that if \(\mathcal{E} = \mathcal{A}_0\) is an algebra and \(\rho = \mu_0\) is a pre-measure, then \(\mathcal{A}_0 \subset \mathcal{A}\). This exercise provides a counterexample when \(\mathcal{E}\) and \(\rho\) are general.
、



\section*{Hahn–Kolmogorov Collapse (when $\mu_0$ not $\sigma$-finite)}
Let \(X \subset \mathbb{R}\) be the set of dyadic rational numbers, that is, the set of numbers of the form \(\frac{r}{2^n}\), where \(r\) and \(n\) are integers. Let \(\mathcal{A}_0 \subset \mathcal{P}(X)\) be the collection of finite unions of intervals of the form \((a, b] \cap X\), where \(-\infty \leq a < b \leq \infty\).

\noindent \textbf{(a) Prove that \(\mathcal{A}_0\) is an algebra.}
\begin{proof}
\begin{enumerate}
    \item \(\varnothing \in \mathcal{A}_0\), since it is the empty union of intervals of the given form.
    \item \textbf{Closed under complements}:  Let \(A\in \mathcal{A}_0\). Then \(A\) is a finite union of intervals of the form \((a_i,b_i]\cap X\).  So  \begin{align}
     A^c \cap X  &= X\setminus A \\
     &=  X\setminus \bigcup_{i=1}^n \bigl((a_i,b_i]\cap X\bigr) \bigr) \\
     &=
 X\cap \bigl(\bigcap_{i=1}^n \bigl((a_i,b_i]\cap X\bigr)^c \bigr) \\
    &  =  X\cap \bigl(\bigcap_{i=1}^n \bigl((-\infty, a_i] \cup (b_i, \infty]\bigr) \bigr)
\end{align} 
Note that finite intersection of intervals of the form \((-\infty,a_i]\), \((b_i,\infty]\) is still of this form. Hence \(A^c\cap X\in \mathcal{A}_0\).
    \item \textbf{Closed under finite unions}: Suppose \(A_1\) and \(A_2\) are finite unions of intervals \(\bigl((a_i,b_i]\cap X\bigr)\), then \(A_1\cup A_2\) is still a finite union of intervals of that form. (They either merge into one such interval, so are disjoint.) Hence \(A_1\cup A_2\in \mathcal{A}_0\).  The same reasoning extends to any finite union.
\end{enumerate}
\noindent \textbf{This finishes the proof that \(\mathcal{A}_0\) is an algebra on \(X\).}\\\\
\end{proof}

\noindent \textbf{(b) Prove that the \(\sigma\)-algebra on \(X\) generated by \(\mathcal{A}_0\) equals \(\mathcal{P}(X)\).}
\begin{proof}
Since $<\mathcal{A}_0> \subset \mathcal{P}(X)$, it suffices to show that $\mathcal{P}(X) \subset <\mathcal{A}_0>$.
\noindent Note that $X$ is countable, so any set in $\mathcal{P}(X)$ is a countable union of singleton sets. Thus it suffices to show that any singleton set $\{x\}$ where $x\in X$ is in $<\mathcal{A}_0>$, since if so, then any countable union of singleton sets from $\mathcal{P}(X)$ is also in $<\mathcal{A}_0>$, with implies that $\mathcal{P}(X) \subset <\mathcal{A}_0>$\\
\noindent Let $x \in X$. Then we have: 
   \[
     \{x\} \;=\; \bigcap_{n=1}^\infty \bigl( (x - \frac{1}{2^{n}},\,x]\cap X \bigr),
   \]
 since $x$ is in the RHS set, and for any $y <x$, we can find a $n \in \mathcal{B}$ such that $x - \frac{1}{2^n} > y$.\\
 \noindent This finishes the proof that  \(<\mathcal{A}_0> = \mathcal{P}(X)\).
\end{proof}


\noindent \textbf{(c) Define \(\mu_0 : \mathcal{A}_0 \to [0, \infty]\) by \(\mu_0(\emptyset) = 0\) and \(\mu_0(A) = \infty\) for \(A \neq \emptyset\). Prove that \(\mu_0\) is a pre-measure on \(\mathcal{A}_0\)}
\begin{proof}
It suffices to show the countable disjoint additivity.\\
Let $(A_i)_{i\in\mathcal{N} }$ be a sequence of disjoint sets in $\mathcal{A}_0$.\\
Case 1: all $A_i = \varnothing$, then $\sqcup_{i\in \mathcal{N} } A_i = \varnothing$, so $\mu_0(\sqcup_{i\in \mathcal{N} } A_i) = \sum_{i\in\mathcal{N} }\mu_0(A_i) = 0$.\\
Case 2: $A_k \not = \varnothing$ for some $k$, then $\mu_0 (A_k) = \infty$ and $\sqcup_{i\in \mathcal{N} } A_i \not= \varnothing$. Thus $\sum_{i\in\mathcal{N} }\mu_0(A_i) \geq \mu_0(A_k)= \infty = \mu_0(\sqcup_{i\in \mathcal{N} } A_i )$.\\
The two cases cover all circumstances, finishing the proof.\\
\end{proof}


\noindent \textbf{(d) Prove that there exist infinitely many different measures \(\mu\) on \(\mathcal{P}(X)\) whose restriction to \(\mathcal{A}_0\) equals \(\mu_0\).}
\begin{proof}
    Given $n \in \mathbb{N}$, We define the "n-timed counting measure" on a $\sigma$-algebra $S$ as:
$$
    \mu_{count_n}(E) := \begin{cases}
        n\times  \#(E) \;\;, \text{ if }  E \text{ is finite }  \\
        \infty \;\; ,\text{ if }  E \text{ is infinite }
    \end{cases}
 $$

 \noindent \textbf{Claim 1: For any set $X$ and any $\sigma$-algebra $S$ on $X$, the "n-timed counting measure" is a well-defined measure on $S$, for all $n \in \mathbb{N}$.}\\
Proof of claim 1: $\mu_{count_n}(\varnothing) = 0$ since $\card(\varnothing)  =0$, and countable disjoint additivity trivially follows from the rule of counting.\\
\noindent \textbf{Claim 2: for any $n \in \mathbb{N}$, $\mu_{count_n}(E) $ on \(\mathcal{P}(X)\) restricted to \(\mathcal{A}_0\) equals \(\mu_0\).}
Proof of claim 2: Let $E \in \mathcal{A}_0 \setminus \varnothing$, then $E$ contains at least one interval of the form \((a, b] \cap X\), where \(-\infty \leq a < b \leq \infty\). Sicne $a < b$, there are infinitely many elements in \((a, b] \cap X\), so $\mu_{count_n}(E) = \infty$.\\
\noindent This finishes the proof of the original statement.\\
\end{proof}

\noindent \textbf{(e) Explain why (d) does not contradict the uniqueness part of the Hahn–Kolmogorov theorem (see Theorem 1.14 in Folland).}\\
\begin{solution}
    \noindent This is because Hahn–Kolmogorov theorem requires $\mu_0$ to be $\sigma$-finite to extend uniquely on $<\mathcal{A}_0>$. But $\mu_0$ here is not $\sigma$-finite.\\
\end{solution}






\section*{Nur für Verrückte (Only for nuts)}
(It’s really not necessary to attempt these problems. Do not, under any circumstances, hand them in!)

1. Let \((X, \mathcal{A}, \mu)\) and \((Y, \mathcal{B}, \nu)\) be measure spaces. Define a morphism from \((X, \mathcal{A}, \mu)\) to \((Y, \mathcal{B}, \nu)\) to be a map \(f : X \to Y\) that is measurable, that is, \(f^{-1}(B) \in \mathcal{A}\) for all \(B \in \mathcal{B}\), and moreover measure preserving, in the sense that \(\mu(f^{-1}(B)) = \nu(B)\) for all \(B \in \mathcal{B}\).

(a) Prove that measure spaces with measure-preserving maps as morphisms form a category. Denote this category by \(C_3\).

(b) Denote by \(C_1\) the category of sets, and by \(C_2\) the category of measurable spaces (see HW1). Consider the evident forgetful functors \(C_3 \to C_2\) and \(C_2 \to C_1\). Are these functors faithful? Are they full? Are they essentially surjective?