\chapter*{Homework 8: on $L^p$ spacecs (50/50)}

\vspace*{5mm}
\begin{center}
\textit{Some of the following questions will be graded. Do them, and do hand them in}.
\end{center}

\section*{一个 Barely in $L^1$ 的函数}
Find a function $f\in L^1(\mathbb{R}^{2025})$ such that $f\not\in L^p(U)$ for any $p>1$ and any nonempty open subset $U\subset \mathbb{R}^{2025}$. \textit{Hint}: see HW5(g).
\begin{solution}
Recall Hw 5(g):For $\alpha\in(0,1)$, define $g_\alpha\colon \mathbb{R}\to \mathbb{R}$ by $g_\alpha(x)=(1-\alpha)x^{-\alpha}$ for $0<x<1$ and $g_\alpha(x)=0$ otherwise. Let $(x_n)_n$ be an enumeration of the rational numbers, and define $f\colon \mathbb{R}\to[0,\infty]$ by \[
    f(x)=\sum_{n=1}^\infty2^{-n}g_{1-n^{-n}}(x-x_n)
  \]
We have proved $f$ has the following properties:
\begin{itemize}
    \item $f$ is Lebesgue integrable and $\int_\mathbb{R} |f|\; d m = \int_\mathbb{R} f\; d m<\infty$;
    \item \(\int_I f^p\; d m=\infty \quad \text{for all }p >1, \text{for all open interval } I\)
\end{itemize}
Now we continuing this definition of $f$, and further define:\begin{align*}
    F: \mathbb{R}^{2025} &\to \mathbb{R} \\
(x_1,\cdots, x_{2025})    &\mapsto \prod_{j=1}^{2025} f(x_j)
\end{align*}
\textbf{Claim 1: }$F \in L^1(\mathbb{R}^{2025})$.\\
To prove this, we just need this lemma.
\begin{lemma}{(Folland 2.5 exercise 51)}
If $f$ is $\mathcal{M}$-measurable, $g$ is $\mathcal{N}$-measurable, then $fg$ is $(\mathcal{M}\otimes \mathcal{N})$-measurable.\\
Particularly, if $f \in L^1(\mu)$, $g \in L^1 (\nu)$, then $fg \in L^1(\mu \times \nu)$ and \[
\int fg \; d(\mu\times \nu) = \bigg(f \; d\mu \bigg) \bigg( g \; d\nu \bigg)
\]
\end{lemma}
It seems like we have not proved this yet so here let's prove it.
\begin{proof}
    of Lemma: Define \[
    h : = fg
    \]
Note \[
   p:  (u,v) \mapsto uv
    \]
from $\mathbb{C}^2\to \mathbb{C}$ is a product of two coordinate maps, thus is measurable since coordinate map is measurable, and product of two measurable functions is measurable. \\
And \[
\pi:    (x,y)\mapsto (f(x),g(y))
    \] from $X\times  Y\to \mathbb{C}^2$ is $(\mathcal{M}\otimes \mathcal{N},\mathbb{C}^2)$-measurable, since for any measurable rectangle $B_1 \times B_2 \in \mathbb{C}^2$, we have \[
    \pi^{-1}(B_1 \times B_2)  = f^{-1}(B_1) \times g^{-1}(B_2) \in \mathcal{A}\otimes \mathcal{B} \quad  \text{as a measurable rect}
    \]
    Thus $h = \pi \circ p$ is $(\mathcal{M}\otimes \mathcal{N})$-measurable, as a \textbf{composition of two measurable functions.} \\
To show the second statement, it suffices to assume $f,g$ takes positive real values, since otherwise we can decompose $f,g$ into their real and imaginary parts, and for each part decompose them into positive part minus negative part.\\
Take two seq of simple functions approximating $f,g$ respectively from below, say: \[
   s_n(x)  := 
   \sum_{k=1}^K a_k\,\chi_{A_k}(x),  
   \quad
   t_n(y)  =
   \sum_{\ell=1}^L b_l \,\chi_{B_l}(y)
\]
their product on \(X\times Y\) is \[
   s_n(x)\,t_n(y)  = 
   \sum_{k=1}^K\sum_{l=1}^L 
       a_k\,b_l \chi_{A_k\times B_l}(x,y)
\]
By definition of the product measure \(\mu\times\nu\), we have
\[
   (\mu\times\nu)\bigl(A_k\times B_l\bigr) = 
   \mu(A_k)\,\nu(B_l)
\]
Hence \begin{align*}
     \int_{X\times Y} s_n(x)\,t_n(y)\,d(\mu\times\nu) &= 
   \sum_{k,l} a_k\,b_l\,\mu(A_k)\,\nu(B_l) \\ &= 
   \Bigl(\sum_{k} a_k\,\mu(A_k)\Bigr)\,\Bigl(\sum_{l} b_l\,\nu(B_l)\Bigr) \\
   &= \bigg( \int_X s_n\,d\mu \bigg)
 \bigg(  \int_Y t_n\,d\nu\bigg)
\end{align*}
Since \(s_n(x)\nearrow f(x)\) and \(t_n(y)\nearrow g(y)\), we also have $s_n t_n \nearrow fg$, thus by \textbf{MCT} we have: \[
\lim_n  \int_X s_n\,d\mu  = \int_X f,\quad \lim_n  \int_Y t_n\,d\nu  = \int_Y g
\] and \[
\lim_n \int_{X\times Y} s_n(x)\,t_n(y)\,d(\mu\times\nu)  =  \int_{X\times Y}   fg \; d(\mu\times \nu)
\]
Then, since the right side are two finite positive reals, we have: \[
   \int_{X\times Y} f(x)\,g(y)\,d(\mu\times\nu) =
   \Bigl(\int_X f\,d\mu\Bigr)\,   \Bigl(\int_Y g\,d\nu\Bigr) < \infty
\]
Thus $h = fg \in L^1(\mu \times \nu)$
\end{proof}
After proving the Lemma, we can extend it to the product of any finite number of functions. Applying it, we get \[
F \in L^1(\mathbb{R}^{2025})
\]
Then, we take arbitrary open set $U \subset \mathbb{R}^{2025}$ and arbitrary $p>1$, and fix it.\\
Claim 2: $F \not \in L^p (U)$.
Sine $U$ is open in $\mathbb{R}^{2025}$, it must contain an open ball, thus must contain an open box (e.g., the one internally connected in the open ball), say $I_1 \times \cdots \times I_{2025}$.\\
Suppose for contradiction that $F \in L^p (U)$.\\
Then by monotonicity of integration: \[
  \int_{I_1\times\cdots\times I_{2025}} |F|^p \,d(x_1,\ldots,x_{2025}) \leq \int_U |F|^p \,d(x_1,\ldots,x_{2025})  < \infty
\]
Then by Fubini's Thm we have:
\[
     \int_{I_1\times\cdots\times I_{2025}}
       \prod_{j=1}^{2025} |f(x_j)|^p
     \,d(x_1,\ldots,x_{2025}) = \prod_{j=1}^{2025}
     \int_{I_j} |f(x_j)|^p \,dx_j < \infty
   \]
Since for each $I_j$, we in hw 5 proved that: \[
\int_{I_j} |f(x_j)|^p\; dx_j=\infty 
\]
This contradicts with what we got. Thus we must have $F \not \in L^p (U)$.\\
This finishes the proof.
\end{solution}




\section*{$L^p$ norm version of LDT}
  Let $1\le p<\infty$. Suppose that $f\in L^p(\mathbb{R})$. 
  Prove that   \[
    \lim_{r\to 0} \frac1{2r} \int_{x-r}^{x+r} |f(y)-f(x)|^p \;d y=0
  \]
for a.e. $x$. \\
(Hint: Follow the proof of the Lebegue Differentiation Theorem when $p=1$, i.e. approximate $f$ by $g\in C_c(\mathbb{R})$ satisfying $\|f-g\|_p<\epsilon$. At some point, use Minkowski's inequality; note that we have $|a+b|\le |a|+|b|$, but we don't have $|a+b|^p\le |a|^p+|b|^p$ for $p>1$.) 

\begin{proof}
\textbf{  Claim 1: The statement is true for $f \in C_c^0(\mathbb{R}^n)$}.\\
Proof of Claim 1:Let \( f \in C_c^0(\mathbb{R}) \), then it is uniformly continuous on any compact set, thus uniformly continuous on an open ball, since its closure is compact.\\
Therefore, let \( \epsilon > 0 \), then there exists \( \delta > 0 \) such that \[
|y - x| < \delta \implies|f(y) - f(x)| < \epsilon
\]
Thus
\[
|f(y) - f(x)|^p < \epsilon^p \quad \text{whenever } \quad|y - x| < \delta
\]
Now fix \( x \in \mathbb{R} \), and take \( r < \delta \). Then,
\[
\frac{1}{2r} \int_{x - r}^{x + r} |f(y) - f(x)|^p \, dy < \frac{1}{2r} \int_{x - r}^{x + r} \epsilon^p \, dy = \epsilon^p
\]
Since this holds for all \( r < \delta \), we get: \[
\limsup_{r \to 0} \frac{1}{2r} \int_{x - r}^{x + r} |f(y) - f(x)|^p \, dy \le \epsilon^p
\]
Since \( \epsilon > 0 \) was arbitrary, this proves claim 1:
\[
\lim_{r \to 0} \frac{1}{2r} \int_{x - r}^{x + r} |f(y) - f(x)|^p \, dy = 0
\]
Next we will prove the general case.\\
\textbf{Step 1: Translate the problem into proving the measure of disqualified points is zero, for which we can use arbitrary error bound.}\\
Define for each $x\in \mathbb{R}, r>0$:
$$Q(x,r) : =   \int_{x - r}^{x + r} |f(y) - f(x)|^p \; dy = \| f \chi_{B_r(x)}-f(x)\chi_{B_r(x)}\|_{p}^p$$
And then we define for each $x\in \mathbb{R}$:
\[
Q(x) : =  \limsup_{r\to 0+} \frac{Q(x,r)^{1/p}}{(2r)^{1/p}}
\]
Then what we want to show is just: \[ m(\{x: Q(x) > 0\}) = 0\]
which is equivalent to show: \[
m(\{x : Q(x) \geq \alpha \}) = 0 \quad \text{for all } \alpha>0
\]
Fix $\alpha >0$. It suffices to show: for any $\epsilon >0$, we have: \[
m(\{x : Q(x) \geq \alpha \}) < \epsilon
\]
Now fix $\epsilon > 0$.  Take $g\in C_c^0(\mathbb{R})$ s.t. $\|f-g\|_p < \epsilon$. This can be done, by the density of $ C_c^0(\mathbb{R})$ in $L^p(m)$.\\
\textbf{Step 2: Bound the $\lim_{r\to 0} \frac1{2r} \int_{x-r}^{x+r} |f(y)-f(x)|^p \;d y$ by $\epsilon$-controllable expressions, using Minkowski's ineq; thus bound the measure of disqualified points by two $\epsilon$-controllable sets}\\
Define for each $x\in \mathbb{R}, r>0$:
$$Q(x,r) : =   \int_{x - r}^{x + r} |f(y) - f(x)|^p \; dy = \| f \chi_{B_r(x)}-f(x)\chi_{B_r(x)}\|_{p}^p$$
This is nonnegative. And since $|f-f(x)|$ is measurable and $L^p$ (since $|f|$ is $L^p$), $|f-f(x)|^p$ is $L^1$, and thus, recall we proved in lecture that $Q(x,r)$ is jointly continuous in $r$ and $x$.\\
By triangular ineq \[Q(x,r)^{1/p} \leq    \bigg( \int_{x-r}^{x+r} \big(|f(y ) - g(y) | + |g(y) -g(x)| + |g(x) - f(x)|\big) ^p \; dy \bigg)^{1/p}\]
Then by Minkowski's ineq: 
\begin{align*}
    Q(x,r)^{1/p} &\leq  \| f \chi_{B_r(x)} - g  \chi_{B_r(x)}  \|_p + \| g \chi_{B_r(x)} -g(x)  \chi_{B_r(x)} \|_p + \| g(x)  \chi_{B_r(x)}  - f(x)  \chi_{B_r(x)}  \|_p 
\end{align*}
Thus \begin{align*}
    \limsup_{r\to 0+} \frac{Q(x,r)^{1/p}}{(2r)^{1/p}} & \leq \limsup_{r\to 0+}\frac{\| f \chi_{B} - g  \chi_{B}  \|_p }{(2r)^{1/p}}   +\limsup_{r\to 0+} \frac{\| g \chi_{B} -g(x)  \chi_{B} \|_p}{(2r)^{1/p}} +\limsup_{r\to 0+} \frac{\| g(x)  \chi_{B}  - f(x)  \chi_{B}  \|_p}{(2r)^{1/p}} \\
    & = \limsup_{r\to 0+}\frac{\| f \chi_{B} - g  \chi_{B}  \|_p }{(2r)^{1/p}}    +\limsup_{r\to 0+} \frac{\| g(x)  \chi_{B}  - f(x)  \chi_{B}  \|_p}{(2r)^{1/p}} 
\end{align*}
Since we already proved the middle one of the three norms is zero, as continuous funciton with cpt supp.\\
Step 2: Reduce the statement to 
For simplication of notation, we also define for each $x\in \mathbb{R}$: \[
M_1(x) :=  \limsup_{r\to 0+}\frac{\| f \chi_{B_r(x)} - g  \chi_{B_r(x)}  \|_p }{(2r)^{1/p}},\quad M_2(x) : = \limsup_{r\to 0+} \frac{\| g(x)  \chi_{B_r(x)}  - f(x)  \chi_{B_r(x)}  \|_p}{(2r)^{1/p}} 
\]
By the ineq we obtained, we have:
\[
\{ x : Q(x) \geq \alpha\} \subset  \{x: M_1(x) \geq \frac{\alpha}{2} \}   \cup \{x: M_2(x) \geq \frac{\alpha}{2} \}
\]
Since if we have both $M_1(x) < \frac{\alpha}{2}$ and $M_2(x) < \frac{\alpha}{2}$, we cannot have $Q(x) \geq \alpha$.\\
Thus \[
m\{ x : Q(x) \geq \alpha\} \leq m\{x: M_1(x) \geq \frac{\alpha}{2} \}  +  m\{x: M_2(x) \geq \frac{\alpha}{2} \}
\]
\textbf{Step 3: Bound $m\{x: M_1(x) \geq \frac{\alpha}{2} \} $ using HL max Thm.}\\
Note \[
\frac{\| f \chi_{B} - g  \chi_{B}  \|_p }{(2r)^{1/p}}   = \bigg(   \frac{1}{2r} \int |f \chi_{B} - g  \chi_{B}|^p \bigg)^{\frac{1}{p}}
\]
And we can express it as HL max function of   \[
\sup_{r}  \frac{1}{2r} \int |f \chi_{B} - g  \chi_{B}|^p = H(f \chi_{B} - g  \chi_{B})^p
 (x)\]
 We want \[
m \bigg \{ x: \bigg(H(f \chi_{B} - g  \chi_{B})^p (x)\bigg)^{1/p} > \frac{\alpha}{2} \bigg \}  =  m \{  x :H(f \chi_{B} - g  \chi_{B})^p
 (x) >  (\frac{\alpha}{2})^p  \} 
 \]
 
 And by HL max Thm: \[
 m \{  x :H(f \chi_{B} - g  \chi_{B})^p
 (x) >   (\frac{\alpha}{2})^p \} \leq   \frac{2^p 3^n }{\alpha^p} \int (|f-g|\chi_B)^p \leq   \frac{2^p 3^n }{\alpha^p} \int |f-g|^p \leq  \frac{2^p 3^n }{\alpha^p} \epsilon^p
 \]
 \textbf{Step 4: Bound $m\{x: M_2(x) \geq \frac{\alpha}{2} \} $ using Markov's ineq.} \\
  Notice that $M_2(x)$ is independent with $r$:\[
 \frac{\| g(x)  \chi_{B_r(x)}  - f(x)  \chi_{B_r(x)}  \|_p}{(2r)^{1/p}}  =  \frac{\bigg(  (f(x) - g(x))^p \, 2r\bigg)^{1/p} }{(2r)^{1/p}} = (f(x) - g(x))^p 
 \]
 Thus  \[
 m\{x: M_2(x) \geq \frac{\alpha}{2} \}   = m \{ x: (f(x) -g(x))^p \geq \frac{\alpha}{2} \}
 \]
 Therefore by Markov's ineq: \[
  m\{x: M_2(x) \geq \frac{\alpha}{2} \}  = m \{ x: (f(x) -g(x))^p \geq \frac{\alpha}{2} \} \leq \frac{2}{\alpha} \int (f(x) -g(x))^p   = \frac{2}{\alpha}  \epsilon^p
 \]
 Put it all together we have: \[
 m\{ x : Q(x) \geq \alpha\} \leq  \bigg(\frac{2^p 3^n } {\alpha^p}  + \frac{2}{\alpha}  \bigg)\epsilon^p
 \]
 Since $\epsilon$ is arbitrary, we finally proved that \[
  m\{ x : Q(x) \geq \alpha\} = 0 \quad \text{for any } \alpha
 \]
 finishing the proof.
\end{proof}




 





\section*{\textbf{generalization of Hölder}: bootstrapped Hölder}
Prove the following generalization of Hölder's inequality. Let 
$0<s<\infty$ and $0<p_1,\dots, p_n< \infty$ be such that 
\[
  \frac1{p_1}+\frac1{p_2}+\dots+\frac1{p_n}=\frac1{s};
\]
then
\[
  \| f_1f_2\cdots f_n\|_s\le \|f_1\|_{p_1}\|f_2\|_{p_2}\cdots \|f_n\|_{p_n}.
\]

\begin{proof}
We prove by induction, applying Hölder's inequality each time.\\
base case: If \( n = 1 \) then the result is Hölder's inequality, as proved.\\
Inductive step: Suppose the inequality holds for all $s,p_1,\cdots, p_{n-1}$ such that the equality holds, then we assume there are $n$ positive reals $p_1, \cdots, p_n$ and some $s>0$ s.t. \[
  \frac1{p_1}+\frac1{p_2}+\dots+\frac1{p_n}=\frac1{s}
\]
WTS the ineq also hold.\\
We set: \[
  \frac{1}{r} := \frac{1}{p_1} + \frac{1}{p_2} + \cdots + \frac{1}{p_{n-1}}
\]
Then we have \[
\frac{1}{r} + \frac{1}{p_n} \;=\; \frac{1}{s}
\]
By the induction hypothesis applying to the \(n-1\) functions \(f_1, \dots, f_{n-1}\), we have
\[
  \|f_1 f_2 \cdots f_{n-1}\|_r \leq 
  \|f_1\|_{p_1}\,\|f_2\|_{p_2}\,\cdots\,\|f_{n-1}\|_{p_{n-1}}
\]
Now we define: \[g(x) := f_1(x)f_2(x)\cdots f_{n-1}(x) ,\quad h(x) =: f_n(x)\]
Applying the classical Hölder inequality with conjugate exponents \(r\) and \(p_n\), we have:
\[   \|g h\|_s =
  \|f_1 f_2 \cdots f_{n-1} \cdot f_n\|_s \leq 
  \|f_1 f_2 \cdots f_{n-1}\|_r \cdot \|f_n\|_{p_n}.
\]
Putting it all together, we obtain: 
\begin{align*}
 \|g h\|_s =
  \|f_1 f_2 \cdots f_{n-1} \cdot f_n\|_s  & \leq   |f_1 f_2 \cdots f_{n-1}\|_r \,\|f_n\|_{p_n} \\ &\leq  \Bigl(\|f_1\|_{p_1}\cdots \|f_{n-1}\|_{p_{n-1}}\Bigr)\,\|f_n\|_{p_n} \\ 
  &= \|f_1\|_{p_1} \cdots \|f_n\|_{p_n}
\end{align*}This completes the inductive step, and thus the proof of the generalized Hölder inequality. 
\end{proof}





\section*{Translated a function by $t$: $f^t \to f$ in $L^p$ ($1\leq p < \infty$), but not in $L^\infty$}
  For any measurable function $f\colon\mathbb{R}\to\mathbb{R}$, set  \[
    f^y(x):=f(x-y),\quad x\in \mathbb{R}
  \]
  \begin{itemize}
  \item[(i)]    Suppose that $f$ is continuous with compact support. Prove that $\lim_{y\to0}\|f^y-f\|_\infty=0$.
  \item[(ii)]    Suppose that $f\in L^p(\mathbb{R})$ for some $p\in[1,\infty)$. Prove that $\lim_{y\to0}\|f^y-f\|_p=0$.
  \item[(iii)]    Prove by example that~(ii) is false for $p=\infty$.
  \end{itemize}


\begin{proof}
    \textbf{of (a): } \\
   Suppose \(f\) is continuous with compact support \(K\subset \mathbb{R}\), then it is uniformly continuous. \\
Let $\epsilon > 0$ and fix it. By uniform continuity, there exists \(\delta > 0\) such that \[\bigl|x - z\bigr| < \delta \implies |f(x) - f(z)| < \epsilon\]
For given $y$, we have: \[
     \|f^y - f\|_\infty = \text{ess} \sup_{x\in \mathbb{R}} \bigl|f^y(x) - f(x)\bigr| \leq
     \sup_{x \in \mathbb{R}} \bigl|f^y(x) - f(x)\bigr| = 
     \sup_{x \in \mathbb{R}} \bigl|f(x - y) - f(x)\bigr|
   \]
   Then for \(\lvert y \rvert < \delta\): for any \(x\), \(\lvert x-y - x\rvert = |y| < \delta\). Thus by uniform continuity, must have
   \(\bigl|f(x-y) - f(x)\bigr| < \epsilon\). Thus we got: \[
     \|f^y - f\|_\infty \le  \epsilon
     \quad \forall |y| < \delta
   \]
Since \(\epsilon\) is arbitrary, this proves that
\[
  \lim_{y \to 0} \|f^y - f\|_\infty = 0
\]
\end{proof}

\begin{proof}
    \textbf{of (b):}\\
Since \(C_c(\mathbb{R})\) is dense in \(L^p(\mathbb{R})\) for \(1 \le p < \infty\), we can take a seq of continuous functions with compact support, say \((\varphi_n)\), s.t. \(\varphi_n \to f\) in \(L^p\).\\
Then for each $y\in \mathbb{R}$, we can define \[\varphi_n^y(x) := \varphi_n(x - y)\]
From (a) we have, for each $n$: \[
 \lim_{y \to 0} \|\varphi_n^y - \varphi_n\|_\infty = 0
\]
Note that since each $\varphi_n$ have compact $K$ whose measure is finite, we have: \[
  \|\varphi_n^y - \varphi_n\|_p = \int |\varphi_n^y - \varphi_n |^p \; dm \leq \int \sup_x|\varphi_n^y - \varphi_n|^p \; dm = \|\varphi_n^y - \varphi_n\|_\infty^p   m(K)
  \]
Thus, \[
    \lim_{y \to 0} \|\varphi_n^y - \varphi_n\|_\infty = 0 \implies  \lim_{y \to 0} \|\varphi_n^y - \varphi_n\|_p = 0
   \]
Also, by translation invariance of Lebesgue measure, for each $y$ we have: \[
\|f^y - \varphi_n^y\|_p  =  \|f - \varphi_n\|_p
\]
Therefore for each \(y\), we can bound 
\begin{align*}
      \|f^y - f\|_p &\leq 
       \|f^y - \varphi_n^y\|_p + 
       \|\varphi_n^y - \varphi_n\|_p + 
       \|\varphi_n - f\|_p \\
       &\ = 2 \|\varphi_n - f\|_p + \|\varphi_n^y - \varphi_n\|_p
\end{align*}
The construction of bound has finished. Now Let $\epsilon > 0$ and fix it. We first choose \(n\) large enough so that \[|\varphi_n - f\|_p < \frac{\epsilon}{3}\]
and for the fixed \(n\), we choose $\delta$ s.t. for all $|y| < \delta$ we have \[\| \varphi_n^y - \varphi_n\|_p < \frac{\epsilon}{3}\]
Then we have: \[
  \|f^y - f\|_p  \leq  \epsilon \quad \forall  |y| < \delta
\]
Since $\epsilon$ is arbitrary, this proves that \[
  \lim_{y \to 0} \|f^y - f\|_p = 0
\]
\end{proof}

\begin{proof}
    \textbf{of (c):}\\
    \begin{comment}
    This part is wrong.
        We consider:
\[
  f(x) \;=\;
  \begin{cases}
    1, & x  \in \mathbb{R} \setminus \mathbb{Q}  \\[6pt]
    0, & x  \in \mathbb{Q}
  \end{cases}
\]
This function is in \(L^\infty(\mathbb{R})\) because \(\lvert f(x)\rvert \le 1\) for all \(x\). Its essential supremum is \(1\), since $ m( \mathbb{R}\setminus \mathbb{Q}) = m(\mathbb{R}) >0$. So\[
\| f\|_\infty = 1
\]
Let \(y\neq 0\), we have:  \[
    f^y(x) = f(x-y) = 
    \begin{cases}
      1, & (x-y) \in \mathbb{R} \setminus \mathbb{Q}\\
      0, & (x-y)  \in \mathbb{Q}
    \end{cases}
  \]
Note that, translating all reals by a rational number, all irrationals are still irrational;  translating all reals by an irrational number can make only one number (the negative of that irrational number) rational.\textbf{ Thus In both cases, $\{x-y \in \mathbb{R}\setminus \mathbb{Q}\}$ occupy the full measure of $m(\mathbb{R})$.}
Thus we have, \[
    \|f^y - f\|_\infty = 1
    \quad\text{for all } y
  \]
This is an counterexample showing that we do not necessarily have $\lim_{y\to0}\|f^y-f\|_\infty=0$.
    \end{comment}
We consider 
\[
   f(x) := \chi_{(0,1)}
\]
We have \[
\| f \|_\infty = 1
\]
and the sup is taken on $x \in (0,1)$.\\
Then for any $y$, we have: We have
\[
   \bigl\lvert f^y(x) - f(x)\bigr\rvert   =
   \bigl\lvert \chi_{(0,1)}(x - y) - \chi_{(0,1)}(x) \bigr\rvert = \bigl\lvert \chi_{(y,y+1)}(x ) - \chi_{(0,1)}(x) \bigr\rvert 
\]
Thus for all $y > 0$, on the open set $(1,y+1)$ which has positive measure, we have \( \bigl\lvert f^y(x) - f(x)\bigr\rvert  = 1
\); \\
For all $y<0$, on the open set $(y,0)$ which has positive measure, we have \( \bigl\lvert f^y(x) - f(x)\bigr\rvert  = 1
\);
Thus the function $\|f^y - f\|_\infty$ with respect to $y$ actually has a jump discontinuity at $0$, since it is $0$ at $y=1$ and $1$ elsewhere.\\
This serves as an counterexample that we do not necessarily have $\lim_{y\to0}\|f^y-f\|_\infty=0$.
\end{proof}
\begin{remark}
    这里可以体现 $L^\infty$ convergence 的严格性, 从本质上比其他 $L^p$ convergence 都要高一级别. 
\end{remark}




\section*{Criterion for $L^p$-convergence: a.e. conv $+$ 积分值 conv}
  Suppose that $1\le p<\infty$ and that $f_n,f\in L^p$ for some measure space $(X,\mathcal{A},\mu)$.
  Prove that if $f_n\to f$ a.e. and $\|f_n\|_p\to\|f\|_p$, then $\|f_n-f\|_p\to0$. Is the converse true?
  \textit{Hint}: revisit the ``\textbf{Generalized DCT}'' problem on HW5.
  \begin{proof}
Recall we have proved 
\begin{theorem}{Generalized DCT}
    Let $(X, \mathcal{A}, \mu)$ be a measure space, and $f_n, g_n, f, g\in L^1$, $n\in \mathbb{N}$. Suppose that 
  \begin{itemize}
  \item[(a)]$\lim_{n\to\infty} f_n(x)=f(x)$ and $ \lim_{n\to\infty} g_n(x)=g(x)$ for a.e. $x$;
  \item[(b)] $|f_n(x)|\le g_n(x)$ a.e. for every $n\in \mathbb{N}$;
  \item[(c)]$g_n\colon X\to [0, \infty]$ and $\lim_{n\to \infty} \int g_n \; d \mu = \int g\; d\mu$.
  \end{itemize}
Then we have: \[
    \lim_{n\to \infty} \int f_n \; d\mu = \int f \; d\mu
  \]
\end{theorem}
which is the case $p=1$. Now we prove the general case with the help of the case $p=1$. We notice that $f_n \to f$ in $L^p$, is just to prove the function $|f_n -f |^p \to 0$ in $L^1$, that's how we can use the generalized DCT.\\
Assume the hypothesis. 
Since $x^p$ is convex for $p \geq 1$, we have for any $x,y$: \[
\bigg(\frac{x+y}{2} \bigg)^p \leq \frac{x^p + y^p}{2}
\]
Thus  \[
(x+y)^p \leq 2^{p-1} (x^p + y^p)
\]
Therefore for each \(n\) and almost every \(x\), we have: \[
     |f_n(x) - f(x)|^p  \leq (|f_n(x)| +| f(x)| ) ^p
  \le 2^{\,p-1}\bigl(\lvert f_n(x)\rvert^p + \lvert f(x)\rvert^p\bigr)
   \]
Hence \[
     |f_n - f|^p  \leq 2^{\,p-1}\bigl(|f_n|^p + |f|^p\bigr)
   \]
We define for each $n$: \[
 g_n := 2^{\,p-1}\bigl(|f_n|^p + |f|^p\bigr)
\]
Since \(f_n \to f\) a.e., we have \(|f_n|^p \to |f|^p\) a.e. Thus \[
     g_n(x)\;=\;2^{p-1}\bigl(|f_n(x)|^p + |f(x)|^p\bigr)\overset{n\to \infty}{\longrightarrow}
     2^{p-1}\bigl(|f(x)|^p + |f(x)|^p\bigr) =2^p\,|f(x)|^p
=:g(x)
   \]Note that \[
     \int g_n \, d\mu   =
     2^{p-1}\,\bigl(\|f_n\|_p^p + \|f\|_p^p\bigr)
   \]
Since \(\lVert f_n\rVert_p \to \lVert f\rVert_p,\) we have \[
\lim_{n\to \infty}     \int g_n\, d\mu = 
     2^{p-1}\,\bigl(\|f\|_p^p + \|f\|_p^p\bigr) = 
     2^p\,\|f\|_p^p  =\int g\,d\mu
   \]
Now we have \textbf{(1)} $g_n \to g$, \textbf{(2) }$\int g_n \to \int g$, and \textbf{(3)} $g_n$ is an upper bound for $    |f_n - f|^p$. Then we can apply generalized DCT to the function seq $    |f_n - f|^p$:
\[
\lim_{n\to \infty} \|f_n - f\|_p^p =  \lim_{n\to \infty}  \int\bigl|f_n(x)-f(x)\bigr|^p \,d\mu = 
  \int 0 \,d\mu = 0
\]
Thus
\[
\lim_{n\to \infty}   \|f_n - f\|_p = 0^{1/p} = 0
\]
This finishes the proof that \(f_n \to f\) in \(L^p\).
\end{proof}
\begin{solution}
    The converse does not hold.\\
  We recall the typewriter function on $[0,1]$:
\[
f_{n,k}(x) = \begin{cases}
1, & x \in \left[\frac{n-1}{2^k}, \frac{n}{2^k}\right] \\
0, & \text{otherwise}
\end{cases}
\]
We index over $k \in \mathbb{N}$, and for each $k$ we index over $n=1$ to $2^k$. That is, for given $k$, \( f_n \) is the indicator function of the \( n \)-th dyadic interval.\\
Then \[
\|f_n\|_p = \left( \int_{[0,1]} |f_n(x)|^p dx \right)^{1/p} = \left( \text{length of the dyadic interval} \right)^{1/p} \le 2^{-k/p}
\]
Therefore, since each \( f_n \) has support of shrinking length, we get:
\[
\|f_{n,k}\|_p \to 0 \quad \text{as } k \to \infty
\]
but for each $x$, $f_{n,k}(x) = 1 $ for infinitely many $(n,k)$. so \( f_n(x) \) does not converge to 0 for any \( x \in [0,1] \).
  \end{solution}





\begin{center}
  \textit{Nur f\"ur Verr\"uckte}
\end{center}
(It's \textbf{really} not necessary to attempt these problems. Do not, under any circumstances, hand them in!)
\begin{enumerate}
\item Prove that the category of measurable spaces (see HW1) admits finite products, and that the product of $(X,\mathcal{A})$ and $(Y,\mathcal{B})$ equals $(X\times Y,\mathcal{A}\otimes\mathcal{B})$. 
\item  Now consider the category of measure spaces (see HW2). Consider two 
  measure spaces $(X_i,\mathcal{A}_i,\mu_i)$, $i=1,2$, and set $X=X_1\times X_2$, $\mathcal{A}=\mathcal{A}_1\otimes\mathcal{A}_2$, and $\mu=\mu_1\times\mu_2$.
  \begin{itemize}
  \item[(a)]    Prove that the projection maps $X\to X_i$ are measurable, and that they are measure preserving iff $\mu_j(X_j)=1$ for $j=1,2$. Thus $(X,\mathcal{A},\mu)$ is \emph{not} the categorical product of $(X_i,\mathcal{A}_i,\mu_i)$ in general.
  \item[(b)]    Prove that even if $\mu_i(X_i)=1$, the measure space $(X,\mathcal{A},\mu)$ is \emph{not} the categorical product of $(X_i,\mathcal{A}_i,\mu_i)$ in general.
    \textit{Hint}: consider the case when the $X_i$ consist of two elements, for example $X_i=\{\mathfrak{o}_i,\mathfrak{v}_i\}$.
  \end{itemize}
\end{enumerate}