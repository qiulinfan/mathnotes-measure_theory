\chapter{outer measure 与 completion of a measurable space}

\section{complete measure space and outer measure [Fol 1.3, finished; 1.4]}

\begin{definition}{nul set, subnull set, almost everywhere}
对于 measure space $(X, \mathcal{M}, \mu)$
\begin{enumerate}
    \item 我们称 $A \in \mathcal{M}$ 为一个 \textbf{null set}, 如果 $\mu(A) = 0$;
    \item 我们称 $B \sub \mathcal{M}$ 为一个 \textbf{subnull set}, 如果存在某个 null set $A$ containing it.
    \item 我们称一个 statement about $X$ 是 \textbf{almost everywhere (a.e.)} 的, 如果这个 statement 除了在某个 null set 上之外, 在 $X$ 上处处成立.
\end{enumerate}
\end{definition}


\begin{definition}{complete measure space}
    我们称 $(X,\mathcal{M}, \mu)$ 是一个 complete measure space, 如果它其中的任意 subnull set 都是 null set. (即它 measurable)
\end{definition}
\begin{remark}
    我们知道, 根据 measure 的 monotonicity, subnull set 的 measure, 如果存在, 一定是 $\leq$ 它所在的 null set 的, 即一定 $=0$. 所以 complete measure space 的实际意思是: 这个 measure space 里, 任意 null set 的所有子集都是 measurable 的, 即所有足够小的集合都在这个 $\sigma$-algebra 里.
\end{remark}




\begin{example} 一个 not complete 的 measure space 的例子:
$$
X = \{1,2\}, \mathcal{M} = {\emptyset, X}, \mu(\forall) = 0.
$$
这个例子中, $\{1\}, \{2\}$ 这两个集合不是 measurable 的, 但是却是 nullset (全集) 的子集.
\end{example}



\begin{theorem}{every measure space can be completed}
    Suppose $(X, \mathcal{M},\mu)$ is a measure space.\\
    Let 
    \[
    \cN := \{\text{all null sets in }   \mathcal{M} \}
    \]
    Claim:
    \[
    \ol{M}  := \{   E\cup F \mid E \in \mathcal{M}, F \sub N \text{ for some } N \in \cN \}
    \]
    is a $\sigma$-algebra, 并且在 $\ol{\mathcal{M}}$ 上存在一个 unique 的 extension $\ol{\mu}$ of $\mu$.
\end{theorem}
\begin{proof}
    这一部分的 proof 以及 remark 在 hw2. 这里, $\overline{M}$ 称为 \textbf{completion of $\mathcal{M}$ with respect to $\mu$}, 以及 $\overline{\mu}$ 称为 \textbf{completion of $\mu$.}
\end{proof}




\subsection{outer measure}
\begin{definition}{outer measure}
    An outer measure on $X$ is a function $\mu^*: \mathcal{P}(X) \rightarrow {[0,\infty)}$ such that
    \begin{enumerate}
        \item $\mu(\varnothing) = 0$
        \item monotone ($A \subset  B \implies \mu^*(A) \leq \mu^*(B)$)
        \item countable subadditive ($\mu^*(\bigcup_{i=1}^\infty E_i)  \leq \sum_{i=1}^\infty \mu^*(E_i)$)
    \end{enumerate}
\end{definition}
\begin{remark}
    我们对比 measure 和 outer measure 的定义:
    measure 的条件比 outer measure 强在:
    \begin{enumerate}
        \item measure 是定义在一个严格的 $\sigma$-algebra 上的, 而 outer measure 则是定义在整个幂集上的. 
        \item measure 要求 disjoint countable additivity, outer measure 并不要求
    \end{enumerate}
\end{remark}

在这两个条件的缩减下, 我们规定 outer measure 具有 monotonicity 和 countable subadditivity. 注意: measure 本身也有这个性质, 这是 measure 的 countable additivity 的推论. \\
outer measure 的意义在于, 我们的 measure 只定义在 $\sigma$-algebra 上, 而我们想要给每个子集都赋予一个近似于测度的东西. 

\subsection{induce outer measure out of a "elementary length function"}
\begin{theorem}{construct outer measure out of an "elementary  length function" }\label{construct outer measure out of a "elementary length function"}
    另 $\cE \sub \cP(X)$ 为一个包含 $\varnothing, X$ 的集合, 并定义 $\rho: \cE \rar [0,\infty)$ 为一个满足 $\rho(\varnothing) = 0$ 的函数, 则
\[
\mu^*(A )  = \inf \{    \sum_{i=1}^\infty \rho(E_i) \mid E_i \in \cE \text{ for each i and }  A \sub \bigcup_{i=1}^\infty E_i    \}
\]
is an outer measure.
\end{theorem}
\begin{proof}
\begin{enumerate}
    \item 取所有 $E_j = \varnothing$, 得到 $\mu^*(\varnothing) = 0$
    \item monotonicity 显然, 因为如果 $A \sub B$, 那么 $A$ 取 inf 的这个集合是包含于 $B$ 的, 因而取到的 inf 是小于等于的.
    \item  证明 ctbl subadditivity, 我们使用经典的 $\epsilon  / 2^i$ argument. 这个 statement 直观上是显然的, 因为对一个 seq of sets, 每一个里面都有一个 seq of covering, 那么这个 seq of seq of covering 总体也是这个 seq union 的一个  covering. 不过我们不能这么说, 因为这里有一个 inf 操作的换序. 所以我们令 $\epsilon >0$, 对于每个 $A_i$ 的 covering $(E_{i,k})_{k\in\bN}$, 我们令 $\sum_k \rho(E_{i,k}) \leq \mu^*(A_i) + \epsilon / 2^i$,  最后可以得到 $\mu^*(\bigcup_i A_i) \leq \sum_i \mu^*(A_i)$. 由于 $\epsilon$ arbitrary, 得证.
\end{enumerate}
\end{proof}
\begin{example}
    我们取 $\cE$ 为 $\mathbb{R}$ 上所有的 intervals, 并取 $\rho $ 为 interval 的 length, 就得到了一个外测度. (也就是 Lebesgue outer measure)
\end{example}


\section{$\mu^*$-measurability and Carathéodory's Theorem [Fol 1.4]}


\subsection{$\mu^*$-measurable}
\begin{definition}{$\mu^*$-measurable}
    Given outer measure $\mu^*$, 我们称 $A \sub X$ 是 $\mu^*$-measurable 的, if:
    $$
    \mu^*(E) = \mu^*(E \cap A) + \mu^*(E \cap A^c)
    $$
\end{definition}
\begin{remark}
countable subadditivity 蕴含的信息是: 如果我们把一个集合 divide 成几部分, \textbf{其 outer measure 有可能 increase.}  而 $\mu^*$-measurable 的含义是: 任何一个其他集合, 分割为和 $E$ 重合以及和 $E$ 的两部分之后, 其 measure 都不会增大.\\
\noindent \textbf{Note: }\textbf{by subaddivity, must have $\mu^*(E) \leq \mu^*(E\cap A) + \mu^*(E\cap A^c)$}, 而 $\mu^*$-measurable 的集合, 则有 equality 总是成立.\\
\noindent 同时注意: 这个行为对于 complement 是对称的.
\end{remark}

\begin{remark}
 outer measure 是对于整个 power set 中每一个集合都赋予的, 并且其性质 ctbl subadditivity 严格弱于 countable additivity. 
 我们自然想到: 是否有一个 power set 的子集, 其不仅是一个 $\sigma$-algebra, 并且其上满足 countable additivity? 如果存在, 那么我们就从 outer measure induce 出了 measure. 
 \\ \noindent 再加上之前的用随意的 length function 来 induce outer measure 的方法, 我们就可以通过一个随意的 length function $\rar $ outer measre $\rar$ measure. (eg: 从 box length induce 出 Legesgue outer measure, 再 induce 出 Lebesgue measure).\\
 \noindent 而实际上这个想法是正确的. 只要把 $\mu^*$ 的范围限制在所有 $\mu^*$-measurable sets 上, 就形成了 $\sigma$-algebra, 并且其 restriction 是一个 measure,  甚至是一个 complete measure.
\end{remark}

\subsection{Carathéodory's Theorem}

\begin{theorem}
\label{Carathéodory's Theorem}
对于任意的 outer measure $\mu^*$, 
\[
\mathcal{M} := \{ \text{all } \mu^* \text{-measurable sets}    \}
\]\textbf{is a $\sigma$-algebra}.\\
并且, $\mu^* |_\mathcal{M}$ \textbf{is a complete measure.}
\end{theorem}
\begin{proof}
我们首先证明这个 $\mathcal{M}$ 是一个 $\sigma$-algebra
\begin{enumerate}
    \item  $\varnothing \in \mathcal{M}$ by def.
    \item $\mathcal{M}$ closed under complement, by def of $\mu^*$-measurablity. (它对于 complement 是对称的.)
    \item 为证明 $\mathcal{M}$ closed under countable union, 我们首先 prove it for two sets.
    假设 $A, B \in \mathcal{M}$, 且 disjoint. 
    Let $E \sub X$.
    我们已知 
    \begin{equation}
        \mu^*(E) = \mu^*(E \cap A) + \mu^*(E \cap A^c)
    \end{equation}
 \textbf{我们 WTS: $\mu^*(E) = \mu^*(E \cap (A\cup B)) + \mu^*(E \cap (A\cup B)^c)$}\\
\noindent 我们对于 $E \cap A$, $E\cap A^c$ 可以得到: \begin{equation}
    \mu^*(E \cap A) = \mu^*(E\cap A \cap B) + \mu^*(E \cap A \cap B^c)
\end{equation}
\end{enumerate}
\begin{equation}
    \mu^*(E \cap A^c) = \mu^*(E \cap A \cap B) + \mu^*(E\cap A^c \cap  B^c)
\end{equation}

By  $A \cup B = (A \setminus B) \sqcup (A \cap B) \sqcup (B\setminus A)$, 可以得到:
\begin{equation}
   \mu^*(E \cap (A\cup B)) \geq \mu^*(E\cap  A \cap  B) + \mu^*(E \cap  A \cap  B^c) + \mu^*(E \cap A^c  \cap B)
\end{equation}
结合以上四个 equations 可以得到
\begin{equation}
    \mu^*(E) \geq \mu^*(E \cap (A\cup B)) + \mu^*(E \cap (A \cup B^c))
\end{equation}
又 $\leq$ by countable subadditivity 成立, 我们得证 closed under two union (从而 inductively closed under any finite union, $\mathcal{M}$ 因而是一个 algebra).\\
\begin{remark}
    (Note: 这里我会想: 证明了这个 statement for any union of two sets 不就是证明了它对 any union 都成立吗? 实则不然, 因为 set union 的从属关系并不是可以从对任意 $n$ 成立推广到对无穷成立, 因为这里的无穷是一个真实存在的 sequence, 而我们可以从"任意 $n$ 成立推广到对无穷成立" 的是比较数值大小, 因为 infinite series sum 的定义就是 limit, 而 set union 并没有 limit. 所以这里不能够直接得证.)\\\\
\end{remark}
\noindent (Continuing the proof:)
\noindent 现在我们再把这个 closed under finite union 推广到 closed under countable union, 以映证 $\mathcal{M}$ 是一个 $\sigma$-algebra. 注意到 \textbf{STS (suffices to show): $\mathcal{M}$ closed under countable disjoint union}. 因为任意不 disjoint 的两个集合都可以拆分成三个 disjoint 的集合.\\
\noindent 我们令 $(A_i)$ 为一个 $\mathcal{M}$ 中的 disjoint sequence, 并定义 $B_n := \bigcup_{i=1}^n A_i$, 我们由上一步的结论知道, $B_n \in \mathcal{M}$ for all $n$.  
\noindent Define $B := \bigcup_{i=1}^\infty A_i$,  Let $E\sub X$, WTS: $\mu^*(E ) = \mu^*(E \cap B) + \mu^*(E\cap B^c)$.
\\
\noindent 考虑 $\mu^*(E \cap B_n ) = \mu^*(E \cap  B_n \cap A_n) + \mu^*(E \cap B_n \cap A_n^c) = \mu^*(E \cap  A_n) + \mu^*(E \cap B_{n-1})$, 因为 inductively 可得到:
\begin{equation}
    \mu^*(E \cap  B_n) = \sum_{i=1}^n \mu^*(E \cap A_i)
\end{equation}
\noindent 从而:
\begin{equation}
    \mu^*(E) = \mu^*(E \cap  B_n) + \mu^*(E \cap  B_n^c) \geq \sum_{i=1}^n \mu^*(E\cap A_i) + \mu^*(E \cap B^c)
\end{equation}
\noindent by monotonicity ($\mu^*(E \cap B_n^c) \geq \mu^*(E \cap B^c)$), 这里是一个 infinite sum, 并且 true for every $n$, 因而可以推广到 infinity, 得到 
\begin{equation}
    \mu^*(E) \geq \sum_{i=1}^\infty \mu^*(E \cap A_i) + \mu^*(E  \cap B^c) \geq \mu^*(\bigcup_{i=1}^\infty (E \cap A_i)) + \mu^*(E  \cap B^c) = \mu^*(E \cap B) + \mu^*(E \cap B^c) \geq \mu^*(E)
\end{equation}
\end{proof}
\noindent\textbf{This finishes the proof of $\mathcal{M}$ being a $\sigma$-algebra.} 我们同时发现,  $\mu^*|_\mathcal{M}$ 是一个 \textbf{complete measure} on $\mathcal{M}$ 是一个 trivial fact after the proof, 因为 taking $B = E$, 可以得到 
\begin{equation}
    \mu^*(B) = \sum_{i=1}^\infty \mu^*(A_i)
\end{equation}
\noindent 并且 by monotonicity, 对于任意的 $\mu^*(A) = 0$, 任取 $E \sub X$, 都有
\begin{equation}
    \mu^*(E )  \leq \mu^*(E \cap A) + \mu^*(E \cap A^c) = \mu^*(E \cap A^c) \leq \mu^*(E)
\end{equation}
因而
\[
\mu^*(E )  = \mu^*(E \cap  A) + \mu^*(E\cap A^c)
\]
得到 $A \in \mathcal{M}$. 从而得证这是一个 complete measure.\\

\begin{remark}
    证明 Carathéodory's Theorem 的 punchline 在于: 我们令 $(A_i) \in \mathcal{M}$ be a sequence, $B_n$ be its partial union for $n$ terms, 可以得到$$\mu^*(E \cap B_n ) = \mu^*(E \cap B_n \cap A_n) + \mu^*(E \cap B_n \cap A_n^c) = \mu^*(E \cap A_n) + \mu^*(E \cap B_{n-1})$$, 因为 inductively 可得到:
\begin{equation}
    \mu^*(E \cap B_n) = \sum_{i=1}^n \mu^*(E \cap  A_i)
\end{equation}
\noindent 这个 statement 对于 $\mathcal{M}$ 是 $\sigma$-algebra 以及 $\mu^*|_{\mathcal{M}}$ 是 measure 的证明都很重要. 我们在 outer measure 的定义中, 只声明了 countable subadditivity, 而我们需要证明的是 countable diskjoint additivity, 也就是需要把不等式变成一个等式. 
\\\noindent 为此我们看到 $\mu^*$-measurable 的定义 (Carathéodory condition) 中的等号, 并从中找到这个等式关系: \textbf{通过 disjoint set sequence 上 inductively 对于前一项使用 Carathéodory condition, 得到 disjoint additivity.} (笔者的感觉是 Carathéodory condition 的直观看似不明显, 但是如果把一个 disjoint union 自身作为 $E$, 并把自身的某项作为 $A$, 就非常明显, 表示的是 disjoint measure sum 就是 measure of disjoint union.)
\end{remark}


\section{premeasure and Hahn-Kolmogrov extension Theorem [Fol 1.4, finished]}
我们发现: 有些子集簇上的 "length" 很明显, 并且也符合 measure 的定义, 但是这个子集簇却并不构成一个 $\sigma$-algebra. 比如:
\begin{example}
    $\{ \text{all half-open, half-closed intervals}\} \sub \mathbb{R}$ 上, 以 interval 的 length 作为 measure, 很显然符合 measure function 的定义, 但是 $\{ \text{all half-open, half-closed intervals}\} \sub \mathbb{R}$ 并不是一个 $\sigma$-algebra, 因为它可以通过 ctbl union 出 open interval, 并不在这个子集簇中. 不过, 这是一个 algebra.\\
\end{example}
因此, 我们想要一个方法来 \textbf{extend a "measure" function on an algebra, to a measure on a $\sigma$-algebra.}

\begin{definition}{premeasure}
给定 $\cP(X)$ 上的一个 \textbf{algebra} $\mathcal{A}_0$, 我们称 \(\mu_0: \mathcal{A}_0 \rar [0,+\infty]\)  为一个 premeasure, if
\begin{enumerate}
    \item \(\mu_0(\varnothing)  = 0\)
    \item \(\mu_0\) ctbl disjoint additive in $\mathcal{A}_0$
\end{enumerate}
\end{definition}

\begin{remark}
premeasure 就是定义在 algebra instead of $\sigma$-algebra 上的 measure. 显然, 通过和 measure 相同的方式可证明, premeasure 在 $\mathcal{A}_0$ 上是 \textbf{monotone 以及 ctbl subadditive 的.  }  
\end{remark}


\subsection{induce outer measure out of a premeasure: preserving $\mu_0$ on $\mathcal{A}_0$}
\begin{proposition}
\label{construct outer measure out of a premeasure}
    Any premeasure can induce an outer measure:
    \begin{equation}
        \mu^*(E) = \inf \{  \sum_{i=1}^\infty \mu_0(A_i) \mid A_i \in \mathcal{A}_0, E \sub \bigcup_{i=1}^\infty A_i   \}
    \end{equation}
    并且, we have:
    \begin{equation}
        \mu^*|_{\mathcal{A}_0} = \mu_0
    \end{equation}
    并且 \textbf{every set in $\mathcal{A}_0$ is $\mu^*$-measurable.}
\end{proposition}
\begin{proof}
    \textbf{这个 outer measure 的 construction directly follows from} \ref{construct outer measure out of a "elementary length function"}.\\
    \noindent \textbf{Proof that $\mu^*$ restricted to $\mathcal{A}_0$ is $\mu_0$}: 令 $E \in \mathcal{A}_0$, 假设 $E \sub \bigcup_{i=1}^\infty A_i$, 我们令 $B_n := E \cap (A_n \setminus \bigcup_{i=1}^{n-1} A_i)$, 即把 covering intersecting $E$ 变成 disjoint covering $(B_n)$, 从而由 $\mu_0$ 的 ctbl disjoint additivity 可得, 这一个新 covering 的 measure sum $\sum_{i=1}^\infty \mu_0(B_i) := \mu_0(E)$. 并且由于 $\mathcal{A}_0$ 是一个 algebra, 这些 $B_n$ 也在 $\mathcal{A}_0$ 里面, 从而它满足 monotonicty, then $\mu_0(E) = \sum_{i=1}^\infty \mu_0(B_i) \leq \sum_{i=1}^\infty \mu_0(A_i) $\\
    \noindent \textbf{Proof that every set in $\mathcal{A}_0$ is $\mu^*$-measurable}: Fix $A \in \mathcal{A}_0$, 我们取任意 $E \sub X$.
    Let $\epsilon > 0$, by def of the outer measure, 存在一个 seq $\{ B_i\}_{i=1}^\infty \sub \mathcal{A}_0$, 使得 $E \sub \bigcup_{i=1}^\infty B_i$ 并且 $\sum_{i=1}^\infty \mu_0(B_i) \leq \mu^*(E) + \epsilon$. 有 disjoint additivity of $\mu_0$ 可得, $\sum_{i=1}^\infty \mu_0(B_i) = \sum_{i=1}^\infty \mu_0(B_i \cap A) + \sum_{i=1}^\infty \mu_0(B_i\cap A^c)$. 从而 $\mu^*(E) \geq \mu^*(E \cap A) + \mu^*(E\cap A^c)$, 得证. (实际上这是个 trivial argument, 通过$\epsilon$ argument 来严格证明.)
\end{proof}
\begin{remark}
    这一 simple proposition 表明的是, $\mu_0$ induce 出的 outer measure 在 $\mathcal{A}_0$ 上 \textbf{presearve $\mu_0$ 的 measure 与 measurability.}
\end{remark}






\subsection{Hahn-Kolmogrov Theorem}
\begin{definition}{$\sigma$-finite measure}
Let $(X,\mathcal{M}, \mu)$ be a measure space.\\
如果 $\mu(X) < \infty$, 则称 $\mu$ 是 finite 的.\\
如果存在一个 sequence $(E_i)$ in $\mathcal{M}$ 使得 $\bigcup_{i} E_i = X$ 并且每个 $\mu(E_i) < \infty$, 则称 $\mu$ 是 $\sigma$-finite 的.
\end{definition}
\begin{remark}
一个 finite measure 说明 $\mathcal{M}$ 中的所有集合的 measure 都 finite.
\end{remark}



\begin{theorem}{Hahn-Kolmogrov Theorem}
\label{Hahn-Kolmogrov Theorem}
给定一个 premeasure $\mu_0$ on algebra $\mathcal{M}_0$ of $X$, 以及其 induced outer measure $\mu*$, 我们令 
$$
\mathcal{M} := <\mathcal{M}_0>
$$
表示 $\sigma$-algebra generated by the algebra $\mathcal{M}_0$.\\
并令
$$
\mu := \mu^* |_\mathcal{M}
$$
then we have:
\begin{enumerate}
    \item $(X,\mathcal{M}_0, \mu_0)$ extends to $(X,\mathcal{M},\mu)$\\
    即: $\mu  |_{\mathcal{M}_0} = \mu_0$
    \item $\mu | _\mathcal{M}$ 是 \textbf{the largest extension of $\mu_0$ to $\mathcal{M}$} (即: 对于任意其他的 $\mathcal{M}$ 上的 measure $\nu$ that extends $\mu_0$ to $\mathcal{M}$, 都有 $\nu(E) \leq \mu(E)$ for all $E \in \mathcal{M}$);\\
    并且 \textbf{if $\mu_0$ is $\sigma$-finite}, 则 $\mu$ 是 \textbf{the unique extension} of $\mu_0$ to $\mathcal{M}$.
\end{enumerate}
\end{theorem}
\begin{proof}
\textbf{Proof of $(X,\mathcal{A}_0, \mu_0)$ extends to $(X,\mathcal{M},\mu)$:}\\
这个 Statement directly follows from \ref{Carathéodory's Theorem}(Carathéodory's Theorem) 以及上一个 proposition \ref{construct outer measure out of a premeasure}. \\
\noindent 1. 我们首先用 $\mu_0$ induce 出 $\mu^*$, 再 restrict $\mu^*$ to $ \mathcal{M}^* :=\{ \text{all } \mu^* \text{-measurable sets}  \}$, 得到一个 $\sigma$-algebra $\mathcal{M}^*$.\\
\noindent 注意此时: 由上一个 proposition \ref{construct outer measure out of a premeasure} 可得 $\mathcal{M}_0$ 中所有集合都是 $\mu^*$-measurable 的, thus $M_0 \sub \mathcal{M}^*$, 由于 $\mathcal{M}^*$ 是一个 $\sigma$-algebra, 由 \ref{inclusion properties of generated sigma-algebra} 可得: $\mathcal{M} := <\mathcal{M}_0> \sub \mathcal{M}^*$. \\
\noindent 2. 由 Carathéodory's Theorem 可以得到: $\mu^* | _{\mathcal{M}^*}$ 是一个 measure, 从而 $\mu :=\mu^* |_{\mathcal{M}}$ 也是一个 measure(等于把 $\mu^* | _{\mathcal{M}^*}$ 限制在了一个更小的 sub-$\sigma$-algebra 上).\\
\noindent\textbf{(Note: this is a trivial fact that if $M^*$ is a $\sigma$-algebra and $M \subset M^*$is also a $\sigma$-algebra, then $\mu |_{M}$ is a measure if given that $\mu$ is a $\sigma$-algebra on $M^*$)}\\\\
\noindent \textbf{Proof of $\mu$ being the largest extension of $\mu_0$ to $\mathcal{M}$:}
\noindent 假设 $\nu$ 是一个 $\mathcal{M}$ 上的 $\sigma$-algebra s.t. $\nu|_{\mathcal{M}_0} = \mu_0 $.\\
\noindent Let $E \sub \mathcal{M}$. (WTS: $\nu(E) \leq \mu(E)$, 即$\nu(E) \leq \mu^*(E)$ .)\\
\noindent 由外测度 \(\mu^*\) 的定义, 对于任意 \(\epsilon>0\), 存在一列集合 \(\{A_i\}_{i=1}^\infty \subset \mathcal{A}_0\) 满足
\[
E\subset \bigcup_{i=1}^\infty A_i \quad \text{且} \quad \sum_{i=1}^\infty \mu_0(A_i) \le \mu^*(E)+\epsilon.
\]
由于 \(\nu\) 在 \(\mathcal{A}_0\) 上和 \(\mu_0\) 一致,即
\[
\nu(A_i) = \mu_0(A_i) \quad \forall i,
\]
因此,
\[
\sum_{i=1}^\infty \nu(A_i) = \sum_{i=1}^\infty \mu_0(A_i) \le \mu^*(E)+\epsilon
\]
利用 \(\nu\) 的 additivity 和 monotoncity 得
\[
\nu(E) \le \nu\Bigl(\bigcup_{i=1}^\infty A_i\Bigr) \le \sum_{i=1}^\infty \nu(A_i) = \sum_{i=1}^\infty \mu_0(A_i) \le \mu^*(E)+\epsilon
\]

由于 \(\epsilon\) arbitrary, 得到
\[
\nu(E) \le \mu^*(E)
\]


\noindent (证明思路: 在 $\mathcal{M}$ 上 $\mu$ 就等于 $\mu_0$ induce 的外测度, 对于其他的 extended measure, 其作用在一个集合上的测度一定小于等于任意的 $\mathcal{M}_0$ covering 的 premeasure 和, 而我们可以通过控制这个 covering 的测度和与它的外测度的差距(since inf), 从而使得这个测度小于等它的外测度加一个无限小的 $\epsilon$, 从而得证.) \\\\

\noindent \textbf{Proof of $\mu$ being the unique extension of $\mu_0$ to $\mathcal{M}$, provided that $\mu_0$ is $\sigma$-finite}:\\
\noindent (recall $\mu_0$ is $\sigma$-finite 即 $\mu_0(X) < \infty$) It remains to show that $\nu(E) \geq \mu^*(E)$.

\noindent Continuing 上一个 proof, we have:
$$
\mu^*(E) \leq \mu^*(\bigcup_{i=1}^\infty A_i) = \nu(\bigcup_{i=1}^\infty A_i) = \nu(E) + \nu(\bigcup_{i=1}^\infty A_i \setminus E)
$$
$$
\leq \nu(E) + \mu^*(\bigcup_{i=1}^\infty A_i \setminus E)
$$
我们只要 controling $\mu^*(\bigcup_{i=1}^\infty A_i \setminus E) = \mu^*(\bigcup_{i=1}^\infty A_i ) - \mu^*(E) = \epsilon $ 逼近 0, 即可得到反向的不等式关系.\\
\noindent (证明思路: 我们证明了 $\nu(E) \leq \mu^*(E)$ 之后, 注意到 covering set 和 $E$ 之间的差集的 $\nu$-measure 自然也小于等于这个差集的 $\mu^*$-measure, which can approximate 0.)
\\\\

\end{proof}


\begin{remark}
\noindent 1. 我们首先容易定义 $X$ 上的一个 algebra $\mathcal{M}_0$ 和一个 algebra 上的 premeasure $\mu_0$; \\\\
    \noindent 2. 然后用 inf of covering sum 来 induce 出一个 $\cP(X)$ 上的 outer measure $\mu^*$, 而后我们限制 $\mu^*$ 到 $\mu^*|_{\mathcal{M}^*}$ (where $\mathcal{M}^*$ 表示所有的 $\mu^*$-measurable sets), by Carathéodory's theorem 这就 induce 出了一个 complete measure. \\\\
    \noindent 3. 我们可以再取 $\mathcal{M}^*$ 的一个 sub $\sigma$-algebra $\mathcal{M} := <\mathcal{M}_0>$, 限制在这个集合上的 $\mu^*|_{\mathcal{M}}$ 自然也是一个 measure, 并且是 $\mathcal{M}_0$ extend 到 $\mathcal{M}$ 上的 lartest measure. By Hahn-Kolmogrov Thm, 这个 measure 如果是 $\sigma$-finite 的则是 $\mathcal{M}_0$ extend 到 $\mathcal{M}$ 上的 unique measure.\\
    \noindent (Notice: \textbf{自然地, $(X, \mathcal{M}^*, \mu^* |_{\mathcal{M}^*})$ 是 $(X, \mathcal{M}, \mu^*|_{\mathcal{M}})$ 的一个 completion.})
    
\end{remark}
