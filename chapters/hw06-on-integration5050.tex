\chapter{on integration(50/50)}
\begin{center}
\textit{None of the following questions will be graded. Do them, but do not hand them in}.
\end{center}

\section{Dirac measure: $\int f \; d\delta_{x_0}= f(x_0) $}
  Let $(X,\mathcal{A})$ be a measurable space, and $x_0\in X$ a point. 
  Let $\delta_{x_0}$ be the Dirac measure at $x_0$, i.e. for $E\in \mathcal{A}$, $\delta_{x_0}(E)=1$ if $x_0\in E$ and $\delta_{x_0}(E)=0$ if $x_0\notin E$. 
Show that every measurable function $f\colon X\to \mathbb{R}$ is integrable and 
\[
	\int f \; d\delta_{x_0}= f(x_0) 
\]
\textit{Remark}: what is often called a Dirac delta function is actually this Dirac measure. 

\section{measure space 的 extension 保留 measurable function 的可测性和积分}  
  Let $(X,\mathcal{A},\mu)$ and $(X,\mathcal{B},\nu)$ be measure spaces on the same set $X$. Suppose that $(X,\mathcal{B},\nu)$ is an extension of $(X,\mathcal{A},\mu)$. 
  \begin{itemize}
  \item[(a)] Show that if a function $f$ on $X$ is $\mathcal{A}$-measurable, then it is $\mathcal{B}$-measurable.
  \item[(b)] Show that if a function $f$ on $X$ is $\mathcal{A}$-measurable and $f\in L^1(\mathcal{A}, \mu)$, then $f\in L^1(\mathcal{B},\nu)$ and $\int f \; d\mu= \int f\; d \nu$. 
  \end{itemize}


\section{almost everywhere defined measurable function}
  Carefully think through the notion of an ``almost everywhere defined'' measurable (or integrable) function.
  How can we deduce the ``almost everywhere'' versions of the main convergence theorems (MCT, FL, DCT) from their ``everywhere'' counterparts?
  Propositions 2.11 and 2.12 in~[Folland] are useful here (these appeared on HW4).
  
\section{new measure from old: $\nu(A):=\int_A f\; d\mu \implies\int g \; d \nu= \int gf \; d\mu$}
  Let $(X, \mathcal{A},\mu)$ be a measure space. Let $f\colon X\to[0,\infty]$ be an $ \mathcal{A}$-measurable function.
Define $\nu\colon  \mathcal{A}\to[0,\infty]$ by  $\nu(A)=\int_A f\; d\mu= \int f\chi_A\; d\mu$ for $A\in \mathcal{A}$.  
\begin{itemize}
\item[(a)]Prove that $\nu$ is a measure on $(X, \mathcal{A})$. 
\item[(b)]Prove that $\int g \; d \nu= \int gf \; d\mu$ for every $ \mathcal{A}$-measurable function $g\colon X\to [0, \infty]$. 
\textit{Hint}: Start with the case when $g=\chi_E$; then treat the case when $g$ is a simple function; finally consider the case when $g$ is a general nonnegative function.
\item[(c)]Now  consider the case $(X, \mathcal{A},\mu)=(\mathbb{R},\mathcal{B}(\mathbb{R}), m)$, where $m$ is Lebesgue measure. 
Each nonnegative function $f\colon \mathbb{R} \to[0,\infty]$ induces a Borel measure $\nu_f(A)= \int_A f \; d m$ by (a).
\begin{itemize}
\item[(i)]Which functions $f$ induce a locally finite Borel measure? In that case, what is the distribution function for $\nu_f$?
\item[(ii)] Do all locally finite Borel measures arise from some $f$?
\item[(iii)] Can you interpret (b) as a change of variables formula? 
\end{itemize}
\end{itemize}



\section{Truncations in $L^1$: 通过 $\int f_n$ 或者 $\int_{X_n} f$ 的极限 (bounded function / subset) 得到 $\int_X f$}
Let $(X,\mathcal{A},\mu)$ be a measure space and $f\colon X\to\mathbb{C}$ an integrable function.
\begin{itemize}
\item[(a)] (Horizontal truncation) Suppose that $X=\bigcup_{n=1}^\infty X_n$ for some $X_1\subset X_2\subset \cdots$ 
with $X_n\in \mathcal{A}$. Prove that 
\[
  \int_X f \,d\mu = \lim_{n\to \infty} \int_{X_n} f\,d\mu
\]
\item[(b)] (Vertical truncation) Prove that
\[
  \int f\,d\mu = \lim_{n\to \infty} \int f\chi_{\{|f|\le n\}}\; d\mu
\]
\end{itemize}
\textit{Remark}: a similar question for nonnegative measurable functions appeared in HW4.

\section{$L^1$-convergence from dominated convergence}
Let $(X,\mathcal{A},\mu)$ be a measure space, and $f_n, f$, measurable functions on $X$, $n\in \mathbb{N}$. 
Suppose that $f_n\to f$ a.e.\ and there is an integrable nonnegative function $g$ such that $|f_n(x)|\le g(x)$ a.e.\ for all $n$. Prove that 
$f_n\to f$ in $L^1$, i.e.\
\[
  \lim_{n\to \infty} \int |f_n-f| =0.
\]
\textit{Hint}: use DCT.

\section{Lebesgue integrals and affine transformations}
  Let $f$ be a Lebesgue integrable function on $\mathbb{R}$. 
  Prove that 
  \[
    \int f(rx+s)\; d m(x)=\frac1{|r|}\int f(x)\; d m(x)
  \]
  for all real numbers $r,s$ with $r\ne0$.
  
  \textit{Hint}: approximate using simple functions $f$.


\section{even moments of Gaussian distribution}
    Using Multivariable Calculus (and the fact that Riemann integrals coincide with Lebesgue integrals) one can show that
    \[
      \frac1{\sqrt{2\pi}} \int_{-\infty}^\infty e^{-t\frac{x^2}{2}} \;d x = \frac1{\sqrt{t}}
    \]
    for every $t>0$.
%    Here, the integral is the improper Riemann integral $\lim_{a\to -\infty, b\to \infty} \int_a^b e^{-t\frac{x^2}{2} } \dd x$.
    Prove, by (justified!) differentiating with respect to $t$, that
    \[
      \frac1{\sqrt{2\pi}} \int_{-\infty}^\infty x^{2n} e^{-\frac{x^2}{2}} = (2n-1)!!
      := \frac{(2n)!}{2^n n!}
    \]
    for $n\in \mathbb{N}$.

    \textit{Remark}: here the integrals are as defined in this course.
    \textit{Remark}: in probability theory, these are the even moments of the standard normal distribution.

\section{Generalized DCT}
  Let $(X, \mathcal{A}, \mu)$ be a measure space, and $f_n, g_n, f, g\in L^1$, $n\in \mathbb{N}$. Suppose that 
  \begin{itemize}
  \item[(a)]$\lim_{n\to\infty} f_n(x)=f(x)$ and $ \lim_{n\to\infty} g_n(x)=g(x)$ for a.e. $x$;
  \item[(b)] $|f_n(x)|\le g_n(x)$ a.e. for every $n\in \mathbb{N}$;
  \item[(c)]$g_n\colon X\to [0, \infty]$ and $\lim_{n\to \infty} \int g_n \; d \mu = \int g\; d\mu$.
  \end{itemize}
  Prove that
  \[
    \lim_{n\to \infty} \int f_n \; d\mu = \int f \; d\mu.
  \]
\textit{Hint}: Follow the proof of the DCT, based on FL.



\section{Criterion for $L^1$-convergence}
  Let $(X, \mathcal{A}, \mu)$ be a measure space. 
  Let $f_n, f$ be integrable functions on $X$, $n\in \mathbb{N}$. 
  Suppose that $\lim_{n\to\infty} f_n(x)=f(x)$ a.e. Prove that \[
    \lim_{n\to\infty} \int |f_n-f| \; d\mu =0\quad\text{iff}
    \quad \lim_{n\to \infty} \int |f_n| \; d\mu = \int |f|\; d\mu\]
  \textit{Hint}: use the generalized DCT.






  
\vspace*{10mm}
\newpage
\begin{center}
\textit{Some of the following questions will be graded. Do them, and do hand them in}.
\end{center}

\section{Formal equivalence between MCT and FL}
  Let $(X,\mathcal{A},\mu)$ be a measure space and $L^+=L^+(X,\mathcal{A})$ the space of measurable functions $f\colon X\to[0,\infty]$. \\
  Let $I\colon L^+\to[0,\infty]$ be a function that is increasing in the sense that $f\le g$ implies $I(f)\le I(g)$. Prove that the following properties are equivalent:
  \begin{itemize}
  \item[(a)]$I$ is continuous along increasing sequences: if $f_n\in L^+$, and $f_n\le f_{n+1}$ for $n\in\mathbb{N}$, then $\lim I(f_n)=I(\lim f_n)$.
  \item[(b)] if $f_n\in L^+$, $n\in\mathbb{N}$, then $\liminf_nI(f_n)\ge I(\liminf_nf_n)$.
  \item[(c)] $I$ is lower semicontinuous: if $f_n,f\in L^+$, and $\lim_nf_n=f$, then 
    $I(f)\le\liminf_nI(f_n)$.
  \end{itemize}
  Here $\lim_nf_n=f$ means that $\lim_nf_n(x)=f(x)$ for all $x\in X$, and similarly for $\liminf f_n$.
\textit{Remark}: the equivalence between~(a) and~(b) shows that \textbf{the Monotone Convergence Theorem and Fatou's Lemma are equivalent.}

\begin{proof}
    \textbf{of (\(\textbf{a} \implies \textbf{b}\)):}\\
    Suppose \(I\) is continuous along increasing sequences. WTS: \[
  \liminf_{n} I(f_n) \;\ge\; I\!\bigl(\liminf_{n} f_n \bigr)
\]for any sequence \((f_n)\) in \(L^+\).\\
Define for each $k\in\mathbb{N}$ \[
    g_k \;:=\; \inf_{n \ge k}\, f_n
  \] Then for all $k \in \mathbb{N}$, $g_k$ is a measurable function. Also notice that by definition, $\{g_k\}$ is an increasing sequence, and  \[
    \lim_{k \to \infty} g_k(x) \;=\; \liminf_{n \to \infty} f_n(x)
  \]
for each \(x \in X\).\\
Applying \((\textbf{a})\) to \(g_k\): since \(g_k \uparrow \lim_{k} g_k\), we get \begin{equation}
      \lim_{k \to \infty} I(g_k)
    \;=\;
    I\Bigl(\lim_{k\to\infty} g_k\Bigr)
    \;=\;
    I\bigl(\liminf_{n \to \infty}  f_n\bigr)
 \end{equation}
By def of $g_k$, we have:  \[
    g_k \;\le\; f_n \quad \text{for all } n \ge k
  \]
Since  \(g_k \le f_n\) implies \(I(g_k) \le I(f_n)\), we also have: \[
    I(g_k) \;\le\; \inf_{n \ge k} \,I(f_n)
  \]
Taking the limit as \(k \to \infty\), we get 
\begin{equation}
\lim_{k\to\infty} I(g_k)
    \;\le\;
    \lim_{k\to\infty} \inf_{n \ge k}\, I(f_n)
    \;=\;
    \liminf_{n \to \infty} I(f_n)    
\end{equation}
Combining (5.1) and (5.2), we obtain: \[
    I(\liminf_{n} f_n)
    \;=\;
    \lim_{k} I(g_k)
    \;\le\;
    \liminf_{n} I(f_n).
  \] which is exactly what we want.\\\\\end{proof}
\begin{proof}
(\(\textbf{b} \implies \textbf{c}\)):  We now assume \((\textbf{b})\) and prove that \(I\) is lower semicontinuous, i.e. WTS: \[
  f_n \to f \quad \text{pointwisely} \quad \Longrightarrow \quad
  I(f) \;\le\; \liminf_{n} I(f_n).
\]
Given \(f_n \to f\) pointwise, we have
\[
  f(x)
  \;=\;
  \lim_{n} f_n(x)
  \;=\;
  \liminf_{n} f_n(x) \quad \forall x
\]
Hence for the sequence \(\{f_n\}\), the pointwise limit of \(f_n\) is exactly \(\liminf_{n} f_n\). \((\textbf{b})\) gives:
\[   \lim_{n} f_n(x) =   \liminf_{n} I(f_n) \;\ge\; I(\liminf_{n} f_n)=  I(f)\]
This is precisely the definition of lower semicontinuity, proving \((\textbf{b}) \implies (\textbf{c})\).\\\\
\end{proof}
\begin{proof}
of (\(\textbf{c} \implies \textbf{a}\)): \\
Assume $I$ is lower semi-continuous, i.e.  If \(f_n \to f\) pointwise, then \[
    I(f) \;\le\; \liminf_{n} I(f_n)
  \] Let \((f_n)\) be a sequence in \(L^+\) such that \(f_n \uparrow f\), i.e. \[
  f_1 \le f_2 \le \cdots
  \quad\text{and}\quad
  \lim_{n\to\infty} f_n(x) \;=\; f(x) \quad\text{ptwisely for all }x
\]
WTS (a): \(\lim_{n} I(f_n) = I(f)\). \\

Since $f_n$ is an increasing seq, \(f_n \le f\) for each $n$, and since \(I\) is monotone, we have \[
     I(f_n) \;\le\; I(f)
     \quad \forall n
   \]
Hence \[\limsup_{n} I(f_n) \;\le\;I(f) \]
And by \(\mathbf{(c)}\), since \(f_n \to f\) pointwisely, we have \[I(f) \le \liminf_{n} I(f_n)\]
Combining (1) and (2), we get
\[
  \liminf_{n} I(f_n)
  \ge I(f) \ge
  \limsup_{n} I(f_n)
\]
This we also has $ \liminf_{n} I(f_n) \leq 
  \limsup_{n} I(f_n)$, this shows that \(\lim_{n} I(f_n)\) exists and equals \(I(f)\).  This is exactly the statement of (a). Thus \(\mathbf{(c)} \implies \mathbf{(a)}\).\\\\
\end{proof}
Here we finished the proof that the three properties are equivalent. In particular, the equivalence of (a), (b) shows the equivalence of Fatou's Lemma and MCT.



\section{Convergence on subsets}
  Let $(X, \mathcal{A}, \mu)$ be a measure space. Let $f_n\colon X\to [0, \infty]$ be a measurable function for each $n\in \mathbb{N}$. 
Suppose that there is a function $f\colon X\to [0, \infty]$ such that 
\[
  \text{$\lim_{n\to \infty} f_n(x) = f(x)$ for every $x\in X$ and 
    $ \lim_{n\to \infty} \int f_n  = \int f$}
\]
\begin{itemize}
\item[(a)] Assume that $ \int f<\infty$. Show that $ \lim_{n\to \infty} \int_E f_n =\int_E f$ for every $E\in \mathcal{A}$. 
\textit{Hint}: Use Fatou twice. %for $f_n1_E$ and one more time for $f_n1_{E^c}$.) 
It may be useful to note that even though $\liminf (\alpha_n+\beta_n)\ge \liminf \alpha_n + \liminf \beta_n$ in general, if $\lim \alpha_n$ exists, then $\liminf (\alpha_n+\beta_n)=  \lim \alpha_n + \liminf \beta_n$ for sequences of extended real numbers $\alpha_n, \beta_n$.
\item[(b)]Find an example of $f_n\colon \mathbb{R}\to [0, \infty]$ on the measure space $(\mathbb{R}, \mathcal{B}(\mathbb{R}), m)$ showing that (a) does not necessarily hold if $ \int f=\infty$.
\end{itemize}

\begin{proof}
    \textbf{of (a):} \\
    By Fatou’s Lemma, since \(f_n \to f\) pointwise and all \(f_n\) are nonnegative,
\[
\liminf_{n\to\infty} \int_E f_n    =    \liminf_{n\to\infty} \int f_n \chi_E  \geq \int f\chi_E  =
\int_E f
\]
For the same reason, \[
     \liminf_{n\to\infty} \int_{E^c} f_n \,\ge\, \int_{E^c} f
   \]
Since \[
\int f \; d\mu = \int _X f \; d\mu  = \int _E f \; d\mu + \int _{E^c} f\; d\mu
\], we have: \begin{align}
    \int  f \; d\mu - \int _E f \; d\mu &= \int _{E^c} f \; d\mu \\
     &\leq \liminf_n \int_{E^c} f_n \; d\mu \\
     &= \liminf_n (\int f_n \; d\mu - \int_E f_n \;d \mu)\\
     & = \lim_{n \rightarrow \infty} \int f_n \; d\mu +  \liminf_n (-\int_E f_n \;d \mu) \\
     & = \lim_{n \rightarrow \infty} \int f_n \; d\mu - \limsup_n \int_E f_n \;d \mu \\
     & = \int f \; d\mu - \limsup_{n} \int_E f_n \;d \mu
\end{align}
Rearranging the terms, gives: \[
\int _E f  \geq \limsup_n \int_E f_n \;d \mu
\]
Combining with the statement given by Fatou's Lemma: \[
\liminf_{n\to\infty} \int_E f_n      \geq  
\int_E f
\]
We then have: 
\[
    \liminf_{n\to\infty} \int_E f_n =\int_E f \geq  \limsup_n \int_E f_n
   \]
Since also by definition of limsup and liminf we have: \[
\liminf_{n\to\infty} \int_E f_n \leq \limsup_{n\to\infty} \int_E f_n
\]
We have:  \[
\liminf_{n\to\infty} \int_E f_n  =  \limsup_{n\to\infty} \int_E f_n =  \lim_{n\to\infty} \int_E f_n = \int_E f
\]
This completes the proof.
\end{proof}
\begin{solution}
    \textbf{of (b):}
Define for each $n \in \mathbb{N}$ \[
f_n(x) := \chi_{[n,n+1]}  + \chi_{(-\infty, 0]}
\]
Then we have: \[
\int f_n(x) = 1 + \infty = \infty 
\]
for each $n$. So \[
\lim_{n\to\infty} \int f_n(x) = \infty
\]
And the pointwise limit of $f_n$ is \[
f(x) : = \lim_{n\to \infty} f_n(x) =   \chi_{(-\infty, 0]} \]
So the integral of $f$ is also: \[
\int \lim_{n\to \infty} f_n(x) = \int f(x) = \infty
\]
But consider the subset $E = [0,\infty)$, we have: \[
\int_E f_n  = \int \chi_{[n,n+1]}  = 1 \quad \text{for all } n
\]
So \[
\lim_{n\rightarrow \infty} \int_E f_n  = 1
\]while \[
\int_E f = 0 \not = \lim_{n\rightarrow \infty} \int_E f_n
\]
This completes the counterexample.
\end{solution}





\section{Some integrals}
    Use the DCT to evaluate the following limits:
  \begin{itemize}
  \item[(a)]\[\
      \lim_{n\to \infty} \int_0^\infty \frac{n\sin\left(\frac{x}{n}\right)}{x(1+x^2)} \; d x
    \]
  \item[(b)] \[
      \lim_{n\to\infty} \int_0^n x^m \left( 1- \frac{x}{n} \right)^n \; d x,
    \]
    where $m$ is a non-negative integer. (The integrals are Lebesgue integrals.)
  \end{itemize}

\begin{solution}
    \textbf{of (a):}\\
    Define \[
    f_n := \begin{cases}
  \frac{n\sin\left(\frac{x}{n}\right)}{x(1+x^2)}  ,\quad        x > 0\\
  0, \quad x\leq 0
    \end{cases}
    \]
Recall that for all $x  \in \mathbb{R}$, we have: \[
|\sin(x)| \leq |x|
\]
So for all $n$, and for all $x>0$, we have: \[
|f_n (x)| = \Bigg| \frac{n\sin\left(\frac{x}{n}\right)}{x(1+x^2)} \Bigg|  =   \frac{n 
 \sin\left(\frac{x}{n}\right)   }{x(1+x^2)}\leq   
\frac{n \frac{x}{n} }{x(1+x^2)}   = \frac{1}{1+ x^2}
\]
So by taking: \[
g(x) := \begin{cases}
  \frac{1}{1+x^2}  ,\quad        x > 0\\
  0, \quad x\leq 0
    \end{cases}
\]
We have: \[
g(x) \geq |f_n(x)| \quad \forall x\in \mathbb{R}, \forall n
\]
Since $g$ is continuous a.e. (except on $x=0$), it is a measurable function. And it is Riemann integrable. We can do Riemann integration of $g$: \[
\int_{0}^{\infty}
\frac{1}{1 + x^2} \, dx
\;=\;
\left[\arctan(x)\right]_{0}^{\infty}
\;=\;
\frac{\pi}{2} < \infty
\]
Also, for each $x > 0$, since \[
\lim_{n\to \infty} \frac{\sin(\frac{x}{n})}{\frac{x}{n}} = 1
\]
We have for each $x > 0$: \[
\lim_{n \to \infty} f_n(x) = \frac{1}{1+ x^2 } \lim_{n\to \infty} \frac{\sin(\frac{x}{n})}{\frac{x}{n}}   =  \frac{1}{1+ x^2 }
\]
Thus the pointwise limit of $f_n$ is:  \[
f(x):=\lim_{n \to \infty} f_n(x) = \begin{cases}
  \frac{1}{1+x^2}  ,\quad        x > 0\\
  0, \quad x\leq 0
    \end{cases}
\]
(Notice it coincides with the $g$ that we chose as bound.) We also have: 
\[ \int_{0}^{\infty} f(x) \; dx = \frac{\pi}{2}
\]
Then by DCT, 
\[
\lim_{n\to \infty} \int_0^\infty \frac{n\sin\left(\frac{x}{n}\right)}{x(1+x^2)} \; d x =\lim_{n \to \infty} \int_{0}^{\infty} f_n(x) \; dx =\int_{0}^{\infty} \lim_{n\to \infty} f_n(x) \; dx = \int_{0}^{\infty} f(x) \; dx =   \frac{\pi}{2}
\]
This finishes the calculation.
\end{solution}


\begin{solution}
    \textbf{of (b):}\\
Define for each $n \in \mathbb{N}$ \[
f_n(x) = x^m \left(1 - \frac{x}{n}\right)^n 
\quad \text{for} \quad 0 \le x \le n
\]
and $f_n(x) = 0$ for $x > n$.\\
Then the integral we wish to evaluate can be written as \[
 \lim_{n\to \infty} \int_0^n x^m \left(1 - \frac{x}{n}\right)^n \, dx  = \lim_{n\to \infty} 
\int_0^\infty f_n(x) \, dx
\]
We first evaluate the ptwise limit function $f:= \lim_{n\to \infty} f_n(x)$.\\
For \(x = 0\): \[
   f_n(0) \;=\; 0^m \left(1 - \frac{0}{n}\right)^n = 0^m \cdot 1 =0^m e^{-x} \quad \forall n
   \]
For \(0 < x < \infty\): \[
   f_n(x) \;=\; x^m \left(1 - \frac{x}{n}\right)^n
   \]
for all large enough $n$. \\
Recall the standard limit \(\lim_{n \to \infty} \left(1 - \tfrac{x}{n}\right)^n = e^{-x}\), hence \[
  f(x) := \lim_{n \to \infty} f_n(x) = \lim_{n \to \infty}   x^m \left(1 - \frac{x}{n}\right)^n = 
   x^m e^{-x}
   \]
Thus \[
f(x) = \begin{cases}
    0, \quad x< 0 \\
       x^m e^{-x}, \quad x \geq 0
\end{cases}
\]
Now we determine the dominating function $g$.\\
Consider the same function as $f$:
\[
g(x) :=  \begin{cases}
    0,\quad x< 0\\
    x^m  e^{-x}, \quad x\geq 0
\end{cases}
\]
We now prove this same function $g$ works.\\
Let $n \in \mathbb{N}$.\\
It is sure that for $x > n$, $g(x) \geq |f_n(x)|$ since $f_n(x) = 0$.\\
So consider $x\in[0,n]$.\\
Recall the inequality: \[ \ln(1-t)\le -t \quad \forall t\in[0,1]\]
Thus we have: 
\[
\left(1 - \frac{x}{n}\right)^n \leq
e^{-\frac{x}{n}n} =
e^{-x}
\]
Therefore,
\[
0 
\;\le\;
x^m \left(1 - \frac{x}{n}\right)^n 
\;\le\;
x^m e^{-x}
\quad
\text{for all } 0 \le x \le n
\]
Thus in all cases,
\[|f_n(x)|=f_n(x) \leq 
x^m e^{-x} =
g(x)
\]
Recall:
\[
\int_0^\infty x^m e^{-x}\,dx = \Gamma(m+1) = m!
\]
is \textbf{finite} for all nonnegative integers \(m\). Thus \(g\) is \textbf{integrable}. Then \textbf{$g$ is indeed a dominating function for $(f_n)$.} \\
Applying the DCT, we exchange the limit and the integral:
\[
\lim_{n \to \infty} \int_0^\infty f_n(x)\, dx =
\int_0^\infty \lim_{n\to\infty} f_n(x)\, dx =
\int_0^\infty x^m e^{-x}\, dx
\] thus
\[
\lim_{n \to \infty} \int_0^n x^m \left(1 - \frac{x}{n}\right)^n \, dx =
\int_0^\infty x^m e^{-x}\, dx = \Gamma(m+1) = m!
\] This finishes the evalutation of this integral.
\end{solution}




  \section{Continuity of translations}
  Let $f\in L^1(\mathbb{R}, \mathcal{L}, m)$. For $x\in\mathbb{R}$, set $f_s(x)= f(x-s)$. 
  Prove that $s\mapsto f_s$ is a continuous map from $\mathbb{R}$ to $L^1$. In other words, prove that if $t\in\mathbb{R}$, then 
  \[
    \lim_{s\to t} \int |f_s-f_t| \; d m =0 
  \]
  \textit{Hint}: approximate $f$. 
\begin{proof}
We write: \[
||f-g||_1 := \int |f-g| \;dm
\] for $f,g\in L^1(\mathbb{R}, \mathcal{L}, m)$.
Let $\epsilon > 0$.\\
Recall that \( C_c(\mathbb{R}) \) is dense in \( L^1(\mathbb{R}) \). So there exists a function \( g \in C_c(\mathbb{R}) \) such that
\[
\| f - g \|_1 < \frac{\epsilon}{3}
\] Since \( g \) is continuous and compactly supported, it is \textbf{uniformly continuous}. Denote $ K :=\supp (g) $. 
There exists \(\delta > 0\) such that for all \( x \in \mathbb{R} \),
\[
|s - t| < \delta \implies |g(x - s) - g(x - t)| < \frac{\epsilon}{3 \cdot m(K)}
\]
Integrating the difference over this support gives:
\[
\| g_s - g_t \|_1 \leq \frac{\epsilon}{3 \cdot m(K)} \cdot m(K) = \frac{\epsilon}{3}
\]
Recall that $L^1(\mathbb{R}, \mathcal{L}, m)$ is a normed vector space with $||\cdot||_1$ as the norm. So by the triangle inequality of a norm, we have:
\[
\| f_s - f_t \|_1 \leq \| f_s - g_s \|_1 + \| g_s - g_t \|_1 + \| g_t - f_t \|_1\]
By the translation invariance of Lebesgue measure, we have: 
\[
\| f_s - g_s \|_1 = \| f - g \|_1 < \frac{\epsilon}{3} \quad \text{and} \quad \| g_t - f_t \|_1 = \| g - f \|_1 < \frac{\epsilon}{3}
\]By choosing \(\delta\) such that \(\| g_s - g_t \|_1 < \frac{\epsilon}{3}\), we get
\[
\| f_s - f_t \|_1 < \frac{\epsilon}{3} + \frac{\epsilon}{3} + \frac{\epsilon}{3} = \epsilon
\]
Since $\epsilon$ is arbitrary, this proves that for any \( t \in \mathbb{R} \), \[
\lim_{s \to t} \int |f_s - f_t| \, dm =  ||f_s - f_t||_1  = 0
\]finishing the proof of continuity of the map \( s \mapsto f_s \).
\end{proof}






\section{An interesting integrable function}
  For $\alpha\in(0,1)$, define $g_\alpha\colon \mathbb{R}\to \mathbb{R}$ by $g_\alpha(x)=(1-\alpha)x^{-\alpha}$ for $0<x<1$ and $g_\alpha(x)=0$ otherwise. Let $(x_n)_n$ be an enumeration of the rational numbers, and define $f\colon \mathbb{R}\to[0,\infty]$ by
  \[
    f(x)=\sum_{n=1}^\infty2^{-n}g_{1-n^{-n}}(x-x_n)
  \]
  Prove that $f$ has the following properties:
  \begin{itemize}
  \item[(a)]$f$ is Borel (and hence Lebesgue) measurable;
  \item[(b)] $f$ is Lebesgue integrable, that is $\int_\mathbb{R} f\; d m<\infty$;
  \item[(c)]  there exist uncountably many $x\in\mathbb{R}$ such that $f(x)<\infty$;
  \item[(d)]  $f$ is discontinuous at every point $x\in\mathbb{R}$ where $f(x)<\infty$;
  \item[(e)]  $f$ is unbounded on any nonempty open interval $I=(a,b)$, that is $\sup_If=\infty$;
  \item[(f)]   the statements in~(d) and~(e) remain true even if we redefine $f$ on a set of (Lebesgue) measure zero.
  \item[(g)]$\int_I f^p\; d m=\infty$ for all $p>1$ and all intervals $I=(a,b)$.
  \end{itemize}


\begin{proof}
    \textbf{ of (a):}\\
  We define \[ \alpha_n : = 1 - n^{-n}
  \] and \[
    h_n(x) :=  2^{-n}g_{\alpha_n}(x - x_n)
    \] and \[
    f_k(x) := \sum_{n=1}^k 2^{-n}g_{\alpha_n}(x-x_n) = \sum_{n=1}^k  h_n(x)
    \] to simplify the expression.\\
Then we have: \[
f(x) = \lim_{k \rightarrow \infty} f_k(x)
\]
 Notice that, since each $g_{\alpha_n}$ is nonnegative, $f_k(x)$ is a \textbf{increasing} sequence of functions, so for any $x \in \mathbb{R}$, $ \lim_{k\to \infty} f_k(x)$ exists in $\overline{\mathbb{R}}$. This shows the well-definedness of $f  = \lim_{k\to \infty} f_k$. \\
Now we \textbf{claim: each $h_n(x)$ is Borel measurable.}\\
By translate invariance and scaling invariance of Borel measurability, to prove the claim, it \textbf{suffices to prove that each $g_{\alpha}$ is Borel measurable for any $\alpha \in (0,1)$}.\\
\pic[0.3]{assets/hw5-image-20250214195825512.png}
If $a < 0$, we have:  \[
g_\alpha^{-1}((a,\infty)) =  \mathbb{R}
\]
if $ 0 \leq a \leq  1-\alpha$, then we have \[
g_\alpha^{-1}((a,\infty)) = (0,1)
\]
if $ a > 1-\alpha$, then we have \[
g_\alpha^{-1}((a,\infty)) = (0,(\frac{1-\alpha}{a})^{1/\alpha})
\]
This proves that $g_\alpha$ is Borel measurable for any $\alpha \in (0,1)$.\\
Thus each $f_k$ being a \textbf{finite sum of Borel measurable functions}, is Borel measurable.\\
Then $f $ as \textbf{the limit of Borel measurable function sequence} $(f_k)$, is Borel measurable.\\\\
\end{proof}
\begin{proof}
    \textbf{of (b):}\\
    We define: \[
    h_n(x) :=  2^{-n}g_{\alpha_n}(x - x_n)
    \] in order to simplify the expression.\\
By translation invariance of Lebesgue measure, we have for any $\alpha_n$, : \[
     \int_{\mathbb{R}} g_{\alpha_n}(x - x_n)\,dm =
     \int_{\mathbb{R}} g_{\alpha_n}(x)\,d m_t
     = (1-\alpha) \cdot \frac{1 - 0}{1-\alpha} = 1
   \]
So by homogeneity of integral,  \[
\int_{\mathbb{R}} h_n(x) \; d m =  \int_{\mathbb{R}} 2^{-n}g_{\alpha_n}(x - x_n)\,dm = 2^{-n}  \int_{\mathbb{R}} g_{\alpha_n}(x - x_n)\,d m  = \frac{1}{2^n}
\]
Thus we have: \[
\sum_{n=1}^\infty \int_{\mathbb{R}} |h_n(x)| = \sum_{n=1}^\infty \int_{\mathbb{R}} h_n(x) =  \frac{1/2}{1-1/2} = 1 < \infty
\] by sum of geometric series.
Since this sum of integral of the sequence is finite, we can apply \textbf{theorem 2.25 on Folland, to exachange the order of limit and integral}, and have: \[
 \int_{\mathbb{R}}  \sum_{n=1}^\infty h_n(x)   = \sum_{n=1}^\infty \int_{\mathbb{R}} h_n(x)  =1
\]
Hence,
\[
  \int_{\mathbb{R}} f \,d m
 =  \int_{\mathbb{R}}  \sum_{n=1}^\infty h_n(x)   dm = \sum_{n=1}^\infty  \int_{\mathbb{R}}  h_n(x)   dm  = 1
\]
So \(\int_{\mathbb{R}} f < \infty\).  This proves \(f\in L^1(\mathbb{R})\).\\\\
\end{proof}



\begin{proof}
\textbf{of (c):}
\begin{lemma}
    For $f \in L^+(\mu)$,   if $f(x) =+\infty$ on a set $S$ where $\mu(S) >0$, then $\int f = \infty$
\end{lemma}
Proof for Lemma: trivially follows from definition. We can pick make a sequence of simple functions $(\phi_n)$, setting $\phi_n |_S = n$ (doable since $f|_S = \{\infty\}$) then we have: \[
\int \phi_n \; d\mu \geq  \int  n  \chi_S = n
\]
So the limit of integral of this simple function sequence is $\infty$. \\\\
Then (c) follows from the lemma: suppose for contradiction that there exist only countably many $x\in\mathbb{R}$ such that $f(x)<\infty$, we denote this this by $C$, then on $\mathbb{R} \setminus C$ which has positive measure (since $C$ has measure 0), $f(x) = \infty$. So by lemma, $\int f = \infty$, contradicting with the fact that $\int f = 1$ proven in (b). So there exist uncountably many $x\in\mathbb{R}$ such that $f(x)<\infty$.\\\\
\end{proof}


\begin{proof}
\textbf{of (e):}
Fix an interval \( I \). By the density of rational numbers in any interval, there exists some rational $x_N \in I$. Note that though $g_{\alpha_N} (x_N) = 0$, $g_{\alpha_N} (x)$ can be arbitrarily large near $x_N$.\\
Fix $M > 0$.\\
It suffices to pick some $x$ s.t.
\[
2^{-N}g_{\alpha_N } (x-x_N) = \frac{1-\alpha_N}{2^N}   (x-x_N)^{-\alpha_N} > M
\]
So by taking any\[
x \in (x_N, x_N + (\frac{2^N M}{1-\alpha_N})^{\alpha_N}) \cap I
\]
then it is done.\\
Since we already have $2^{-N}g_{\alpha_N } (x-x_N) > M$, we have \[
f(x) > 2^{-N}g_{\alpha_N } (x-x_N) > M
\]
Since $M$ is arbitrary, this proves that the value of $f$ on $I$ can be unboundedly large, finishing the proof that \[
\sup_I f = \infty
\]
\end{proof}


\begin{proof}
\textbf{of (d):}
Notice that we first proved (e) and then let's prove (d) using the conclusion of (e).\\
Let $x \in \mathbb{R}$ s.t. \(f(x)<\infty\).\\
Suppose $f$ is continuous at $x$, then by definition, there exists an open neighborhood $B_\delta(x) = (x-\delta, x+ \delta)$ s.t. $|f(y) - f(x)| < \frac{1}{83}$ for all $y \in B_\delta (x)$.\\
But since the neighborhood is an interval, we have:\[
\sup_{(x-\delta, x+ \delta)} f = \infty
\] by (e). This two facts contradicts. So by contradiction we have proved that $f$ is discontinuous at $x$.\\
So we can conclude that $f$ is discontinuous at any point $x$ s.t. $f(x) < \infty$.\\\\
\end{proof}



\begin{proof}
\textbf{of (f):} 
Let $I$ be an interval.\\
Suppose we have redefined $f$ on a measure $0$ set. We pick a rational $x_N \in I$ (It does not matter whether the new $f$ is defined there.)\\
For arbitrary $M>0$, we can still always find an $x$ s.t. $x \in (x_N, x_N + (\frac{2^N M}{1-\alpha_N})^{\alpha_N}) \cap I$ that \textbf{keeps its original $f(x)$}, which guarantees that $f(x) > M$, implying $\sup_I f = \infty$. This is because, if not so, then it means that we have modified the whole interval $(x_N, x_N + (\frac{2^N M}{1-\alpha_N})^{\alpha_N}) \cap I$, \textbf{which is not a measure zero set}, \textbf{conflicting with the statement} "redefining $f$ on a measure zero set".
So (e) must still hold true.\\
For (d), we apply the same trick as original, getting an open interval around $x$ s.t. $|f(y) - f(x)| < \frac{1}{83}$ for all $y \in B_\delta (x) = (x-\delta, x+ \delta)$. And by the restated (d), even if we modified a set of measure zero on $(x-\delta, x+ \delta)$, we still reaches the the same conclusion that $\sup_{(x-\delta, x+ \delta)} f = \infty$, thus causing the same contradiction.\\
This finishes the proof.\\\\
\end{proof}



\begin{proof}
\textbf{of (g):}
WTS: \(\int_I f^p \, d m = \infty\) for all \(p>1\) and every interval \(I\)
\textbf{Claim: for each $n$,  \(g_{\alpha_n}^p\) \emph{fails} to be in \(L^1\) when \(p>1\), i.e its integral is $\infty$.}
Fix $p>1$.\\
Since by translation invariance of Lebesgue integral,:
\[
  \int_{\mathbb{R}} \Bigl(2^{-n}g_{\alpha_n}(x - x_n)\Bigr)^p \, d m
  \;=\;
  2^{-np}\int_{\mathbb{R}} g_{\alpha_n}(x)^p \, d m
\]
where
\[
  g_{\alpha_n}(t)^p 
  \;=\;
  \bigl(n^{-n}t^{-\,\alpha_n}\bigr)^p
  \;=\;
  n^{-np}\,t^{-\,p\alpha_n}
  \;=\;
  n^{-np}\,t^{-\,p\,(1 - n^{-n})}
\]
Since \(p > 1\), there eixst $N$ such that for all $N\geq n$, the exponent \(-p(1-n^{-n})\) is less than \(-1\), causing \(\int_0^1 t^{-p + p\,n^{-n}}\,dt = +\infty\) for sufficiently large \(n\).  Multiplying by the constant \(n^{-np}\) does not remove the infinity. \\
Hence for large enough $n$, each individual summand has an infinite integral, then by monotonicity of integral, \[
f^p(x) = (\sum_n 2^{-n}g_{\alpha_n}(x - x_n))^p \geq 2^{-N}g^p_{\alpha_N}(x - x_N)
\]also has an infinite integral, finishing the proof.\\\\
\end{proof}








\begin{center}
  \textit{Nur f\"ur Verr\"uckte}
\end{center}
(It's \textbf{really} not necessary to attempt these problems. Do not, under any circumstances, hand them in!) 

\begin{enumerate}
    \item Make an accurate sketch of the graph of the function in the last problem.
\end{enumerate}