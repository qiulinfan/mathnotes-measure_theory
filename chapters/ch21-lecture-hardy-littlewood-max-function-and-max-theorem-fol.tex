\chapter{Hardy-Littlewood max function and max theorem [Fol 3.4]}
目前我们 finish 了 Folland 的 Ch1, Ch2.\\
我们将先跳过 Radon-Nikodym differentiation theory, 在讲完 $L^p$ space theory 后再回到 Radon-Nikodym differentiation theory. 但是我们将会先将 differentiation theory 中的一个特殊部分: HL max theorem 和 Lebesgue differentiation theorem, 因为它们在 $L^p$ space theory 中需要被用到.\\
此 lec 对应: Folland 3.4( 1)\\

Differentiation theorey 的 overview: \\
Radon-Nikodym derivative 并不是 classical calculus 的扩展 (classical calculus 表示变量之间的相对变化), 而是专门针对: 同一个 measure space 上, 一个测度对于另一个测度 (要求它们之间绝对连续) 的变化率. \[
d\nu =  f  d\mu
\]
从而 \[
\nu(A)  = \int_A f \; d\mu
\]
这使得我们可以更改一个积分 with respect to 的测度. \\\\
其核心定理: Radon-Nikodym Theorem, 表示了在一般的 measure space 上, 这两个测度满足一定条件下, 这个 Radon-Nikodym derivative 的存在性; 而 LDT 提出一种\textbf{在 Euclidean space 上, 求 Radon Dikodym derivative 的方法.}\\
LDT 本身是一种将积分信息转换为点态信息的手段. 它表示\textbf{对于 locally integrable 的函数, 局部积分平均值可以收敛到函数本身, a.e.}\\ 
因而在知道 $\nu$ 和 $\mu$ 的情况下, LDT 提供了类比经典微积分中 "导数是局部变化率的极限" 的观点: 在 Euclidean space 上, Radon Nikodym derivative 等于局部均值: \[
f(x)  = \lim_{r \to 0 } \frac{\nu(B(r,x))}{\mu(B(r,x))} \;\; \text{ for a.e. } x
\]

Radon Nikodym derivative 的应用: 比如在概率论中, \textbf{pdf/pmf 都是 cdf 对于 Lebesgue measure 的 Radon Nikodym derivative}; 在贝叶斯推理中,给定先验 prior 和观测数据的分布, 后验分布 posterior 的密度可以通过 Radon-Nikodym 导数计算.
下面介绍概念:\\\\

\section{$L^1_{loc}$ and local average}
\begin{definition}{locally integrable}
如果 measurable  $f:\mathbb{R}^n \to \mathbb{C}$ 在任意 bounded subset of $\mathbb{R}^n$ 上的 integral 都 $< \infty$, 则称 function $f:\mathbb{R}^n \to \mathbb{C}$ 是 locally integrable 的, 写作 $f\in L^1_{loc}(m)$.
\end{definition}
\begin{remark}
    等价于 $f$ \textbf{在任意 compact set 上}的 integral 都 $< \infty$: \[
    \int_K f\; dm < \infty
    \]for all $K$ ; 或者$f$ \textbf{在任意 open ball 上}的 integral 都 $< \infty$: \[
    \int_{B(r,x)} f \; dm< \infty
    \] for all $x,r$.\\
    这是比 $f \in L^1(m)$ 更弱的条件. For example: 连续函数一定是 $L^1_{loc}$ 的.
\end{remark}

\begin{example}
    考虑 \[
    f(x) := |x|^p, \quad x\in \mathbb{R}^n
    \]
   (使用 polar coord) 可验证: \[
    f\in L^1_{loc}(m) \quad \Longleftrightarrow  \quad p > -n
    \]
\end{example}
 

\begin{definition}{average}
    对于 $f\in L^1_{loc}(m)$, 以及 bounded and Lebesgue measurable $E\subset \mathbb{R}^n$ with $m(E) > 0$, 我们定义: \[
    \avint_E f : = \frac{1}{m(E)}\int_E f\; dm
    \] 为 $f$ 在 $E$ 上的 \textbf{average value}.\\
特别地, 当 $E$ 为一个 ball $B(r,x)$ 时, 我们可以写作: \[
A_r f(x) := \avint_{B(r,x)} f
\]表示它在 $x$ 为中心的 $r$ 为半径的 ball 上的 average.
\end{definition}


\begin{lemma}
对于任意 $f\in L^1_{loc}$, $A_r f(x)$ 都是 jointly continuous in $r$ and $x$ 的. ($r>0, x\in \mathbb{R}^n$)
\end{lemma}
\begin{proof}
    Suppose $(x_j, r_j) \to (x,r)$ in $\mathbb{R}^n \times \mathbb{R}_{>0}$.\\
    于是 for sure: \[
    m(B(x_j, r_j)) \overset{j\to \infty}{\longrightarrow} m(B(x,r))
    \]
并且 by DCT (取 $\chi_{B(r_0+1, x_0)}$ 作为 bound) 可以得到: \[
\int f \chi_{B(x_j,r_j)} \to \int \chi_{B(x,r)}
\]
\end{proof}

\section{Hardy-Littlewood maximal function}
\begin{definition}{Hardy-Littlewood maximal function}
对于 $f \in L^1_{loc}$, 我们定义它的 HL maximal function 为: \[
    Hf(x) : = \sup_{r>0} A_r |f| (x)
    \]
\end{definition}
HF maximal 函数 $Hf(x) $表示 $f$ 的绝对值函数在 $x$ 处能取到最大的 local average.

\begin{theorem}{HL maximal function 是 measurable 的}
    对于任意 $f \in L^1_{loc}$, $Hf$ 都是 measurable 的.
\end{theorem}
\begin{proof}
    Follows from lemma. \[
    (Hf)^{-1} ((a,\infty)) = \bigcup_{r>0} (A_r |f|)^{-1} ((a,\infty))
    \] 是 open 的, 因为 $A_r|f|$ ctn, ctn function 下 open set 的 preimage 也 open.
\end{proof}

\begin{corollary}
    如果 $f:\mathbb{R}^n \to [0,\infty)$ 是 lower semictn 的, 那么它一定 Borel (thus Lebesgue) measurable.
\end{corollary}




\subsection{Vitali-type convering lemma}
对于一个 ball $B = B(x,r)$ 以及一个 constant $c$, 我们定义: \[
cB := B(x,cr)
\]
\begin{lemma}{Vitali-type convering lemma}
    For given collection of balls $\{B_j \subset \mathbb{R}^n\}_{j=1}^k$, 存在 \textbf{disjoint} subcollection $\{B_{j_1},\cdots, B_{j_m}\}$ 使得\[
\bigcup_{j=1}^k B_j  \subset \bigcup_{i=1}^m (3B_{j_i}) 
    \]
    (于是, \[
    m(\bigcup_{j=1}^k B_j) \leq 3^n m(\bigcup_{i=1}^m (3B_{j_i}))
    \])
\end{lemma}
\begin{proof}
    Greedy Algrithm: 直接按照半径大小排序, 取出最大的 disjoint subcollection.\\
    Prove without words: 
    \pic[0.4]{assets/ch3-pics-image-20250313171222780.png}
    (每次都选择下一个和前面所有更大的球不 intersect 的最大球; 在这个过程中, 所有和前面更大的球有 intersection 的球都被被包括在该球的三倍球里.)
\end{proof}


\subsection{Hardy-Littlewood maximal theorem}
\begin{theorem}{HL maximal theorem}
For $L^1(m^n)$, take constant $C := 3^n$, 则对于任意 $f \in L^1(m^n)$, 都有: \[
m(\{x: Hf (x) > \alpha\}) \leq \frac{C}{\alpha} \int |f| 
\]
\end{theorem}
\begin{proof}
    Set \[
    E_\alpha : = \{x: Hf (x)> \alpha   \} \subset \mathbb{R}^n
    \]
    因而 by def of $Hf$, 对于任意的 $x \in E_\alpha$ 都存在 $r_x$ 使得 \[
    m(B(x,r_x))  < \frac{1}{\alpha} \int_{B(x,r_x)} |f|
    \]
对于 compact $K\in E_\alpha $, 一定存在 finite subcovering $\{B(x_i,r_i)\}$ covers $K$.\\
于是 Apply Vitali-type covering Lemma:\[
m(K) \leq \sum_i m(3B(x_i,r_i)) = 3^n \sum_i m(B(x_i,r_i)) \leq \frac{3^n}{\alpha} \int |f|
\]
于是 by inner regularity, taking sup over all compact subsets 得证.
\end{proof}
\begin{remark}
    这是 HL max Thm 的一部分, 另一部分是 $L^p$ space 下的.  这一部分表示了 HL maximal operator 的 \textbf{weakly boundedness}.\\
    Notice: $H$ 是一个从 $L^p$ space (此处为 $L^1$)到它自身的 (nonlinear) operator.\\
    recall, 一个 operator 的 \textbf{(strongly) boundedness} 表示: \[|| T f \|\leq C \| f \|, \quad \text{for all } f  \text{ (in the function space)}
\]
而 weakly boundedness 表示一定的可控制性: 越大的函数值, 能取到这个函数值的点的占比 (with respect to measure) 越小. \\
(will prove: \textbf{对于 $p>1$, $H$ 具有 strongly boundedness}.)
\end{remark}

\begin{remark}
    我们可以 compare the HL ineq 和 Markov ineq. Markov ineq 表示, 对于任意 $f \in L^1(\mu)$ 都有 \[
    m(\{f> \alpha\}) \leq \frac{1}{\alpha} \int |f|
    \]
\end{remark}