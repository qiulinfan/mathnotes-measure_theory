\chapter*{Homework 1: on $\sigma$-algebra (39/40)}

\section*{Borel vs Open}
Let $X$ be a metric space such that every subset of $X$ is Borel set. Does it follow that every subset of $X$ is open? Give a proof or a counterexample.
\begin{solution}
    It is not true.\\
    Every subset of $X$ is Borel set $\Leftrightarrow \mathcal{P}(X) \subset \mathcal{B}_X$. And We know $\mathcal{B}_X \subset \mathcal{P}(X)$, so it is equivalent to saying that $\mathcal{B}_X = \mathcal{P}(X)$.\\
    So consider this counterexample: $\mathbb{Q}$ with the Euclidean metric.\\
    Claim: every singleton set in $\mathbb{Q}$ is closed, thus in $\mathcal{B}_\mathbb{Q}$.
    This is because this only sequence in a singleton set is the point itself repeating, thus converging to itself, in the singleton set. This proves the claim.\\
    And since $\mathbb{Q}$ is countable, every subset of $\mathbb{Q}$ is a countable union of singleton sets, thus by property of $\sigma$-algebra, every subset of $\mathbb{Q}$ is in $\mathcal{B}_{\mathbb{Q}}$. Thus: $$
    \mathcal{B}_\mathbb{Q} = \mathcal{P}(\mathbb{Q})
    $$
    But clearly, \textbf{not every subset in $\mathbb{Q}$ is open.} Consider any singleton set, $\{1\}$ as an example. Any open ball centered at $1$ is not contained in $\{1\}$, thus contradicting the statement.
    
\end{solution}

\section*{Restriction of a $\sigma$-algebra to a Subset}
Let $X$ be a set, and $Y \subset X$ a subset.
\begin{itemize}
    \item[(a)] Given a $\sigma$-algebra $\mathcal{A}$ on $X$, prove that
    \[
    \mathcal{A}|_Y := \{E \cap Y \mid E \in \mathcal{A}\}
    \]
    is a $\sigma$-algebra on $Y$.
    \item[(b)] Given a $\sigma$-algebra $\mathcal{B}$ on $Y$, prove that there exists a $\sigma$-algebra $\mathcal{A}$ on $X$ such that $\mathcal{A}|_Y = \mathcal{B}$.
    \item[(c)] Is the $\sigma$-algebra $\mathcal{A}$ in (b) unique? Give a proof or a counterexample.
\end{itemize}
\begin{remark}
这表示任何一个 measurable space 都可以对其中的一个 subspace 取一个 submeasurable space
\end{remark}

\begin{proof}
    \begin{itemize}
    \item[(a)] 
    \begin{enumerate}
        \item Since $\emptyset \in\mathcal{A}$, $\emptyset \cap Y = \emptyset$, we have $\emptyset \in\mathcal{A}|_{Y}$
        \item Let $F \in \mathcal{A}|_{Y}$, we must have $E \in \mathcal{A}$ s.t. $E\cap Y = F$. Since $E\in \mathcal{A}$, we have $X\setminus E \in \mathcal{A}$, so $X \setminus E \cap Y \in \mathcal{A}|_{Y}$. Since $E \cap Y = F$ and $Y = (E \cap Y) \sqcup ((X \setminus E) \cap Y)$, it implies $(X \setminus E) \cap Y = Y\setminus F$, therefore $Y \setminus F \in \mathcal{A}|_{Y}$.
        \item Let $F_1, F_2,\cdots$ be a sequence of subsets in $\mathcal{A}|_{Y}$. Then for each $i\in \mathbb{N}$, we have $F_i = E_i \cap Y$ for some $E_i \in \mathcal{A}$. Then $\bigcup_{i=1}^\infty F_i = \bigcup_{i=1}^\infty (E_i\cap Y) = (\bigcup_{i=1}^\infty E_i) \cap Y \in \mathcal{A}|_{Y}$ since $\bigcup_{i=1}^\infty E_i \in \mathcal{A}$. 
    \end{enumerate}
    
    \item[(b)] Let $\mathcal{B}$ be a $\sigma$-algebra on $Y$.
    
    
    prove that there exists a $\sigma$-algebra $\mathcal{A}$ on $X$ such that $\mathcal{A}|_Y = \mathcal{B}$.
    Consider let 
    $$
    \mathcal{A} := \{\,E \subset X \mid E \cap Y \in \mathcal{B}\}
    $$
    Then 
    $$
    \mathcal{A} |_{Y} = \{ E \cap Y \mid E,Y\subset X,E \cap Y \in \mathcal{B}  \} = \mathcal{B}
    $$            
    We then prove that this is a $\sigma$-algebra on $X$. \\
    \begin{enumerate}
        \item  $\emptyset \cap Y = \emptyset$ so $\emptyset \in \mathcal{A}$. 
        \item \textbf{Closed under complement}: Let $E \in \mathcal{A}$, we have $E \cap Y \in \mathcal{B}$, so $Y \setminus (E\cap Y ) = Y\setminus E \in \mathcal{B}$.\\
        Then $(X\setminus E) \cap Y  = Y \setminus E 
        \in \mathcal{B}$, so $X\setminus E \in \mathcal{A}$. 
        \item \textbf{Closed under countable union}: Let $E_1,E_2,\cdots$ be a sequence in $\mathcal{A}$, then \(E_n \cap Y \in \mathcal{B}\). for each $n$. 
        Hence
     \[
       \left(\bigcup_{n=1}^\infty E_n\right) \cap Y
       \;=\; \bigcup_{n=1}^\infty (E_n \cap Y)
       \;\in\; \mathcal{B},
     \]
     since \(\mathcal{B}\) is a \(\sigma\)-algebra on \(Y\).  Therefore, \(\bigcup_{n=1}^\infty E_n \in \mathcal{A}\).  
     
    \end{enumerate}

    
    \item[(c)] 
    This is not unique. \\
    \noindent Counterexample:
    \[
    X = \{ 0,1,2\}, Y = \{0\} \subset X
    \]
    Consider
    \[
    A_1 := \mathcal{P}(X), A_2 := \{\emptyset, \{0\}, \{1,2\}, X \}
    \] are valid $\sigma$-algebra on $X$.\\
    \noindent Then we have $A_1 |_Y = A_2 |_Y = \{\emptyset,\{0\}   \}$, while $A_1$ is different from $A_2$.


    
    
\end{itemize}
\end{proof}


\section*{Invariance Properties of the Borel $\sigma$-algebra on $\mathbb{R}^n$}
\begin{itemize}
    \item[(a)] Prove that $\mathcal{B}(\mathbb{R}^n)$ is translation invariant, i.e., if $A \subset \mathbb{R}^n$ is a Borel measurable set, then
    \[
    t + A := \{t + x \mid x \in A\}
    \]
    is a Borel measurable set for every $t \in \mathbb{R}^n$. (Hint: For any fixed $t$, show that $A = \{B \subset \mathbb{R}^n : t + B \in \mathcal{B}(\mathbb{R}^n)\}$ is a $\sigma$-algebra.)
    \item[(b)] Prove that $\mathcal{B}(\mathbb{R}^n)$ is scaling invariant, i.e., if $A \subset \mathbb{R}^n$ is a Borel measurable set, then
    \[
    \lambda A = \{\lambda x \mid x \in A\}
    \]
    is a Borel measurable set for every $\lambda \in \mathbb{R}$.
\end{itemize}

(1)
\begin{proof}

Fix \(t \in \mathbb{R}^n\). Define
\[
\mathcal{A} :=\{\, B \subseteq \mathbb{R}^n : t + B \in \mathcal{B}(\mathbb{R}^n) \}.
\]
We want to show that \(\mathcal{A} = \mathcal{B}(\mathbb{R}^n)\). We first show that $\mathcal{A}$ is a $\sigma$-algebra.

1. \(\emptyset     \in  \mathcal{A}\) since \(t + \emptyset = \emptyset \in \mathcal{B}(\mathbb{R}^n)\).

2. $\mathcal{A}$ is closed under complement: Let \(B \in \mathcal{A}\), then \(t + B \in \mathcal{B}(\mathbb{R}^n)\). The complement \((t + B)^c\) is also in \(\mathcal{B}(\mathbb{R}^n)\). Observe
\[
   t + B^c  = t + \mathbb{R}^n \setminus B = (t+\mathbb{R}^n) \setminus (t+B) = \mathbb{R}^n  \setminus (t+B) =  (t+B)^c
   \]
\noindent Since \(t + B\) is Borel, its complement is Borel, hence \(t + B^c\) is Borel, so \(B^c \in \mathcal{A}\).

3. $\mathcal{A}$ is closed under countable unions: Let \(B_k \in \mathcal{A}\) for \(k = 1, 2, \dots\), then \(t + B_k \in \mathcal{B}(\mathbb{R}^n)\). Thus
   \[
   t + \bigcup_{k=1}^{\infty} B_k
   \;=\;
   \bigcup_{k=1}^{\infty} (t + B_k)
   \;\in\;
   \mathcal{B}(\mathbb{R}^n).
   \]
\noindent Hence \(\bigcup_{k=1}^{\infty} B_k \in \mathcal{A}\).
\noindent These three properties show that \(\mathcal{A}\) is a \(\sigma\)-algebra. \\

\noindent Since \(t + U\) is open if \(U\) is open in $\mathbb{R}^n$, \(\mathcal{A}\) contains all open sets. Since \(\mathcal{B}(\mathbb{R}^n)\) is the smallest \(\sigma\)-algebra containing all open sets in \(\mathbb{R}^n\), we have:\(
\mathcal{B}(\mathbb{R}^n) \;\subseteq\; \mathcal{A}
\) 
\noindent Hence suppose \(A \in \mathcal{B}(\mathbb{R}^n) \), then $A \in \mathcal{A}$, so $t+A \in   \mathcal{B}(\mathbb{R}^n)$. This completes the proof of translation invariance.
\end{proof}

(2)
\begin{proof}
Fix \(\lambda \in \mathbb{R}\). 
Case 1: $\lambda = 0$, then $\lambda A = \{ 0\}$ if $A \not = \emptyset$, and $\lambda A = \emptyset$ otherwise. Both $\{0\}$(closed set) and $\emptyset$ is Borel set.

Case 2: $\lambda \not = 0$. We define
\[
\mathcal{A} :=\{\, B \subseteq \mathbb{R}^n :  \lambda B \in \mathcal{B}(\mathbb{R}^n) \}.
\]

We want to show that \(\mathcal{A} = \mathcal{B}(\mathbb{R}^n)\). We first show that $\mathcal{A}$ is a $\sigma$-algebra.

1. \(\emptyset     \in  \mathcal{A}\) since \(\lambda\emptyset = \emptyset\).

2. $\mathcal{A}$ is closed under complement: Let \(B \in \mathcal{A}\), then \(\lambda B \in \mathcal{B}(\mathbb{R}^n)\), then \((\lambda B)^c\) is also in \(\mathcal{B}(\mathbb{R}^n)\). Observe $(\lambda B)^c = \lambda B^c$, so $\lambda B^c \in \mathcal{B}(\mathbb{R}^n)$, therefore $B^c \in \mathcal{A}$.
3. $\mathcal{A}$ is closed under countable unions: Let \(B_k \in \mathcal{A}\) for \(k = 1, 2, \dots\), then \(\lambda B_k \in \mathcal{B}(\mathbb{R}^n)\). Thus
   \[
    \lambda \bigcup_{k=1}^{\infty} B_k
   \;=\;
   \bigcup_{k=1}^{\infty} (\lambda B_k)
   \;\in\;
   \mathcal{B}(\mathbb{R}^n).
   \]
\noindent Hence \(\bigcup_{k=1}^{\infty} B_k \in \mathcal{A}\).
\noindent These three properties show that \(\mathcal{A}\) is a \(\sigma\)-algebra. \\

\noindent Since $\lambda \not = 0$, \(\lambda U\) is open iff \(U\) is open in $\mathbb{R}^n$, thus \(\mathcal{A}\) contains all open sets, so
\(
\mathcal{B}(\mathbb{R}^n) \;\subseteq\; \mathcal{A}
\), 

Hence if \(A\in \mathcal{B}(\mathbb{R}^n)\),  we have \(A  \in  \mathcal{A}\), therefore $\lambda A \in \mathcal{B}(\mathbb{R}^n)$. This completes the proof of translation invariance.
\end{proof}






\section*{Hex and Such}
Let $A \subset [0,1]$ be the set of real numbers in $[0,1]$ having a hexadecimal expansion with the digit 5 appearing infinitely many times, and the ‘digit’ E appearing at most finitely many times. Prove that $A$ is a Borel set. (Hint: see p. 2 of Folland’s book.)
\begin{proof}
Define:
   \[
   B := \{ x \in [0,1] \mid \text{the digit '5' appears infinitely many times in the hex expansion of }x \}.
   \]\[
   C := \{ x \in [0,1] \mid \text{the digit 'E' appears at most finitely many times in the hex expansion of }x \}.
   \]

\noindent Then clearly
\[
A = B \cap C.
\]
\noindent Hence \textbf{it suffices to show that \(B\) and \(C\) are Borel sets}, since intersection of two Borel sets is a Borel set.
\noindent And thus it \textbf{suffices to show that $B^c$ and $C$ are Borel sets}. Note 
$$B^c = \{ x \in [0,1] \mid \text{the digit '5' appears at most finitely many times in the hex expansion of }x \}$$, so the proof for $B^c$ and $C$ are about the same.
\noindent We now show $B^c$ is a Borel set:
\noindent  We define
\[
     C_{d_1 d_2 \cdots d_n} \;:=\; \bigl\{\, x\in[0,1] : \text{the first }n\text{ hexadecimal digits of }x \text{ are }d_1, d_2, \ldots, d_n \bigr\},
   \]
where each \(d_i\) is one of the 16 hexadecimal digits \(\{0, 1, 2, \dots, 9, A, B, C, D, E, F\}\).  
 Then the set contains all real numbers between $\frac{d_1d_2\cdots d_n}{16^n}$ and $\frac{d_1d_2\cdots d_n + 1}{16^n}$, so actually it is an interval:

   \[
     C_{d_1 d_2 \cdots d_n} =   \left[\frac{d_1d_2\cdots d_n}{16^n},\;\frac{d_1d_2\cdots d_n + 1}{16^n}\right)
   \]
 Since it is an interval, it is a Borel set on \([0,1]\).
\noindent And we define:
\[
D_N = \{ x : \text{from digit \(N\) onward, there are no '5's} \}.
\]
Then we have
\[
B^c = \bigcup_{N=1}^{\infty} D_N,
\]
So it suffices to prove that each $D_N$ is Borel set, since a countable union of Borel sets is Borel set.

\noindent   \textbf{Claim : any $D_N$ is a Borel set.}
To prove this, we fix  an $N$ and define for each $n \geq N$
\[
     E_n 
     \;=\; 
     \{\,x \in [0,1]: d_n(x)\neq 5\}.
   \]
 \noindent  Then we have 
$$
E_n = \bigcup_{d_i \in \{1,\cdots,F\}   \forall 1\leq i \leq n, d_n \not = 5}  C_{d_1d_2\cdots d_n}
$$
 Thus \textbf{each $E_n$ is a Borel set} since it is a finite union of Borel set, which shows that $D_N$ is Borel set, since 
 \[
     D_N 
     \;=\;
     \bigcap_{k = N}^{\infty} E_k.
   \]
This finishes the proof that $B^c$ is a Borel set, and by a similar argument, $C$ is a Borel set, and thus $A = B \cap C$ is a Borel set. 
   
\end{proof}

\section*{Admissible Annuli generating $\mathcal{B}(\mathbb{R}^n)$} 
Define an admissible annulus in $\mathbb{R}^2$ to be a set of the form
\[
\{(x, y) \in \mathbb{R}^2 \mid r^2 < (x - a)^2 + (y - b)^2 < R^2\},
\]
where $a, b \in \mathbb{Q}$, $r, R \in \mathbb{Q}_{>0}$, and $r < R$.

\begin{itemize}
    \item[(a)] Prove that there are only countably many admissible annuli.
    \item[(b)] Prove that every open subset of $\mathbb{R}^2$ is a countable union of (not necessarily disjoint) admissible annuli.
    \item[(c)] Prove that the Borel $\sigma$-algebra on $\mathbb{R}^2$ is generated by the collection of admissible annuli.
\end{itemize}

(1) \begin{proof} Let 
\[ A:=   \{     \text{all admissible annulis in $\mathbb{R}^2$}   \}\]
And we define 
\begin{align}
     f:\mathbb{Q}^4   & \to A \\
    (a,b,r,R)  & \mapsto \{(x, y) \in \mathbb{R}^2 \mid r^2 < (x - a)^2 + (y - b)^2 < R^2\}
\end{align}
Since a Annuli defined by this $(a,b,r,R)$ is unique, this is a well-defined function; and since every admissible annulis can be defined by an element of $\mathbb{Q}^4$, this map is surjective. Therefore $\card(A) \leq \card(\mathbb{Q}^4)$, so $A$ is countable.
\end{proof}
(2) 
\begin{proof}
\noindent \textbf{Claim 1: every open set in $\mathbb{R}^2$ is a countable union of open balls, each centered at some $q\in \mathbb{Q}^2$.}\\
\noindent Proof for Claim 1:\\
\noindent Let $U$ be an open set in $\mathbb{R}^2$.
\noindent Define
\[\mathbb{Q}_U  :=    U \cap \mathbb{Q}^2\]
\noindent By definition, every point in $U$ have an open ball centered at it that is completely contained in $U$, so we pick such ball $B_{r_x}(x)$ for each $x \in U$.
\noindent Since \(\mathbb{Q}^2\) is dense in \(\mathbb{R}^2\), for each \(x \in U\) and each corresponding \(r_x\), we can find a rational point \(q_x \in \mathbb{Q}^2\) such that \(|q_x - x| < \frac{r_x}{3}\). (Or more generally, as small as we wish.)

\noindent Let \(r_{q_x} > 0\) be chosen so that 
   \(
     r_{q_x} = \frac{r_x}{3},
   \)
Then observe that $x \in   B(q_x, r_{q_x}) $
     \[
     B(q_x, r_{q_x}) \subsetneq B(x, r_x) \subset U
   \]
\noindent   which follows from the triangle inequality.
   \pic[0.2]{assets/hw1(1).png}

For each $q \in \mathbb{Q}_U$, we define:
\[ r_{q,sup} := \sup\{  r_{q_x}  \mid q  \text{ is chosen by } x         \}     \]
Now we have:
\[  U \subset \bigcup_{q \in U_q}  B_{r_{q,sup}} (q)  \]
\noindent  This is because for each each $x \in U$, $x \in B_{r_{q_x}}(q_x) \subset B_{r_{q_x,sup}}(q_x)$

And we also have the other direction:
\[ \bigcup_{q \in U_q}  B_{r_{q,sup}} (q)   \subseteq U\]
since every $B_{r_q}(q)$ is guaranteed to be the subset of some ball around some $x \in U$. 
All togethe we have
\[
 U = \bigcup_{q \in U_q}  B_{r_{q,sup}} (q)
\]
This finishes the proof of claim 1.

\textbf{Claim 2: every open ball centered at some $q\in \mathbb{Q}^2$ is a countable union of admissible annulises with the same center, together with another admissible annulis whose center is also rational.}
Proof for Claim 2: 
\noindent Let $q = (a,b) \in \mathbb{Q}^2$.\\
We have
\[    
   B\bigl(q, R)  \setminus \{q\}
   \;=\; 
   \bigcup_{n=1}^{\infty} \Bigl\{(x,y) : (R-\tfrac{1}{n})^2 < (x-a)^2 + (y-b)^2 < R^2\Bigr\}
\]
\textcolor{red}{-1, 这里写的略有问题, 因为 $R$ 不一定是 rational 的, 不过我们可以用 density of $\mathbb{Q}$ in $\mathbb{R}$ 来写.}
It remains to cover the center. Let $q' := (a',b') \in \mathbb{Q}^2$ such that $R/6 < |q' - q| < R/3$, $r' := R/6$ and $R' := R/2$ . Then the annuli $A(a',b',r',R')$ defined by the four parameters is contained in the $ B\bigl(q, R)$ and it covers $\{q\}$. 
Therefore
\[    
   B\bigl(q, R)
   \;=\; 
 (  \bigcup_{n=1}^{\infty} \Bigl\{(x,y) : (R-\tfrac{1}{n})^2 < (x-a)^2 + (y-b)^2 < R^2\Bigr\}) \cup A(a',b',r',R')
\]
\pic[0.2]{assets/hw1(2).png}
This finishes the proof of Claim 2.\\
Combining Claim 1 and Claim 2, we can conclude that \textbf{every open subset of $\mathbb{R}^2$ is a countable union of admissible annuli.}\end{proof}

\noindent (3) 
\begin{proof}

As defined,
\[
\mathcal{B}(\mathbb{R}^2) = <\mathcal{T}_{metric}> = <\{\text{all open sets in }\mathbb{R}^2\}>
\]
 Let 
\[ A:=   \{     \text{all admissible annulis in $\mathbb{R}^2$}   \}\]
Every admissible annuli is open in $\mathbb{R}^2$, so 
\[
A \subset \{\text{all open sets in }\mathbb{R}^2\}
\]
and since $\mathcal{B}(\mathbb{R}^2)$ is a $\sigma$-algebra, we have 
\[
<A> \subset <\{\text{all open sets in }\mathbb{R}^2\}> =\mathcal{B}(\mathbb{R}^2)
\]by the proposition proved in class.
\noindent And by (2), any open set is a countable union of admissible annulis, therefore every open set is in $<A>$ since any countable union of sets in a $\sigma$-algebra is still in the set. So
\[   
 \{\text{all open sets in } \mathbb{R}^2\} \subset <A>
\]
This finishes the proof that 
\[
<A> = <\{\text{all open sets in }\mathbb{R}^2\}> =\mathcal{B}(\mathbb{R}^2)
\]
\end{proof}





\section*{Nur für Verrückte}
(It’s really not necessary to attempt these problems. Do not hand them in!)

\begin{itemize}
    \item[(1)] Let $X$ be a set, and define two operations on $\mathcal{P}(X)$:
    \begin{itemize}
        \item The “product” of two subsets $E, F \subset X$ is the intersection $E \cap F$.
        \item The “sum” of two sets $E, F \subset X$ is the symmetric difference $E \Delta F$.
    \end{itemize}
    \begin{itemize}
        \item[(a)] Prove that these operations endow $\mathcal{P}(X)$ with the structure of a commutative ring. What are the additive and multiplicative units? Prove that this ring is idempotent.
        \item[(b)] Let us say that a nonempty subset $A \subset \mathcal{P}(X)$ is a ring if it is closed under differences and finite unions. In other words, if $E, F \in A$, then $E \setminus F \in A$ and $E \cup F \in A$. Prove that a subset $A \subset \mathcal{P}(X)$ is an algebra iff it is a ring containing $X$.
        \item[(c)] Prove that a nonempty subset $A \subset \mathcal{P}(X)$ is a ring iff it is a subring of $\mathcal{P}(X)$. Also prove that it is an algebra iff it is a subring containing the multiplicative identity.
    \end{itemize}
    \item[(2)] Let $(X, \mathcal{A})$ and $(Y, \mathcal{B})$ be measurable spaces. Say that a map $f : X \to Y$ is measurable (with respect to the $\sigma$-algebras $\mathcal{A}$ and $\mathcal{B}$) if $f^{-1}(E) \in \mathcal{A}$ for every $E \in \mathcal{B}$.
    \begin{itemize}
        \item[(a)] Prove that measurable spaces with measurable maps as morphisms form a category.
        \item[(b)] Try convincing an analyst that (a) is useful.
    \end{itemize}
\end{itemize}