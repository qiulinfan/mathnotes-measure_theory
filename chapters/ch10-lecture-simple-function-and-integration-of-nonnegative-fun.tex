\chapter{simple function and integration of nonnegative functions [Fol 2.1, finished; 2.2]}

\section{indicator and simple function}
\begin{definition}{characteristic (indicator) function}
    Given $E \subseteq X$, 我们定义:
    $$
    \chi_E(x) := \begin{cases}
        1 \quad,x\in E\\
        0 \quad, x\not \in E
    \end{cases}
    $$
\end{definition}

\begin{lemma}
如果 $(X,\mathcal{M})$ 是一个 measurable space, 那么一个 indicator function
\begin{center}
   \textbf{ $\chi_E$ on $X$ 是 measurable 的 $\Longleftrightarrow$ $E \in  \mathcal{M}$}
\end{center}
\end{lemma}
indicator function measurable 当且仅当它 indicate 的集合是 measurable 的.







\begin{definition}{simple function}
    一个 simple function on measurable space $(X,\mathcal{A})$ 是一个 $\mathcal{A}$-measurable function $\phi:X \rightarrow \mathbb{C}$, taking only finitely many values.

即: $\phi(X) = \{c_1,\cdots, c_k\}$
\end{definition}

\begin{proposition}{使用 \textbf{a sum of indicator functions of measurable sets} 来定义 simple function}
对于 simple function $\phi:X \rightarrow \mathbb{C}$ s.t. $\phi(X) =\{ c_1,\cdots,c_n\} $, 我们也可以定义它为:  $$
    \phi(x) = \sum_{j=1}^n c_j \chi_{E_j}
    $$
其中, $E_j  =  \phi^{-1}(\{c_j\})$.
我们称之为: the \textbf{standard representation of simple $\phi$.}
\end{proposition}
这是因为, 单点集在 $\mathcal{B}(\mathbb{C})$ 上是 measurable 的, \textbf{由于 $\phi$ measurable, 我们得到 $E_j \in \mathcal{M}$.}
\begin{remark}
    对于 simple function $$
    \phi(x) = \sum_{j=1}^n c_j \chi_{E_j}
    $$ 一定有 $$
    \bigsqcup_{j=1}^n E_j = X
    $$其中通常有一个 $E_j$ 上 $\phi$ 的值是 0.
\end{remark}



\begin{lemma}
    如果 $\phi, \psi: X \rightarrow \mathbb{C}$ 是 simple functions, 那么 
    \begin{itemize}
        \item  $\phi + \psi$
        \item $\phi \psi$
        \item $|\phi|$
        \item $k\phi$ $\forall k \in \mathbb{C}$
    \end{itemize}
都是 simple functions.

特别地, 如果  $\phi, \psi: X \rightarrow  \mathbb{R}$, 那么 $
\max(\phi,\psi), \min(\phi,\psi)
$ 也是 simple functions.
\end{lemma}
\begin{proof}
    trivial.
\end{proof}




\section{measurable function is a limit of simple functions}
\begin{theorem}{approximating a nonneg measurable function by simple function}
\label{approximating a nonneg measurable function by simple function}
    任意的 measurable $f:X \rightarrow [0,\infty]$ 都是 \textbf{pointwise limit} of an \textbf{increasing sequence of simple functions} $\{\phi_n:X\rightarrow [0,\infty]\}_{n\in\mathbb{N}}$.
\end{theorem}
\begin{proof}
    这个构造看起来有点复杂但是其实非常直观.

    对于 $n\in \mathbb{N}$, 我们都 index $0 \leq k \leq 2^{2n}-1$

    然后对每个 $k$ 取:
    $$
    E_n^k := f^{-1}((\frac{k}{2^n},\frac{k+1}{2^{n+1}}])
    $$
    以及:
    $$
    F_n := f^{-1}((2^n,\infty])
    $$

    即, 我们把 $(0,2^n]$ 这一部分值域切成了 $2^{2n}$ 份, 再把 $(2^n,\infty]$ 这一部分值域单独列成一份.
    
    这 $2^{2n} +1$ 份值域的切片, 我们对每一份所对应的 function graph, 都取它对应的 Preimage 上的 indicator function 乘以 $\frac{k}{2^n}$, 这段值域的最小值的 constant 函数, 于是一定会得到一个 well approximation:

$$
\phi_n := \sum_{k=0}^{2^{2n}-1} k\frac{k}{2^n} \chi_{E_n^k} + 2^n \chi_{F_n}
$$
易得, $$\phi_n \leq \phi_{n+1} \leq f$$
for all $n$. 并且\textbf{在 $X \setminus F_n = \{ x\mid f(x) \leq 2^n  \}$ 上我们有}:

$$
0 \leq f-\phi_n \leq \frac{1}{2^n}
$$

    
    随着 $n$ 增大, 最终这个近似会覆盖整个 image, (除非具有非零测数量的无穷间断点, 那样的话最后结果也是无穷), 并且值域的划分越来越精细, 最后会得到: 
    \begin{itemize}
        \item \textbf{$\phi_n \rightarrow f$ pointwisely}
        \item \textbf{在 $f$ bounded 的定义域 $\{x \mid f(x) <\infty\}$ 上, $\phi_n \rightarrow f$ uniformly.}
    \end{itemize}

    \pic[0.6]{assets/ch2-pics-simple.png}    
\end{proof}
\begin{remark}
我们在构造 simple function 的时候这样用到 measurability:
这里的每个 $\phi_n$ 是 simple function, 是由于 $f$ measurable, 以至于每个 \textbf{$E_n^k, F_n$ 作为 interval 的 preimage, 都是 measurable sets. } 
\end{remark}

\begin{corollary}{approximating a complex-valued measurable function by simple function}
对于任意的 measurable $f:X\rightarrow \mathbb{C}$, 都存在 a seq of simple functions $$0 \leq |\phi_1| \leq |\phi_2| \leq \cdots \leq |f|$$ 使得 
\begin{itemize}
    \item \textbf{$\phi_n \rightarrow f$ pointwisely}
    \item \textbf{$\phi_n \rightarrow f$ uniformly on $\{ x \mid |f(x)| < \infty\}$}
\end{itemize}
\end{corollary}

\begin{proof}
    我们可以把 $f$ 拆为 $\im f, \re f$, 然后再把它们分别拆为 $\im f^+ - \im f^-$, 以及 $\re f^+ - \re f^-$. 得到四个 real-valued nonng functions.
\end{proof}




\section{integration of non-neg functions}

\begin{definition}{$L^+$ space and integration on it}
给定一个 measure space $(X, \mathcal{M},\mu)$
    我们定义: $$
    L^+(\mu)  := \{  \textbf{measurable functions }  f: X \rightarrow [0,\infty]   \} $$
对于所有的 \textbf{simple functions $\phi = \sum_{j=1}^n a_j \chi_{E_j} \in L^+(\mu)$}, 即所有非负的 simple functions, 我们定义\textbf{ the integral of $\phi$ with respect to $\mu$} by:
$$\int \phi d \mu  \;(= \int_X   \phi d\mu  ) := \sum_{i=1}^n a_j \mu(E_j)$$
对于任意的 $f \in L^+(\mu)$, 我们定义 \textbf{the integral of $f$ with respect to $\mu$ }by: 
$$ \int f d \mu \;(= \int_X  fd\mu ) := \sup \{   \int \phi d\mu \mid 0\leq \phi \leq f, \phi  \text{ simple} \}$$
\end{definition}

\begin{remark}
因而对于 general 的非负可测函数, 我们通过 \ref{approximating a nonneg measurable function by simple function} 得知, 我们可以用 simple function 来近似它. 从而, 我们使用 simple function 的积分的极限来定义 general measurable function 的积分.

而 simple function 的积分, 即等于它下方的面积. 因而我们发现, 这个积分的定义和 $\mathbb{R}^n \rightarrow \mathbb{R}$ 上 Rieamnn 积分有很大的相似之处, 不同在于一个竖切定义域一个横切值域.

之后我们也会证明, 在 $\mathbb{R}^n \rightarrow \mathbb{R}$ 上, 所有 Riemann 可积的函数也 Lebesgue 可积, 并且得到的结果相同.

这一积分的定义是对 Riemann 积分的推广.
\end{remark}

\begin{remark}
    measure theory 中的积分理论是把从 $\mathbb{R}^n$ 出发的函数 推广到了从抽象的测度空间出发的函数; 而还有其他的积分理论, 比如微分形式上的积分则是把实值函数的积分推广到了 oriented smooth manifolds 上, 不仅可以积分 scalars 还可以积分向量场. 这些积分理论的共同点是对 $\mathbb{R}^n\rightarrow \mathbb{R}$ 上的函数的积分是 coincide 的.

笔者感觉积分理论就是在一个抽象空间上,通过一个抽象的密度函数(被积函数) 以及体积指标(measure function), 得到一个抽象质量。由于这个理念本身是从 $\mathbb{R}^n$ 上 generalize 的,因而各种不同的积分理论在 $\mathbb{R}^n$ 上的积分总是 coincide 的
\end{remark}



\begin{definition}{integration on a subset}
    对非负 \textbf{simple functions $\phi = \sum_{j=1}^n a_j \chi_{E_j} \in L^+(\mu)$}, 我们定义 \textbf{the integral of $\phi$ on $A\in \mathcal{M}$ with respect to $\mu$} by:
$$
\int_A \phi \;d\mu  := \int \phi \chi_A \; d\mu
$$
对于 general 的 $f \in L^+(\mu)$, 我们也从而定义:
$$ \int_A  fd\mu  := \sup \{   \int_A \phi d\mu \mid 0\leq \phi \leq f, \phi  \text{ simple} \}$$
\end{definition}
\begin{remark}
$$
\int_A \phi \;d\mu  := \int \phi \chi_A \; d\mu = \sum_j a_j \chi_{A \cap E_j}
$$
\end{remark}




\begin{proposition}{integral of simple functions 的性质}
Let $\phi, \psi$ be simple functions in $L^+(\mu)$, 有:
\begin{itemize}
    \item \textbf{homogeneity:} 对于任意非负 $c$, 有 $\int c\phi = c\int \phi$
    \item \textbf{linearity:} $\int (\phi + \psi) = \int \phi + \int \psi$
    \item \textbf{monotonicity:} $\phi \leq \psi \implies \int \phi \leq \int \psi$
    \item \textbf{induced measure}: $A \mapsto \int_A \phi \; d\mu$ 是一个 $\mathcal{M}$ 上的 measure.
\end{itemize}
\end{proposition}
\begin{proof}
   \textbf{ homogeneity}  trivial .

  \textbf{  linearity:} Let  $$
    \phi = \sum_{i=1}^n a_i \chi_{E_i} \quad , \psi = \sum_{j=1}^n b_j \chi_{F_j}
    $$ 则有: $$
    E_j = \bigsqcup _k (E_j \cap F_k)\quad , F_k = \bigsqcup_j (E_j \cap F_k) $$ for each $j,k$. 从而有 $$ \int \phi + \int \psi = \sum_{j,k} (a_j + b_k) \mu(E_j \cap F_k)  $$

    \textbf{Monotonicity}: trivial.

    induced measure: 只需要证明 countable additivity, 于是我们让 $A$ be the union of a disjoint seq in $\mathcal{M}, 有:  $$$  \int_A  \phi = \sum_j a_j \mu(A \cap E_j) = \sum_{j,k}  a_j \mu(A_k \cap E_j)    = \sum_k \int_{A_k} \phi$$
\end{proof}
\begin{remark}
    本身, 我们已经基于一个 measure 作为 "体积密度", 来定义一个 simple function 按照这个体积密度得到的积分, 而它在每个可测集上的积分又可以定义另一个 measure; 
    
    这个 measure 表示 "某个集合和 $E_1, \cdots, E_n$ 的交集在这个体积密度以及 simple function 放缩下有多大".
\end{remark}



那么对于 general 的 $f\in L^+(\mu)$, 有刚才的四条性质成立吗? \textbf{显然, monotonicity 和 homogeinity 是成立的}, 但是我们会发现, 很难证明
$$
\int f \; d\mu + \int g \;d\mu = \int(f+g) \;d \mu
$$
$\leq $ 是容易证明的, 但是 $\geq$ 有点困难. 为了证明 $\geq$ 这个方向, 我们需要下面这个重要定理:




\section{MCT}
\begin{theorem}{monotone convergence theorem}
\label{MCT}
Let $\{f_n\}_{n\in\mathbb{N}}$ be a seq in $L^+(\mu)$, 并且有 $f_n \leq f_{n+1}$ for each $n$. \\
我们 define:$$f := \lim_n f_n \; \; (= \sup_n f_n)$$, 则一定有 $$\int f = \lim_{n\rightarrow \infty } \int f_n$$
 \end{theorem}
\begin{proof}
首先 Note 几个事情: 
1. 这个极限函数 \textbf{$f$ 是 well-defined 的} (可能 $\infty$), by \textbf{numerical sequence 的 monotone bounded convergence theorem}. 

2. 同样地, 由于 $\int f_n \leq \int f_{n+1} \leq \int f$, 这个 \textbf{$\lim \int f_n$ 也是存在的}.

3. 并且, $f$ 也是一个可测函数, 因为 by 上个 lecture 的定理: \textbf{可测函数序列的极限也是可测函数.}

现在进行证明: 
By monotonicity of integral, $$\lim \int f_n \leq \int f$$ 

是 natural 的. 因而只需要证明另一方向.

By def, $\int f = \sup \{ \int \phi \mid \phi \leq f\}$ where $\phi $ is simple.
因而 it \textbf{suffices to show: 对于任意 simple $\phi \leq f$, 都有 $\lim \int f_n \geq \int \phi$.}

我们 fix 一个 $0 \leq \phi \leq f$. WTS: $$\lim_n \int f_n \geq \int \phi$$

要证明 $\lim \int f_n \geq \int \phi$, 我们再把它转化成证明: $$\forall \alpha \in (0,1)\;\;\lim_n \int f_n \geq \alpha \int \phi$$
我们取 $$ E_n := \{x \mid f_n(x) \geq \alpha \phi  \} = f^{-1} ([\alpha \phi,\infty]) \in \mathcal{M}$$

容易发现, $E_n \subseteq E_{n+1}$ for each $n$. 并且 Claim: $\bigcup_n E_n = X$.  (这就是为什么要做取 $\alpha$ 这个意义不明的行为) 这是因为 $\alpha < 1$, 并且 $f_n$ converge pointwisely to $f$, by measurable function 的 limit behavior.  \textbf{而由于 simple function $\phi$ 是 bounded 的, 从而 $f_n$ 会 uniformly 向上接近(以至于超过) $\phi$. 取 $\alpha$ 是为了保证, 一定存在一个 $n$ 使得 $E_n = X$ }

于是我们有:
$$
\int f_n \geq \int f_n \chi_{E_n}  \geq \int \alpha \phi \chi_{E_n} = \alpha \int_{E_n} \phi
$$

我们此处又可以用到一条冷门的性质: \textbf{由于 $E \mapsto \int_E \phi$ 是一个 measure on $(X,\mathcal{A})$, by continuous from below, 有:}
$$
\lim_n \int_{E_n} \phi  = \int \phi
$$
从而有$$\lim_n \int f_n \geq \alpha \int \phi$$ finishing the proof.
\end{proof}
\begin{remark}
    这是一个非常重要的定理. 它表示了\textbf{非负可测函数的极限的积分等于积分的极限, 可以把取极限和积分这两个操作进行换序.}
\end{remark}

以下为一个应用 MCT 得到的结论.

\begin{example}
    取 $$(\mathbb{N}, \mathcal{P}(\mathbb{N}), \mu_{counting})$$
    于是$$L^+(\mu) = \{f:\mathbb{N}\rightarrow [0,\infty]  \}$$
    是所有的从自然数到 reals 的函数. (因为我们取了 power set 作为 $\sigma$-algebra)

   \noindent  注意到任何一个这样的函数都可以被 $$\phi_n :=  \sum_{j=1}^n f(j) \mu(\{j\}) =  \sum_{j=1}^n f(j)   $$ 来逼近. 从而 $$
    \int f = \sum_{j=1}^\infty f(j)  \in [0,\infty]
    $$ 
如果取一个从下逼近 $f$ 的可测函数序列 $\{f_n\}_{n\in\mathbb{N}}$, 那么 by MCT, 我们总有:
$$
\sum_1^\infty f_n (j) \nearrow \sum_1^\infty f (j) 
$$
\end{example}


\subsection{(countable) linearity of integral}
\begin{corollary}
$$f,g \in L^+(\mu) \quad \Longrightarrow \quad \int(f+g) = \int f + \int g$$
\end{corollary}
\begin{proof}
使用 approximation by simple functions 以及 MCT.
取 $$\phi_n \nearrow f, \quad \psi_n \nearrow g$$, 从而 $$\phi_n + \psi_n \nearrow f+g$$, 从而我们有 $$ \int (f+g)  \overset{MCT}{=}  \lim_{n} \int (\phi_n + \psi_n)$$
从而由 simple function 的 Linearity 得到:$$ \int (f+g) = \lim_n\int \phi_n + \lim_n \int \psi_n$$
并且由于 $$ \int \phi_n \nearrow \int f, \int \psi_n \nearrow \int g$$
我们得到:
$$
\int (f+g) \geq \int f + \int g
$$
另一方向 trivial.
\end{proof}


\begin{remark}
    由此可见, \begin{center}
\textbf{$f\mapsto \int f$ 是 $\mathbb{R}$-linear 的映射.}
    \end{center}
\end{remark}

\subsection{Tonelli for sum and integrals}
\begin{corollary}{Tonelli for sum and integrals}
\label{Tonelli for sum and integrals}
for $\{f_i\}_{i\in\mathbb{N}}$ in $L^+(\mu)$, 有:
$$
\int \sum_{i=1}^\infty f_i = \sum_{i=1}^\infty \int f_i
$$
\end{corollary}\begin{proof}
Apply MCT to 
$$
g_n = \sum_{i=1}^n f_i
$$ 可得证.
\end{proof}
\begin{remark}
    这是 linearity of integral 的 countable version.\\
    由此可见 MCT 的用处很大.
\end{remark}