\chapter{Tonelli's Thm [Fol 2.5]}

 我们将 focus on the case $n=2$: $(X,\mathcal{A}, \mu)$, $(Y, \mathcal{B}, \nu)$, 考虑 \[
 (X \times Y, \mathcal{A} \otimes \mathcal{B}, \mu \times \nu)
 \]
 从而, 它可以推广到任何 finite 个 measure space 的 product 上.

\section{$E\subset X \times Y $ 的 section }
\begin{definition}{$x$-section, $y$-section}
给定 product space 上的集合 $E \subset X \times Y$, 对于 $x \in X$, $y\in Y$, 我们定义:
\[E_x : = \{  y\in Y \mid (x,y) \in E \}\]
\[
E^y := \{x\in X \mid (x,y) \in Y  \}
\]
\pic[0.3]{assets/ch2-pics-image-20250220190452148.png}
给定从 product space 出发的函数 $f: X \times Y \to \mathbb{C}$, 对于 $x \in X$, $y\in Y$, 我们定义: \[
f_x (y) := f^y(x) := f(x,y)
\]
表示固定住一个变量, 另一个变量的变化.
\end{definition}

\begin{example}
对于任意的 $E \subset X \times Y$如果定义: \[
    f:= \chi_E
    \]
那么有: \[
f_x = \chi_{E_x},\quad f^y = \chi_{E^y}
\]
对于 \textbf{rectangle:} $E = A \times B $, $A \in \mathcal{A}, B \in \mathcal{B}$, 有 $$E_x = \begin{cases}
        \varnothing ,\quad x\not\in A \\
        B ,\quad x\in A
    \end{cases}$$
\end{example}



\begin{proposition}
(a) $$E \in \mathcal{A} \otimes \mathcal{B} \implies \begin{cases}
        E_x \in \mathcal{B},\quad \forall x \in X\\
        E^y \in \mathcal{A}\quad, \forall y \in Y
    \end{cases}$$ (b) $$ f \text{ is } \mathcal{A} \otimes \mathcal{B}\text{-measurable} \implies \begin{cases}
        f_x \text{ is } \mathcal{B}\text{-measurable} \;\;\forall x \\
         f^y \text{ is }  \mathcal{A}\text{-measurable} \;\;\forall y
    \end{cases}$$
\end{proposition}

\begin{proof}
    (a) Let $$\mathcal{E}:= \{    E \subset X \times Y \mid  E_x \in \mathcal{B}\;\; \forall x \in X \quad\text{ and }\quad E^y \in \mathcal{A}\;\;\forall y \in Y  \}$$
Claim: $\mathcal{E}$ 包含了所有的 rectangles, 并且 $\mathcal{E}$ a $\sigma$-algebra.\\
容易证明这一点. 从而, 由 $\mathcal{A} \otimes \mathcal{B}$ 的定义 (为包含所有 rectangles 的最小 $\sigma$-algebra) 得  $\mathcal{A} \otimes \mathcal{B} \subset \mathcal{E}$, 从而 (a) 成立\\
并且由于 (check) $f_x^{-1} (V) = (f^{-1}(V) )_x$ (Similar for $f_y$),  (a)$\implies$(b).
\end{proof}
\begin{remark}
    这里三件需要记住的事情:
    \begin{itemize}
        \item \textbf{section 和取 preimage 可以交换顺序}
        \item 对于一个 product measurable set, 任意 \textbf{section 也 measurable}
        \item 对于一个 product measurable function, 任意 \textbf{section function 也 measurable}
    \end{itemize}
\end{remark}
\begin{remark}
    记录一下这里的证明方法, 以前见的比较少. 这里\textbf{证明条件 A 推出条件 B} 的方法: \textbf{证明所有满足条件 B 的元素构成的集合 包含了 所有满足条件 A 得元素构成的集合.} \\
    这一方法的好处在于: 可以运用所有满足条件 B 的元素构成的集合的整体性质, 比如是 $\sigma$-algebra 等.
\end{remark}


\begin{definition}{monotone class}
Given a set $X$, a collection $C \subset \mathcal{P}(X)$ is called a monotone class, if it is closed under \textbf{countable increasing unions} and \textbf{countable decreasing intersections}
\end{definition}
当然, 一个 $\sigma$-algebra 是一个 monotone class. monotone class 是一个比 $\sigma$-algebra 更弱的定义.

\section{tool needed to show Tonelli: Monotone Class Lemma}
\begin{lemma}{monotone class lemma}
Let $\mathcal{A} \subset \mathcal{P}(X)$ be an algebra.\\ 
Define $\mathcal{C}$ 为包含 $\mathcal{A}$ 的最小的 montone class.\\
Claim: \[
<\mathcal{A}> = \mathcal{C}
\]
\end{lemma}
\begin{proof}
    $\mathcal{C} \subset <\mathcal{A}>$ is trivial.\\
    $<\mathcal{A}> \subset \mathcal{C}$: STS $\mathcal{C}$ 是一个 $\sigma$-algebra. see p.66. 具体做法比较 tricky, 但是思路是先证明 $\mathcal{C}$ 是一个 algebra (这一部分较难. 我们对于 $E \in \mathcal{C}$, define $\mathcal{C}(E)$ 为 $\mathcal{C}$ 中所有和它的交和差也仍然在 $\mathcal{C}$ 中的 $F$ 构成的集合, 并发现这个子集 $\mathcal{C}(E)$ 也同样是一个 monotone class. 从而 $\mathcal{C} = \mathcal{C}(E)$ );
    
然后对于任意的 seq, 其 前 $n$ 项的 finite union $(\cup_{i=1}^nE_i)$seq 是一个 increasing seq, 其 limit 等于原 seq limit, 是属于 $\mathcal{C}$ 的.
\end{proof}
\begin{remark}
    即, 对于一个已经是 algebra 的集合, 它生成的 monotone class 等于它生成的 $\sigma$-algebra.\\
    这个 lemma 的好处在于, \textbf{我们在证明了一个集合是 algebra 后, 证明它是一个 $\sigma$-algebra, 只需要证明它 closed under ctbl increasing union 和 decreasing interection 即可. 我们可以利用这种 set seq 的单调性.}
\end{remark}

但是如果我们只知道 $\mathcal{C} \supset \mathcal{A}$, 没有 "包含 $\mathcal{A}$ 的最小的 monotone class" 这个条件怎么办? 那也没关系, 我们很自然得出  
\begin{corollary}
    Let $\mathcal{A} \subset \mathcal{P}(X)$ be an algebra, $\mathcal{C} \supset \mathcal{A}$ be a monotone class, 那么一定有 \[
    \mathcal{C} \supset \big \langle \mathcal{A}\big \rangle
    \]
\end{corollary}
因为 $\big \langle \mathcal{A}\big \rangle  =$ "包含 $\mathcal{A}$ 的最小的 monotone class" $\subset \mathcal{C}$.


\section{Tonelli for sets: integrating a section to get product measure}
\begin{theorem}{Tonelli for sets}
Let $(X,\mathcal{A}, \mu)$, $(Y, \mathcal{B}, \nu)$ be\textbf{ $\sigma$-finite} measure spaces.\\
Take $E \in \mathcal{A} \otimes \mathcal{B}$. Then:  \[
x \mapsto \nu(E_x), y \mapsto \mu(E_y) \text{ are \textbf{measurable} }
\] 并且 \[
(\mu \times \nu)(E) = \int \nu(E_x)  \; d\mu(x) = \int \mu(E^y) \; d \nu(y)
\]
\end{theorem}
\begin{remark}
这个定理说明的是: 在 $\sigma$-finite 的 measure spaces 构成的 product measure space 中, 任意 product measurable set $E$, 把 $x$ 映射到 $E$ 的 $x$-section 的 measure (\textbf{"扫描" 这个集合在一个方向上的宽度变化) 的行为是可测的.}\\
并且, 我们可以把 $E$ 的 measure 用 \textbf{``对每个 $x$, 在 $Y$ 上取 $E$ 的 $x$-section 的 measure, 并对这一行为在 $X$ 上进行积分"}来刻画. 这就把 product measure 拆分了开来. 其 following 是 Tonelli \textbf{``把 $n$ 个 measure spaces 的 product 上的积分拆成 $n$ 个积分}" 的强大定理.
\pic[0.3]{assets/ch2-pics-image-20250225215218999.png}
\end{remark}
\begin{proof}
Define:    $$\mathcal{C} = \{ E \subset X \times Y \mid x \mapsto \nu(E_x), y \mapsto \mu(E_y) \text{ are measurable } \forall (x,y)\in E \text{ and} \cdots (2)  \}$$
\textbf{Claim 1: }$\mathcal{C}$\textbf{ contains 所有的 rectangles.}
\textbf{Proof of Claim 1:} 考虑 $E = A\times B \in \mathcal{A}\times \mathcal{B}$, 即为一个 rectangle. 上一 lec 中, 我们 by def confirm: $\mathcal{A}\times \mathcal{B}\subset  \mathcal{A}\otimes \mathcal{B} $.\\
那么对于任意的 $(x,y) \in E$, 我们有: \(\nu(E_x) =  \nu(B)\), 对于所有的 $(x,y) \not \in E$, 则有 \(\nu(E_x) =  \varnothing\).\\
所以对于任意的 $x$, $\nu(E_x) =  \chi_A(x) \nu(B)$, 同理 $\mu(E^y) = \chi_A(x) \mu (A)$. \\
由此得到 $x\mapsto \nu(E_x), y\mapsto \mu(E^y)$ 是 measurable 的, 并且 \[
\mu \times \nu (E) = \mu(A) \times \nu(B) = \mu(A) \int \chi_B(y) \;d \nu(y) = \int \mu(E^y) \; d  \nu(y) 
\]
同理, \(\mu\times \nu (E) = \int \nu (E_x) \; d\mu(x)   \). 从而得证. 从而, \textbf{对于任意 union of finite disjoint rectangles, 这个结论也成立}, by additivity in definition. 因而 $\mathcal{C}$ \textbf{为一个 algebra.}\\\\
Note: 由于 $\mathcal{A} \otimes \mathcal{B}$ 为包含所有 rectangles 的最小 $\sigma$-algebra, 我们\textbf{只需要证明 $\mathcal{C}$ 为一个 $\sigma$-algebra}, 那么它一定包含 $\mathcal{A} \otimes \mathcal{B}$. 并且 by Monotone Class Lemma, \textbf{STS: $\mathcal{C}$ 为一个 monotone class.}\\\\
\textbf{Claim 2: $\mathcal{C}$ 为一个 monotone class.}
令 $\{E_n\}$ 为一个 increasing seq in $\mathcal{C}$, 定义其 union 为 $E := \bigcup_{n=1}^\infty E_n$. 并定义: \[f_n(y) := \mu((E_n)^y)\]
根据 $\mathcal{C}$ 的 definition, 每个 $f_n$ 都是 measurable 的, 并且我们容易证明: \[f_n \nearrow f(y) := \mu((E)^y) \text{ ptwise.}\]于是使用 MCT, 容易得到 \[
\int \mu(E^y) \; d\nu(y) = \lim \int \mu((E_n)^y) \; d\nu(y) = \lim \mu \times \nu(E_n) \overset{\text{CFB}}{=} \mu \times \nu (E)
\]
从而 $E \in \mathcal{C}$. \\
It remains to show: $\mathcal{C}$ closed under ctbl decreasing intersection. 不过这里我们涉及到一个 decreasing sequence 中间突然从 infinite measure 变为 finite measure 的问题, 所以我们从这里开始要分 $\mu ,\nu$ finite 和 not finite (but still $\sigma$-finite) 的两种情况讨论. finite measure 不用担心上述这一问题.\\
\textbf{Case 1:  $\mu ,\nu$ finite}, 于是令 $\{E_n\}$ 为一个 decreasing seq in $\mathcal{C}$, 和 increasing 的情况 similar, 得到 \(\mu((E_n)^y)=: f_n  \searrow f(y) := \mu((E)^y) \text{ ptwise.}\), 从而  by DCT (取 $\mu(X)$ 为 donimating function), 得到 \[
\int \mu(E^y) \; d\nu(y) = \lim \int \mu((E_n)^y) \; d\nu(y) = \lim \mu \times \nu(E_n) \overset{\text{CFA}}{=} \mu \times \nu (E)
\] 从而我们证明了在  $\mu,\nu$ 为 finite measure 的情况下, $\mathcal{C}$ 为一个 monotone class, 从而为一个 $\sigma$-algebra, 从而 $\mathcal{M} \otimes \mathcal{N} \subset \mathcal{C}$. \\\\
\textbf{Case 2: $\mu,\nu$ $\sigma$-finite measure}: 我们可以把 $X \times Y$ 写作 union of a seq of finite measure sets $\{ X_i \times Y_i\}_{i\in \mathbb{N}}$, 从而也是 a union of increaasing seq of finite measure sets $\{ X_j \times Y_j\}_{j\in \mathbb{N}}$. (取 $X_j \times Y_j  = \bigcup_{i=1}^j X_j \times Y_j$) 从而对于任意的 $E \in \mathcal{M}\otimes \mathcal{N}$, $$E = \lim_{j\to\infty} (E \cap (X_j\times Y_j))$$
对于每个 $E \cap (X_j\times Y_j)$, 我们可以应用前一结论, 得到 \[
\mu \times \nu (E \cap (X_j\times Y_j)) = \int \chi_{Y_j}(y) \mu(E^y \cap X_j) \; d\nu(y)
\]
从而\textbf{应用 MCT}, 得到 \[
\mu \times \nu (E ) = \int \mu(E^y) \; d\nu(y)
\]
 同理 \(\mu \times \nu (E ) = \int \mu(E_x) \; d\mu(x)\), 从而 $E \in \mathcal{C}$, 得证.
\end{proof}


\section{Tonelli's Theorem}

\begin{theorem}{Tonelli}
 Let $(X,\mathcal{A}, \mu)$, $(Y, \mathcal{B}, \nu)$ be $\sigma$-finite measure spaces.\\
条件: 令 $f \in L^+ (X \times Y)$,\\
结论:$$g(x) := \int f(x,y) \; d\nu\; \in L^+(X)\quad h(y) := \int f(x,y) \; d\mu \; \in L^+(Y) $$ (显然) 并且
\begin{align}
    \int f \; d(\mu\times \nu) &= \int \Big[   \int f(x,y) \; d\nu(y)  \Big] d\mu(x)\\
    &=  \int \Big[   \int f(x,y) \; d\mu(x)  \Big] d\nu(y)
\end{align}
\end{theorem}\begin{remark}
    Tonelli for sets 表示了 product measure 的计算方式: 通过对 $x$-section 的 $y$ measure, 在 $x\in X$ 上进行积分可得到. 这就把 product measure 拆成了单个 measure 与积分.\\
    而 Tonelli's Theorem 表示\textbf{对一个非负 product measurable function 积分可以转化成对逐个 measure 积分.} \\
recall: 非负 measurable 函数的积分, 就是一个 seq of simple functions 的积分的 sup, 而 simple function 的积分, 就是\textbf{几个 measurable set 的 measure 的加权和}.
因而 Tonelli's Theorem 基本上 naturally follows from Tonelli for sets.
\end{remark}
\begin{proof}
首先,\textbf{ 对于 $f$ 是 simple function 的 case, 直接 follows from Tonelli for sets.} (mentioned in remark.)\\
对于 general case: $f \in L^+(X\times Y)$, 令 $\{f_n\}$ 为一个 seq of simple functions ptwisely converging to $f$.\\
于是 \[
\int g \; d\mu = \lim \int g_n \; d\mu  = \lim \int f_n \;  d(\mu\times \nu) = \int f \; d(\mu\times \nu)
\]
\[
\int h \; d\mu = \lim \int h_n \; d\mu  = \lim \int f_n \;  d(\mu\times \nu) = \int f \; d(\mu\times \nu)
\]
by MCT.
\end{proof}