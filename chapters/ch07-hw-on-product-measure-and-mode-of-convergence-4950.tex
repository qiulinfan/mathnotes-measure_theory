\chapter{on product measure and mode of convergence (49/50)}
\begin{center}
\textit{Some of the following questions will be graded. Do them, and do hand them in}.
\end{center}

\section{Order of integration: $ \int_0^\infty\int_x^\infty e^{-y^2/2}\;d y \;d x=1$}
  Use Tonelli's Theorem and 1-variable calculus to give a rigorous proof for the equality
  \[
    \int_0^\infty\int_x^\infty e^{-y^2/2}\;d y \;d x=1
  \]
\begin{proof}
 Define \[
 f(x,y) := \begin{cases}
   e^{-y^2/2}, & \text{if } 0 \le x \le y,\\
   0, & \text{otherwise}.
   \end{cases}
 \] Then we have \[
     \int_0^\infty\int_x^\infty e^{-y^2/2}\;d y \;d x=   \int \Big[   \int f(x,y) \; d m(y)  \Big]  \; dm(x) 
 \]
 Since $ f (x,y) = e^{-y^2/2}$ is \textbf{nonnegative} and \textbf{continuous}, it is measurable and thus in $L^+(X \times Y)$, where $X = Y = (\mathbb{R},\mathcal{L}, m)$ is $\sigma$-finite.\\
    Thus we can apply Tonelli's theorem: \begin{align}
      \int \Big[   \int f(x,y) \; d m(y)  \Big]  \; dm(x) &= \int f \; d(m(x)  \times m(y))     \\
& =    \int \Big[   \int f(x,y) \; d m(x)  \Big]  \; dm(y) \\
& =  \int \Big[   \int f(x,y) \; d m(x)  \Big]  \; dm(y) \\
& = \int \Big[   \int e^{-y^2/2} \; d m(x)  \Big]  \; dm(y) 
    \end{align}Where\[
     \int e^{-y^2/2} \; d m(x)   = \int_{[0,y]} e^{-y^2/2} \; dx =  ye^{-y^2/2}
    \]
Thus  \begin{align}
      \int \Big[   \int f(x,y) \; d m(y)  \Big]  \; dm(x) &=\int \Big[   \int e^{-y^2/2} \; d m(x)  \Big]  \; dm(y) \\
      &= \int ye^{-y^2/2} \; dm(y) \\
      & = \int_{[0,\infty)} ye^{-y^2/2} \; dy 
\end{align}
Make the substitution \(t = \frac{y^2}{2}\), then we have
\[
\int_0^\infty y\, e^{-y^2/2}\,dy =
\int_0^\infty e^{-t}\,dt =
\Bigl[ - e^{-t} \Bigr]_{0}^{\infty} = 1
\]
This finishes the proof that \[
 \int_0^\infty\int_x^\infty e^{-y^2/2}\;d y \;d x = 1
\]
\end{proof}




\section{integration of a function $=$ Area under the curve}
  Let $(X,\mathcal{A},\mu)$ be a $\sigma$-finite measure space, and let $f\in L^+(X)$. Consider the subset  $G_f\subset X\times[0,\infty)$ consisting of all points $(x,y)$ with $y<f(x)$.
  \begin{itemize}
  \item[(a)]Prove that $G_f$ is $\mathcal{A}\otimes\mathcal{B}_\mathbb{R}$-measurable.
  \item[(b)]Prove that $(\mu\otimes m)(G_f)=\int f\; d\mu$.
  \end{itemize}  
\begin{remark}
    这个 $G_f$ 即为 $f:X \to \mathbb{R}$ 的 graph 下的 area, 
\end{remark}
\begin{proof}
    \textbf{of 2(a):}\\
\[y < f(x)\quad\Longleftrightarrow\quad
\exists\,q\in\mathbb{Q},\; y < q < f(x)\]
Hence
\[G_f =
\bigcup_{q\in\mathbb{Q},\,q>0}\Bigl(\{x : f(x) > q\}\times \{y : y < q\}\Bigr)
\]
Since \(\{x : f(x) > q\}\in \mathcal{A}\) (by the measurability of \(f\)) and \(\{y : y < q\}\in \mathcal{B}_\mathbb{R}\), each set in the union is a measurable rectangle, thus measurable in the product measurable space \(X \times \mathbb{R}\). Since a countable union of measurable sets is measurable in the product \(\sigma\)-algebra, 
We have \[G_f \in \mathcal{A}\otimes \mathcal{B}_\mathbb{R}\]
\end{proof}
\begin{proof}
    \textbf{of 2(b)}:\\
Since \(f \ge 0\), and \(\sigma\)-finiteness of $X$ is assumed, \(\sigma\)-finiteness of $Y$ is known, \\ 
we can apply Tonelli’s theorem to compute:
\begin{align}
    (\mu \otimes m)(G_f) &= 
\int_{X\times [0,\infty)} \chi_{G_f}(x,y)\, d(\mu\otimes m)\\
&= \int_X  \Big[ \int_ {[0,\infty)} \chi_{G_f}(x,y)\, d m(y)\Big] \;d \mu(x)
\end{align}
By definition of \(G_f\), \(\chi_{G_f}(x,y) = 1\) if and only if \(y < f(x)\), and \(0\) otherwise. Hence, for each fixed \(x\),
\[
\int_{[0,\infty)} \chi_{G_f}(x,y)\,dm(y) =
\int_{[0,\infty)} \chi_{\{ y < f(x)\}}\,dm(y) = 
\begin{cases}
f(x), & \text{if } f(x) < \infty,\\
\infty, & \text{if } f(x) = \infty
\end{cases}
\]
Therefore
\[
\int_{[0,\infty)} \chi_{G_f}(x,y)\,dm(y) = f(x) \;\; a.e.
\]
Applying Tonelli’s theorem again yields
\[
(\mu \otimes m)(G_f) = \int_X  \Big[ \int_ {[0,\infty)} \chi_{G_f}(x,y)\, d m(y)\Big] \;d \mu(x) = 
\int_X f(x)\, d\mu(x)
\]
Thus we conclude that
\[
(\mu \otimes m)(G_f) = 
\int_X f \, d\mu \]
\end{proof}




\section{Oscillations:  $f_n(x)=(\sin(\pi n x))^n \to f = 0$ in measure}
  Consider the sequence $f_n(x)=(\sin(\pi n x))^n$, $n=1,2,\dots$, on the interval $[0,1]$. Prove that there exists a set $E\subset[0,1]$ such that $m(E^c)\le 2^{-597}$ and a sequence $1\le n_1<n_2<\dots$ such that $|f_{n_j}(x)|\le j^{-597}$ for all $x\in E$ and all $j\ge1$. \textit{Hint}: use E. Consider convergence in measure
\begin{proof}
\textbf{Claim 1: It suffices to show that $f_n $ converges in measure.}\\
Proof of Claim 1: Suppose $f_n $ converges in measure to $f = 0$, then by Folland 2.30, there exists a subseq $(f_{n_k}) \overset{k \to \infty}{\longrightarrow} f= 0$ a.e. .And since $[0,1]$ has \textbf{finite measure} $1$, \textbf{by Egoroff's Theorem},  for any $\epsilon>0$ there exists $E \subset [0,1]$ s.t. $\mu(E^c) < \epsilon$ and $(f_{n_k}) \overset{k \to \infty}{\longrightarrow} f= 0$ \textbf{uniformly} on $E$. \\
Then we take $\epsilon : = 2^{-597}$and coresponding $E$.\\
And for each $j \in \mathbb{N}$, we let $\delta_j = j^{-597}$. By the uniform convergence property of $(f_{n_k})$, we can take $N_j$ s.t. $|f_{n_k}(x)| < \delta_j $ for all $x \in E$ whenever $n_k \geq N_j$.\\
Therefore, $E$ and the sequence $(f_{N_j})$  satisfty the requirements in the context.\\
This shows that, \textbf{as long as we can show $(f_n)$ converges in measure} to $f = 0$, the statement is proved.\\\\
Let $f_n (x) : = \sin( n\pi x)^n $ for $n\in \mathbb{N}$.\\
\textbf{Claim 2:} $f_n $ \textbf{converges in measure.}\\
Proof of Claim 2: The idea is that the exponent $n$ makes the sequence converge faster than the linear growth of $nx$ that shortens a period and messes up the sin values.\\ 
Fix $\epsilon  > 0$. (WLOG $\epsilon < 1$.) WTS: \[
m(\{ x: |\sin (n \pi x) \geq \epsilon^{1/n}  \}) \to 0 \quad \text{ as } n\to \infty
\]
We know that $\sin(n\pi x) = 1$ iff $x = \frac{2k-1}{2n}$ for some $k = 0,\cdots, 2n-1$.
Consider $x \in [0,\frac{1}{2n})$, let $|\sin(n \pi x_0 ) | : = \epsilon^{1/n}$.\\
Denote \[
\delta_n :=\Big |\frac{1}{2n} - x_0 \Big| 
\]
Then we can express the measure as: \[
m(\{ x: |\sin (n \pi x) \geq \epsilon^{1/n}  \}) = 2n  \delta_n
\]
Notice that by the monotonicity of arcsin function, we can solve for $x_0$ as:\[
x_0  =\frac{1}{n\pi} \arcsin (\epsilon^\frac{1}{n})
\]
Thus \[
\delta_n = \frac{1}{2n} = \frac{1}{n\pi} \arcsin (\epsilon^\frac{1}{n})
\]
Thus  \begin{align}
\lim_{n\to \infty}m(\{ x: |\sin (n \pi x) \geq \epsilon^{1/n}  \}) &=  \lim_{n\to\infty} 2n  \delta_n     \\
&= 1-  \lim_{n\to\infty}  \frac{2}{\pi}\arcsin (\epsilon^\frac{1}{n}) \\
&= 1-\frac{2}{\pi} \cdot \frac{\pi}{2} \\
&=0
\end{align}
Since $\epsilon$ is arbitrary, this finishes the proof that $f_n \to f = 0$ in measure.\\
Thus combining Claim 1, the whole statement is proved.
\end{proof}





\section{Indicator functions 是 $L^+$ 的一个 closed subset}
  Let $(X,\mathcal{A},\mu)$ be any measure space. Let $M\subset L^+$ be the set of indicator functions $\chi_E$, where $E\in\mathcal{A}$ and $\mu(E)<\infty$. Prove that $M$ is a closed subset of $L^1$. In other words, prove that $M\subset L^1$, and that if $f_n\in M$, $f\in L^1$, and $\int|f_n-f|\to0$, then $f\in M$.

\begin{proof}
Let $(f_n := \chi_{E_n})_{n\in\mathbb{N}}$ be a seq of indicator functions in $L^+$ s.t. $\int|f_n-f|\to0$ for some $f\in L^1$.  \\
Define for all $k\in \mathbb{N}$ $$
A_k :=\{x\colon |f(x)|>\frac{1}{k},|f(x)-1|>\frac{1}{k}\}
$$
Fix one $k\in\mathbb{N}$, bt monotonicity of integration in $L^1$, we have
$$\int|f-\chi_{E_n}|\geq \int_{A_k}|f-\chi_{E_n}|\geq   \int_{A_k} \frac{1}{k}   \geq     \frac{\mu(A_k)}{k}$$
Thus \[
\mu(A_k) \leq k \int |f-\chi_{E_n}|
\]
Since $\chi_{E_n}\to f$ in $L^1$, it follows that $\mu(A_k)=0$.\\
Since $A_k$ is arbitrary, by ctbl sub additivity, \[
\mu(\bigcup_{k=1}^{\infty}A_k) \leq \sum_{k=1}^{\infty}\mu(A_k)= 0 
\]
Define \[
A:=\{x\colon f(x)\not=0,1\}
\]
By the definition of $A_k$, we have the equality: \[
A=\bigcup_{k=1}^{\infty}A_k
\]
Thus $\mu(A) = 0$, which means that $f(x)\in\{0,1\}$ a.e., showing that $f$ is a.e. an indicator function, in the same equivalence class of some indicator function in $L^1$, thus we have $f \in M \subset L^1$. This finishes the proof that $M$ is a closed subset of $L^1$.
\end{proof}





\section{a complete metric space of measurable functions (other then $L^1(\mu)$) }
  Suppose that $(X,\mathcal{A},\mu)$ is a measure space such that $\mu(X)<\infty$. Set $\chi(t)=\frac{t}{1+t}$ for $t\ge 0$.\\
  Given measurable functions $f,g\colon X\to\mathbb{C}$, set\[
    \rho(f,g):=\int\chi(|f-g|)\; d\mu
  \]
  \begin{itemize}
  \item[(a)]Prove that $\rho$ induces a metric, also denoted $\rho$, on the space \[
      L:=\{f\colon X\to\mathbb{C}\ \text{measurable}\}/\!\!\sim,
    \]
    where $f\sim g$ iff $f=g$ a.e. \textit{Hint}: prove that $\chi(s+t)\le\chi(s)+\chi(t)$ for $s,t\ge0$.
  \item[(b)]Prove that if $f_n,f\in L$, then $\rho(f_n,f)\to0$ iff $f_n\to f$ in measure.
  \item[(c)] Prove that $(L,\rho)$ is a complete metric space.
  \end{itemize}    
\begin{remark}





\textbf{对于任何 measure $\mu$, $L^1(\mu)$ 都是一个 complete metric space (因为它是 Banach space)}; 这里, 我们略微修改了 $L^1(\mu)$ 的 metric, 嵌套了一个函数, 但是它\textbf{仍然是一个 complete metric space.}
\end{remark}
\begin{proof}
   \textbf{ of 5(a):}
$\chi(t)=\frac{t}{1+t} = 1- \frac{1}{1+t}$ is an increasing function on $t \geq 0$.\\
\textbf{Claim: for all $s,t \geq 0$, we have $ \chi(s) + \chi(t)  \leq \chi(s+t)$.}\\
\textbf{Proof of claim: }\\
Let $s,t \geq 0$, we have
\[
   \chi(s) + \chi(t) =
   \frac{s}{1+s} \;+\; \frac{t}{1+t} =
   \frac{s(1+t) + t(1+s)}{(1+s)(1+t)} =
   \frac{s + st + t + ts}{(1+s)(1+t)} = 
   \frac{s + t + 2st}{(1+s)(1+t)}
\]
while
\[\chi(s+t) =
   \frac{s+t}{1 + s + t}
\]
Note
\begin{align}
        (s+t)(1+s)(1+t)  = (s+t)(1 + s + t + st) &=  s + t + s^2 + 2st + t^2 + s^2t + st^2\\
          (s+t + 2st)(1 + s + t) &=s + t + s^2 + 4st+ t^2 + 2s^2t + 2st^2
\end{align}
We have: \[
  (s+t)(1+s)(1+t) \leq   (s+t + 2st)(1 + s + t)
\]
Since $(1 + s + t)$ and $(1+s)(1+t)$ are positive, we can rearrange the ineq to be  \[
   \frac{s+t}{1+s+t} \leq
   \frac{s+t+2st}{(1+s)(1+t)}
\] which is exactly \[   \chi(s) + \chi(t)  \leq \chi(s+t)\]
as needed.\\\\
First, $\rho$ is a well-defined function on the quotient set, since if \(f \sim g\) and \(f' \sim g'\) then \(|f-g| = |f'-g'|\) a.e. Consequently,
\[
   \chi\bigl(|f-g|\bigr) \;=\; \chi\bigl(|f' - g'|\bigr)
   \quad \text{a.e.}
\]and hence \[
   \int_X \chi\bigl(|f-g|\bigr)\,d\mu  = 
   \int_X \chi\bigl(|f' - g'|\bigr)\,d\mu
\]
Now we prove that \(\rho\) is a metric:
\begin{itemize}
    \item \textbf{Nonnegativity}: \(\rho(f,g) \ge 0\) is immediate since \(\chi(\cdot)\ge0\) and \(\mu\) is a measure; and since $\chi(h) = 0 $ iff $h=0$ a.e., we have \(\rho(f,g) = 0\) iff $f = g$ a.e., that is, $f = g \in L^1(\mu)$
    \item \textbf{Symmetry}: \(\rho(f,g)=\rho(g,f)\) follows immediately from \(\chi(|f-g|) = \chi(|g-f|)\).
    \item \textbf{Triangle inequality}:    For any three functions \(f,g,h\), we have pointwise \[
     |f(x)-h(x)| 
     \;\le\;
     |f(x)-g(x)| + |g(x)-h(x)|.
   \]
   Then applying the subadditivity of \(\chi\) proved above, we have: \[
     \chi\bigl(|f(x)-h(x)|\bigr)  \leq 
     \chi\bigl(|f(x)-g(x)| + |g(x)-h(x)|\bigr) \leq 
     \chi\bigl(|f(x)-g(x)|\bigr)  +
     \chi\bigl(|g(x)-h(x)|\bigr)
   \]
   Integrating both sides over \(X\) gives  \[
     \rho(f,h)  =
     \int_X \chi\bigl(|f-h|\bigr)\,d\mu \leq 
     \int_X \chi\bigl(|f-g|\bigr)\,d\mu  +
     \int_X \chi\bigl(|g-h|\bigr)\,d\mu =
     \rho(f,g) +\rho(g,h)
   \]

\end{itemize} 
Therefore, \(\rho\) is a metric on \( L \;=\; \{f \colon X\to \mathbb{C} \text{ measurable} \}/\!\!\sim\) as desired.
\end{proof}


\begin{proof}
   \textbf{ of 5(b)}: \\
\textbf{Claim 1: \(\rho(f_n,f)\to 0\) \(\implies\) \(f_n\to f\) in measure}\\
Suppose \(\rho(f_n,f)\to 0\).  Let $\epsilon > 0$.\\
Since \(\chi(t)=\tfrac{t}{1+t}\) is\textbf{ strictly increasing} in \(t\): \[
    |f_n-f|> \epsilon \Longleftrightarrow 
    \chi\bigl(|f_n-f|\bigr) \;>\; \chi(\epsilon)\;=\;\frac{\epsilon}{1+\epsilon}
  \]
Hence \[
    \{|f_n-f| > \epsilon\} = \bigl\{\chi(|f_n-f|) > \tfrac{\epsilon}{1+\epsilon}\bigr\}
  \]
Since the function is nonnegative, by Chebyshev:\[
 \mu(  \{|f_n-f| > \epsilon\} ) =    \mu\Bigl(\bigl\{\chi(|f_n-f|)>\tfrac{\epsilon}{1+\epsilon}\bigr\}\Bigr) \leq 
    \frac{1}{\,\tfrac{\epsilon}{1+\epsilon}\,}
    \int\chi\bigl(|f_n-f|\bigr)\,d\mu =  \frac{\rho(f_n,f)}{\chi(\epsilon)}
  \]
By assumption, $\rho(f_n,f) \to 0$, thus \[
    \mu\bigl(\{|f_n-f|>\epsilon\}\bigr) \leq 
    \frac{\rho(f_n,f)}{\chi(\epsilon)}\longrightarrow\;0
  \]
Since $\epsilon$ is arbitrary, it proves that \(f_n\to f\) in measure.\\\\

\textbf{Claim 2: \(f_n\to f\) in measure \(\implies\) \(\rho(f_n,f)\to0\)}\\
Now assume \(f_n\to f\) in measure. \\
Let $\delta > 0$.\\
Observe that for any \(\epsilon>0\): 
\begin{itemize}
    \item \(|f_n-f|\le \epsilon \implies  \tfrac{|f_n-f|}{1+|f_n-f|}\le \tfrac{\epsilon}{1+\epsilon}\).
    \item \(|f_n-f|\ge \epsilon \implies  \tfrac{|f_n-f|}{1+|f_n-f|}\leq 1 \)
\end{itemize}
Hence by choosing any arbitrary $\epsilon$, we can bound the integral by: \[
  0  \leq  \int_X \frac{|f_n-f|}{1+|f_n-f|}\,d\mu \leq
  \int_{\{|f_n-f|\le \epsilon\}} \frac{\epsilon}{1+\epsilon}\, d\mu +
  \int_{\{|f_n-f|> \epsilon\}} 1 \,d\mu
\]
For the first term:
\[
  \int_{\{|f_n-f|\le \epsilon\}} \frac{\epsilon}{1+\epsilon}\, d\mu = 
  \frac{\epsilon}{1+\epsilon}\;\mu\bigl(\{|f_n-f|\le \epsilon\}\bigr) \leq
  \frac{\epsilon}{1+\epsilon}\;\mu(X)
\]
Because \(\mu(X)\) is finite, we can choose $\epsilon$ s.t. \(\tfrac{\epsilon}{1+\epsilon}\mu(X) < \delta / 2\).\\
Once \(\epsilon\) is fixed, by convergence in measure there exists \(N\) such that for all \(n\ge N\),
\[
  \mu\bigl(\{|f_n-f|> \epsilon\}\bigr) < \delta /2
\]
Then for any $n\geq N$, we have:
\[
  \rho(f_n,f) = 
  \int_X \chi(|f_n-f|)\,d\mu \leq
  \mu(X)\,\frac{\epsilon}{1+\epsilon}  +
  \mu\bigl(\{|f_n-f|> \epsilon\}\bigr) < \delta
\]
 Hence \[\rho(f_n,f)\overset{n\to \infty}{\longrightarrow} 0\]
\end{proof}


\begin{proof}
   \textbf{ of 5(c): }\\
   Suppose $(f_n)$ is a Cauchy seq in $(L,\rho)$, i.e. for any $\epsilon > 0$, exists some $N>0$ s.t. \(\rho(f_m, f_n)  < \epsilon\) whenever $n,m \geq N$.\\
   WTS: $(f_n)$ converges, i.e. $\rho(f_n,f)\to0$. \\
By (b) we know \textbf{it suffices to show that $f_n\to f$ in measure}.\\
And by Folland 2.30, \textbf{ STS: $(f_n)$ is Cachy in measure}.\\
Let $\epsilon > 0$. Let $\delta > 0$.\\
by Chebyshev:\[
 \mu(  \{|f_n-f_m| > \epsilon\} ) =    \mu\Bigl(\bigl\{\chi(|f_n-f_m|)>\tfrac{\epsilon}{1+\epsilon}\bigr\}\Bigr) \leq 
    \frac{1}{\,\tfrac{\epsilon}{1+\epsilon}\,}
    \int\chi\bigl(|f_n-f_m|\bigr)\,d\mu =  \frac{\rho(f_n,f_m)}{\chi(\epsilon)}
  \]
So since $(f_n)$ is a Cauchy, there exists $N > 0$ s.t. $\rho(f_n,f_m) < \chi(\epsilon) \delta$ whenever $n,m \geq N$, thus  $ \mu(  \{|f_n-f_m| > \epsilon\} ) \leq \delta$ whenever $m,n \geq N$.\\
This proves that $(f_n)$ is Cachy in measure, thus $f_n\to f$ in measure, and thus $(f_n)$ converges, showing that every Cachy seq converges in $(L,\rho)$. Therefore $(L,\rho)$ is a complete metric space. 
\end{proof}





\begin{center}
    Nur f\"ur Verr\"uckte (Only for nuts).
\end{center}

(It's \textbf{really} not necessary to attempt these problems. Do not, under any circumstances, hand them in!)
\begin{enumerate}
\item Prove that the category of measurable spaces (see HW1) admits finite products, and that the product of $(X,\mathcal{A})$ and $(Y,\mathcal{B})$ equals $(X\times Y,\mathcal{A}\otimes\mathcal{B})$. 
\item Now consider the category of measure spaces (see HW2). Consider two 
  measure spaces $(X_i,\mathcal{A}_i,\mu_i)$, $i=1,2$, and set $X=X_1\times X_2$, $\mathcal{A}=\mathcal{A}_1\otimes \mathcal{A}_2$, and $\mu=\mu_1\times\mu_2$.
  \begin{itemize}
  \item[(a)]Prove that the projection maps $X\to X_i$ are measurable, and that they are measure preserving iff $\mu_j(X_j)=1$ for $j=1,2$. Thus $(X,\mathcal{A},\mu)$ is \emph{not} the categorical product of $(X_i,\mathcal{A}_i,\mu_i)$ in general.
  \item[(b)]Prove that even if $\mu_i(X_i)=1$, the measure space $(X,\mathcal{A},\mu)$ is \emph{not} the categorical product of $(X_i,\mathcal{A}_i,\mu_i)$ in general.
    \textit{Hint}: consider the case when the $X_i$ consist of two elements, for example $X_i=\{\mathfrak{o}_i,\mathfrak{v}_i\}$.
  \end{itemize}
\end{enumerate}