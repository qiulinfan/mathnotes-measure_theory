\chapter{integration of real and complex functions-I [Fol 2.3]}

我们目前只定义了 non-negative $\overline{\mathbb{R}}$-valued measurable function 的积分, 而我们想要完整地定义: $\overline{\mathbb{R}}$-valued measurable function 的积分 $\int f \in \extR$, 以及 $\mathbb{C}$-valued  measurable function 的积分 $\int f \in \mathbb{C}$.

recall: 对于任意 $\extR$-valued $f$, \[f = f^+ - f^-\]

\textbf{因而我们希望 define:}\[ \int f = \int f^+ -\int f^-\]
但是其中有一个 undefined 的问题: 我们要避免 $\infty - \infty$ 这一类的问题. 因而我们无法对所有的可测函数进行积分, 而是定义 "integrable" 的可测函数.

\begin{lemma}
    \[
    \begin{cases}
        \int f^+ < \infty \\
        \int f^- < \infty
    \end{cases} \Longleftrightarrow \int|f| < \infty
    \]
\end{lemma}
\begin{proof}
    trivial.
\end{proof}
 正负部分都可控, 肯定是当且仅当绝对值函数可控.

 我们接下来将定义可积函数的空间是: 所有绝对值积分非无穷的函数. (怎么和预期不一样...这样的话这个空间在积分运算下的值域就是 $\mathbb{R}$ 而不是 $\extR$ 了. 我期待的是为了避免无穷之间相减的 undefined behavior 只需要正负部分有一个积分非无穷就行了. 但是我们要求的是都不是无穷.  不过既然这么定义了肯定有其道理.)




\section{$\tilde{L}(X, \mu, \mathbb{C})$ and $L^1(X, \mu, \mathbb{C}$)}
\begin{definition}{real-valued integrable function}
Given measure space $(X,\mathcal{M},\mu)$,  \textbf{measurable $f : X \rightarrow \extR$ 被称为 integrable} 的, 如果它满足 \[ \int |f| < \infty\] 并定义其 integral 为: \[\int f = \int f^+ - \int f^-\]
\end{definition}
\begin{definition}{complex-valued integrable function}
    Further, 我们定义 \textbf{measurable $f:X \rightarrow \mathbb{C}$ 是 integrable 的}, 如果它同样满足: \[\int |f| < \infty\]
\textbf{注意到这个条件等价于 \(\re f, \im f\) integrable, 因为}
\[|f| \leq |\re f| + |\im f| \leq 2|f|\]
我们定义其 integral 为: \[\int f = \int \re f+ i \int \im f\]
\end{definition}
\begin{remark}
    所以说,\textbf{ 实值函数的积分要计算两个, 复值函数的积分要计算四个}. (好麻烦.)
\end{remark}


\begin{proposition}
    所有的 real-valued integrable functions 构成一个 $\mathbb{R}$-vector space, 并且 integral 是一个 linear functional on it.
    
    所有的 complex-valued integrable functions 构成一个 $\mathbb{C}$-vector space, 并且 integral 是一个 linear functional on it.
\end{proposition}
\begin{proof}
    trivial.
\end{proof}
下面我们可以定义这个 vector space 并在上面进行一定研究. 此处为一个 temporary 的记号:
    
\begin{definition}{$\tilde{L}(X, \mu, \mathbb{R})$ 以及$\tilde{L}(X, \mu, \mathbb{C})$ space}
给定 measure space $(X, \mathcal{M},\mu)$
    我们定义 \[\tilde{L}(X,\mu, \mathbb{R}) := \{  \text{all (extended) real-valued integrable functions on } X\} \] 以及 \[\tilde{L}(X, \mu, \mathbb{C}) := \{  \text{all complex-valued integrable functions on } X\} \]
\end{definition}
\begin{remark}
    这基本接近我们最终的可积空间的定义了. 只需要再 quotient 掉所有的 a.e. 相等的函数就可以. 在此之间, 我们首先在这临时的空间上证明一些结论.

   \textbf{ 我们基本不使用 \(\tilde{L}(X,\mu, \mathbb{R})\), 因为它是 \(\tilde{L}(X,\mu, \mathbb{C})\) 的 subspace, 而且大部分结论基本都在更 general 的 \(\tilde{L}(X,\mu, \mathbb{C})\) 上成立.}
\end{remark}
\begin{remark}
    这个 $\mathbb{C}$-vector space 的 dimension 是多少呢: \\
    如果 $X$ 是一个 finite set, 那么 \(\tilde{L}(X,\mu, \mathbb{C})\)  的 dimension 是 $|X|$, 因为 $e_i : x_j \mapsto \delta_{ij}$ 是一个 basis; 同样的, 如果 $X$ countable, 那么 \(\tilde{L}(X,\mu, \mathbb{C})\) 的 dimension 也是 countably infinite 的; 如果 $X$ uncountable, 那么 \(\tilde{L}(X,\mu, \mathbb{C})\) 的 dimension 也是 uncountable 的.\\
    比如, \(\tilde{L}(\mathbb{R}^n,\mu, \mathbb{C})\) 的 dimension 就是 uncountable 的.
\end{remark}

\begin{proposition}
    \(\tilde{L}(X, \mu, \mathbb{C})\) 上, $f\mapsto \int f$ 为一个 linear functional.
\end{proposition}
因为积分是 linear 的, as we have proved.


\begin{proposition}
$$f \in \tilde{L}(X,\mu, \mathbb{C}) \implies |\int f| \leq \int |f|$$
\end{proposition}
\begin{proof}
    For real-valued case, $$ \Big| \int f  \Big| =\Big|\int f^+ - \int f^- \Big|  \leq \Big|\int f^+\Big| +  \Big|  \int f^- \Big|  = \int f^+  + \int f^- = \int |f|$$
For complex-valued case,
Set $$\alpha = \frac{\int f}{|\int f|}$$
    于是有 $\alpha \in \mathbb{C}$ 且 $|\alpha| = 1$. \textbf{Note: 一个绝对值为 1 的 complex number 的倒数是它的 conjuate.} \\
   因而:
   $$
  \Big|\int f \Big| = \overline{\alpha} \int f = \int \overline{ \alpha } f  \in \mathbb{ R}
   $$
   从而 $$ \Big| \int f \Big|  = \int \overline{\alpha} f = \int \re (\overline{\alpha } f) \leq \int |\re (\overline{\alpha} f)  | \leq \int |\overline{\alpha} f| = \int |f|$$
\end{proof}



\begin{definition}{integratal restricted to a measurable set}
    if $f \in \tilde{L}(X, \mu, \mathbb{C})$, $E \in \mathcal{A}$ ($\mu$ 的 $\sigma$-algebra), 我们 define: $$\int_E f   \, d \mu := \int f  \chi_E \, d \mu$$\end{definition}
\begin{remark}
    容易验证, restricted to a measurable set 的积分也是 linear 且 monotone 的.
\end{remark}



\begin{proposition}{可积函数几乎处处相等的等价条件}
    if $f,g \in \tilde{L}(X, \mu, \mathbb{C})$, 则 TFAE: 
    \begin{itemize}
        \item $f=g $ a.e.
        \item $\int |f-g| = 0$
        \item $\int _E f = \int _E g$ for all $E \in \mathcal{A}$
    \end{itemize}
\end{proposition}
\begin{proof}
    $(i) \Longleftrightarrow (ii) $: by last time proposition.\\
    $(ii) \implies (iii)$: 因为 $$
    \Bigg|\int_E f - \int _E g  \Bigg|  = \Bigg| \int (f-g_)\chi_E  \Bigg| \leq \int |f-g| \chi_E \leq \int|f-g| = 0
    $$
    $(iii) \implies (ii)$: 令 $u := \Re (f-g)$, $v := \Im (f-g)$, 则 $$
    \int |f-g| = \int u^+ + \int u^- + i\int v^+ + i\int v^-
    $$
\textbf{这四个积分都是正值. }容易发现如果 $u^+$ 在一个 positive measure set $E$ 上非 0, 那么 $\int_E u^+ > 0$ , 那么 $\int |f-g| > 0$. (其他三个积分同理.)
\end{proof}
\begin{remark}
    $\int |f-g| = 0$ 是一个比 $\int f-g = 0$ 更强的条件. $\int f-g = 0$ 可以是非零集有交错并且正负抵消, 而 $\int |f-g| = 0$ 则表示 a.e. 相等.
    \end{remark}



\begin{remark}
    有这个定理得:\textbf{ 我们可以 integrate $f:X\rightarrow \mathbb{C}$ a.e. defined}.\\
    即: $$
    f: E^c \rightarrow \mathbb{C}\quad , \quad \mu(E) = 0
    $$
    其中的一种情况是: $$
    f: X \rightarrow \overline{\mathbb{R}} \quad s.t. \quad |f| < \infty  \;\;\ a.e.
    $$
\end{remark}

并且我们发现, a.e. 相等的两个可积函数 $f,g \in \tilde{L}(X, \mu, \mathbb{C})$ 在任意可测集上的积分都相等. 于是这两个函数在 $\tilde{L}(X, \mu, \mathbb{C})$ 中的表现是相等的. 因而我们可以把 a.e. 相等的这种关系 quotient 掉, 简化这个空间:


\begin{definition}{$L^1(\mu)$ space}
    我们定义 $L^1(X, \mu, \mathbb{C})$, 或简称为 $L^1(\mu)$, 为:$$
    \tilde{L}(X, \mu, \mathbb{C}) / \sim 
    $$
其中 $\sim$ 表示一个 equivalent class: $f\sim g$ if $f=g$ a.e. (等价于 $\int |f-g| = 0$)
 \end{definition}
$L^1(\mu)$ 中的每个函数之间彼此至少都在一个正测度集上相互不同. 这减去了分析上考虑几乎处处相等的集合的顾虑, 对于处处相等的函数, 我们认为它们在 $L^1(\mu)$ 上直接相等. 并且, 我们有: $$
f \mapsto \int f
$$
在 $L^1(\mu)$ 上是一个 well-defined function.




\section{DCT}


\begin{lemma}
    令 $(f_n)$ 为 a seq of \textbf{a.e. defined measurable functions} on $X$., s.t. \[
    f(x) := \lim_{n\to \infty} f_n(x) 
    \] \textbf{exists a.e.}\\
    Claim: \textbf{$f$ is measurable.}
\end{lemma}
\begin{remark}
    Measurability is well preserved by taking limit, 并且更改一个零测集上函数的 definedness 不会改变这个 behavior. (这是一个很宽的条件了)
\end{remark}



\begin{theorem}{dominated convergence theorem}
    \label{DCT}
    Let $(f_n)$ be a seq of functions in $L^1(\mu)$, s.t. \begin{itemize}
        \item $f_n \rightarrow f $ a.e.
        \item 存在 $g \in L^1(\mu)$ s.t. $|f_n| \leq g $ a.e. for all $n$. 
    \end{itemize}
    Claim: $f \in L^1(\mu)$ 并且 \[
    \int f = \lim_n \int f_n
    \]
\end{theorem}
\begin{proof}
    首先由于 $f_n \to f$ a.e.,  by lemma 可以得到 $f$ 是 measurable 的.\\
    并且 $$|f_n| \leq |g| \text{ a.e. } \implies  |f| \leq |g| \text{ 
a.e.}$$ 于是 \[\int |f| \leq \int |g| < \infty \] 即 $f \in L^1$. (从而 $|f|$ 至多在一个 measure zero set 上无穷).\\
并且 $ g(x) \pm f_n(x) \geq 0 $ a.e. 这一点很重要, 因为从而我们可以对 $g+f_n$, $g-f_n$ 使用 Fatou's Lemma: 
\begin{align}
    \int g + \int f = \int (g+f) &= \int (g + \lim_{n\to\infty} f_n) \\
    &= \int \lim_{n\to\infty} (g+f_n) \\
    &  \overset{\text{by Fatou}}{\leq }  \liminf_n \int (g+f_n) \\
    &= \int g + \liminf_n \int f_n
\end{align}从而 (由于 $\int g < \infty$)\[
\int f \leq \liminf_n \int f_n
\]
以及 similarly get: \[
\int g - \int f \overset{\text{by Fatou}}{\leq } \liminf_n \int (g-f_n) = \int g - \limsup_n \int f_n
\]
从而: \[
\int f  \geq \limsup_n \int f_n
\]
(这里注意, negate 一个 numerical seq 后 liminf 变 limsup. 由此可见 Fatou'e Lemma 其实是很强大的, 只需要对 $\int g + \int f$ 和 $\int  g - \int f$ 各用一次就可以得到: )\[
\int f = \lim_{n\to \infty } \int f_n
\]
\end{proof}

\begin{remark}
DCT 是 MCT 在 $L^1$ 上的推广. MCT 只作用于非负的可测函数, 并且要求序列递增. 而 DCT 则作用于更加广泛的情况.\\

DCT 增加的要求是存在一个 $L^1$ 的 (a.e.) bound function, 以及极限 a.e. 存在于 extened $\mathbb{R}$. 这两个要求都是合理的, 一个控制了函数的上下浮动程度, 一个控制了序列的收敛性.\\

而进一步, 我们可以把 "存在 $g \in L^1$ s.t. $|f_n| \leq |g| $ a.e. for all $n$." 这一 条件放宽到 : 存在一个 seq $(g_n)$ 以及 $g$ in $L^1$, 使得 \begin{itemize}
        \item $|f_n| \leq g_n$
        \item  $g_n \to g$ a.e.
        \item $\int g_n \to \int g$
    \end{itemize}
Proof 在 hw 5.
\end{remark}

\begin{example}
    Suppose $u:[0,1] \to [0,1]$ is Lebesgue measurable. \\
    考虑这一 seq of function: $( u^n)$.\\
    容易发现 $u^n \rightarrow \chi_{\{u = 1\}}$ p.w.
    我们可以用 $g = 1$ 作为 bound function. 从而得到: \[
\int f=    \lim_{n\to\infty} \int f_n = \int_{\{u=1\}}1 = m(\{\mu =1 \})
    \]
\end{example}


\begin{example}
    compute \[
    I = \lim_{n\to \infty} \int_{[0,1]} \frac{1 + n x^2}{(1+ x^2)^n}
    \]
令 $f_n(x) : = \frac{1 + n x^2}{(1+ x^2)^n}$, 有: $f_n(x) \to 0$ as $n\to \infty$ for $x \in (0,1]$;\\
并且考虑 $g=1$, 作为 bound.\\
因而有 $I = 0$
\end{example}