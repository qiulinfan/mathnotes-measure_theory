\chapter{Lebesgue measure on $\mathbb{R}^n$}
\section{Lebesgue measure in $\mathbb{R}^n$ [Fol 2.6]}
今日: Lebesgue measure in $\mathbb{R}^n$ 的 \begin{itemize}
    \item regularity
    \item behavior under affine transformation
    \item behavior under diffeomorphism
\end{itemize}

\subsection{Lebesgue measure in $\mathbb{R}^n$}
这是 product measure 最常见的应用和例子. 
\begin{definition}
    $(\mathbb{R}^n, \mathcal{L}^n, m)$ Lebesgue measure is \textbf{completion of } $(\mathbb{R}^n, \mathcal{B}_{\mathbb{R}^n}, m|_{borel})$. 
\end{definition}
where $ \mathcal{B}_{\mathbb{R}^n} =  \mathcal{B}_{\mathbb{R}} \otimes \cdots \otimes \mathcal{B}_{\mathbb{R}}  $
\(\mathcal{L^n }  = \{ \text{Leb meas sets} \}  \supset  \mathcal{B}_{\mathbb{R}^n}\)
Write: \[  \int f \;d m^n   \quad 
\]


\begin{theorem}{Fubini-Tonelli for $m^n$}
    Suppose $f \in L^+(\mathbb{R}^n)$ or $L^1(\mathbb{R}^n)$
\begin{align}
\int f \; dm^n &= \int \cdots \int f(x_1, \cdots, x_n) 
\; dx_1 \cdots dx_n        \\
& = \int \cdots \int f(x_1, \cdots, x_n) 
\; dx_n \cdots dx_1
\end{align}
\end{theorem}

\begin{example}
    Show: \[
    \int_0^\infty e^{-sx} \frac{\sin^2(x)}{x} \; dx = \frac{1}{4} \log(1+ 4s^{-2})
    \]
for $s > 0$, by integrating $e^{-sx} \sin 2xy = f(x,y)$ over the rectangle $x \in (0,\infty), y \in (0,1)$.\\
Sketch: $f \in L^1$ (since it is ctn on $\mathbb{R}$)
以及 \[
|f| \leq e^{-sx}, \quad \int_{\mathbb{R}} e^{-sx} < \infty
\]
可计算得 \[
\int_0 ^1 \sin 2xy \; dy = \frac{1}{2x} \sin^2 x
\]
而后 compute \[
\int_0 ^1    e^{-sx} \sin 2xy \; dy
\] by integration by part for twice.
\end{example}




\subsection{regularities of Lebesgue measure in $\mathbb{R}^n$ }

\begin{theorem}{regularities of $\mathcal{L}^n$}
If $E \subset \mathcal{L}^n$, 则有: 
\begin{itemize}
    \item \textbf{outer regularity}: \[ m(E) = \inf \{  m(U) \mid U  \text{ open}   \supset E\}  \]
    \item \textbf{inner regularity}: \[ m(E) = \sup \{  m(K) \mid K  \text{ compact}   \subset E\}  \]
    \item if $m(E) < \infty$, 则对于任意 $\epsilon > 0$, 都存在 disjoint rectangles $R_1, \cdots R_N$ with sides that are open intervals (literally rectangles) s.t. \[ m(E  \Delta \bigcup_{j} R_j ) < \epsilon \]
\end{itemize}
\end{theorem}
\begin{proof}
 \textbf{   for (a,b) i.e. regularities: }\\
 Fix $\epsilon > 0 $. By construction, 存在 finite disjoint union of rectangle $T_j$ for each $j$, 使得 \[
 E \subset \bigcup_{j=1}^\infty T_j \quad \text{ and } \quad \sum_{j=1}^\infty m(T_j) \leq m(E) + \epsilon
 \]
By outer regularity of $m^1$, 存在 $U_j \supset T_j$ open rect s.t. $m(U_j) \leq m(T_j) + \epsilon / 2^j$
Then: \[
E \subset U := \bigcup_{j=1}^\infty U_j \quad \text{and} \quad m(U) \leq \sum_{j=1}^\infty m(U_j)
\]
Construct $K$ as in dim $1$ (DIY) $\leq m(E) + 2\epsilon$.\\
(完整 Pf 可见 395 笔记, 此略)
\end{proof}
\begin{proof}
 \textbf{   for (c):} \\
Notation as above.\\
\[
m(E) < \infty \implies m(U) < \infty \implies m(U_j)  < \infty \;\; \; \forall j
\]
Sides of $U_j$ are disjoint union of ctbly many open finite intervals.\\
因而存在 open rectangle $V_j \subset U_j$ for each $j$ that are finite disjoint union of finite open intervals s.t. \[
m(U_j  \setminus   V_j) < \epsilon / 2^j
\]
Now pick $R_1, \cdots, R_N$ from honest rectangles (即 sides 都是 intervals 的 rectangle) insides $V_j$ (DIY).
(完整 Pf 可见 395 笔记, 此略)
\end{proof}




\begin{corollary}
For $f \in L^1(m)$, 
    if $f \in L^1 (m)$ and $\epsilon > 0$ then \begin{itemize}
        \item 对于任意 $\epsilon>0$, 都存在 $\phi = \sum_{j=1}^N c_j \chi_{R_j} $ s.t. \[\int |\phi - f|\; dm < \epsilon  \]其中 each $c_j \in \mathbb{C}$, $R_j$ 是 rectangles with sides as finite open intervals.
        \item 存在 $\phi \in C_c^0(\mathbb{R}^n)$ s.t. \[\int |f - \phi| \; dm < \epsilon \]
    \end{itemize}
\end{corollary}
\begin{proof}
Similar to 1 dim case, 可以证明 $\{\text{all step functions}\}$, $C_c^0(\mathbb{R}^n)$ 是 dense subspace of $L^1(m)$.
\end{proof}



\subsection{approximating an open set $E\subset \mathbb{R}^n$ by countable disjoint interior cubes}
对于 $k \in \mathbb{Z}$, 令 $\mathcal{Q}_k$ be the collection of cubes whose side length is $\frac{1}{2^k}$ 且 vertices 在 lattice $(2^{-k} \mathbb{Z})^n$ 中, 即精细度为 $\frac{1}{2^k}$ 的网格中的所有 cubes.\\
\pic[0.4]{assets/ch2-pics-image-20250311142400747.png}
对于 $E\subset \mathbb{R}^n$, 我们定义: \[
\underline{A}(E,k) := \bigcup \{Q \in \mathcal{Q_k} : Q \subset E\},\quad \overline{A}(E,k) := \bigcup \{Q \in \mathcal{Q_k} : Q \cap E \not = \varnothing\}
\]
即, 一个是被包含在 $E$ 中的所有格子, 一个是最小的覆盖 $E$ 的所有格子.
并定义: \[
\underline{A}(E) : = \bigcup_{k=1}^\infty \underline{A}(E,k),\quad \overline{A}(E) : = \bigcup_{k=1}^\infty \overline{A}(E,k)
\]以及
\[
\overline{\kappa}(E) := \lim_{k\to\infty}m(\underline{A}(E,k)), \quad \underline{\kappa}(E) := \lim_{k\to\infty}m(\overline{A}(E,k))
\]
By CFB, CFA 容易得到: \[
\overline{\kappa}(E) = m(\overline{A}(E)) ,\quad \underline{\kappa}(E) = m(\underline{A}(E)) 
\]
Note: 这里的 $\underline{A}(E,k), \; \overline{A}(E,k), \; \underline{A}(E), \; \overline{A}(E)$ 都是 union of cubes with disjoint interiors.
\begin{lemma}{approximate an open set by disjoint interior cubes}
Let $E\subset \mathbb{R}^n$ be open.\\
Claim: $E = \underline{A}(E)$
\end{lemma}
\begin{proof}
    Folland 2.43.
\end{proof}

\begin{corollary}
$E \subset \mathbb{R}^n$ 是 Lebesuge measurable 的 $\Longleftrightarrow$ $\overline{\kappa}(E) = \underline{\kappa}(E)$
\end{corollary}



\subsection{behavior under affine transformation}
Affine transformation 即 linear transformation + translation.
\subsection{Lebesgue measure and integral is invariant under translation}
对于 $a \in \mathbb{R}^n$, 一个 translation $t: \mathbb{R}^n \to \mathbb{R}^n, x \mapsto x+ a$ 是 ctn 的并且 \[
t_a^{-1} = t_{-a}
\]

\begin{theorem}{Lebesgue measure and integral is invariant under translation}
(a) 任取 $a \in \mathbb{R}^n$, 
\[
E \in \mathcal{L}^n \implies  t_a(E) \in \mathcal{L}^n \quad \text{ and }\quad m(t_a(E)) = m(E)
\]
(b) if $f: \mathbb{R}^n \to \mathbb{C}$ is Leb measurable, then so is $f \circ t_a$. \\
More, if $f \in L^+$ or $f \in L^1$, then $f \circ t_a \in L^1$ 并且  \[
\int (f \circ t_a) \; dm = \int f \; dm
\]
\end{theorem}
\begin{remark}
    集合的 measure 以及 measurable function 的积分在 translation 下保持不变.
\end{remark}
\begin{proof}
    (Folland 2.42)\\
    (a)
    $t_a $ ctn $\implies$ $t_a(\mathcal{B}_{\mathbb{R}^n}) \subset \mathcal{B}_{\mathbb{R}^n}$, 因而  $t_a(\mathcal{B}_{\mathbb{R}^n})=\mathcal{B}_{\mathbb{R}^n}$
    $E$ rectangle, so $E = E_1 \times \cdots \times E_n $, each in $\mathcal{B}_\mathbb{R}$
    $m(E) = \prod_1^n m(E_i)$, $t_a(E) = \prod t_{a_i} (E_i)$ 
    因而  \[
    m(t_a(E)) = \prod m(t_{a_i} (E_i)) = \prod m(E_i) \subset m(E)
    \]
    BY HK uniqueness, get \[
    m(t_a(E)) = m(E) \quad \forall E \in \mathcal{B}_{\mathbb{R}^n}
    \]
    if $N \subset \mathbb{R}^n$ subnull set, so is $t_a(N)$. 因而 \[
    m(t_a(E)) = m(E) \quad \forall E \in \mathcal{L}^n 
    \]
    (b) Pick $B \in \mathcal{B}_\mathbb{C} \implies f^{-1}(B) \in \mathcal{L}$.
    因而 $f^{-1}(B) = E \cup N$, $E \in \mathcal{B}_{\mathbb{R}^n}$, $N$ null set
    因而 \begin{align}
        (f\circ t_a)^{-1}(B) & = t_a^{-1}( f^{-1}(B)) \\&= t_a^{-1}(E) \cup t_a^{-1}(N) \text{ (one Borel, one null)} \\
        &= t_{-a}(f^{-1}(B))
    \end{align}
当 $f= \chi_E$ 时, 积分 reduce to measure, 即 (a);
因而  \[
\int (f \circ t_a) \; dm = \int f \; dm
\] also holds for simple $f$, by linearity.\\
从而 by def, 也 hold for $f \in L^+$ 和 $f \in L^1$.
\end{proof}






\subsection{Lebesgue measure and integration is scaled $|\det T|$ under linear map }

\begin{theorem}{Lebesgue measure and integration is scaled $|\det T|$ by linear map}
For $T \in GL(n,\mathbb{R})$ (即 linear map $T: \mathbb{R}^n \to \mathbb{R}^n$ 且可逆)
(a) 如果 $f: \mathbb{R}^n \to \mathbb{C}$ is Lebesgue measurable, then so is $f \circ T$.\\
Moreover if $f \in L^+$ or $f \in L^1$, then $f \circ T \in L^+$, $f \circ T \in L^1$ respectively. And \[
  \int f \; dm = |\det T| \int f \circ T \; dm
\]
(b) \[ E \in \mathcal{L}^n  \implies T(E) \in \mathcal{L}^n  \quad \text{and}\quad    m(T(E)) = |\det T| m(E)  \]
\end{theorem}

\begin{proof}
Note: 对于 $T,S \in GL(n,\mathbb{R})$, 如果  \[\int f  = |\det T| \int f\circ T\quad  \text{ and } \quad \int f = |\det S| \int f\circ S\]  , 那么则有 \[
\int f = |\det (T \circ S) | \int f \circ (T\circ S) (x) 
\]
which trivially follows from computation. (and $\det (S \circ T) = \det S \times \det T$ for any linear map $S,T$.)\\
recall that:
\begin{lemma}{row reduction}
\textbf{任意 invertible linear map 可以被拆分为 finite 个 elementary linear maps.} ( $T_1$: scale 一行; $T_2$:  交换两行; $T_3$:  一行加上另一行的倍数).
\end{lemma}
于是, 我们只需要 prove the theorem for elementary linear maps 就可以了. 而 elementary linear maps 的 cases 则 easily follows from Fubini-Toneilli.\\
\textbf{Let $f$ be Borel measurable.}\\
对于 $T_2$: 交换两行 (其 det 为 -1), 我们改变 the order of integration for two coordinates, 因而 integration 不变; \\
对于 $T_1$: scale 一行 by const $c$ (其 det 为 $c$), 我们在一个 coordinate 上积分值翻 $c$ 倍, 因而整体积分值翻 $c$ 倍. 这里用到了 $\mathbb{R}\to\mathbb{R}$ 的 Lebesgue integral 的已证明结论:\[
\int f(t) \; dt = |c| \int f(ct) \; dt
\]
对于 $T_3$: 一行加上另一行的倍数 (其 det为 1), 我们 recall $\mathbb{R}\to\mathbb{R}$ 的 Lebesgue integral 的 translation invariance: \[
\int f(t +a ) \; dt = \int f(t) \; dt
\]
因而整体积分值不变.\\
从而\textbf{我们证明了 (a) for Borel measurable $f$}.\\
从而, (b) for Borel set $E$ trivially follows from (a), by taking indicator function.\\
而对于 (b) 的 $E$ Lebesgue measurable case, $E = B \cup N$ for some Borel set $B$ 以及 subnull set $N$, 从而 $m(E) = m(B)$.\\
\textbf{从而 (b) proved. }\\
而 (a) 的 $f$ Lebesugue measurable 的 case, by def \textbf{reduces to $f = \chi_E$ where $E$ is Lebesgue measurable set}, 于是 follows from the (b).
\end{proof}

\subsection{Lebesgue measure is invariant under rotation (and reflection)}
\begin{corollary}{Lebesgue measure is invariant under rotation}
    对于 rotation 和 reflection (即 orthogonal transformation), 即 $TT^* = I_n$ 的 linear map $T$, 有 $m(T(E))  = m(E)$.
\end{corollary}
\begin{proof}
    $TT^* = I_n \implies |\det (T)| = 1$. 
\end{proof}
\begin{remark}
    $A \in GL(n,\mathbb{R})$ 为一个 orthogonal transformation (可写作 $A \in O(n)$) 的定义是它 preserve norm.\\
我们知道, $A \in O(n)$ 当且仅当 $A^* = A^{-1}$.\\
有两种情况: rotation ($\det A = 1$) 和 reflection ($\det A = -1$).\\
\end{remark}



\section{Change of Variable Thm on $\mathbb{R}^n$[Fol 2.6, finished]}
\subsection{COV}
\begin{theorem}{general change of variable theorem}
Suppose $\Omega \subset \mathbb{R}^n$ \textbf{open}, $G: \Omega \to \mathbb{R}^n$ 为一个 $C^1$ \textbf{diffeomorphism}.\\
Claim: 
\begin{itemize}
    \item[(a)] 如果 $f:G(\Omega)\to \mathbb{C}$ 上是 Lebesgue measurable 的, 则 $f \circ G:\Omega \to \mathbb{C}$ 也是 Lebesgue measurable 的. 并且, 如果 $f \in L^+(G(\Omega ),m)$ 或者 $f \in f \in L^1(G(\Omega ),m)$, 则有\[ \int _{G(\Omega)}  f\; dm = \int_\Omega (f\circ G) \, |\det DG |\; dm    \]
    \item[(b)]  如果 $E\subset \Omega$ 是 Lebesgue measurable set, 则 $G(E)$ 也是 Lebesgue measurable set, 并且 \[ m(G(E)) = \int_E |\det DG| \; dm  \]
\end{itemize}
\end{theorem}
\begin{proof}
首先, 类似于上一个 lecture 中的各个证明, 只需要 prove for Borel measurable functions 和 Borel sets 就可以了. 我们分为五步证明.\\
\textbf{Step 1: 我们首先证明, 在 $E$ 为一个 closed cube 的情况下} (我们转而用 $Q$ 来表示它), 有 \[
m(G(Q)) \leq \int_Q |\det DG(x)| \; dx
\]
\textbf{Proof of Step 1}: \[Q = \{x : ||x-a||_{\sup} \leq h\}  \]
By MVT 容易得到, 对于任意的 $x\in Q$, 有: \[
||G(x)-G(a)||_{\sup} \leq h \cdot ({\sup}_{y\in Q} ||DG(y)||_{\sup})
\]
(by bounding each entry.)\\
从而, 我们发现  $G(Q)$ \textbf{是 contained in 一个边长是 $h \cdot {\sup}_{y\in Q} ||DG(y)||_{\sup} $ 的 cube 的}.\\
从而有: \[
m(G(Q)) \leq ({\sup}_{y\in Q} ||DG(y)||)^n m(Q)
\]
在 invertible $T$ 的作用下, $T^{-1}\circ G$ 仍然是一个 diffeomorphism, 从而 
\begin{align}
    m(G(Q)) &= |\det T| m(T^{-1}(G(Q))) \\
    &\leq |\det T|({\sup}_{y\in Q} ||T^{-1}DG(y)||)^n m(Q)
\end{align}
Let $\epsilon >0$.\\
由于 $DG$ 是 continuous 的, $DG(x)^{-1} DG(y)$ 也是 ctn 的 (从而 \textbf{uni.ctn.} in the compact cube), 我们对于任意 $\epsilon > 0$ 都可以找到一个 $\delta >0 $ 使得 对于任意的 $y,z \in Q$ s.t. $||y-z||_{\sup} \leq \delta$, 都有 
\[ ||DG(x)^{-1} DG(y)||    \leq 1+ \epsilon\]
于是我们可以把 $Q$ 切分成 interior disjoint 的 closed subcubes $Q_1,\cdots,Q_N$, 标记其各个中心为 $x_1,\cdots x_N$, 其每个的 side length 都至多为 $\delta$,  从而有 $G(Q) \subset \bigcup_{j=1}^N m(G(Q_j)) $. 
于是
\begin{align}
    m(G(Q)) &\leq \sum_{j=1}^N m(G(Q_j)) \\
    & \leq \sum_{j=1}^N |\det DG(x_j)| \, \big( {\sup}_{y\in Q_j} ||DG(x_j)^{-1} DG(y)||_{\sup}\big)^n m(Q_j)\\
    &\leq (1 + \epsilon) \sum_{j=1}^N |\det DG(x_j)| \, m(Q_j)\\
&    \rightarrow  (1+\epsilon)\,|\det DG(x)| \,m(Q) \quad \text{ as } \quad\delta\to 0 \\
& \rightarrow |\det DG(x)| \,m(Q)=\int_Q |\det DG(x)| \; dm \quad \text{ as } \quad\epsilon\to 0
\end{align}
证明了这一结论, 我们就完成了这个 proof 的一大半.\\\\
\textbf{Step 2: }Prove \[m(G(U)) \leq \int_U|\det DG(x)| \; dm\] for open $U$ 的 case.\\
\textbf{Proof of Step 2}: Directly follows from 上一 lecture 的这个 statement: 任意 open $E \subset\mathbb{R}^n$ 都是 countable disjoint interior cubes 的 union.\\\\
\textbf{Step 3:} Prove \[m(G(E)) \leq \int_E |\det DG(x)| \; dm\] for $E$ Borel 的 case.\\
\textbf{Proof of Step 3:} Apply step 2 的结论, 使用 MCT for $L^+$ case, 使用 DCT for $L^1$ case.
至此, 我们完成了 (b) 的证明的一个方向, 由此可以完成 (a) 的不等式的一个方向:\\\\
\textbf{Step 4}: 证明 \[
\int _{G(\Omega)} f\; dm \leq \int_{\Omega  }f\circ G \,|\det DG(x)| \; dm
\]
simple function 的 case reduces to measure, 而 $L^+$ 的 case follows from MCT.\\\\
\textbf{Step 5}: 不等式的另一方向: 其实很简单, 因为 diffeomorphism 的 inverse 仍然是 diffeomorphism, 所以 apply inverse 可得.\\
注意, 这只是 for Borel $E$ 和 $L^+$ Borel measurable $f$, 不过我们容易接着推导出 Lebesgue measurable $E$ 的情况和 $f \in L^+(m)$ 的情况; 从而再接着推导出  $f\in L^1(m)$ 的情况.
\end{proof}
\begin{remark}
这个证明写得比较潦草. 详情见 Folland 2.47.\\
但是大概思路都比较简单. 其中比较困难的是 Step 1 中的各种 error bounds. 很麻烦.\\
\end{remark}


\subsection{application of COV: polar coordinate}
\begin{definition}{mapping from Euclidean coord to polar coord}
 我们定义: \[
\Phi: \mathbb{R}^n \setminus \{0\}  \rightarrow \ (0,\infty) \times S^{n-1}
\]by: \[
x \mapsto (r\in \mathbb{R},\theta \in\mathbb{S^{n-1}})
\]
其中, \[
r = |x| ,\quad \theta =  \frac{x}{|x|} \in S^{n-1}
\]
   
\end{definition}
这是一个很直观的坐标变换, 即一个 diffeomorphism.\\

\begin{definition}{a Borel measure on $(0,\infty) \times S^{n-1}$}
    我们定义 \[
    m_*(E) := m(\Phi^{-1}(E))
    \]
\end{definition}
这是一个通过坐标变换的 preimage 的 Borel measure 定义的新的 Borel measure.\\

\begin{theorem}
Define Borel measure $\rho$ on $(0,\infty)$ by: \[
\rho(E) = \int_E r^{n-1} \; dr
\]
存在 unique 的 Borel measure $\sigma_{n-1}$ on $S^{n-1}$, 使得 for Borel measurable $f:\mathbb{R}^n \to \mathbb{C}$ 且 $f\geq 0$ or $f\in L^1(m)$, 有 \begin{align}
    \int_{\mathbb{R}^n} f(x) \; dm &\overset{COV}{=} \int_{(0,\infty)\times S^{n-1}} f(r\theta) \; dm_*\\
    &\overset{Fubini}{=}\int_0^{\infty}  \int_{S^{n-1}} f(r\theta) \; d\sigma \, d\rho\\
    &= \int_0^{\infty}  r^{n-1} \int_{S^{n-1}} f(r\theta) \; d\sigma \, dr
\end{align}
\end{theorem}
\begin{proof}
    见 Folland 2.49.
\end{proof}

\begin{remark}

这里 $S^{n-1}$ 的 unique measure $\sigma$ 的计算公式是: \[
\sigma(E) = n\cdot m\bigg(\Phi^{-1}\big( (0,1)\times E  \big)\bigg) = n\cdot m\{r\theta \mid   0<r \leq 1, \theta \in E\}
\]
这很容易直观: \pic[0.4]{assets/ch2-pics-image-20250312031159838.png}
这里 $n=2$, \textbf{$m(E_1)$ 表示的单位圆下, $E$ 的弧长下的扇形面积, 而 $\sigma(E)$ 表示 $E$ 的 arc length.}\\
(类比, 在 $n=3$ 的情况下, $m(E_1) $ 表示单位球下, $E$ 的球面下的锥形体积, $\sigma(E)$ 表示 $E$ 在 $S^2$ 中的球面面积.)
\end{remark}

\begin{remark}
    对于 $E = S^{n-1}$ 即全集的情况 , 这个 measure 有固定的计算公式. \[
    \sigma(S^{n-1}) = \frac{2\pi^{\frac{n}{2}}}{\Gamma(\frac{n}{2})}
    \]
\end{remark}
\begin{example}
    $\sigma(S^1) = 2\pi$, $\sigma(S^2) = 4\pi$.
\end{example}



\begin{example}
    使用 polar coordinate 计算积分: \[
    \int _{\mathbb{R}^n} e^{-a|x|^2} \; dx = (\frac{\pi}{a})^{\frac{n}{2}}
     \]
这是因为: \[
I_2 = 2\pi \int_0^\infty re^{-ar^2} \; dr = \frac{\pi}{a}
\]
而由于 \[
e^{-a|x|^2} = \prod_{j=1}^n e^{-ax_j^2}
\]
我们得到 \[
I_n  = (I_1)^n
\]
特别地, \[
I_2  = I_1^2, \quad \text{thus } I_1  = (\frac{\pi}{a})^{\frac{1}{2}}
\]
\end{example}