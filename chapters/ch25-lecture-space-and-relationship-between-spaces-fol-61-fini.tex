\chapter{$L^\infty$ space, and relationship between $L^p$ spaces ($0\leq p \leq \infty$) [Fol 6.1, finished]}
对应 Folland 6.1(3), finishing 6.1.\\
我们已经完成了对 $1\leq p < \infty$ 的 $L^p$ space 的构建. 现在, 我们来构建最后一块拼图: $L^\infty$ space.
\section{$L^\infty$ space}
我们考虑这个启发式的例子: \[
X := \{ 1,2,\cdots, n\},\quad\mathcal{A} := \mathcal{P}(X),\quad \mu = \mu_{counting}
\]
于是: \[
L^p (\mu)   = \{(a_1,\cdots, a_n) : \| (a_1,\cdots, a_n) \|_p = \big( \sum |a_i|^p \big)^{1/p} < \infty\}= \mathbb{C}^n
\]
我们发现: \[
\| (a_1,\cdots, a_n) \|_p \to \max_j |a_j| \quad \text{as}\quad p\to \infty
\]
因为 $p$ 取得越大, 最大的 entry 的 contribution 占比就越突出.\\
对于这样的 $L^p$ space, 我们可以定义 $\sup$ norm, 定义为最大的 entry.\\
即便 $X$ 是 countable 的, 这个定义也可以定义为 $\sup_j |a_j|$, make sense.\\
那么如果我们想要给任意的 measure space 定义 sup norm 呢? 我们可以考虑 \[
\| f\|_\infty : = \sup_{x\in X} |f(x)| \; ?
\]
实际上我们有更好的定义方式: 

\begin{definition}{essential supremum}  \[
    \| f\|_\infty : = \inf \{ a\geq 0 : \mu\{x:|f(x)|>a\}  = 0 \}
    \]
    也可以写作: \[
    \text{ess} \sup_{x\in X} |f(x)|
    \]
\end{definition}
\begin{remark}
essential sup 是一个比较容易搞错的定义. \\
一个 function 的 essential supremum 即: 这个 function 几乎处处的 sup.\\
它 $\leq \sup f$ , 因为它允许在零测集上存在一些点的函数值大于它.\\
这是合理的, 因为积分可以不考虑零测集.\\
\end{remark}
\begin{remark}
  对于零测集只有空集的 measure space 上的函数, 比如  对于 $\ell^\infty(\mathbb{N})$ 上的函数, 其 essentail supermum 即 supermum.\\
  对于\[
   \sup_{x\in X} |f(x)| 
  \]我们也有一个称呼, 称其为 \textbf{uniform norm}. 即: \[
\|f\|_u :=   \sup_{x\in X} |f(x)| 
  \]
\end{remark}

\begin{definition}{$L^\infty$ space} \[
    L^\infty(\mu) : = \{ f:X \to \mathbb{C} \text{ measurable} : \| f\|_\infty < \infty \} / \sim
    \]
    where $\sim$ 表示 a.e. 相等的函数的 equiv class.
\end{definition}
\begin{remark}
    注意: \textbf{$f\in L^\infty(\mu)$, 并不等价于 $f$ a.e. bounded!}\\
实则 recall: $f$ a.e. \textbf{bounded  是 $f \in L^p (\mu)$ for any $1\leq p \leq \infty$ 的必要条件}, 否则, 函数积分不可能 $<\infty$, 函数 $p$ 次方的积分更加不可能 $<\infty$.\\
$f\in L^\infty(\mu)$ 是一个很严格的条件, 当然严格强于 $f$ a.e. bounded. \\
比方说: \textbf{$f = \frac{1}{x}$, 只有在 $0$ 这一个点上 $f$ 是 unbounded 的, 但是它的 essential supermum 仍然是 $\infty$}, 因为不可能通过去掉一个 measure $0$ set 来使它 bounded. \\
无法找到一个 $M$, 使得 $f$ 在几乎处处都小于 $M$. 你只能控制, $f$ 在 $(0,1/M)$ 上小于 $M$, 这个集合的测度随 $M$ 增大越来越小, 但是永远都是正测度.
(同样这个函数也不属于任何 $L^p(m)$.)\\
一个函数 essential supermum $<\infty$, 即 $\in L^\infty$, 则必须要它 unbounded 的这个行为是可以忽略不计的, 不能是明显的. 比如它在 $\mathbb{Q}$ 上 unbounded. 如果是在一个点上连续 blow up, 那么它就不可能 $\in L^\infty$. 类似于这里的  $f = \frac{1}{x}$.\\
\end{remark}
\begin{remark}
    我们在本节课还会证明, 如果 measure space $X$ has finite measure, 那么有 \[
    L^\infty(X) \subset \cdots \subset L^p(X)\subset \cdots  \subset L^{q} (X)\subset \cdots \subset L^1(X)
    \]
for 任意的 $p \geq q$.\\ 这表明的是, 在一定要求下, $L^\infty$ 是要求最严格的 space.
\end{remark}


下面是一个比较典型的例子:
\subsection{$\ell^\infty$ space}
\begin{definition}{$\ell^\infty$} \[
    \ell^\infty : = \{  (a_j)_1^\infty : \| (a_j)\|_\infty : = \sup_j |a_j|  <\infty \}
    \]\end{definition}
\begin{example}
    \[
    f = x\chi_{\mathbb{Q}} \in L^\infty (m)
    \]with \[
    \| f\|_\infty = 0
    \]
    因为整个 $\mathbb{Q}$ 都是零测的.
\end{example}

\begin{remark}
$\ell^\infty$ 其实就是: \[
X: = \mathbb{N},\quad \mathcal{A}: = \mathcal{P}(X),\quad \mu = \mu_{counting}
\]的 measure space 上的 $L^\infty(\mu)$. \[
\ell^\infty =L^\infty (\mathbb{N}, \mathcal{P}(\mathbb{N}), \mu_{counting} )
\]
一个 seq 就是一个从 $\mathbb{N}$ to $\mathbb{C}$ 的函数, 把每个 entry map to 一个 complex number.\\
\textbf{而对于 counting measure 作为 measure 的 measure space 上, 唯一的零测集就是空集}, 因为哪怕只取一个元素, 这个子集的测度也是 1.\\
比如, 我们只取三个 entry $1,2,8$, 看 $\{|a_n|\}_1^\infty \setminus \{|a_1|,|a_2|, |a_3|)\}$ 中的 $\sup$ value, 也不符合 essential supremum 的定义.\\
因而我们发现, \textbf{对于 唯一的零测集就是空集 的 measure space, for example, 任何以 counting measure 作为 measure 的 measure space, 其 essential sup norm 就是普通的 sup value norm.}\\
比如 \[
\mathbb{C}^1,\mathbb{C}^2,\mathbb{C}^3,\cdots,\ell^\infty
\]
\end{remark}

\subsection{$L^\infty$ 的基本性质: as a NVS; Hölder's ineq on it; dense subsets }

\begin{lemma}
如果 $f \in L^\infty(\mu) $ 则: 
\begin{itemize}
    \item 一定有 $|f(x)| \leq \|f\|_\infty$ for a.e. $x$.
    \item 存在一个 bounded 函数 $g$, 使得 $f=g$ a.e.
\end{itemize}
\end{lemma}
\begin{proof}
    显然.
\end{proof}
\begin{remark}
    是否有在某个零测集上 unbounded 但是却 $L^\infty$ 的函数? 答案是肯定的:\[
f(x) =
\begin{cases}
\frac{1}{x}, & x \in \mathbb{Q} \cap (0,1] \\
0, & \text{otherwise}
\end{cases}
\]
有 $\|f\|_\infty = 0$.
\end{remark}

\begin{theorem}
    \begin{itemize}
        \item \[ \|fg\|_\infty \leq \|f\|_1 \|g\|_\infty\]
可以把它看作 \textbf{Hölder 的一部分特殊情况}, 因为可以看作 \[ \frac{1}{1} + \frac{1}{\infty} = 1\] 从而补充完整了 Hölder ineq for $1 \leq p,q\leq \infty$
        \item $L^\infty$ 是一个 \textbf{normed vector space}, equipped with $\| \cdot \|_\infty$
        \item simple functions are dense in $L^\infty$
    \end{itemize}
\end{theorem}
\begin{proof}
    容易证明. 
\end{proof}
\begin{remark}
    注意, $L^\infty$ 和 $L^p$ 有一个出入点是: \textbf{$C_c^0(\mathbb{R}^n)$ 并不是 $L^\infty (\mathbb{R}^n, m)$ 上的 dense subspace!}\\
\end{remark}


\subsection{$L^\infty$-convergence 作为 (finite measure space 下) 最强的 $L^p$ convergence: 等价于 uni. conv a.e.}
\begin{theorem}{convergence in $L^\infty$ $\iff$uniform convergence a.e.}
\[  f_n \to f \; \text{ in } L^\infty \iff \text{exists null set } E\subset X\; s.t. f_n\to f \text{ uniformly on } E^c \]
(注意, 这\textbf{不是 conv almost uniformly}, 而是一个比 almost uniformly\textbf{ 更强}的条件:\textbf{ conv uniformly almost everywhere}, 因为 almost uniformly 只要求对于任意的 $\epsilon$, 都存在一个 measure 小于 $\epsilon$ 的 $E$, 使得在 $E^c$ 上 uni conv 即可.)
\end{theorem}
\begin{remark}
    这一条 convergence 十分惊人. 因为\textbf{对于普通的 $L^p$ space, converge in $L^p$ 和 a.e. convergence 并没有任何的互推关系}; 但是对于 $L^\infty$ convergence, 我们却可以把它\textbf{等价于 uniform convergence almost everywhere}, which is 一个\textbf{比 a.u convergence 更强, 比 a.e. convergence 更强的逐点 convergence}. 可以看出 $L^\infty$ convergence 是比任何 $L^p$ convergence 都要强一个层次的收敛性质.\\
    这一点
\end{remark}
\begin{proof}
⇐: Suppose $f_n \to f$ uni. a.e; WTS: $f_n \to f$ in $L^\infty$
 $f_n \to f$ uni. a.e 即: 存在零测集 \(E\subset X\), \(f_n \to f\) on $E^c$.\\
Let $\epsilon > 0$.\\
$f_n \to f$ uni. a.e 表明, 存在 \(N\) 使得 for all \(n \ge N\) 有 \[
\forall x \in E^c,\quad |f_n(x) - f(x)| < \epsilon
\]
by def, exactly is: \[
\|f_n - f\|_{L^\infty} = \operatorname{ess\,sup}_{x \in X} |f_n(x) - f(x)| \le \epsilon
\]
This shows that \(\|f_n - f\|_{L^\infty} \to 0\), 即 \(f_n \to f\) in \(L^\infty\).\\

⇐: Suppose $f_n \to f$ in $L^\infty$; WTS: $f_n \to f$ uni. a.e.Denote: \[
\epsilon_n := \|f_n - f\|_{L^\infty}
\]
By assumption, $\epsilon_n \to 0$. Define for each $n$:
\[
A_n := \{x \in X : |f_n(x) - f(x)| > \epsilon_n\}
\]
By def \(\|f_n - f\|_{L^\infty} = \text{ess sup}_{x} |f_n(x) - f(x)| \le \epsilon_n\), 于是 \( \mu(A_n) = 0\)
那么令:  \[
E := \bigcup_{n=1}^\infty A_n
\]by subadditivity of measure 有 \(\mu(E) = 0\).
于是对于任意 $\epsilon_n$, 都有
\[
 |f_n(x) - f(x)| \le \epsilon_n \to 0,\quad \text{for all } x \in E^c
\]
由于 $\epsilon_n \to 0$, showing that outside $E$, 有  \(\|f_n - f\|_{L^\infty} \to 0\).
\end{proof}
\begin{remark}
这两个 convergence 直觉上是自然相等的.\\
但是这并不能够说明 $   L^\infty(\mu)\text{-convergence}$ 就是强于任何 $L^p(\mu)\text{-convergence} $ 的. 因为即便是 uniform 的 ptwise conv 也无法推出 $L^p$ conv. \\
特殊情况是, 如果整个 base space $X$ 是 finite measure 的, 则可以推出\[
     L^\infty(\mu)\text{-convergence} \implies  L^p(\mu)\text{-convergence} \implies  L^q(\mu)\text{-convergence} \implies\cdots
    \]
    whenever $p > q$. (可证明)\\
    但是对于无限测度空间, 这种推论未必成立.
\end{remark}



\subsection{$L^\infty$ as Banach space}
\begin{theorem}{$L^p$ ($1\leq p \leq \infty$) is Banach}
For any measure space $(X,\mathcal{A},\mu)$, $L^p(\mu)$ is Banach for all $1\leq p \leq \infty$
\end{theorem}
\begin{proof}
我们已经 proved 了 $1\leq p < \infty$ 的 case, 现在 prove $p = \infty$ 的 case.\\
By \ref{Equiv Condition for space being Banach}, 我们知道 STS: every abs conv series conv in $L^\infty$.\\
我们 suppose $f_k \in L^\infty$ 有 \[
\sum_{k=1}^\infty \| f_k \|_\infty < \infty
\]
WTS: \(\sum_{k=1}^\infty f_k \) converges.\\
Set: \[
E_k := \{x: |f_k(x)| > \|f_k||_\infty\}
\]
于是有 \[
\mu(E_k) = 0 \quad \text{for each }k
\]
因而 setting \[
E : = \bigcup_{k=1}^\infty E_k
\]有 \[
\mu(E) = 0
\]
note: \[
x \in E^c \implies \sum_{k=1}^\infty |f_k(x)| \leq \sum_{k=1}^\infty \| f_k \|_\infty < \infty
\]
从而, \[
g: = \sum_{k=1}^\infty  f_k
\]在 $E^c$ 上是 well-defined 的, 且 bounded by $\sum_{k=1}^\infty \| f_k \|_\infty $.\\
对于 $x\in E$, 我们可以随便设置值, 比如 $\pi$, 然后 define $g(x)= \pi$ on $x\in E$. 然后对于 each $n$, 我们 set: \[
g_n(x) : = \begin{cases}
    \sum_{k=1}^n f_k(x),\quad x \in E^c \\
    \frac{1}{\pi},\quad x\in E
\end{cases}
\] 从而
\begin{align*}
    \|g_n - g \|_\infty \leq \sup_{x\in E^c} |g_n (x) -g (x)|  
   &\leq \sup_{x\in E^c} |\sum_{n+1}^\infty f_k(x)| \\
   &\leq \sup_{x\in E^c} \sum_{n+1}^\infty |f_k(x)| \\
    &\leq \sum_{n+1}^\infty  \|f_k\|_\infty \to 0\\
\end{align*}
\end{proof}
\begin{remark}
My reflection: 不 Banach 的 normed vector space 是什么样子的呢? 即, 这个 space 中存在某些 series, 其对应的 norm series absolutely conv 但是它却不 converge to 一个元素呢? \\
我们考虑空间 \( c_{00} \),它是所有 finite supp 的 seq 组成的空间:
\[ c_{00} := \{ x = (x_1, x_2, \dots) \in \mathbb{R}^\mathbb{N} \mid \text{only finite } x_i \neq 0 \}
\]
with \( \ell^1 \) norm: \[
\|x\| = \sum_{i=1}^\infty |x_i|
\]
 \( c_{00} \) 是一个 normed vector space, 但不是 Banach space, 它的完备化是 \( \ell^1 \).
我们考虑 series, with: \[
x_n = e_n / 2^n
\]
其中 \( e_n = (0, \dots, 0, 1, 0, \dots) \), 第 \( n \) 个位置是 1, 其余是 $0$, 是这个 NVS 的 standard basis.\\
显然每个 \( x_n \in c_{00} \),并且: \[
\|x_n\| = \frac{1}{2^n} \quad \Rightarrow \quad \sum_{n=1}^\infty \|x_n\| = \sum_{n=1}^\infty \frac{1}{2^n} = 1
\]这是一个 absolutely convergent series,  但其和
\[
\sum_{n=1}^\infty x_n = \left(\frac{1}{2}, \frac{1}{4}, \frac{1}{8}, \dots\right) \notin c_{00}
\]
$L^p$ space 的 Banach 性表示了其\textbf{极限存在的稳定性}. recall, Banach 即 complete NVS, 而 \textbf{complete 是比 closed 更强的条件}. \\
因而\textbf{任何一个 $L^p$ 函数列, 如果 Cauchy / converge in $L^p$ norm, 那么它的极限一定在 $L^p$ 里.}
\end{remark}



\section{relationship between $L^p$ spaces}

\subsection{$L^{m}(\mu) \subset L^{n}(\mu), 0 < n\leq m\leq \infty$, for measure finite space)}
刚才我们已经 state 了, 但还没有证明: 
\begin{theorem}{ inclusion relation between $L^p$ spaces (when base space is finite measure)}
    如果 measure space $X$ has finite measure, 那么有 \[
    L^\infty(X) \subset \cdots \subset L^m(X)\subset \cdots  \subset L^{n} (X)\subset \cdots 
    \]
for 任意的 $m \geq n$.
\end{theorem}
这是我们首次把 $p<1$ 也 include 进我们的讨论.

这个 statement 即: 对于 from finite measure space to $\mathbb{C}$ 的 function $f$, 它的 $\|f\|_m< \infty$ 是比 $\|f\|_n<\infty$ 更强的条件. \\
尤其, 除去 $L^\infty$ 的情况, 它更直接的意思是: 对于 $0 <  n \leq m <\infty$ 而言, $f$ 的绝对值的 $m$ 次方的积分 $<\infty $ 是比 $f$ 的绝对值的 $n$ 次方的积分 $<\infty$ 要更强的条件. 

这其实是一件比较直观的事情. 因为对于 $|f| \geq 1$ 的部分, 
\[
\int_{|f| \geq 1} |f|^{large} \geq \int_{|f| \geq 1}  |f|^{small}
\]
而对于 $|f| <1$ 的部分,\[
\int_{|f| < 1} |f|^{large} \leq \int_{|f|< 1}  |f|^{small}
\]
然而\textbf{由于整个 space 的 measure 是 finite 的, $|f| <1$ 的部分并不影响}. 因为 \[
\int_{|f| < 1} |f|^{large} \leq \int_{|f|< 1}  |f|^{small} \leq \int_{|f|< 1}  1 \leq \mu(X)
\]
因而, 对于 $\mu(X)<\infty$ 的情况, 显然有 $\|f\|_{large}< \infty$ 是比 $\|f\|_{small}<\infty$ 更强的条件.\\

\textbf{(实际上, 如果只有 measure finite 的 $x$ 上 $|f(x)| < 1$, 那么即便 $\mu(X)=\infty$, $\|f\|_{large}< \infty$ 也是比 $\|f\|_{small}<\infty$ 更强的条件; 而如果有 measure infinite 的 $x$ 上 $|f(x)| <1$, 那么有可能 $\|f\|_{large}< \infty$ 是比 $\|f\|_{small}<\infty$ 更弱的条件)}\\
My point: 虽然说 $|f(x)|^{large}$ 比起 $|f(x)|^{small}$ 是更大还是更小取决于 $|f(x)|$ 是否 $\geq 1$ or $<1$, 但是 $\geq 1$ 的值是可以 unbounded 的, 而 $<1$ 的值再怎么通过小次方变得更大, 也超不过 $1$. 因而 $|f(x)| \geq 1$ 的部分通常更能函数积分值的有限性, 除非在一个 measure infinite 的集合上 $|f(x)| <1$.

这里有一个更加严格的证明:
\begin{proof}
首先, 对于 $m = \infty$ 的 case, 如果 $f\in L^{m} = L^{\infty}$, 那么取任意 $1\leq n < \infty$ 都有: \[
\int |f|^n \leq \int \|f\|_\infty^n = \|f\|_\infty ^ n  \mu(X) < \infty
\]
 其次, 对于正常的  $m < \infty$ 的 case, 我们使用 Hölder: 
如果 $f\in L^m$, 那么对于任意 $n<m$, 我们可以构造出 Hölder conjugate $\frac{m}{n}$ 和 $\frac{m}{m-n}$,从而:
\begin{align*}
      \int |f|^n &= \int |f|^n \cdot 1 \\
      &\leq \bigg(  \int (|f|^n)^{\frac{m}{n}}  \bigg)^{\frac{n}{m}}\bigg(  \int 1^{\frac{m}{m-n}}  \bigg)^{\frac{m-n}{m}}\\
      & = \|f\|_m^n \mu(X)^{\frac{m-n}{m}} < \infty 
 \end{align*}
 从而 $$\| f\|_m <\infty \implies \|f \|_n < \infty$$
 这一 proof 利用 Hölder conjuate, 通过构造包含 $\frac{m}{n}$ 的 Hölder conjugate, 把 $ \int |f|^n$ 改成了 $ \|f\|_m$ 的 expression.
\end{proof}
以下是一个经典的例子:
\begin{example}
考虑\textbf{ measure finite 的 measure space $(0,1)$}: 通过经典的 Calculus 我们知道:
\[ f(x) = \frac{1}{x^m} \in L^p(0,1) \quad \text{ for all } p < \frac{1}{m}\]
但是对于任意的 $m$, 都有: \[
 f(x) = \frac{1}{x^m} \not\in L^p(0,1)
\]
而我们再看一个 \textbf{measure infinite 的 measure space $(1,\infty)$ 上的反例}, 采用同一个函数:
$$f(x) = \frac{1}{x^m},\quad x\in (1,\infty)$$
这个时候, $p$ 越大, $\int |f|^p =  \|f \|_p^p$ 反而越小, 通过经典的 Calculus 我们知道:我们知道
而对于 \[
f(x) = \frac{1}{x^m} \in L^p(1,\infty)\quad \text{for all } p > \frac{1}{m}
\]并且 $f \in L^\infty (1,\infty)$, 因为 $\|f\|_\infty  = 1$.\\
这个空间上的这个函数正对应了我们刚才讨论的, 如果有 infinite measure 数量的 $x$ 上 $|f(x)| <1$, 那么很可能 $\|f\|_{large}< \infty$ 是比 $\|f\|_{small}<\infty$ 更弱的条件
\end{example}


\subsection{control arbitrary $\|f \|_m $ 和 $\|f \|_n$ 的大小比例, in measure finite space}
\begin{remark}
刚才我们的推导中, \begin{align*}
      \int |f|^n \leq  \|f\|_m^n \mu(X)^{\frac{m-n}{m}} < \infty 
 \end{align*}
 两边开 $p$ 方, 可以得到一个不等式:  
 \begin{theorem}
     对于 measure finite space $X$, 对于任意的 $0< n\leq m \leq \infty$, 有: \[
     \|f\|_n \leq  \|  f \|_m \,\mu(X)^{\frac{1}{n} - \frac{1}{m}}
     \]\end{theorem}
 这也是一个有用的不等式. 它在 measure finite space 上, 对于任意的可测函数, 控制了两个任意的 function $p$-norm (虽然 for $p<1$ 不能严格地称为 norm) 之间的大小关系。
\end{remark}


\subsection{$(L^n\cap L^r) \subset L^m \subset (L^n + L^r)$, 对任意 $0< n < m < r \leq \infty$} 

\begin{proposition}
对于 measurable $f: X \to \mathbb{C}$, \[
t \mapsto \| f\|_{\frac{1}{t}} 
\]is \textbf{log-convex}.\\
equivalently 即: 对于任意的 $0< n < m < r \leq \infty$, 都有 \[
\|f \|_m \leq \|f \|_n ^\lambda  \cdot \| f\|_r ^{1-\lambda}
\]
where \[
\lambda := \frac{\frac{1}{m} - \frac{1}{r}}{\frac{1}{n} - \frac{1}{r}} \in (0,1),\quad i.e. \bigg(\frac{1}{m}\bigg) = \lambda\bigg( \frac{1}{n}\bigg) + (1-\lambda) \bigg(\frac{1}{r}\bigg)
\]
\end{proposition}
\begin{remark}
    log convex 即: 这个函数的 $\log$ 函数是 convex 的. 即对于任意 $x,y$, 以及 $[x,y]$ 上的任意一点, 即 $\lambda x + (1-\lambda) y $ for some $\lambda \in [0,1]$, 都有: \[
    \log f (\lambda x + (1-\lambda) y) \leq \lambda  \log f(x) + (1-\lambda) \log f(y)
    \] 即: \[
    f (\lambda x + (1-\lambda) y)  \leq f(x)^\lambda f(y)^{1-\lambda}
    \]
例如: $e^x, e^{x^2}, x^x$ 都是 log-convex 的. convex 函数的几何意义是 \textbf{"函数值小于等于两端的线性插值"}, 中点值 $\leq $两端值的\textbf{算术平均}, 而 log-convex 函数的几何意义是: , 中点值 $\leq $两端值的\textbf{几何平均}.\\
这里, 两端点是 $\frac{1}{r} <\frac{1}{n}$, 而中间的取点则是 $\frac{1}{m}$. log convexity 性质表明: \[
\|f \|_m \leq \|f \|_n ^\lambda  \cdot \| f\|_r ^{1-\lambda}
\]
\end{remark}
\begin{proof}
    For $r = \infty$, then $\lambda = \frac{n}{m}$.\\
    Since \[
    |f |^m  =  |f |^n \cdot | f|^{m-n} \leq |f |^n \cdot \| f\|_\infty ^{m-n} \quad a.e.
    \]
    可以得到 \[
    \int |f |^m \leq \bigg(\int |f|^n \bigg) \cdot \|f \|_\infty ^{m-n}  =  \|f ||_n ^n \cdot  \|f \|_\infty ^{m-n}
    \]
    从而 Taking $q$th root 得到结果:\[
    \| f \|_m \leq \|f \|_n ^{n/m} \|f \|_\infty ^{1-n/m}
    \]
    For $r < \infty$: 我们采用 conjugate exponents: \[
    \frac{n}{\lambda m}, \frac{r}{(1-\lambda)m}
    \]
    这是因为:  \[
     \bigg(\frac{1}{m}\bigg) = \lambda\bigg( \frac{1}{n}\bigg) + (1-\lambda) \bigg(\frac{1}{r}\bigg) \implies 1 = \lambda\bigg( \frac{m}{n}\bigg) + (1-\lambda) \bigg(\frac{m}{r}\bigg)
    \]
    从而 Applying Hölder: \begin{align*}
         \int |f|^m &= \int |f|^{\lambda m  } |f|^{(1-\lambda)m}\\
         &\leq \bigg( \int {|f|^n}\bigg)^ {\frac{\lambda m}{n}} \bigg(\int |f|^r \bigg)^{\frac{(1-\lambda)m}{r}} \\
         & = \| f\|_n ^{\lambda m} \cdot \|f \|_r ^{(1-r) m}
    \end{align*}
    Taking $q$ th root 得到结果.
\end{proof}
\begin{remark}
    Hölder's ineq 仍然是这里重要的一步. 我们这里需要利用 convexity 表述中的 "point on a line segment" 条件来构造一个 conjugate.
\end{remark}
\begin{remark}
    此处我们可以由这个 proposition 直接得到一个推论: \

\begin{corollary}
    对于任意的 $0< n < m < r \leq \infty$, 都有  \[
   (L^n \cap L^r) \subset L^m
    \]
\end{corollary}
\end{remark}

\begin{example}
    令 $A$ 为任意集合, $0\leq p < q \leq \infty$, 有: \[
 \|f \|_q \leq \|f \|_p \quad \text{and thus} \quad \ell^p(A) \subset \ell^q(A)
    \]
    这是因为 \[
    \|f \|_\infty ^p  = \sup_\alpha |f(\alpha)|^p \leq \sum_\alpha |f(\alpha)|^p = \|f \|_p^p
    \]
    于是 for $q\not = \infty$ case  \[
    \|f \|_q \leq \|f \|_p^\lambda \|f\|_\infty^{1-\lambda }  \leq \|f\|_p
    \]
(另一 case, trivial.)\\
我们发现 $\ell^p$ 空间, $p$ 越小要求反而越严格. \\
这是因为 $\ell^p$ 空间中一个函数就是一个 seq, 其 $p$-norm 就是各项的 $p$ 次方和, 再开 $p$ 次方根.\\
对于一个 seq, 如果它的累和 series 收敛, 它的各项肯定是 \textbf{eventually 收敛的}, 那么这些除了有限项外的这些项的绝对值都是 $<1$ 的, 那么\textbf{ $p$ 越大,  它们 $p$ 次方和只会越小}. 这正对应了我们之前说的 \textbf{"$|f(x)| < 1$ 的点主导函数" 的情况.}\\\\
\end{example}

相对于这个inclusion 关系, 我们还有另外一个 inclusion 关系: 
\begin{proposition}{每个 $L^m$ 函数都是一个 $L^n$ 函数和一个 $L^r$ 函数的和 ($0< n < m < r \leq \infty$)}
 对于任意的 $0< n < m < r \leq \infty$, 都有  \[
   L^m\subset  (L^n + L^r)
    \]
\end{proposition}
这个 inclusion 关系有一种调和的感觉在里面. 它 roughly mean 给定一个函数, 它可以拆成一个更加容易积的函数和一个更加不容易积的函数, 并且我们很大程度上可以控制这两个函数的可积性.\\
但其实很简单, 就是用我们之前的 $|f(x)|<1$ 和 $\geq 1$ 的点作为区分, 把函数的定义域分成两部分. 如果 $|f|$ 的 m 次方是可积的, 那么更小的 $n$ 次方, 对于 $|f(x)| \geq 1$ 的部分肯定也是可积的; 更大的 $r$ 次方, 对于 $|f(x)|< 1$ 的部分肯定也是可积的;

\begin{proof}
Suppose $f \in L^m$. Let \[
E:= \{ x:|f(x)|>1\} 
\]
let \[
g: = f\chi_E,\quad h := f \chi_{E^c}
\]
于是 $g \in L^n$ for all $0< n \leq m$, $h \in L^r$ for all $r \geq m$ and $r= \infty$.
\end{proof}