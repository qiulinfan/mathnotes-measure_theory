\chapter{Problem solving}
Recall: Given mspace $(X,\mathcal{A},\mu)$ 以及 $f:X \to \mathbb{C}$ mble, 我们可以 define distribution function: \[
\lambda_f : (0,\infty) \to [0,\infty]
\]
by \[
\lambda_f (\alpha) = \mu(\{\mathbb{}|f| > \alpha \} )
\]
Chebyshevs ineq: \[
\lambda_f(\alpha) \leq  \bigg(\frac{\| f\|_p}{\alpha} \bigg)^p
\]for $0< p < \infty$.\\
Today: Problem Solving



\begin{proposition}
    对于任意 $0< p < \infty$, 我们有: \[
    \int_X |f|^p \, d\mu = \int_0^\infty p \alpha^{p-1} \lambda_f (\alpha ) \, d\alpha
    \]
\end{proposition}
左边是 integral on $X$, 右边是 integral on $\mathbb{R}$.

\begin{proof}
    Sketch: 
    Step 1: $f$ simple $\implies$ $|f|$ simple.\\
    
\end{proof}

Write \[
|f| = \sum_{j=1}^N c_j \chi_{A_j} 
\]where $A_j$ disjoint, $c_1 > c_2 > \cdots >  c_N > 0$
This implies: \[
\int |f| ^p \, d\mu = \sum_{j=1}^N c_j^p r_j,\quad r_j = \mu(A_j)
 \]
Then
\begin{align*}
\lambda_f (\alpha) = \begin{cases}
    \sum_{j=1}^N r_j,& 0 < \alpha < c_N\\
    \sum_{j=1}^{n-1} r_j,&c_n \leq \alpha  < c_{n-1}, 2\leq n \leq N\\
    0,& \alpha \geq c_1
\end{cases}    
\end{align*}
从而 
\begin{align*}
    \int_0^\infty p \alpha^{p-1} \lambda_f (\alpha) d\alpha & = (\sum_{j=1}^N r_j) \int_0^{c_N} p \alpha^{p-1} d\alpha + \sum_{n=2}^N (\sum_{j=1}^{n-1} r_j ) \int_{c_n}^{c_{n-1}} p\alpha^{p-1} \, d\alpha\\
    & = 
\end{align*}


Step 2: $f$ general.\\
Use: $\exists$ simple functions $g_n \geq 0$ s.t. $g_n \nearrow |f|$.\\
 MCT $\implies $ \[
 \int_X |f|^p \,d \mu = \lim_{n\to \infty} \int_X g_n^p \, d\mu
 \]
 Also, \[
\lambda_{g_n} \overset{\text{CFB} }{\nearrow  }  \lambda_f \quad \text{pointwisely on} (0,\infty)
 \]
从而  MCT $\implies $ \[
\lim_{n\to \infty} \int_0^\infty p\alpha^{p-1} \lambda_{g_n}(\alpha) \, d\alpha \to \int_0^\infty p \alpha^{p-1} \lambda_f (\alpha ) \, d\alpha
\]
$\lambda_f(\alpha) = \mu(\{ |f| > \alpha\})$, 以及 $\{ |f| > \alpha \} = \bigcup_1^\infty \{g_n > \alpha \} $ increasing union.




\begin{example}
    Let $f:[0,1] \to \mathbb{R}$ be abs ctn. Suppose $f(0)  = 0$ 以及 $f^1 \in L^2([0,1])$.\\
    Show that the limit \[
    \lim_{x \to 0^+} x^{-1/2} f(x)
    \]exists, 并 compute it.\\
    What could the limit be? Must be $0$.\\
\begin{solution}
    Use FTOC, can recover $f$ from $f'$.\\
    \[
    f(x) = f(0) + \int_0^x f'(t) \, dt,\quad 0\leq x \leq 1
    \]
    使用 Hölder with $p=q =2$ (Cauchy-Swartz): \[
    |f(x)| \leq \int_0^x |f'(t)| \, dt = \int_0^x |f'(t) | 1\, dt  \leq \bigg(\int_0^x |f'(t)|^2\bigg)^\frac{1}{2} x^{\frac{1}{2}}
    \]
    从而 \[
    x^{-1/2} |f(x)| \leq \int_0^x |f'(t) | ^2 \, dt 
    \]
    Use fact: \(    g = L^1 (X,\mathcal{A},\mu) \implies \forall \epsilon > 0, \exists \delta > 0\) s.t. for all $\mu(E) < \delta $ we have $\int_E |g| \, d\mu < \epsilon$.\\
  (Proof of this fact: use approx by simple functions 可得).\\
  然后 use approx by simple functions, apply to $g = |f'|^2$, $\mu = m$, $E = [0,x]$, 于是得到 \[
  \int_0^x |f'(t)| ^2 dt \overset{x\to 0}{\longrightarrow} 0
  \]
\end{solution}
\end{example}


\begin{example}
    Let $f: \mathbb{R}^n \to \mathbb{R}$ be a function.\\
    Assume: 对于 $\forall \epsilon > 0$, 都存在 Lebesgue mble functions $g,h\in L^1 (m)$  s.t. \[
    g(x) \leq f(x) \leq h(x) \quad \forall x \in \mathbb{R}^n
    \]并且 \[
    \int_{\mathbb{R}^n} (h-g) \, dm < \epsilon 
    \]
    Prove that: $f$ 也是 Lebesgue mble 的, 并且 $f\in L^1(m)$.\\
   \begin{proof}
By assumption: Given $k \in \mathbb{N}$, 存在 $g_k,h_k \in L^1(\mathbb{R}^n)$ s.t.  \[   g_k \leq f \leq h_k ,\quad \int (h_k - g_k) < \frac{1}{k}\]
Idea: $f = \limsup g_k = \liminf h_k$ ? \\
我们应该 try to prove: for a.e. $x$ 都有 $ 0\leq h_k(x) - g_k (x) \to 0$.\\
Use Fatou's Lemma: \[
\int \liminf_{k\to \infty} (h_k - g_k) \leq \liminf_{k\to \infty} \int (h_k - g_k)  = 0
\]而 $h_k - g_k \geq 0$, 因而 This means: \[
 \liminf_{k\to \infty} (h_k - g_k) = 0\quad \text{for a.e. } x
\]
且我们知道\[
\liminf_{k\to \infty} (h_k - f)\leq \liminf_{k\to \infty}  (h_k - g_k) = 0\quad \text{for a.e. } x
\]
从而 \[
f(x) = \liminf_{k\to \infty} h_k(x) \quad \text{for a.e. } x
\]
This proves that, $f$ is Lebesgue measurable.
   
   \end{proof}
\end{example}









\begin{example}
    Prove that: \[
    \lim_{n\to \infty} \int_E \sin (nx)\, dx = 0
    \]for every bounded Borel set $E \subset \mathbb{R}$.\\
\begin{proof}
    Step 1: $E = (a,b)$ 是一个 interval.\\
    \begin{align*}
        \int_E \sin (nx)\,d x  & = \bigg[ -\frac{1}{n} \cos (nx)   \bigg]_a^b
    \end{align*}
    从而 \[
       \bigg|   \int_E \sin (nx)\,d x  \bigg| \leq \frac{2}{n} \overset{n\to \infty}{\longrightarrow } 0
    \]
    Step 2: $E$ 是一个 finite union of disjoint open intervals.\\
    Same as Step 1.\\
    Step 3: General Case.\\
    Fix $\epsilon > 0$.\\
    Then by outer regularity: 存在 some $U$ 为 finite disjoint union of open intervals, 使得 \[
    m(U \Delta E ) < \epsilon
    \]
    从而 \[
\bigg| \int_E f_n = \int_U f_n   \bigg| < \bigg|\int_{U \Delta E} f_n\bigg| \leq m(U \Delta E) < \epsilon
    \]因而\[
 \bigg|\int_{ E} f_n\bigg|  <   \bigg|\int_{ U} f_n\bigg| + \epsilon
    \]for all $n$. 并且 By step 2: \[
    \limsup_{n\to \infty} \bigg | \int _U f_n \bigg| + \epsilon = 0 + \epsilon
    \]
    因而 \[    \limsup_{n\to \infty} \bigg | \int _E f_n \bigg| \leq \epsilon \]
    Since $\epsilon$ arbitrary, 得证.
\end{proof}
\end{example}






\begin{example}
    Let $E \subset \mathbb{R}$ be a Borel set, with $m(E) > 0$.\\
    Set $f: \mathbb{R}\to \mathbb{R}$ be mble, nonneg, 并且 $\int f > 0$.\\
    Prove that: 存在 $t \in \mathbb{R}$ s.t. $$\int_{E + t} f > 0$$
    \begin{proof}
\textbf{Claim 1: STS to assume $f$ simple.}\\
Proof of Claim 1: 对于 $f$, can find seq of simple functions $0 \leq f_n \leq f$, s.t. $f_n \nearrow f$.\\
By MCT, 
    \end{proof}
\end{example}