\chapter{ $ NBV \;\&\; AC $ 空间, 以及其上的 FTC for Lebesgue integral [Fol 3.5, finished] }
\section{$NBV$ 及其性质}
\begin{definition}{NBV}
    For $F:\mathbb{R}\to \mathbb{C}$, 我们定义: $F \in NBV$, if $F \in BV$ 且 $F$ right ctn, $F(-\infty) = 0$.
\end{definition}
\begin{remark}
这个要求中 $F(-\infty) = 0$ 这一条并不要紧, 因为我们知道 for $F\in BV$, $F(-\infty) = c$ for some const $c$ 是一定的 (因为 $T_F(-\infty) =0$), 因而 right ctn 的 $F\in BV$ 减去一个常数一定是 $NBV$ 的.
    $F \in NBV$ 的
\end{remark}

\begin{proposition}
    $NBV \subset BV$ 是一个 linear subspace.
\end{proposition}
\begin{remark}
我们容易发现\[
    F \in BV \iff T_F \in BV \iff F_+, F_-  \in BV
    \]
又因为, $T_F (-\infty) = 0$ 对于 $F\in BV$ 总是成立, 且我们知道 $F$ right ctn $\implies T_F $ right ctn; 于是: \[
    F \in BV \text{ and right ctn} \implies T_F \in NBV \implies  F_+, F_- \in NBV
    \]
\end{remark}



我们已经知道:
$$\{\text{positive regular Borel measures on }\mathbb{R}\} \simeq \{\text{distribution functions }F:\mathbb{R}\to\mathbb{R} \}$$
Notice: 这一点 is achieved by 
$$
F_\mu(x ) := \begin{cases}
    \mu((0,x]) \quad  , x \geq 0 \\
     -\mu((x,0]) \; , x < 0
\end{cases}
$$
等同于: $$\mu_F ((a,b]) = F(b) - F(a)$$注意: positive regular Borel measure 和 distribution functions 都有一个共同点: 它们在 bounded set 上是 bounded 的, 但是整体可以 unbounded.\\

而现在我们证明:
\subsection{$\{\text{complex Borel measures on }\mathbb{R}\} \simeq NBV$}
注意: 一个 complex Borel measure 和一个 $F \in NBV$ 都是 finite 的.

\begin{theorem}{$\{\text{complex Borel measures on }\mathbb{R}\} \simeq NBV$}
\begin{enumerate}
    \item 对于 $\mathbb{R}$ 上的 complex measure $\mu$,  defining \[F(x) : = \mu((-\infty,x])\] 则有: \[F \in NBV\]
    \item 对于 $F \in NBV$, 一定存在某个 unique complex measure $\mu_F$ on $\mathbb{R}$, 使得 \[ \mu((-\infty,x]) = F(x) \quad \forall x \]
\end{enumerate}
\end{theorem}
\begin{proof}
    (1): 当 $\mu$ 是 positive 的情况下, 是显然的. complex 的情况就是 re/im 部分分别叠加即可.\\
    (2): 同样, WLOG 我们可以假设 $F$ 是 real-valued 的.\\
    $F\in NBV\implies F = F_+ - F_-$, 这两个都是 bounded increasing functions 且 NBV, 从而存在两个 finite signed measure 满足: \[  \mu_{\pm}((-\infty,x])  = F_{\pm}(x) - F_{\pm}(-\infty)  \] 再定义 \[\mu : = \mu_+ - \mu_-\] 即可
\end{proof}
\begin{remark}
    这里其实和 positive regular measure 关联 distribution function 的 way:
$$
F_\mu(x ) := \begin{cases}
    \mu((0,x]) \quad  , x \geq 0 \\
     -\mu((x,0]) \; , x < 0
\end{cases}
$$
是等价的, 只不过\textbf{差了一个常数 $\mu((-\infty,0])$} 而已. \\
之所以我们这里可以直接定义 \[F(x) : = \mu((-\infty,x])\]是因为, \textbf{对于 complex measure (thus finite) 而言, 这个常数 $\mu((-\infty,0])$ 一定是 finite 的. 但是对于 regular positive regular measure 而言, 这个常数可能是无穷} (且很可能). 因而对于 regular positive regular measure 我们采用这种迂回的定义方式来避开无穷值, 但是对于 complex measure 我们可以直接爽快地定义\[F(x) : = \mu((-\infty,x])\]
\end{remark}


\begin{theorem}{$\mu_F$ 的 total variation measure $=$ $\mu_{T_F}$}
    对于任意的 $F \in NBV$, 我们有: \[|\mu_F| = \mu_{T_F} \] Specially when $F$ 是 real-valued 情况下, 那么 $\mu_F$ 是一个 finite positive measure, 且有 \[\mu_{\pm} = \mu_{F_{\pm}}\]
\end{theorem}



Now: Given $F\in NBV$ with associated c.m. $\mu_F$, 什么时候 $\mu_F \perp m$, 什么时候 $\mu_F \ll m$?

\begin{theorem}{characterization of $\mu_F \perp m$ 和 $\mu_F \ll m$, for $F\in NBV$ }
    对于 $F \in NBV$, 我们已经知道: $F'$ $m$-a.e. 存在, 且 $F'\in L^1(m)$.\\
    Now we claim, 有: \[
    \mu_F \perp m \iff F' = 0 \quad m\text{-a.e.}
    \]以及 \[
        \mu_F \ll m \iff F(x) = \int_{-\infty}^x F'(t)\,d t \quad \forall x
    \]
\end{theorem}
\begin{proof}
Let $x\in \mathbb{R}$.  Applying LDT and LRNT, with $E_r: = (x,x+r]$: \[
    \lim_{r\to 0} \frac{\mu_F(E_r)}{m(E_r)} = \lim_{r\to 0} \frac{F(x+r)- F(x)}{r} = F'
    \]
    因而 $F'$ 就是这个 RN derivative. 对于 \[
    \mu_F = \lambda + \rho
    \]
    where $\lambda \perp m, \rho \ll m$, 我们有: \[F(x) : = \mu_F((-\infty,x]) =\lambda ((-\infty,x]) +\rho ((-\infty,x]) \]
我们知道, $    \mu_F \perp m \iff \rho = 0$, 从而 by LDT meets LRNT, we know that \[
F' = 0 \,\, a.e.
\]
而   $    \mu_F \ll m \iff \lambda = 0$, 从而直接 : \[F(x) : = \rho ((-\infty,x]) = F'dm((-\infty,x]) =   \int_{-\infty}^x F'(t)\,d t \]
\end{proof}



\section{$AC$ 及其性质}
\begin{definition}{absolutely continuous function}
我们定义 $F: \mathbb{R}\to \mathbb{C}$ 是 absolutely ctn 的, if 对于任意 $\epsilon > 0$ 都存在 $\delta > 0$ 使得对于任意的 disjoint intervals $(a_1,b_1),\cdots, (a_N,b_N)$, 都有: \[
\sum_1^N |F(b_j) - F(a_j)| <\epsilon \quad  \text{whenever}\quad \sum_1^N |b_j - a_j| <\delta
\]
\end{definition}
\begin{remark}
    absolutely continuous 是比 uniformly continuous 严格更强的条件: 我们只考虑 $N=1$ 而非任意正整数时这个 def 就 reduce 为 uniform ctn.\\
     absolutely continuous  表示了一种更强的控制性: 选取任意一些地方的变化足够小的 $x$, 其引发的 $y$ 的变化一定可控. ($y$ 的变化的可控性完全由 $x$ 的变化量决定, 不由 $x$ 的位置决定)\\
     这一看就和 measure 有关系.
\end{remark}
\begin{definition}{absolutely continuous function on a cpt interval}
我们定义 $F:I \to \mathbb{C}$ 是 absolutely ctn 的, if 对于任意 $\epsilon > 0$ 都存在 $\delta > 0$ 使得对于任意的 disjoint intervals $(a_1,b_1),\cdots, (a_N,b_N) \subset I$, 都有: \[
\sum_1^N |F(b_j) - F(a_j)| <\epsilon \quad  \text{whenever}\quad \sum_1^N |b_j - a_j| <\delta
\]
\end{definition}

\begin{lemma}{$F\in NBV$ abs ctn $\iff$ $\mu_F \ll m$}
    对于 $F\in NBV$, $$F \in AC\iff \mu_F \ll m$$
\end{lemma}
\begin{proof}
    我们 recall, abs ctn 除了 "$m$ 的 nullsets 也一定是 $\mu_F$ 的 null sets" 之外, 还有另一个 characterization: $\mu_F \ll m$ 当且仅当对于任意 $\epsilon > 0$ 都存在 $\delta > 0$ 使得 $ m(E) < \delta \implies|\mu_F(E)| < \epsilon$.\\
    显然, 这个 characterization 和这里的命题有关. 我们发现, $\mu_F  \ll m \implies F \in AC$ 直接 naturally follows from 这个 form. Let $\epsilon  >0$, 存在 $\delta$ 使得 $ m(E) < \delta \implies|\mu_F(E)| < \epsilon$. 那么考虑 $E = \bigsqcup_1^N (a_j,b_j)$ with $m(E) < \delta $, 直接有\[
    |\mu_F(E)|  = \sum_1^N |F(b_j) - F(a_j)| < \epsilon
    \]
从而得证.\\
而反向, 我们考虑 $m(E) = 0$, 并利用 outer regularity 取一个逼近它的 open set (每个是 union of finite disjoint open intervals) seq, 逼近 $E$, with $m(U_1) < \delta$. 由 $F \in AC$ 可以得到 \[
\mu_F(U_j) \leq \mu_F(U_1) < \epsilon
\] for all $j$, 从而 $\mu_F(E) \le \epsilon$. 从而得证, since $\epsilon $ arbitrary.
\end{proof}


\subsection{FTC for Lebesgue integral on $\mathbb{R}$: requires $NBV + AC$}
\begin{corollary}{}
    如果 $f \in L^1(m)$, 那么 \[
    F(x) : = \int_{-\infty}^x f(t) \,d t \in NBV \cap AC,\quad     f = F' \quad a.e.
    \]
Conversely, 如果 $F \in NBV \cap AC$, 那么 \[
    F' \in L^1(m), \quad F(x) = \int_{-\infty}^x F'(t) \, dt
    \]
\end{corollary}
\begin{remark}
forward 方向即: $f\in L^1(m)$ 则它的累积函数 $\int_{-\infty}^x f(t) \,d t $ 具有良好的连续性和变差有界性, 从而满足 FTC: a.e. 有 \[
\frac{d}{dx} \bigg( F(x):=\int_{-\infty}^x f(t) \,d t  \bigg) = f(x)
\]
并且 for $  F(x) : = \int_{-\infty}^x f(t) \,d t $, 这个函数是 $NBV \cap AC$ 的. 这可以推出: \[
F(x) - F(y) = \int_y^x f(t) \, dt
\]

backward 方向即: 如果 $F$ 具有良好的连续性和变差有界性, 那么它的导数满足: \[F(x) = \int_{-\infty}^x F'(t) \, dt    \]
简而言之: \textbf{FTC-I for Lebesgue integral 在整个 $\mathbb{R}$ 上都成立当且仅当 $F \in NBV \cap AC$.}
\end{remark}


\subsection{FTC for Lebesgue integral on a cpt interval: 只需要 $AC$}
比起刚才的 FTC-I, FTC-II 的条件要宽松很多, 只需要 $F$ 在它需要被用到的 compact interval 上 AC 即可以. 这是因为, 我们不需要用到 NBV 只需要 BV, 并且在 cpt interval 上, AC 本身就可以推出 BV.
\begin{lemma}
    如果 $F\in AC([a,b])$, 那么 $F\in BV([a,b])$.\\
    即 \[
    AC([a,b]) \subset BV([a,b])
    \]
\end{lemma}
\begin{proof}
    这是显然的. 我们看到 $F \in AC([a,b])$ 的定义: on $[a,b]$ we have: \[
\sum_1^N |F(b_j) - F(a_j)| <\epsilon \quad  \text{whenever}\quad \sum_1^N |b_j - a_j| <\delta
\]
我们不妨考虑 $\epsilon = 1$, 然后可以划分 $[a,b]$ into 一个个总长度为 $\frac{\delta}{2}$ 的由 disjoint intervals 构成的块 (补空缺没事), 从而 Bound 住这个划分上的 variation by parition \(\sum_1^{N_0} |F(b_j) - F(a_j)| \). 而我们发现: 这个时候我们不论怎么 fine 这个划分, 每个块的总长度总归是不变的, 从而仍然可以使用原先的 bound.\\
我们 recall: for total variation, partition 的选取是 greddy 的. 从而这就足以得证.
\end{proof}
\begin{remark}
这里没有仔细证明, 但是理解这个 idea 即可. 这表现了 locally, $AC$ 是一个比 $BV$ 更强的条件, 因为 total variation 就是 sup of variations over 所有划分, 而 \textbf{$AC$ 的定义正好就是: 无视划分的方法, 只要这个集合的总长度小于 $\delta$, 它上面的 variation by partition 就要小于 $\epsilon$. }
\end{remark}

\begin{theorem}{FTC-II for Lebesgue integral on a cpt interval}
TFAE:
\begin{itemize}
    \item $F\in AC[a,b]$
    \item $F$ 是 diffble a.e. on $[a,b]$ 的, 且 $F'\in L^1([a,b],m)$, 且 \[F(x) - F(a) = \int_a^x F'(t)\, dt\]\textbf{for all $x\in [a,b]$}.
\end{itemize}
\end{theorem}
\begin{proof}
    首先, $F\in AC[a,b]\implies F \in BV[a,b] \implies F$ diffble a.e. on $[a,b]$. \\
    Notice that: 这里我们只考虑 $F|_{[a,b]}$, 于是我们可以将其他部分的值都设为 smooth 的, 并 normalize it: 把 $F(x) := 0 $ for $x< a$, $F(x): = b$ for $x > b$.\\
    又 $F \in AC \implies F$ right ctn for sure, 我们 then have: $F \in NBV$, 从而
\end{proof}



\subsection{characterization for Lipschitz ctn}
\begin{theorem}{characterization for Lipschitz ctn}
    对于 $F:\mathbb{R}\to \mathbb{C}$, we have: \[
    F \text{ Lipschitz ctn with const } M \iff  F \in AC \text{ and } |F'(x)|\le M \text{ a.e.}
    \]
\end{theorem}
\begin{proof}
见 HW 12.
\end{proof}

从而我们得到: ctnity 条件的递推关系: \[
\text{ Lipschitz ctn }\implies   \text{ abs ctn }\implies   \text{ uniformly ctn } \implies  \text{ ctn } 
\]
关于连续函数的可导性:  Lipschitz ctn 可以推得 a.e. diffble $+$ bounded derivative; abs ctn 可以推得 a.e. diffble 且在 cpt interval 上 derivative $L^1$; 而往后的 uniform ctn 则不蕴含可导性条件.


而关于和可导性紧密相关的变差性质: 
abs ctn 在 bounded interval 上是比 BV, NBV 更强的条件, 而在 $\mathbb{R}$ 上则并不是.

在 $\mathbb{R}$ 上, NBV + AC 的函数可运用 FTC. 而在 bounded area 上 AC 的函数就可以运用 FTC.