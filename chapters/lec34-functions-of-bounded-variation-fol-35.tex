\chapter{functions of bounded variation: $F\in BV$ [Fol 3.5]}
上一节课我们证明了 Monotone Differentiation Theorem: 它表明的是, 任何 $\mathbb{R}\to \mathbb{R}$ 的 non-decreasing function 都是 differentiable a.e. 的.\\
我们知道一个函数在一整个区间上 differentiable 其实是一个比价严格的条件, 但是 differentiable a.e. 的条件就略好达到一些.\\
Question: 如果一个函数在 $[a,b] $ 上 differentiable a.e., 那么 in a.e. sense, 可以在 $[a,b] $ 上定义它的 derivative $F'$. 那么, 是否一定有 \[
F(b)  - F(a)  = \int_a^b F'(x) \, dx
\]呢? 答案肯定是不一定的. 以下是三个反例: 1. Heaviside function; 2. Cantor function; 3.$F'(x) = 0$ a.e., but not $0$ on a null set.\\
这也很显然: 因为单点的值是无法控制的. 我们只能控制 in sense of a.e. , 因而有 outlier 的 $a,b$ 是很正常的.\\
我们之后将 revisit 这一问题, 给出这个等式成立的 condition.

接下来我们将


\section{total variation function $T_F$ of a function $F$}
\begin{definition}{total variation function}
    给定一个 function $F:\mathbb{R}\to \mathbb{C}$, 我们定义它的 total variation function $T_F$ 为:\begin{align*}
        T_F : \mathbb{R} &\to [0,\infty]\\
        x &\mapsto \sup \{     \sum_{j=1}^n | F(x_j) - F(x_{j+1})| : -\infty <  x_0 < \cdots < x_n  = x    \} 
    \end{align*}
\end{definition}
\begin{remark}
    Total variation 是一个很形象的定义. $T_F(x)$ 表示的是 $F$ 从 $-\infty$ 到当前 $x$ 的这段定义域上, 总的变化量. 它 count into 所有的变化, 包括离散的和连续的, 正方向的和负方向的.\\
    \pic[0.5]{assets/ch3-pics-Screenshot 2025-04-10 at 15.39.23.png}
\end{remark}

\begin{lemma}
 对于任意的 $F:\mathbb{R}\to \mathbb{C}$,  $T_F$ 都是 increasing 的; 并且对于任意 $a<b$, 有: \[
 T_F(b) =  T_F(a) + T_F(a;b)
 \]
 where \[
 T_F (a;b) = \sup\{\sum_{j=1}^n  | F(x_j) - F(x_{j+1})| : b =  x_0 < \cdots < x_n  = a \}
 \]
表示 $F$ 的定义域限制在 $[a,b]$ 上的 total variation. 
\end{lemma}
\begin{proof}
    显然, 由于 total variation 是 increasing 的,  \textbf{我们总是可以 greedyly 选择 partition}. 对于一个 partition, 总是可以插入一个中间点把它分成两半, 而这两半的 sub partition 的 total variation 的和 $\geq$ 原先的 partition 的 total variation.
\end{proof}

\section{space of functions of bounded variation: $BV$ 的基本性质 }

\begin{definition}{function of bounded variation}
 如果 $T_F(\infty) < \infty$, 我们称 $F:\mathbb{R}\to \mathbb{C}$ is \textbf{of bounded variation} 的, 写作 $F \in BV$.
\end{definition}

\begin{definition}{function of bounded variation on an interval}
如果 $T_F(a;b) < \infty$,  我们称 $F:\mathbb{R}\to \mathbb{C}$ is \textbf{of bounded variation} on $[a,b]$, 写成 $F \in BV([a,b])$.
\end{definition}

首先显然, $F\in BV$ 可以 reduce to real-valued 的情况来讨论.
\begin{proposition}\[
    F \in BV \iff \Re f \in BV \text{ and } \Im f \in BV
    \]
\end{proposition}


\subsection{$BV$ as a vector space }

\begin{lemma}{$BV$ 是一个 complex vector space}
    如果 $F,G \in BV$, 那么对于任意的 $a,b\in \mathbb{C}$, we have \[ T_{aF + bG}  \leq |a| T_F + |b| T_G   \]
    从而$$aF + bG \in BV$$
\end{lemma}
\begin{proof}
易得. 显然, 函数的 total variation 是线性可加的.
\end{proof}

\subsection{$F\in BV$ 的 $T_F$ 的 limit behavior}

我们知道,$F\in BV$ if $T_F(\infty) < \infty$. 而关于 $ T_F(-\infty)$, 同样有强结论:
\begin{proposition} \[
F \in BV \implies T_F(-\infty)  = 0
\]
\end{proposition}
\begin{proof}
    Let $\epsilon  > 0$.\\
    从而对于任意的 $x \in \mathbb{R}$, since $F\in BV$ 那么 $T_F$ bounded, $T_F(x)$ 是一个 real number.\\
    因而我们可以找到一组 partition points $x_0 < \cdots < x_n$ 使得 \[
    \sum_{1}^n |F(x_j)- F(x_{j-1})| \geq T_F(x) - \epsilon
    \]从而 \[
    T_F(x) - T_F(x_0) \geq T_F(x) - \epsilon
    \]从而 \[
    T_F(y) \leq \epsilon,\quad \forall y\leq x_0
    \]
Since $\epsilon  > 0$ arbitrary, 这证明了 $T_F(-\infty) = 0$
\end{proof}
\begin{remark}
$F$ bounded variation 的必要条件是它在 $x \to \infty$ 时, 截止 $x$ 处的 variation $\to 0$. 
\end{remark}


\begin{lemma}{$F\in BV$ right ctn $\implies T_F$ right ctn}
$F\in BV$ right ctn $\implies T_F$ 也 right ctn
\end{lemma}
\begin{proof}
Let $x\in \mathbb{R}, \epsilon > 0$.\\
Let \[
\alpha : = T_F(x+) - T_F(x)
\]WTS: $\alpha = 0$.\\
By right ctnity of $F$ 和 $T_F$ increasing, 我们可以选择 $\delta > 0$, 同时满足: $|F(x+ h) - F(x)|<\epsilon$, $T_F(x+h) - T_F(x+) < \epsilon$ whenever $0<h<\delta$.\\
Fix 一个满足 $0<h<\delta$ 的 $h$. 其后的证明见 Folland 104.
\end{proof}




\subsection{属于 $BV,BV(I)$ 的函数 }

\begin{lemma}{哪些函数一定 $BV$ or $BV(I)$}
\begin{enumerate}
    \item 如果 $F:\mathbb{R}\to \mathbb{R}$  bounded 且 increasing, 那么  $F \in BV$ 且 $T_F(x) = F(x) - F(-\infty)$.\\
    \item 如果 $F: \mathbb{R} \to \mathbb{R}$  是 Lipschitz countinuous 的, 那么$F \in BV(I)$ for 任意的 cpt interval $I$
    \item 如果 $F: \mathbb{R} \to \mathbb{R}$  是 differentiable 且 $F'$ bounded 的, 那么 $F \in BV(I)$ for 任意的 cpt interval $I$
\end{enumerate}
\end{lemma}
\begin{proof}
(1) trivial.\\
(2) by def: 考虑 Lipschitz const $M$, 则 $T_F(a;b) \leq M (b-a)$.\\
(3): 这是 (2) 的推论, 因为 recall: by MCT 可得: $F: \mathbb{R} \to \mathbb{R}$  是 differentiable 且 $F'$ bounded $\implies F$ Lipstchiz ctn.
\end{proof}

\begin{proposition}
以下是一些经典的函数的 variational behavior: 
\begin{enumerate}
    \item $f(x) = \sin(x)$: 属于 $BV(I)$ for 任意 cpt $I$, 但不属于 $BV$.
    \item $f(x) = x\sin \frac{1}{x}, f(0) = 0$:  \textbf{属于$BV(I)$ iff $0\not\in I$.}
    \item $f(x) = x^2\sin \frac{1}{x^2}, f(0) = 0$:   \textbf{属于$BV(I)$ iff $0\not\in I$.}
\end{enumerate}
\end{proposition}
\begin{proof}
  (1) 显然;
  (2),(3) 见 HW 11. 其实它们基本相同. 
 ($\implies$): if $0 \not\in I$ then $F \in BV(I)$. 是简单的, we differentiate $F(x)=x \sin (1 / x)$ for $x \neq 0$:
$$
F^{\prime}(x)=\frac{d}{d x}\left(x \cdot \sin \left(\frac{1}{x}\right)\right)=\sin \left(\frac{1}{x}\right)+x \cdot \cos \left(\frac{1}{x}\right) \cdot\left(-\frac{1}{x^2}\right)=\sin \left(\frac{1}{x}\right)-\frac{1}{x} \cos \left(\frac{1}{x}\right)
$$
在不含 $0$ 的区间上, 它是 bounded 的. 于是 by lemma 得证.\\
($\impliedby$): if $F \in BV(I)$ then  $0 \not\in I$. This is equiv to: if $0 \in I $ then $F \not \in BV(I)$.\\
Suppose $0 \in I= [a,b] $ then $a \leq  0$ and $b \geq  0 $, one of which is strict. WLOG we suppose $b > 0$. \\
我们的 idea 是 harmonic series. 考虑
$$
y_n:=\frac{1}{n \pi+\pi / 2} \rightarrow 0^{+}
$$
we have:
$$
F\left(y_n\right)=y_n \sin \left(\frac{1}{y_n}\right)=\frac{1}{n \pi+\pi / 2} \cdot \sin (n \pi+\pi / 2)
$$For odd $n$, $F(y_n) = \frac{-1}{n \pi+\pi / 2}$, for even $n$, $F({y_n}) =  \frac{1}{n \pi+\pi / 2}$. Since $b > 0$, for some $N_0$ we have $y_{N_0} < b$. Then we consider the partition: pick $N \in \mathbb{N}$, and use $ x_0 = 0,x_1 = y_{N_0 + N-1},x_2 =y_{N_0 +N-2},\cdots, x_{N} = y_{N_0},x_{N+1} = b$ as the partition points of $[0,b]$.\\
Then we have \[
\sum_{n=1}^{N+1} |F (x_n) - F(x_{n-1}) | \geq  \sum_{n=N_0}^{N_0 -2+ N}  \frac{1}{\pi n+\pi / 2} + \frac{1}{\pi (n+1)+\pi / 2} \geq 2\sum_{n=N_0}^{N_0 -2+ N} \frac{1}{\pi n+\pi / 2}
\]
As $N \to \infty$, this sum $\sum_{n=1}^{N+2} |F (x_j) - F(x_{j-1}) |  \to \infty$, by the harmonic series.
\end{proof}



\section{Jordan decomposition for $f\in BV$: $f = \frac{1}{2}(T_F+F) - \frac{1}{2}(T_F-F)$}

\begin{lemma}
   如果 real-valued  $F \in BV$, 那么 $T_F + F, T_F - F$ 都是 increasing 的.
\end{lemma}
\begin{remark}
    $T_F + F$ 即: $F$ increasing 的地方加倍 increasing, $F$ decreasing 的地方 const;\\
     $T_F - F$ 即: $F$ decreasing 的地方反向加倍 increasing, $F$ increasing 的地方 const;
     \pic[0.5]{assets/ch3-pics-Screenshot 2025-04-10 at 15.39.58.png}
\end{remark}
\begin{proof}
    任取 $x<y$.\\
    Let $\epsilon > 0$.\\
    Can find $x_0 <  x_ 1 < \cdots < x_N = x$, s.t. \[
    \sum_{1}^N |F(x_j) - F(x_{j-1}) | \geq T_F(x) - \epsilon
    \] 从而 
    \begin{align*}
T_F(y)  &\geq    \sum_{1}^N |F(x_j) - F(x_{j-1}) |  + |F(y) - F(x)| \\
&\geq T_F(x) - \epsilon + |F(y) - F(x)|
    \end{align*}
由于 $\epsilon > 0$ 任意, 可以得到: \[
 T_F(y) -  T_F(x)\geq  |F(y) - F(x)|
\]
因而: \[
( T_F(y)  - F(y)) -  (T_F(x)  -F(x)) \geq  |F(y) - F(x)| - (F(y) - F(x)) \geq 0
\]
\end{proof} 



\begin{theorem}{Jordan decomposition for $ F \in BV $}
  对于 $F: \mathbb{R}\to \mathbb{R}$ (注意是 real-valued): \[
  F \in BV \iff F \text{ 等于两个 bounded increasing functions 的差}
  \]  Specially, \[
    F \in BV \iff  T_F \pm F \text{ bounded}
  \]
  因而 for $F\in BV$, 我们总是可以把它写作 \[
  F = \frac{1}{2}(T_F+F) - \frac{1}{2}(T_F-F)
  \]
  where we call it as the\textbf{ Jordan decomposition} of $F\in BV$. 其中,  $ \frac{1}{2}(T_F+F)$ 被称为 $F$ 的 \textbf{positive variation}; $ \frac{1}{2}(T_F-F)$ 被称为 $F$ 的 \textbf{negative variation}.
\end{theorem}
\begin{proof}
    显然, $F\in BV \implies F$ bdd, 因为\[
    |F(y) - F(x) | \leq T_F(\infty) - T_F(-\infty)
    \]
    For $F\in BV$, we have $T_F(\infty) < \infty , \; T_F(-\infty)  = 0$.\\
    又 $F\in BV \implies T_F$ bdd by def, 我们得到: \[
        F \in BV \implies  T_F \pm F \text{ bounded}
    \]而反向 trivial (bounded function 的差仍然 bounded).
\end{proof}
\begin{remark}
我们可以通过 $F$ 和 $0$ 的 min, max 取到它的正负部分, 把它按正负部分分解成 \[
F = F^+ - F^-
\]
而这里, Jordan decomposition 则是按照它正负方向上的 variation 来分: \[
  F = \frac{1}{2}(T_F+F) - \frac{1}{2}(T_F-F)
\]
$F$ 的 \textbf{positive variation}, \textbf{negative variation} 和 total variation 一样也是很形象.
\begin{figure}
    \centering
    \includegraphics[width=0.5\linewidth]{assets/ch3-pics-Screenshot 2025-04-19 at 02.30.17.png}
    \caption{positive/negative variation} 
    \label{fig:positive/negative variation}
\end{figure}
我们把 \[
F_+ : = \frac{1}{2}T_F + F,\quad F_- : = \frac{1}{2} T_F - F
\]
(这和正负部分的拆分的记号差别在正负号的上下.)
\end{remark}

\subsection{corollaries of Jordan decomposition}
\begin{corollary}
Let $F \in BV$. By Jordan decomposition,  $F$ 等于两个 bounded increasing functions 的差.从而我们\textbf{ by MDT 得}: 
    \begin{itemize}
        \item $F(x+), F(x-)$ 存在 for all $x$; $F(\pm\infty)$ 也存在.
        \item $D_F = \{x: F \text{ disctn at }x \}$ 是 at most ctbl 的.
        \item 定义 $G (x) : = F(x+)$, 则 $F,G$ 都 a.e. differentiable 且 $F' = G'$ $m$-a.e.
    \end{itemize}
\end{corollary}
这里最重要的是: \[
F \in BV \implies F \text{ a.e. differentiable}
\]