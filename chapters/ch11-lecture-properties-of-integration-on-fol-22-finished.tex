\chapter{properties of integration on $L^+(\mu)$ [Fol 2.2, finished]}
\section{Fatou's Lemma}
\begin{theorem}{Fatou's Lemma}
    \label{Fatou's Lemma}
 令 $(f_n)$ be a seq of functions in $L^+(\mu)$, then $$ \liminf_n \int f_n \geq \int \liminf_n f_n$$
\end{theorem}

\begin{proof}
    Set \[g_n := \inf_{m \geq n }     f_n   \]于是 \[g_n \nearrow \liminf_n f_n\]
于是\textbf{ by MCT}, we have: \[  \lim_n \int g_n =  \int \lim_n g_n =      \int \liminf_n f_n  \]
By def, 我们有 $g_n \leq f_n \;\;\forall n$, 于是 by monotonicity, $\int g_n \leq \int f_n$. 因而 \[  \liminf_n \int f_n \geq \liminf_n \int g_n  = \lim_n \int g_n = \int \liminf_n f_n   \]
\end{proof}
\begin{remark}
    对于 increasing 的从而有 limit 的可测 $(f_n)$, 我们可以使用 MCT.

    但是对于任意的可测 $(f_n)$, 我们无法使用 MCT, 不过有弱化的版本 Fatou's Lemma. 它表示\textbf{下极限的积分 小于等于 积分的下极限}.

    这是一个符合直觉的事情, 因为取函数的 pointwise 极限是一个很容易极端的事情. 
    
    积分的极限是一个 numerical seq 的极限, 比较 robust. 而函数的逐点极限是一个比较不稳定的事情, \textbf{在对函数逐点极限的过程中, 它的 "质量" 会存在一个比较大的损失, 因为其中可能包含了 uncountably many 个点的函数值的逐点极限的累积, 而积分的极限只是单个点的逐点极限. 因而大小关系很显然. }
\end{remark}


\begin{example}
    取 $(\mathbb{R},\mathfrak{L}, m)$, 考虑 $L^+(m)$ 上的函数, 即非负 Lebesgue 可测函数.

    下面有几个非常经典的 Fatou's Lemma 的例子:

\noindent 1. \textbf{escape to hat}: \[f_n = \chi_{(n,n+1)}\]
$f_n$ 在 $\mathbb{R}$ 上平移

   \noindent 2. \textbf{escape to width}: \[f_n = \frac{1}{n}  \chi_{(0,n)}\]
    $f_n$逐渐变得平坦
    
 \noindent   3. \textbf{escape to height}: \[ f_n = n \chi_{(0,\frac{1}{n})}\]
$f_n$ 逐渐变成一根针.

这三个例子中都有 $f_n \rightarrow 0$ pointwisely. 因而 $$\int \lim f_n = 0$$, 而 $$\lim \int f_n = 1$$, 因为对于所有 $f_n$ 都有 $\int f_n = 1$
\end{example}


\section{Chebyshev's inequality with corollaries}

\begin{lemma}{Chebyshev's inequality}
\label{Chebyshev's inequality}
对于 measure space $(X,\mathcal{M}, \mu)$, 如果 $f \in L^+(\mu)$ 并且 $c > 0$, 那么
\[
\mu\{ f \geq c\}  \leq \frac{1}{c} \int f
\]
\end{lemma}\begin{proof}
  Let $E := \mu\{f \geq c\}$
    \[
    \int f \geq \int f \chi_{E} \geq \int c \chi_E = c \int \chi_E = c\mu(E)
    \]
\end{proof}
\begin{remark}
    一个可测集的测度, 就等于 constant 1 在它上面的积分, by definition.

    这是一个简单而常用的结论.
\end{remark}








\begin{proposition}{非负函数积分为 0 等价于几乎处处为 0}
    令 $f \in L^+(\mu)$, 有:
    \begin{center}
        $\displaystyle \int f = 0$ $\Longleftrightarrow$ $f = 0$ a.e. (即只在一个零测集上非 0)
    \end{center}
\end{proposition}
\begin{proof}
    forward direction: directly follows from Chebyshev: set $A_n := \{f \geq \frac{1}{n}\}$, 对于任意 $n$ 都有 $\mu(A_n) \leq n\int f = 0$. 从而 by ctn from below, $>0$ 处构成零测集.
    
    backward direction: 对于 simple function, trivial by 积分的定义; 对于 general $f$, 通过 limit 得到 (它下方的所有 simple functions 也 a.e. 为 0 从而积分为 0).
\end{proof}






\begin{corollary}{几乎处处相等的非负函数积分相等}
Let $f,g \in L^+(\mu)$ 且 $f = g$ a.e., 则有  \[ \int f = \int g\]
\end{corollary}
\begin{proof}
    Set $D : = \{ x \mid f(x) \not = g(x)\}$, 则 $\mu(D) = 0$ by def
    \[
    \int f = \int_{D} f + \int_{D^c} f = 0 + \int_{D^c} g = \int g
    \]
\end{proof}





\begin{corollary}{liminf version of MCT}suppose $(f_n)_{n\in\mathbb{N}} $ 是一个 seq of functions in $L^+(\mu)$, 且 $f_n \rightarrow f \in L^+(\mu)$, 则:
$$
\liminf_n \int f_n \geq \int f
$$
\end{corollary}
\begin{proof}
    这是一个条件稍微弱化的 MCT: 把 $f_n\nearrow f$ 的条件改成了 $f_n \rightarrow f$ a.e., 得到的结论也稍弱化.\\
   \textbf{ modify $f_n$ and $f$ on a null set} (thus without chaning the integral) 后, follows directly from \textbf{Fatou's lemma}, 
\end{proof}


\begin{theorem}{积分收敛 $\implies$ 发散点集零测, 以及 support $\sigma$-finite}
    如果 $f\in L^+(\mu)$ 且 $|\int f| < \infty$, 则有: \[   \mu( \{x\in X \mid f(x) = \infty\} ) = 0\] 并且 \[ \{ x \mid f(x) > 0  \}\] is \textbf{$\sigma$-finite}
\end{theorem}

\begin{proof}
    直接 follows from Chebyshev. 取 \[ A_t := \{ x \mid f(x) \geq t\}\] for $t > 0 $.

于是: \[ \{x\in X \mid f(x) = \infty\}  = \bigcap_{n=1}^\infty A_n  \]
By Chebyshev, each $A_n$ 都有: $\mu(A_n) \leq \frac{1}{n} \int t$, 从而 by continuous from above 可得这个交集的 measure 为 0.

又有:\[ \{x\in X \mid f(x) >0 \} = \bigcup_{n=1}^\infty A_{\frac{1}{n}} \]
其中, each set has measure $\leq n\int f \leq \infty$. By def, 这个集合 $\sigma$-finite.
\end{proof}