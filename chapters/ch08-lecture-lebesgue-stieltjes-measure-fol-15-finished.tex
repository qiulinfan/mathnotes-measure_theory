\chapter{Lebesgue-Stieltjes measure [Fol 1.5, finished]}
给定一个 increasing 且 right ctn 的函数 $F$, 我们已经展示了用它作为 distribution function 来 induce 出一个 regular Borel measure $\mu_F$ on $\mathcal{B}(\mathbb{R})$. \\
在构造这个函数时, 我们使用的是用 premeasure $\mathcal{A}_0$ (of all finite unions of h-intervals), 使用 Hahn-Kolmogrov 来 induce outer measure $\mu_F^*$, 再把 restrict 它到 $<\mathcal{A}_0>$, 即 $\mathcal{B}(\mathbb{R})$ 上, 获得的 measure. \textbf{这一个 measure 是一个 Borel measure, 但是它并不 complete. }\\
recall in lec 6: 我们其实可以 complete 这个 measure, 只需要在第二步, 用 premeasure $\mathcal{A}_0$ induce 出 outer measure 后, 不要 restrict 它到 $\mathcal{B}(\mathbb{R})$ 上, 而是 restrict 到取 $\mathcal{M}_\mu := \{ \text{all } \mu_F^* \text{-measurable set\}}$ 上, 得到的就是 completion of $\mu_F$, 即 $$(\mathbb{R}, \mathcal{M}_\mu, \overline{\mu_F})$$
其中, $\mathcal{A}^*$ 是 $<\mathcal{A}_0>$ 即 $\mathcal{B}(\mathbb{R})$ 的 proper super set. \textbf{我们把这个 completed measure 叫做 Lebesgue Stieltjes measure associated with $F$, 并用 $\mu_F$ 来指代它. (刚才, 我们把未完备的 measure 叫做 $\mu_F$, 但现在我们不再使用这个 measure, 而是使用它的 completion, 并转而称它的 completion ($\overline{\mu_F}$) 为 $\mu_F$.)}
$$
\text { Regular Borel measure } \xrightarrow{\text { completion }} \text { LS measure }
$$

\begin{definition}{Lebesgue-Stieltjes measure associated with $F$}
给定一个 distribution function $F$, 我们使用它来定义 h-intervals 的 premeasure $\mu_0$, 并把这个 premeasure induce 出的 outer measure $\mu^*$ 限制在 $$\mathcal{M}_\mu := \{ \text{all } \mu^* \text{-measurable set\}}$$ 上, 由 Carathéodory Thm 得它是 complete 的. 称这个 complete 的 measure $$\mu_F := \mu^* |_{\mathcal{M}_\mu}$$ 为 \textbf{Lebesgue Stieltjes measure associated with $F$. }
\end{definition}
\begin{remark}
根据定义, 对于任意 $E \in \mathcal{M}_\mu$, 它的 LS measure 为:$$
\mu_F(E) = \inf \{ \sum_1^\infty (F(b_i) - F(a_i))   \mid E \sub \bigcup_1^\infty (a_i, b_i]      \}
$$
\end{remark}

\section{inner and outer regularity of LS measure}

虽然我们使用 h-intervals 来 induce 了这个 measure, 但是实际上我们在表示 measure 时,可以用 open intervals 来代替 h-intervals:
\begin{lemma}{open intervals can substitute for h-intervals when computing measure}
固定一个 Lebesgue-Stieltjes measure associated with $F$, 任意 $E \in \mathcal{M}_\mu$, 它的 measure 等于:$$
\mu_F(E) = \inf \{ \sum_1^\infty (F(b_i) - F(a_i))   \mid E \sub \bigcup_1^\infty (a_i, b_i)      \}
$$
\end{lemma}
\begin{proof}
    每个 open interval 都等于 a ctbl disjoint union of h-intervals, 从而是在这个被取 inf 集合内的; 所以只需要证明能取到这个 inf 即可.
    Fix $\epsilon > 0$, 我们根据定义可以取到一个 seq $(a_i, b_i]$ 使得它 measure sum $\leq \mu(E) + \epsilon /2$, 而我们对于每个 $i$, 在 interval 的右边再取一个 $ <\epsilon/2^{i+1}$ 的 $\delta_i$, 就变成了一个 open interval, 并且最后距离这个 h-interval seq 的 measure sum 差距至多 $\epsilon/2$. 从而得证.
\end{proof}


\begin{theorem}{\textbf{outer regularity}}
\label{outer regularity}
对于一个 Lebesgue-Stieltjes measure associated with $F$, 任意 $E \in \mathcal{M}_\mu$, 它的 measure 等于:
\begin{equation}
    \mu_F(E) = \inf \{ \mu_F(U) \mid  U \text{ open , and } E \sub U  \}
\end{equation}
\end{theorem}


\begin{proof}
    Directly follows from lemma. 首先, by monotonicity, 一个包含 $E$ 的开集 $U$ 的 $\mu_F$ 一定比 $E$ 的大. 并且, 对于任意的 $\epsilon > 0$, 都可以找到一个 open covering 使得 measure sum $< \mu_F(E) + \epsilon$, by def.\\
\end{proof}

\begin{theorem}{\textbf{inner regularity}}
\label{inner regularity}
对于一个 Lebesgue-Stieltjes measure associated with $F$, 任意 $E \in \mathcal{M}_\mu$, 它的 measure 等于:
\begin{equation}
    \mu_F(E) = \sup \{ \mu_F(K) \mid  K \text{ compact , and } K \sub E  \}
\end{equation}
\end{theorem}
\begin{proof}
    首先证明 $E$ bounded 的 case. 假设 $E$ bdd. \\
    如果 $E$ closed, 则 $E$ cpt, trivially true. \\
    如果 $E$ open, 那么 $E$ 的 bounadry 是 closed (cpt) 的, 从而 $\partial E \in \mathcal{M}_\mu$ 
    我们 let $\epsilon >0$. 我们对 $\partial E$ 使用 outer regularity, 可以取一个 open set $U$ covering $\partial E$, 并且使得 $\mu_F(U) \leq \mu_F(E) + \epsilon$\\
    此时取 $K := E \setminus U$, 我们发现这是一个 approximating $E$ 的 compact set, 并且有:
    $$
    E = K \sqcup (U \cap E)
    $$
从而:
\pic[0.4]{assets/ch1-pics-image-20250131003019214.png}
而对于 unbounded 的 case, 直接由 
$$
E  = \bigsqcup_j (E \cap (j,j+1])
$$得到.
\end{proof}
\begin{remark}
outer / inner regularity 表示, $\mathbb{R}$ 上一个 (LS-measurable set 的) LS measure 就等于它内部用 cpt set 逼近它的 measure limit; 以及等于它外部用 open set 逼近它的 measure limit.\\
这个性质也可以推广到 $\mathbb{R}^n$ 上.
\end{remark}




\section{Lebesgue-Stieltjes measurable 的等价条件}
\begin{definition}{$G_\delta, F_\sigma$ sets}
Topological space 中, 一个 \textbf{coutable intersection of open sets 被称为一个 $G_\delta$ set}, 一个 \textbf{countable union of closed sets 被称为一个 $F_\sigma$ set}.
\end{definition}
\begin{remark}
    topological space 中, finite intersection of open sets 还是 open set, 但是 countable intersection 则未必; finite union of closed sets 还是 closed set, 但是 countable union 则未必.\\
$G_\delta$ sets 包括了所有的 open sets, 以及一部分扩充; $F_\sigma$ sets 包括了所有的 closed sets, 以及一部分扩充.
\end{remark}



\begin{theorem}{Lebesgue-Stieltjes measurable 的等价条件}
\label{Lebesgue-Stieltjes measurable 的等价条件}
TFAE:
\begin{enumerate}
    \item[i] $$E \in \mathcal{M}_\mu$$
    \item[ii] 存在一个 $G_\delta$ set $V$ 以及一个 measure zero set $N_1$ ($\mu_F(N_1) = 0$) 使得   $$E = V \setminus N_1$$
    \item[iii] 存在一个 $F_\sigma$ set $H$ 以及一个 measure zero set $N_2$ ($\mu_F(N_2) = 0$) 使得 $$E = H \cup N_2$$
    \item[iv] 存在一个 open set $U$ 使得对于任意的 $\epsilon > 0$, 都有 $$\mu^*(U \setminus E) < \epsilon $$
\end{enumerate}
\end{theorem}

\begin{proof}
由 (ii) 和 (iii) 推得 (i) 是 trivial 的. 这是因为 LS measure 是 complete measure, 任意 null set 都是 measurable 的.
由 (i) 推 (ii) 和 (iii): follows from outer 与 inner regularity. 假设 $E$ 是 LS-measurable 的, 我们直接取一个 inner seq of cpt subsets 以及一个 outer seq of open super sets, 使得
\begin{equation}
    \mu_F(U_j) -  \frac{1}{2^i} \leq \mu_F(E) \leq \mu_F(K_j ) + \frac{1}{2^i}
\end{equation}
于是就得到: $V := \intsec_i U_i$, $H := \bigcup_i K_i$, 与 $E$ 的差集都是一个 null set. 并且它们分别为 $G_\delta$ 和 $F_\sigma$ sets.
\end{proof}
\begin{remark}
    $\sigma$-algebra 和 topology 各自只 closed under finite 的交和并, 而 $<\mathcal{B}(\mathbb{R})>$ 则 closed under ctbl 交和并, 从而所有的  $G_\delta$ 和 $F_\sigma$ sets 都在其中. $\mathcal{M}_\mu$ 是一个比 $<\mathcal{B}(\mathbb{R})>$ 更大的集合, 但是其实它其中的元素都可以用 $G_\delta$ 和 $F_\sigma$ sets, 即 $<\mathcal{B}(\mathbb{R})>$ 中的集合来逼近. 这是合理的, 因为 completion 就是把一些 subnull sets 加入到了 $\sigma$-algebra 里.
\end{remark}





\section{Lebesgue measure and its invariance properties}
\begin{definition}{Lebesgue measure}
Lebesgue measure 即 Lebesgue-Stieltjes measure associated with $F(x) = x$.
我们用 $m:=\mu_F$ 来表示它, 并用 $\fL := \mathcal{M}_m$ 来表示所有的 Lebesgue measurable sets.\\
从而 $\mathbb{R}$ 上的 Lebesgue measure space 表示为:
$$
(\mathbb{R}, \fL, m)
$$
\end{definition}
\begin{remark}
    Lebesgue measure 是最 normal 的 Lebesgue-Stieltjes measure, 它 preserve intervals 的长度作为其 measure: 
    $$
    m((a,b]) = b -a
    $$
\end{remark}


\begin{theorem}{$\fL$ preserves translation and scaling}
    if $E \in \fL$ $\Longrightarrow$ $E+s, rE \in \fL$ $\forall s,r \in \mathbb{R}$.\\
    并且, $m(E+s) = m(E), m(rE) = |r| m(E)$
\end{theorem}
\begin{proof}
首先, 如果 $E \in \mathcal{B}(\mathbb{R})$, 那么 by hw 1, 我们证明了 $\mathcal{B}(\mathbb{R})$ 是 closed under translation 和 scaling 的, 因而 $rE, E+s \in \mathcal{B}(\mathbb{R})$.\\
我们 define on $\mathcal{A}_0:=$ $\{\text{finite union of h-intervals}\}$:
$$
m_s(E) := m(E+s)
$$
$$
m^r(E) := m(rE)
$$
显然, 这两个函数 agree with $m, |r|m$. \textbf{由于 $m$ 是 $\sigma$-finite 的, 从而 by Hahn-Kolmogrov, 它 uniquely extend to $\mathcal{B}(\mathbb{R})$}. 因而, $m_s$ 在 $\mathcal{B}(\mathbb{R})$ 上和 $m$ 相等, $m^r$ 在 $\mathcal{B}(\mathbb{R})$ 上和 $|r|m$ 相等. 并且, 我们知道\textbf{ $(\mathbb{R}, \fL, m)$ 是 completion of $(\mathbb{R}, \mathcal{B}(\mathbb{R}),m)$}, 因而 $m_s$ 也同样 complete to $m$ on $\fL$. (同理, $m^r$ 也同样 complete to $|r|m$ on $\fL$)
\end{proof}
\begin{remark}
    我们只要证明两个 measure function 在 premeasure 上相等或称倍数关系, 就能证明它们在 induced (complete) measure 上相等.\\
此外, 有另一种证明方式. After we know $m_s$ 在 $\mathcal{B}(\mathbb{R})$ 上和 $m$ 相等, $m^r$ 在 $\mathcal{B}(\mathbb{R})$ 上和 $|r|m$ 相等, 我们由 \ref{Lebesgue-Stieltjes measurable 的等价条件} Lebesgue-Stieltjes measurable 的等价条件 可知: $\fL$ 上的集合一定是一个 Borel set 并上一个 null set, 由于 null set 的 measure 在经过 translation 和 scaling 后仍然是 0, 同样得证.
 \end{remark}