\chapter{$\sigma$-algebra [Fol 1.2]}
\noindent 我们 (见 my Math 395 notes) 已经证明: 在 $\mathbb{R}$ 上不存在一个 measure function $\mu : \cP(\mathbb{R}) \rar [0,\infty]$ satisfying:
\begin{enumerate}
    \item $\mu(\emptyset) = 0$;
    \item translate invariant
    \item countably additivite
\end{enumerate}

\noindent 因而, 对于比如 $\mathbb{R}$ 的这种无法在其幂集上定义良好的 measure function 的集合, 我们要定义一个 $\mathcal{A} \sub \cP(X)$, 使得我们能在这个 power set 的子集上, 定义一个 make sense 的 measure.\\\\
\noindent 
首先, 为了对于一个任意的集合 $X$ 都能在其上定义 measure, 我们要考虑在 $X$ 的一个什么样的子集簇上有希望定义这样的 measure. 

\begin{definition}{algera, $\sigma$-algebra}
对于 set $X$, $S \sub \cP(X)$ 被称为 $X$ 上的一个 $\sigma$-algebra, if 其满足:
\begin{enumerate}
    \item $\emptyset \in X$;
    \item \textbf{closed under complement}: if $E \in S$ then $X \setminus E \in S$;
    \item \textbf{closed under countable union}: if $E_1,E_2,\cdots \in S$ then $\bigcup_{k=1}^\infty E_k \in S$.
\end{enumerate}
如果第三条并不满足, 而是只满足 \textbf{closed under finite union}, 则称 $S$ 是 $X$ 上的一个 algebra. 当然, $\sigma$-algebra 是比 algebra 严格更强的条件.
\end{definition}

\noindent 我们定义 $X$ 的一个子集簇为一个 $\sigma$-algebra 如果它包含空集并 closed under complement and countable union. 但这并不是 $\sigma$-algebra 的全部性质. 这三个性质还蕴涵了: $\sigma$-algebra 也一定包含 $X$, 且 \textbf{closed under set difference}, \textbf{symmetric difference} 以及 \textbf{countable intersection}. \\
\noindent \textbf{对于 algebra, 它也有以上的所有性质的 finite version.}


\begin{theorem}{$\sigma$-algebra also closed under set difference, symmetric difference and countable intersection}
    Let $S$ be a $\sigma$-algebra on set $X$.\\
    Claim: 
    \begin{enumerate}
        \item $X \in S$
        \begin{proof}
            Directly from def.
        \end{proof}
        \item $D,E \in S \implies D\cup E, D\cap E, D \setminus E \in S$
        \begin{proof}
            union: from def by leaving others as $\emptyset$; \\intersection: $$(D\cap E)^C = D^C \cup E^C \in S$$
            setminus: $$ D\setminus E = D \cap (X \setminus E) \in S$$    
        \end{proof}
        \item $D,E \in S \implies D \Delta S\in S$
        \begin{proof}
        $$
        D \Delta E = (D \setminus E) \bigcup (E \setminus D)
        $$
        \end{proof}
        \item $A_1,A_2,\cdots \in S \implies \intsec_{i=1}^\infty A_i \in S$
        \begin{proof}
            $$(\intsec_{n=1}^\infty )^C = \bigcup_{n=1}^\infty E_n ^C \in S  $$
        \end{proof}
    \end{enumerate}
\end{theorem}

\begin{remark}
    我们发现 $\sigma$-algerbra 很像是 topology. 实际上 $\sigma$-algerbra 和 topology 的区别就是: $\sigma$-algebra 只保证了 closed under countable union 而 topology closed under any union; topology 只保证 closed under finite intersection 而 $\sigma$-algebra closed under countable intersection.
\end{remark}


\begin{lemma}{任意 $\sigma$-algebra 的 intersection 仍是 $\sigma$-algebra}
Let $\{S_\alpha \}_{\alpha \in A}$ be a collection of $\sigma$-algebra on $X$, then $\intsec_{\alpha \in A} S_\alpha$ is a $\sigma$-algebra on $X$. 
\end{lemma}
\begin{proof}
    这是个 trivial proof. 但是它具有一定理解上的启发.\\
    我们对 $\sigma$-algebra 有一个直观理解: 如果我们想把一些集合做成一个 $\sigma$-algebra, 那么首先我们把它们的补集放进这个 $\sigma$-algebra 里, 其次我们把这些集合的 up to countable 的任意组合的并集也放进这个 $\sigma$-algebra 里.\\
    因而即便我们把一些 $\sigma$-algebra 给 intersect 起来, 其中每个集合的补集和这些集合的 up to ctbl 的任意组合的并集也在这个 intersection 里.\\
    这是个重要的直观理解. 我们想到, 如果我们要把一个 sigma-algebra 里的一部分去掉,并保持它仍然是一个 sigma-algebra,那么我们得把这些集合的补集, 以及能够 ctbly union 成这些集合的小集合也去掉, 并对这些小集合也 recursively 进行这个操作.
\end{proof}

\begin{corollary}{unique smallest $\sigma$-algebra containing a collection of subsets}
    Given $\varepsilon  \sub \cP(X)$
    $$
    <\varepsilon> := \intsec_{\varepsilon  \sub S \sub \cP(X), \newline S \text{ is }\sigma \text{ -algebra on } X} S
    $$
\end{corollary}

\begin{definition}{$\sigma$-algebra generated by a subset}
We call  $$
    <\varepsilon> := \intsec_{\varepsilon  \sub S \sub \cP(X), \newline S \text{ is }\sigma \text{ -algebra on } X} S
    $$ \textbf{the $\sigma$-algebra generated by $\varepsilon$ }
\end{definition}