\chapter{product measure and Fubini-Tonelli theorem}
\section{product space and product measure [Fol 1.2, finished; 2.5]}

Goal: Given $(X_i, \mathcal{A}_i, \mu_i)$, construct $X = \prod X_i $, s.t. 
\[
\mu(E_1 \times E_2 )  = \mu_1(E_1) \mu_2(E_2)
\]
So that we can do Fubini (iterated integration) like that in Riemann integral.


\subsection{product $\sigma$-algebra}
\begin{definition}{product $\sigma$-algebra}
Suppose $(X_i, \mathcal{A}_i)$ mble, $1 \leq i\leq n$, the product $\sigma$-algebra $A_1 \otimes \cdots \otimes A_n$ on $X_1 \times \cdots \times X_n$ is the smallest $\sigma$-algebra s.t. the\textbf{ coordinate map }$$ \pi_j: X_1 \times \cdots \times X_n \rightarrow X_j$$\textbf{is measurable.}\\
即 the $\sigma$-algebra generated by: \(
\{  \pi_\alpha (E_\alpha) : E_\alpha  \in \mathcal{A_\alpha} \}
\). \[
A_1 \otimes \cdots \otimes A_n : = <\{  \pi_\alpha (E_\alpha) : E_\alpha  \in \mathcal{A_\alpha} \}>
\]

\end{definition}


我们容易发现:
\begin{proposition}
$$A_1 \otimes \cdots \otimes A_n = <\{  E_1\times \cdots \times E_n \in \mathcal{A}_i \times \cdots \times \mathcal{A}_n  \}>$$
\end{proposition}
\begin{proof}
    By def 易得. (Prop 1.14 in book).\\
\end{proof}
\begin{remark}
  这里只考虑了有限情况, 但是无限的乃至于 ctblly 无限的相似. 取所有可能的 set product 作为 geneating 即可.
\end{remark}



\subsection{product Borel algebra $\subset$ Borel algebra of the product space}
\begin{proposition}
    If $X_1,\cdots, X_n $ are metric spaces. Let $X := X_1 \times \cdots \times X_n$ (with product metric), then: $$\bigotimes_{i=1}^n \mathcal{B}_{X_i} \subseteq \mathcal{B}_X$$
\textbf{and the equality holds if $X_i$ separable $\forall i$.}
\end{proposition}
\begin{proof}
   $$\bigotimes_{i=1}^n \mathcal{B}_{X_i}  \overset{\text{by prop}}{=} < \{ U_1\times \cdots \times U_n  \text{ each open} \} > \subseteq\mathcal{B}_X $$
   Now let $C_i \subseteq X_i$ dense, ctbl. \\
   Set $$\mathcal{E}_i  := \{  B_r(x) \mid x\in C_i \;,\; r\in \mathbb{Q}_{>0}  \} \subseteq \mathcal{B}_{X_i} $$
   Then: \textbf{every open set in $X_1 \times \cdots \times X_n$ is a ctbl union of products $B_1 \times \cdots \times B_n$}, each $B_1 \in \mathcal{E}_i$.
   Then we have: $$B_X = <\{B_1 \times ...\times B_n\}> \subset \bigotimes_1^n B_{X_i}$$
\end{proof}
\begin{remark}
    显然. \textbf{有限 index} 情况下, 在 product topology 中, \textbf{product of open sets仍然是 open set}, 但是 open sets 可以不止是 product of open sets 这些. 因而 \(\bigotimes_{i=1}^n \mathcal{B}_{X_i} \subseteq \mathcal{B}_X\). \\
 并且在 separable 的 topological space 上, 比如 $\mathbb{R}$, 我们有: $\mathbb{R}^n$ 中的任意 open set 都是 a ctbl union of open boxes. 于是 \(\bigotimes_{i=1}^n \mathcal{B}_{\mathbb{R}}  = \mathcal{B}_{\mathbb{R}^n}\)
\end{remark}

\begin{example}
    $$\mathcal{B}_{\mathbb{R}^n} = \mathcal{B}_{\mathbb{R}} \otimes \cdots \otimes \mathcal{B}_{\mathbb{R}}$$
\end{example}


\begin{corollary}
    if $(X,\mathcal{A})$ is a mble space, then $$f:X\rightarrow \mathbb{C}  \;\;(\mathcal{A},\mathcal{B}_{\mathbb{C}})\text{-mble } \Longleftrightarrow \Re f, \Im f \;\;\mathcal{A}\text{-mble}$$
\end{corollary}
\begin{proof}
    略.
\end{proof}

\subsection{construction of product measure}
下面我们构建 product measure: 
Let $(X_i \,,\mathcal{A}_i\, , \mu_i)$, $1 \leq i \leq n$ be mble spaces.\\
    And let $X := X_1 \times ... \times X_n$, $\mathcal{A}:= \mathcal{A}_i \otimes \cdots \otimes \mathcal{A}_n$
Goal: 通过 Hahn-Kromolgrov 来构建 product measure on product mble space.
Idea: Let $$\mathcal{A}' := \{   \text{all finite disjoint unions of rectangles (each measurable)} A_1 \times \cdots \times A_n\}$$
\textbf{Step 1: }
\subsection{all finite disjoint unions of rectangles as an algebra}
\begin{proposition}
   $\mathcal{A}'$ is an algebra.
\end{proposition}
\begin{proof}
The set $\mathcal{E}:= \{\text{rectangles}\} \subseteq \mathcal{P}(X)$ satisfies:
    \begin{itemize}
        \item $\varnothing \in \mathcal{E}$
        \item $E,F \in \mathcal{E} \implies E \cap F \in \mathcal{E}$
        \item $E \in \mathcal{ E} \implies E^c \text{ is a finite disjoint union of recs}$ (画图可知).\\\\
    \end{itemize}
\end{proof}

\textbf{Step 2: }
\subsection{各维度 measure 的 product 作为 rectangle 的 measure, 从而定义 premeasure}
Now define $$\mu' : \mathcal{A}' \rightarrow [0,\infty]$$ as follows: $$ \mu'(\bigsqcup_{k=1}^N E_1^{(k)} \times \cdots \times E_n^{(k)}) = \sum_{k=1}^N \prod_{i=1}^n \mu_i (E_i^{(k)}) $$
Claim 2:
\begin{proposition}
   \textbf{ (1) $\mu'$ is a well-defined premeasure on $\mathcal{A}'$.\\
    (2) If each $\mu_i$ is $\sigma$-finite, so is $\mu'$.}   
\end{proposition}
\begin{proof}
 Sketch: (2) DIY.
    (1) STS(check): if $E = E_1 \times \cdots E_n$ is a finite or ctbl union of rects $E^{(k)} = E_1^{(k)}\times \cdots \times E_n^{(k)} $, then $$\prod_1^n  \mu_i(E_i) = \sum_k \prod_1^n \mu_i(E_i^{(k)})  $$
 Use Tonelli for sums and integrals: \begin{align}
        \prod_1^n \chi_{E_i} (x_i) &= \chi_E (x_1\,, \cdots \,,x_n) \\
        &= \sum_k \chi_{E^{(k)}} (x_1 \,, \cdots \,, x_n) \\
        &= \sum_k \prod_1^n \chi_{E_i^{(k)}} (x_i)
    \end{align}
Integrate w.r.t. $x_1$: 
\begin{align}
\mu_1(E_1) \prod_{i=1}^n \chi_{E_i} (x_i)& = \int \sum_k \prod_1^n \chi_{E_i ^{(k)}} (x_i)    \\
&\overset{\text{Tonelli}}{=}  \sum_k  \int (\prod_1^n   \chi_{E_i ^{(k)}} (x_i) )  d \, \mu_1 (x_1) \\
& = \sum_k \mu_1(E_1^{(k)}) \prod_{i=1}^n \chi_{E_i^{(k)}} (x_i)
\end{align}
And repeat for $i=2,\cdots,n$.\\\\
    
\end{proof}

\subsection{HK extension of the premeasure as definition of product measure}
\textbf{Step 3: 现在已经有了 $\sigma$-finite 的 premeasure, 我们可以应用 HK Thm 构建出完整的 measure.}
Now use HK:
\begin{corollary}
    $\exists$ measure $\mu := \mu_1 \times \cdots \times \mu_n$ on $\mathcal{A} = \mathcal{A}_1 \otimes \cdots \mathcal{A}_n$ extending $\mu'$.\\
    (\textbf{And if each $\mu_i $ are $\sigma$-finite, then product measure 也 $\sigma$-finite, 从而 $\mu$ 是 unique extension.)}
\end{corollary}
由此, 我们从 $\mathcal{A}_1 , \cdots ,\mathcal{A}_n$ 的 measure 中构建出了它们的 product measure.\\\\


\subsection{associativity of product $\sigma$-algebra and $\sigma$-finite product measure}
\begin{corollary}{assotiativity}
总有 $$ \mathcal{A}_1 \otimes \mathcal{A}_2  \otimes \mathcal{A}_3 =  (\mathcal{A}_1 \otimes \mathcal{A}_2)  \otimes \mathcal{A}_3 =   \mathcal{A}_1 \otimes (\mathcal{A}_2  \otimes \mathcal{A}_3 ) $$ 并且, if $\mu_1, \mu_2,\mu_3$ are $\sigma$-finite, then: $$\mu_1 \times \mu_2 \times \mu_3 = (\mu_1 \times \mu_2) \times \mu_3  = \mu_1 \times (\mu_2 \times \mu_3 )$$
\end{corollary}
\begin{proof}
    DIY. 前者 play with def, 后者直接由 $\sigma$-finite 的 premeasure 的 HK extension unique 得到. 
\end{proof}


\subsection{如何证明一个函数 product measurable}

要证明一个函数是 product measurable 的, 只需要证明它对于每个  measurable rectangle 的 preimage 都是 measurable 的即可.
\begin{lemma}
    Suppose $f: X\to Y\times Z$ is a function from a measurable space $(X,\mathcal{A})$ to a product measure space $(Y\times Z, \mathcal{B}_1 \otimes \mathcal{B}_2)$.\\
  Claim: If $f^{-1}(B_1 \times B_2)\in\mathcal{A}$ for each measurable rectangle $B_1 \times B_2 \in \mathcal{B}_1 \otimes \mathcal{B}_2$, then $f$ is an $(\mathcal{A}, \mathcal{B}_1 \otimes \mathcal{B}_2)$-measurable function.
\end{lemma}
\begin{proof}
    因为 product $\sigma$-algebra 由所有的 measurable rectangles 生成. \\
    Hw7 中有另一版的证明作为 lemma.
\end{proof}
在 Hw7 中我们可以通过这个 lemma 得到一个结论: 如果 $E \in \mathcal{A\otimes A}$, 那么 $E$ 的 diagonal 一定 $\in \mathcal{A}$.


对于特殊的函数, 比如两个 measurable function 的乘积, 其一定是 product measurable 的.
\begin{lemma}{easier Fubini}
条件: $(X,\mathcal{A},\mu)$, $(Y,\mathcal{B}, \nu)$ 为 arbitrary measure space (不需要 $\sigma$-finite.), $f:X\to \mathbb{C}$, $g:Y \to \mathbb{C}$ 为 measurable functions.\\
结论: \[
h:= fg  \quad \text{is } (\mathcal{A} \otimes \mathcal{B}) \text{-measurable}
\]
并且如果 $f,g$ 是 $L^1$ 的, 那么 $h \in L^1(\mu \times \nu)$ 并且 \[
\int h \; d(\mu\times \nu) = \Big( \int f \; d\mu \Big)\Big( \int g \; d\nu \Big)
\]
\end{lemma}
\begin{proof}
     in hw 8. 这个 statement 表示一个 $\mathcal{A} \text{-measurable}$ 的函数和一个 $\mathcal{B} \text{-measurable}$ 的函数的乘积是 $(\mathcal{A} \otimes \mathcal{B}) \text{-measurable}$ 的.\end{proof}








\section{Tonelli's Thm [Fol 2.5]}

 我们将 focus on the case $n=2$: $(X,\mathcal{A}, \mu)$, $(Y, \mathcal{B}, \nu)$, 考虑 \[
 (X \times Y, \mathcal{A} \otimes \mathcal{B}, \mu \times \nu)
 \]
 从而, 它可以推广到任何 finite 个 measure space 的 product 上.

\subsection{$E\subset X \times Y $ 的 section }
\begin{definition}{$x$-section, $y$-section}
给定 product space 上的集合 $E \subset X \times Y$, 对于 $x \in X$, $y\in Y$, 我们定义:
\[E_x : = \{  y\in Y \mid (x,y) \in E \}\]
\[
E^y := \{x\in X \mid (x,y) \in Y  \}
\]
\pic[0.3]{assets/ch2-pics-image-20250220190452148.png}
给定从 product space 出发的函数 $f: X \times Y \to \mathbb{C}$, 对于 $x \in X$, $y\in Y$, 我们定义: \[
f_x (y) := f^y(x) := f(x,y)
\]
表示固定住一个变量, 另一个变量的变化.
\end{definition}

\begin{example}
对于任意的 $E \subset X \times Y$如果定义: \[
    f:= \chi_E
    \]
那么有: \[
f_x = \chi_{E_x},\quad f^y = \chi_{E^y}
\]
对于 \textbf{rectangle:} $E = A \times B $, $A \in \mathcal{A}, B \in \mathcal{B}$, 有 $$E_x = \begin{cases}
        \varnothing ,\quad x\not\in A \\
        B ,\quad x\in A
    \end{cases}$$
\end{example}



\begin{proposition}
(a) $$E \in \mathcal{A} \otimes \mathcal{B} \implies \begin{cases}
        E_x \in \mathcal{B},\quad \forall x \in X\\
        E^y \in \mathcal{A}\quad, \forall y \in Y
    \end{cases}$$ (b) $$ f \text{ is } \mathcal{A} \otimes \mathcal{B}\text{-measurable} \implies \begin{cases}
        f_x \text{ is } \mathcal{B}\text{-measurable} \;\;\forall x \\
         f^y \text{ is }  \mathcal{A}\text{-measurable} \;\;\forall y
    \end{cases}$$
\end{proposition}

\begin{proof}
    (a) Let $$\mathcal{E}:= \{    E \subset X \times Y \mid  E_x \in \mathcal{B}\;\; \forall x \in X \quad\text{ and }\quad E^y \in \mathcal{A}\;\;\forall y \in Y  \}$$
Claim: $\mathcal{E}$ 包含了所有的 rectangles, 并且 $\mathcal{E}$ a $\sigma$-algebra.\\
容易证明这一点. 从而, 由 $\mathcal{A} \otimes \mathcal{B}$ 的定义 (为包含所有 rectangles 的最小 $\sigma$-algebra) 得  $\mathcal{A} \otimes \mathcal{B} \subset \mathcal{E}$, 从而 (a) 成立\\
并且由于 (check) $f_x^{-1} (V) = (f^{-1}(V) )_x$ (Similar for $f_y$),  (a)$\implies$(b).
\end{proof}
\begin{remark}
    这里三件需要记住的事情:
    \begin{itemize}
        \item \textbf{section 和取 preimage 可以交换顺序}
        \item 对于一个 product measurable set, 任意 \textbf{section 也 measurable}
        \item 对于一个 product measurable function, 任意 \textbf{section function 也 measurable}
    \end{itemize}
\end{remark}
\begin{remark}
    记录一下这里的证明方法, 以前见的比较少. 这里\textbf{证明条件 A 推出条件 B} 的方法: \textbf{证明所有满足条件 B 的元素构成的集合 包含了 所有满足条件 A 得元素构成的集合.} \\
    这一方法的好处在于: 可以运用所有满足条件 B 的元素构成的集合的整体性质, 比如是 $\sigma$-algebra 等.
\end{remark}


\begin{definition}{monotone class}
Given a set $X$, a collection $C \subset \mathcal{P}(X)$ is called a monotone class, if it is closed under \textbf{countable increasing unions} and \textbf{countable decreasing intersections}
\end{definition}
当然, 一个 $\sigma$-algebra 是一个 monotone class. monotone class 是一个比 $\sigma$-algebra 更弱的定义.

\subsection{tool needed to show Tonelli: Monotone Class Lemma}
\begin{lemma}{monotone class lemma}
Let $\mathcal{A} \subset \mathcal{P}(X)$ be an algebra.\\ 
Define $\mathcal{C}$ 为包含 $\mathcal{A}$ 的最小的 montone class.\\
Claim: \[
<\mathcal{A}> = \mathcal{C}
\]
\end{lemma}
\begin{proof}
    $\mathcal{C} \subset <\mathcal{A}>$ is trivial.\\
    $<\mathcal{A}> \subset \mathcal{C}$: STS $\mathcal{C}$ 是一个 $\sigma$-algebra. see p.66. 具体做法比较 tricky, 但是思路是先证明 $\mathcal{C}$ 是一个 algebra (这一部分较难. 我们对于 $E \in \mathcal{C}$, define $\mathcal{C}(E)$ 为 $\mathcal{C}$ 中所有和它的交和差也仍然在 $\mathcal{C}$ 中的 $F$ 构成的集合, 并发现这个子集 $\mathcal{C}(E)$ 也同样是一个 monotone class. 从而 $\mathcal{C} = \mathcal{C}(E)$ );
    
然后对于任意的 seq, 其 前 $n$ 项的 finite union $(\cup_{i=1}^nE_i)$seq 是一个 increasing seq, 其 limit 等于原 seq limit, 是属于 $\mathcal{C}$ 的.
\end{proof}
\begin{remark}
    即, 对于一个已经是 algebra 的集合, 它生成的 monotone class 等于它生成的 $\sigma$-algebra.\\
    这个 lemma 的好处在于, \textbf{我们在证明了一个集合是 algebra 后, 证明它是一个 $\sigma$-algebra, 只需要证明它 closed under ctbl increasing union 和 decreasing interection 即可. 我们可以利用这种 set seq 的单调性.}
\end{remark}

但是如果我们只知道 $\mathcal{C} \supset \mathcal{A}$, 没有 "包含 $\mathcal{A}$ 的最小的 monotone class" 这个条件怎么办? 那也没关系, 我们很自然得出  
\begin{corollary}
    Let $\mathcal{A} \subset \mathcal{P}(X)$ be an algebra, $\mathcal{C} \supset \mathcal{A}$ be a monotone class, 那么一定有 \[
    \mathcal{C} \supset \big \langle \mathcal{A}\big \rangle
    \]
\end{corollary}
因为 $\big \langle \mathcal{A}\big \rangle  =$ "包含 $\mathcal{A}$ 的最小的 monotone class" $\subset \mathcal{C}$.


\subsection{Tonelli for sets: integrating a section to get product measure}
\begin{theorem}{Tonelli for sets}
Let $(X,\mathcal{A}, \mu)$, $(Y, \mathcal{B}, \nu)$ be\textbf{ $\sigma$-finite} measure spaces.\\
Take $E \in \mathcal{A} \otimes \mathcal{B}$. Then:  \[
x \mapsto \nu(E_x), y \mapsto \mu(E_y) \text{ are \textbf{measurable} }
\] 并且 \[
(\mu \times \nu)(E) = \int \nu(E_x)  \; d\mu(x) = \int \mu(E^y) \; d \nu(y)
\]
\end{theorem}
\begin{remark}
这个定理说明的是: 在 $\sigma$-finite 的 measure spaces 构成的 product measure space 中, 任意 product measurable set $E$, 把 $x$ 映射到 $E$ 的 $x$-section 的 measure (\textbf{"扫描" 这个集合在一个方向上的宽度变化) 的行为是可测的.}\\
并且, 我们可以把 $E$ 的 measure 用 \textbf{``对每个 $x$, 在 $Y$ 上取 $E$ 的 $x$-section 的 measure, 并对这一行为在 $X$ 上进行积分"}来刻画. 这就把 product measure 拆分了开来. 其 following 是 Tonelli \textbf{``把 $n$ 个 measure spaces 的 product 上的积分拆成 $n$ 个积分}" 的强大定理.
\pic[0.3]{assets/ch2-pics-image-20250225215218999.png}
\end{remark}
\begin{proof}
Define:    $$\mathcal{C} = \{ E \subset X \times Y \mid x \mapsto \nu(E_x), y \mapsto \mu(E_y) \text{ are measurable } \forall (x,y)\in E \text{ and} \cdots (2)  \}$$
\textbf{Claim 1: }$\mathcal{C}$\textbf{ contains 所有的 rectangles.}
\textbf{Proof of Claim 1:} 考虑 $E = A\times B \in \mathcal{A}\times \mathcal{B}$, 即为一个 rectangle. 上一 lec 中, 我们 by def confirm: $\mathcal{A}\times \mathcal{B}\subset  \mathcal{A}\otimes \mathcal{B} $.\\
那么对于任意的 $(x,y) \in E$, 我们有: \(\nu(E_x) =  \nu(B)\), 对于所有的 $(x,y) \not \in E$, 则有 \(\nu(E_x) =  \varnothing\).\\
所以对于任意的 $x$, $\nu(E_x) =  \chi_A(x) \nu(B)$, 同理 $\mu(E^y) = \chi_A(x) \mu (A)$. \\
由此得到 $x\mapsto \nu(E_x), y\mapsto \mu(E^y)$ 是 measurable 的, 并且 \[
\mu \times \nu (E) = \mu(A) \times \nu(B) = \mu(A) \int \chi_B(y) \;d \nu(y) = \int \mu(E^y) \; d  \nu(y) 
\]
同理, \(\mu\times \nu (E) = \int \nu (E_x) \; d\mu(x)   \). 从而得证. 从而, \textbf{对于任意 union of finite disjoint rectangles, 这个结论也成立}, by additivity in definition. 因而 $\mathcal{C}$ \textbf{为一个 algebra.}\\\\
Note: 由于 $\mathcal{A} \otimes \mathcal{B}$ 为包含所有 rectangles 的最小 $\sigma$-algebra, 我们\textbf{只需要证明 $\mathcal{C}$ 为一个 $\sigma$-algebra}, 那么它一定包含 $\mathcal{A} \otimes \mathcal{B}$. 并且 by Monotone Class Lemma, \textbf{STS: $\mathcal{C}$ 为一个 monotone class.}\\\\
\textbf{Claim 2: $\mathcal{C}$ 为一个 monotone class.}
令 $\{E_n\}$ 为一个 increasing seq in $\mathcal{C}$, 定义其 union 为 $E := \bigcup_{n=1}^\infty E_n$. 并定义: \[f_n(y) := \mu((E_n)^y)\]
根据 $\mathcal{C}$ 的 definition, 每个 $f_n$ 都是 measurable 的, 并且我们容易证明: \[f_n \nearrow f(y) := \mu((E)^y) \text{ ptwise.}\]于是使用 MCT, 容易得到 \[
\int \mu(E^y) \; d\nu(y) = \lim \int \mu((E_n)^y) \; d\nu(y) = \lim \mu \times \nu(E_n) \overset{\text{CFB}}{=} \mu \times \nu (E)
\]
从而 $E \in \mathcal{C}$. \\
It remains to show: $\mathcal{C}$ closed under ctbl decreasing intersection. 不过这里我们涉及到一个 decreasing sequence 中间突然从 infinite measure 变为 finite measure 的问题, 所以我们从这里开始要分 $\mu ,\nu$ finite 和 not finite (but still $\sigma$-finite) 的两种情况讨论. finite measure 不用担心上述这一问题.\\
\textbf{Case 1:  $\mu ,\nu$ finite}, 于是令 $\{E_n\}$ 为一个 decreasing seq in $\mathcal{C}$, 和 increasing 的情况 similar, 得到 \(\mu((E_n)^y)=: f_n  \searrow f(y) := \mu((E)^y) \text{ ptwise.}\), 从而  by DCT (取 $\mu(X)$ 为 donimating function), 得到 \[
\int \mu(E^y) \; d\nu(y) = \lim \int \mu((E_n)^y) \; d\nu(y) = \lim \mu \times \nu(E_n) \overset{\text{CFA}}{=} \mu \times \nu (E)
\] 从而我们证明了在  $\mu,\nu$ 为 finite measure 的情况下, $\mathcal{C}$ 为一个 monotone class, 从而为一个 $\sigma$-algebra, 从而 $\mathcal{M} \otimes \mathcal{N} \subset \mathcal{C}$. \\\\
\textbf{Case 2: $\mu,\nu$ $\sigma$-finite measure}: 我们可以把 $X \times Y$ 写作 union of a seq of finite measure sets $\{ X_i \times Y_i\}_{i\in \mathbb{N}}$, 从而也是 a union of increaasing seq of finite measure sets $\{ X_j \times Y_j\}_{j\in \mathbb{N}}$. (取 $X_j \times Y_j  = \bigcup_{i=1}^j X_j \times Y_j$) 从而对于任意的 $E \in \mathcal{M}\otimes \mathcal{N}$, $$E = \lim_{j\to\infty} (E \cap (X_j\times Y_j))$$
对于每个 $E \cap (X_j\times Y_j)$, 我们可以应用前一结论, 得到 \[
\mu \times \nu (E \cap (X_j\times Y_j)) = \int \chi_{Y_j}(y) \mu(E^y \cap X_j) \; d\nu(y)
\]
从而\textbf{应用 MCT}, 得到 \[
\mu \times \nu (E ) = \int \mu(E^y) \; d\nu(y)
\]
 同理 \(\mu \times \nu (E ) = \int \mu(E_x) \; d\mu(x)\), 从而 $E \in \mathcal{C}$, 得证.
\end{proof}


\subsection{Tonelli's Theorem}

\begin{theorem}{Tonelli}
 Let $(X,\mathcal{A}, \mu)$, $(Y, \mathcal{B}, \nu)$ be $\sigma$-finite measure spaces.\\
条件: 令 $f \in L^+ (X \times Y)$,\\
结论:$$g(x) := \int f(x,y) \; d\nu\; \in L^+(X)\quad h(y) := \int f(x,y) \; d\mu \; \in L^+(Y) $$ (显然) 并且
\begin{align}
    \int f \; d(\mu\times \nu) &= \int \Big[   \int f(x,y) \; d\nu(y)  \Big] d\mu(x)\\
    &=  \int \Big[   \int f(x,y) \; d\mu(x)  \Big] d\nu(y)
\end{align}
\end{theorem}\begin{remark}
    Tonelli for sets 表示了 product measure 的计算方式: 通过对 $x$-section 的 $y$ measure, 在 $x\in X$ 上进行积分可得到. 这就把 product measure 拆成了单个 measure 与积分.\\
    而 Tonelli's Theorem 表示\textbf{对一个非负 product measurable function 积分可以转化成对逐个 measure 积分.} \\
recall: 非负 measurable 函数的积分, 就是一个 seq of simple functions 的积分的 sup, 而 simple function 的积分, 就是\textbf{几个 measurable set 的 measure 的加权和}.
因而 Tonelli's Theorem 基本上 naturally follows from Tonelli for sets.
\end{remark}
\begin{proof}
首先,\textbf{ 对于 $f$ 是 simple function 的 case, 直接 follows from Tonelli for sets.} (mentioned in remark.)\\
对于 general case: $f \in L^+(X\times Y)$, 令 $\{f_n\}$ 为一个 seq of simple functions ptwisely converging to $f$.\\
于是 \[
\int g \; d\mu = \lim \int g_n \; d\mu  = \lim \int f_n \;  d(\mu\times \nu) = \int f \; d(\mu\times \nu)
\]
\[
\int h \; d\mu = \lim \int h_n \; d\mu  = \lim \int f_n \;  d(\mu\times \nu) = \int f \; d(\mu\times \nu)
\]
by MCT.
\end{proof}




\section{Fubini's Theorem and Lebesgue integral in $\mathbb{R}^n$ [Fol 2.5, finished; 2.6]}
recall Tonelli's Theorem: Given $f \in L^+(X \times Y)$, set $g(x) := \int f_x d\nu $, $h(y) := \int f^y d \mu$. 
Then $g \in L^+(X)$, $h \in L^+(Y)$, 以及有: \[
\int f \; d(\mu \times \nu) = \int g \; d \mu = \int h \; d\nu
\]
展开后可写作: \[
\int f \; d(\mu \times \nu)  = \iint  f(x,y) \; d\nu(y) d\mu(x) = \iint f(x,y) \; d\mu(x) d\nu(y)
\]更加简洁可写作: \[
\int f \; d(\mu \times \nu) = \iint f \; d\nu d\mu  = \iint f \; d\mu d\nu
\]

\begin{corollary}
    if $f \in L^1(X\times Y)$ and $f \geq 0$ then 
    \begin{itemize}
        \item     $g(x) < \infty$ for a.e. $x$
        \item $h(y) < \infty$ for a.e. $y$
    \end{itemize}
\end{corollary}
\begin{remark}
    在 product measure space 上 measurable 的可积函数, 在每个成分上, 都不能有过多的 infinity point.
\end{remark}
Next: Fubini's Theorem.\\
Fubini's Theorem 是 Tonelli's Theorem 对 $\mathbb{C}$-valued 函数 (instead of $\mathbb{R}_{\geq 0}$-valued) 的推广. 但是其实证明很 trivial. 
\subsection{Fubini's Theorem}
\begin{theorem}{Fubini's Theorem}
 条件: $f \in L^1(\mu \times \nu)$,\\
 结论:
    \begin{itemize}
        \item $f_x \in L^1(\nu)$ for a.e. $x$, $f^y \in L^1(\mu)$ for a.e. 
        \item The a.e. defined functions: \[ g(x) := \int f_x \;d\nu \in L^1(\mu),\quad h(x) := \int f^y \;d\nu \in L^1(\nu) \]
        \item \[ \int f \; d(\mu \times \nu)  = \int g \; d\mu = \int h \; d\nu  \;\; (= \iint f\; d\mu d\nu)\]
    \end{itemize}
\end{theorem}
\begin{proof}
\(f = \Re f + i \Im f \), so WLOG can assume $f$ is $\mathbb{R}$-valued.\\
又 $f = f^+ - f^-$, 直接 apply Tonellis's Thm 可得.
\end{proof}
\begin{remark}
    Tonelli and Fubini's Theorem 不仅有用在可以拆分积分以进行计算, 而且有用在积分换序. \\
    实际上, 根据它的条件可以发现, 积分可换序的条件是很宽裕的, 只要这个函数 $f$ 在 $L^+$ 或者 $L^1$ space 中就可以了.
\end{remark}


\begin{example} 
求和换序的合理性:\\
考虑 $$(X, \mathcal{A}, \mu) = (Y , \mathcal{B}, \nu) = (\mathbb{N}, \mathcal{P}(\mathbb{N}), \mu_{counting})$$if $a_{mn} \in \mathbb{C}$ for $(m,n) \in \mathbb{N}^2$ and \[
\infty > \sum_{m,n} |a_{mn}| =: \sup_{F \subset \mathbb{N}^2  \text{finite}} \sum_{(m,n) \in F}  |a_{m,n}|
\]
Thm: 对于任意 $n\in\mathbb{N}$,  $\sum_m a_{mn}$ conv absly to some $b_n \in \mathbb{C}$; \\
同样, 对于任意 $m\in\mathbb{N}$, $\sum_n a_{mn}$ conv absly to $c_m \in \mathbb{C}$.
以及 $\sum_n b_n, \sum_m c_m$ conv absly to $\sum_{m,n} a_{mn}$.\\
即: \[
\sum_{n=1}^\infty \sum_{m=1}^\infty |a_{mn}| = \sum_{m=1}^\infty \sum_{n=1}^\infty |a_{mn}| = \sum_{(m,n) \in\mathbb{N}^2} |a_{mn}|
\]
\end{example}


\subsection{complete Fubini's Theorem}
\begin{remark}
即便 \((X, \mathcal{A}, \mu)\), $ (Y , \mathcal{B}, \nu)$ 都 complete, product space \((X\times Y , \mathcal{A} \otimes \mathcal{B} , \mu\times \nu  )\) \textbf{不一定 complete! (甚至说基本很少 complete)}
\end{remark}

\begin{example}
      考虑 $(X, \mathcal{A}, \mu) = (Y , \mathcal{B}, \nu) = (\mathbb{R}, \mathcal{L}, m)$
       考虑一个 Vitali set. \[
       V \times \{0\} \subset \mathbb{R} \times \{0\} \text{ is a subnull set, not measurable}
       \]
\end{example}

但是如果我们 consider completion: \[
(X \times Y , \overline{\mathcal{A} \otimes \mathcal{B}},  \overline{\mu \times \nu}   ) 
\]
\begin{theorem}{complete Fubini-Tonelli}
对于 complete measure space \((X, \mathcal{A}, \mu)\), $ (Y , \mathcal{B}, \nu)$, 取它们的 product measure space 的 completion: \[
(X \times Y , \overline{\mathcal{A} \otimes \mathcal{B}},  \overline{\mu \times \nu}   ) 
\]
我们将 $\overline{\mathcal{A} \otimes \mathcal{B}}$ 简易写作 $\mathcal{L}$, $\overline{\mu \times \nu}   $ 简易写作 $\lambda$.\\
Suppose $f: X \times Y \to \mathbb{C}$ is $\mathcal{L}$-measurable 并且 $f \in L^+(\lambda)$ or $f \in L^1(\lambda)$, 则有
\begin{itemize}
    \item $f_x$ 是 $\mathcal{B}$-measurable 的 for a.e. $x$ 且 $x \mapsto \int f_x \; d\nu$ 是 measurable 的
    \item $f_y$ 是 $\mathcal{A}$-measurable 的 for a.e. $y$ 且 $y \mapsto \int f_y \; d\mu$ 是 measurable 的
\end{itemize}
并且, 在 $f \in L^1(\lambda)$ 的情况下,  $f_x, f_y$, $x \mapsto \int f_x \; d\nu$, $y \mapsto \int f_y \; d\mu$ 也是 \textbf{integrable} 的, 即 $\in L^1(\lambda)$, 并且 \[
\int f \; d\lambda = \iint  f \; d\mu d\nu =  \iint  f \; d\nu d\mu 
\]
\end{theorem}
\begin{proof}
    exercise. 比较简单.
\end{proof}
\begin{remark}
    这一定理的意思是, 在 $\mu,\nu$ 是 complete measure 的情况下, $\mu \times \nu$ 的 completion  $\overline{\mu \times \nu}   $ \textbf{虽然并不等于 $\mu \times \nu$, 但是 $ L^1(\overline{\mu \times \nu})$ 的函数的积分却可以当作 $ L^1(\mu \times \nu)$ 的函数的积分, 从而分成两个积分.}
    这是因为 因为完备化测度只是增加了一些\textbf{原本测度为零的集合的子集}, 这些集合不会影响积分计算. 这一定理的直接应用是 Lebesgue integral on $\mathbb{R}^n$.
\end{remark}

\subsection{remark: integral of 非负函数等于 area under graph}
\begin{theorem}
    令 $(X,\mathcal{A}, \mu)$ 为一个 arbitrary measure space, $f \in L^+(\mu)$ 为 arbitrary 可测非负函数, 我们定义: \[
    G_f := \{(x,y) \in X \times [0,\infty] : 0 \leq y \leq f(x)\}
    \]
    Claim: $G_f$ 是 $(\mathcal{A}\times \mathcal{B}(\mathbb{R}))$-measurable 的, 并且 \[
    (\mu \times m) (G_f) = \int f \; d\mu
    \]
\end{theorem}
\begin{proof}
    In hw 6.
\end{proof}
\begin{remark}
$G_f$ 即 area under the graph of $f$. 这一 statment 是一个\textbf{正式的表达 of ``integral of 非负函数等于 area under graph"}. 我们也可以推广它到 $L^1$ 上 (正负 part 的差), which is trivial.
\end{remark}
