\chapter*{Homework 4: on measurable functions(36/40)}
\begin{center}
\textit{None of the following questions will be graded. Do them, but do not hand them in}.
\end{center}


\section*{One with Vitali.}
  Let $(X,\mathcal{A})$ be a measurable space, and $E\subset X$ a subset. Prove that $E\in\mathcal{A}$ iff the function $\chi_E$ is measurable. Use this to construct a function  $f\colon\mathbb{R}\to\mathbb{R}$ that is not Lebesgue measurable.
  
\section*{Truncations in $L^+$: 通过 $\int f_n$ 或者 $\int_{X_n} f$ 的极限 (bounded function / subset) 得到 $\int_X f$}
Let $(X,\mathcal{A},\mu)$ be a measure space and $f\colon X\to [0, \infty]$ a measurable function.
\begin{itemize}
\item[(a)] (Horizontal truncation) Suppose that $X=\bigcup_{n=1}^\infty X_n$ for some $X_1\subset X_2\subset \cdots$ 
with $X_n\in \mathcal{A}$. Prove that 
\[
  \int_X f \,d\mu = \lim_{n\to \infty} \int_{X_n} f\,d\mu
\]
\item[(b)] (Vertical truncation)
  Prove that
\[
  \int f\,d\mu = \lim_{n\to \infty} \int \min\{ f,  n\}\,d\mu.
\]
\item[(c)]
  Explain the terminology ``horizontal truncation'' and ``vertical truncation''.
\end{itemize}

\section*{Disregarding null sets.}
  Let $(X,\mathcal{A},\mu)$ be a \emph{complete} measure space.
  \begin{itemize}
  \item[(a)]
    Let $f\colon X\to \overline{\mathbb{R}}$ and $g\colon X\to \overline{\mathbb{R}}$ be functions such that $f=g$ $\mu$-a.e.
    \begin{itemize}
    \item[(i)]
      Prove that $f$ is measurable (i.e.\ $\mathcal{A}$-measurable) iff $g$ is measurable.
    \item[(ii)]
      Prove the same statement when $f$ and $g$ are $\mathbb{C}$-valued, rather than
      $\overline{\mathbb{R}}$-valued.
    \item[(iii)]
      Give examples showing that the condition that $\mu$ be complete is necessary.
    \end{itemize}
  \item[(b)]
    Let $f_n : X \to  \overline{\mathbb{R}}$, $n\in\mathbb{N}$, and $f\colon X\to \overline{\mathbb{R}}$ be functions such that $\lim_{n\to\infty}f_n(x)=f(x)$ for a.e. $x\in X$. 
    \begin{itemize}
    \item[(i)]
      Prove that if $f_n$ is measurable for all $n$, then so is $f$.
    \item[(ii)]
      Prove the same statement when $f_n$ and $f$ are $\mathbb{C}$-valued, rather than
      $\overline{\mathbb{R}}$-valued.
    \item[(iii)]
      Give examples showing that the condition that $\mu$ be complete is necessary.
    \end{itemize}
  \end{itemize}
  \textit{Hint}: this is Proposition 2.11 of [Folland].
  
\section*{Measurable functions and completions.}
  Let $(X,\mathcal{A},\mu)$ be a measure space and let $(X, \bar{\mathcal{A}}, \bar{\mu})$ be its completion. 
Suppose that $f\colon X\to\overline{\mathbb{R}}$ is $\bar{\mathcal{A}}$-measurable. Prove that there is an $\mathcal{A}$-measurable function $g\colon X\to\overline{\mathbb{R}}$ such that $g=f$ $\bar{\mu}$-a.e.,  and hence $\int g  \;d\mu=\int f \;d\bar{\mu}$. 
\textit{Hint}: this is Proposition 2.12 of [Folland].

\section*{Measurability on subsets.}
  Let $(X,\mathcal{A})$ be a measurable space, and $Y\subset X$ a nonempty subset.
  We say that a function $g\colon Y\to\overline{\mathbb{R}}$ is \emph{$\mathcal{A}$-measurable on $Y$} if $g$ is $\mathcal{A}|_Y$-measurable, where the $\sigma$-algebra $\mathcal{A}|_Y$ on $Y$ is defined as in HW1.
  \begin{itemize}
  \item[(a)]
    Prove that if $f\colon X\to\overline{\mathbb{R}}$ is measurable and $Y\subset X$, then $g=f|_Y$ is $\mathcal{A}$-measurable on $Y$. 
  \item[(b)]
    Prove that if $g$ is $\mathcal{A}$-measurable on $Y$ and $Y\in \mathcal{A}$, then $g$ can be extended to an $\mathcal{A}$-measurable function $f$ on $X$. Is the extension unique?
  \item[(c)]
    Let $f\colon X\to\overline{\mathbb{R}}$ be any function, and set $Y=f^{-1}(\mathbb{R})$. Prove that $f$ is measurable iff $f^{-1}(\{\infty\})\in \mathcal{A}$, $f^{-1}(\{-\infty\})\in \mathcal{A}$, and 
    $f|_Y\colon Y\to\mathbb{R}$ is $\mathcal{A}$-measurable on $Y$.
  \end{itemize}

\section*{Suprema of uncountable families.}
  Construct (using the Axiom of Choice, if needed) an \emph{uncountable} family $(f_\alpha)_\alpha$ of real-valued Borel measurable functions on $\mathbb{R}$ such that the function $\sup_\alpha f_\alpha$ is not Lebesgue measurable, let alone Borel measurable.

\section*{Increasing functions again.}
  Let $f\colon\mathbb{R}\to\mathbb{R}$ be an increasing function. Prove that $f$ is Borel measurable. Use this to give an example of a function $f\colon\mathbb{R}\to\mathbb{R}$ that cannot be written as a difference between increasing functions.

\section*{Lebesgue but not Borel.}
  Let $F\colon[0,1]\to[0,1]$ be the function from HW3, whose graph is the Devil's Staircase. Define $G(x)=F(x)+x$. 
  \begin{itemize}
  \item[(a)]
    Prove that $G:[0,1]\to [0,2]$ is an increasing homeomorphism. In other words, $G$ is increasing, bijective, and both $G$ and $G^{-1}$ are continuous. 
  \item[(b)]
    Let $C$ be the middle-thirds Cantor set, and set $K:=G(C)$. Prove that $m(K)=1$. 
  \item[(c)]
    Since $m(K)>0$, we know from HW3 that there is a set $A\subset K$ that is not 
    Lebesgue measurable. Prove that $B=G^{-1}(A)$ is Lebesgue measurable but not 
    Borel measurable. 
  \end{itemize}

\section*{Measurability and absolute values.}
  Let $(X,\mathcal{A})$ be a measure space. 
  Suppose that $f\colon X\to\mathbb{C}$ is a measurable function. Prove that the function $|f|\colon X\to\mathbb{R}$ is also measurable. Is the converse true? 



  
\begin{center}
\textit{Some of the following questions will be graded. Do them, and do hand them in. You may use the results from the exercises above}.
\end{center}



\section*{Measurability of limit loci.}
  Let $(X, \mathcal{A})$ be a measurable space. For each $n\in \mathbb{N}$, let $f_n\colon X\to \mathbb{R}$ be a measurable function. Consider the set
  \[
    E:=\{x\in X\mid\lim_{n\to\infty} f_n(x)\ \text{converges to a real number}\}.
  \]
  Prove that $E$ is a measurable set in two ways:
  \begin{itemize}
  \item[(i)]
    by expressing $E$ in terms of the functions $g(x)=\displaystyle\limsup_{n\to \infty} f_n(x)$ and
    $h(x)= \displaystyle \liminf_{n\to \infty} f_n(x)$;
  \item[(ii)]
    by expressing $E$ in terms of the sets
    \[
      E_{i,j,k}=\{x\mid |f_j(x)-f_k(x)|<\tfrac1{i}\},
    \]
    where $i,j,k\in\mathbb{N}$.
    \textit{Hint}: a sequence $(a_n)_n$ of real numbers converges iff it is a Cauchy
    sequence, i.e. for every  $ \epsilon >0$
    there is $n$ such that for every $j,k\ge n$, $|a_j-a_k|< \epsilon$.
  \end{itemize}
  \textit{Hint}: note that $\pm\infty$ are not real numbers, and please avoid considering $\infty-\infty$; you may want to prove a lemma to the effect that if $g,h\colon X\to    \overline{\mathbb{R}}$ are measurable functions, then the set
  \[
    \{x\in X\mid g(x)=h(x)\in \overline{\mathbb{R}}    \}
  \]
  is measurable; to do this, you may want to consider functions like $\max\{g,\kappa\}$, $\min\{h,\kappa\}$ and $\min\{g,-\kappa\}$, $\min\{h,-\kappa\}$ for large real constants $\kappa>0$.

\begin{proof}
   \textbf{ of method (i):}\\
   Define:
\[
g(x):=\limsup_{n\to\infty} f_n(x) \quad\text{and}\quad h(x):=\liminf_{n\to\infty} f_n(x)
\]
Since each $f_n$ is measurable function, by proposition in lecture (sequential preservation of measurability), \textbf{$g,h$ are measurable.}

And as we know, for any real sequence \((a_n)\),
\[
\lim_{n\to\infty} a_n \text{ exists (as a real number)} \quad\Longleftrightarrow\quad \limsup_{n\to\infty} a_n = \liminf_{n\to\infty} a_n \in \mathbb{R}
\]

Thus, for each \(x\in X\) we have:
\[
x\in E \quad\Longleftrightarrow\quad \limsup_{n\to\infty} f_n(x)=\liminf_{n\to\infty} f_n(x)\in \mathbb{R}
\]

Thus, we can write $E$ as:
\[
E = \{x\in X \mid g(x)=h(x)\in \mathbb{R}   \}
\]

Note: here we want to have a difference function of the two functions, but it is undefined on $\infty - \infty$ type of points. So actually it is not valid to take the difference for functions mapping to $\overline{\mathbb{R}}$. This is why we use the following method instead:

For each \(n \in \mathbb{N}\), we define:
\[
g_n(x):=\min\{\max\{g(x), -n\}, n\} \quad \text{and} \quad h_n(x):=\min\{\max\{h(x), -n\},n\}
\]
Notice that, \textbf{each $g_n, h_n$ is measurable}, since $g,h$ are measurable and constant function is measurable and we have proved in lecture that taking the max, min of two measurable functions is measurable.\\\\
\noindent \textbf{Claim 1.1:} \[
g(x)=h(x)\in\mathbb{R}\quad\Longleftrightarrow\quad \exists N_0>0,\ \forall n\ge N_0,\quad g_n(x)=h_n(x)
\]
\noindent \textbf{proof of claim 1.1:}
Suppose $g(x)=h(x)\in\mathbb{R}$. Let \(M:=\max\{|g(x)|, |h(x)|\} < \infty\), then for any \(n> M\), we have
   \(
   g_n(x)=g(x),  h_n(x)=h(x)\), so \(g_n(x)=h_n(x)\). 

Suppose \(\exists N_0>0,\ \forall n\ge N_0,\quad g_n(x)=h_n(x)\), Then it is clear that  \[
 g(x) =  g_{N_0}(x) = h_{N_0}(x)=h(x) < \infty
   \]
\pic[0.35]{assets/hw4-33121738963658_.pic.jpg}

\noindent \textbf{proof of remaining:}
Therefore we have:
\begin{equation}
    E = \bigcup_{N=1}^\infty \bigcap_{n\ge N} \{x\in X \mid g_{n}(x)=h_{n}(x)\}
\end{equation}
Foe each $n \in \mathbb{N}$, we define
\[
E_n := \{x\in X \mid  g_{n}(x)=h_{n}(x)\}
\]
Since each $g_n, h_n$ is measurable and real-valued (finite), $g_n - h_n$ is measurable and $|g_n - h_n|$ is measurable, so we have for each $m\in \mathbb{N}$,
\[
\{x\in X: |g_{\kappa_n}(x)-h_{\kappa_n}(x)|<1/m\}  = |g_n - h_n|^{-1} ([0,1/m)) \in \mathcal{A}
\]
Thus \[E_n = \bigcap_{m\in \mathbb{N}}|g_n - h_n|^{-1} ([0,1/m)) \in \mathcal{A}\] is a measurable set.
Thus $E$ is a countable union of countable intersections of mea
surable sets, then measurable. 
\end{proof}



\begin{proof}
    \textbf{of method (ii):}\\
Recall: \textbf{a seq of real numbers converges iff it is a Cauchy.}
Now we fix an arbitrary \(i\in\mathbb{N}\) and let \(\epsilon = 1/i\). Define:
\[
E_{i,j,k} = \{x\in X: |f_j(x)-f_k(x)| < 1/i\}
\]
Since each \(f_j\) is measurable, the function \(x\mapsto |f_j(x)-f_k(x)|\) is measurable (since each term in the sequence maps to $\mathbb{R}$ but not $\overline{\mathbb{R}}$), and hence \textbf{each \(E_{i,j,k} = |f_j(x)-f_k(x)|^{-1} ([0,1/i))\) is measurable.}\\\\
\noindent For each \(i\), consider the set of \(x\in X\) for which the sequence \((f_n(x))\) satisfies the Cauchy condition with respect to \(\epsilon=1/i\). That is,
\[
E_i = \Big\{x\in X:  \exists N\in\mathbb{N} \text{ s.t. } \forall j,k\ge N,\; |f_j(x)-f_k(x)|<\frac1{i}\Big\}
\]
We can write \(E_i\) as
\[
E_i = \bigcup_{N=1}^\infty \bigcap_{j,k\ge N} E_{i,j,k}
\]
Since countable unions and intersections of measurable sets are measurable, \textbf{\(E_i\) is measurable.}\\\\
\noindent Now, since \((f_n(x))\) converges in $\mathbb{R}$ i\textbf{ff it is Cauchy, i.e. it is in $E_i$ for each $i\in\mathbb{N}$}, we have:
\[
E = \bigcap_{i=1}^\infty E_i 
= \bigcap_{i=1}^\infty \Big(\bigcup_{N=1}^\infty \bigcap_{j,k\ge N} E_{i,j,k}\Big)
\]
This is a countable intersection of measurable sets, and therefore \(E\) is measurable.
\end{proof}


  
  
  \section*{Measurability of continuity loci.}
    Let $(X,d)$ be a metric space, and $f\colon X\to\mathbb{C}$ any function. Prove that the set of points $x\in X$ such that $f$ is continuous at $x$ is a $G_\delta$-set, and in particular a Borel set.
    \textit{Hint}: consider sets of the form
    \[
      \{x\in X\mid |f(y)-f(z)|\le\tfrac1n\ \text{whenever $\max\{d(y,x),d(z,x)\}\le\delta$}\}
    \]
    and show off your skills with quantifiers.

\begin{proof}
Recall: \(f\colon X\to\mathbb{C}\) \textbf{from a metric space} is continuous at \(x\in X\) iff for every \(\varepsilon > 0\) there exists a \(\delta > 0\) such that\( |f(y)-f(x)| < \varepsilon \text{ whenever } d(y,x) < \delta \).
We can easily check that, \textbf{this condition is equivalent to}: for every \(\varepsilon > 0\) there exists a \(\delta > 0\) such that \(|f(y)-f(z)| < \varepsilon \quad \forall y,z \in B_\delta(x)\), by the relation of diameter and radius of the open ball).\\\\
Thus we have: 
\[
x \in C \Longleftrightarrow \forall n\in\mathbb{N},\ \exists m\in\mathbb{N} \text{ s.t.} y,z \text{ with } d(y,x) < \tfrac{1}{m} \text{ and } d(z,x) < \tfrac{1}{m},\; |f(y)-f(z)| < \tfrac{1}{n}
\]
In other words, \textbf{\(x\) is a continuity point iff it belongs to:}
\[
C = \bigcap_{n=1}^\infty \bigcup_{m=1}^\infty U_{n,m}.
\] where \[
U_{n,m} = \Big\{x\in X \mid  y,z\in B_\frac{1}{m} (x)\implies |f(y)-f(z)| < \tfrac{1}{n}\Big\}
\]
\textbf{Claim: \(U_{n,m}\) is open.}\\
\textbf{Proof of Claim:}\\
Let \(x\in U_{n,m}\). WTS: $\exists$ an \(\varepsilon > 0\) such that \(B_\varepsilon(x) \subset U_{n,m}\).\\
Consider: \(\varepsilon = \tfrac{1}{2m}\). \\
Let \(y \in B_\varepsilon(x)\). Take any two points \(z,w \in X\) satisfying
\[
d(z,y) < \tfrac{1}{2m} \quad \text{and} \quad d(w,y) < \tfrac{1}{2m}
\]
Then by the triangle inequality, we have:
\[
d(z,x) \le d(z,y) + d(y,x) < \tfrac{1}{2m} + \tfrac{1}{2m} = \tfrac{1}{m}
\]
Similarly, \(d(w,x) < \tfrac{1}{m}\).  Since \(x \in U_{n,m}\), it follows that
\[
|f(z) - f(w)| < \tfrac{1}{n}
\]
Thus, the condition defining \(U_{n,m}\) holds for \(y\), meaning \(y \in U_{n,m}\).  This proves that \(B_\varepsilon(x) \subset U_{n,m}\), thus $U_{n,m}$ is open since $x$ is arbitrary.\\\\
Therefore:
\[
C = \bigcap_{n=1}^\infty \bigcup_{m=1}^\infty U_{n,m}
\] is $G_\delta$ since each $ \bigcup_{m=1}^\infty U_{n,m}$ is a union of open sets, thus open; and $C$ is thus a countable intersection of open sets, namely a \(G_\delta\)-set. (thus Borel).

\end{proof}
\begin{comment}
\begin{remark}
  对于任意一个 metric space, 它的所有连续点构成的集合都是 countable 个 open sets 的交集.  
\end{remark}
\end{comment}   
    
\section*{Measurability of differentiability loci.}
Let $f\colon\mathbb{R}\to\mathbb{R}$ be any function. Let us say (as usual) that $f$ is \textbf{\emph{differentiable}} at $x$ if there exists $\lambda\in \mathbb{R}$ such that $\lim_{y\to x}\frac{f(y)-f(x)}{y-x}=\lambda$. 
  
We also declare $f$ to be\textbf{ \emph{strongly differentiable} }at $x$ if there exists $\lambda\in \mathbb{R}$ with the following property: for each $\epsilon>0$ there exists $\delta>0$ such that if $|y-x|\le\delta$ and $|z-x|\le\delta$, then $|f(y)-f(z)-\lambda(y-z)|\le\epsilon |y-z|$.


  \begin{itemize}
  \item[(a)]Does $f$ being differentiable at $x$ imply that $f$ is strongly differentiable at $x$? Give a proof or a counterexample.
  \item[(b)]Prove that the set of points $x\in \mathbb{R}$ at which $f$ is strongly differentiable is a Borel set. \textit{Hint}: consider sets of the form    \[
      E_{\lambda,m,n}:=\{x\in \mathbb{R} \mid |f(y)-f(z)-\lambda(y-z)|\le\tfrac1n|y-z|\
      \text{whenever $\max\{|y-x|,|z-x|\le\tfrac1m$}\}.
      \]
  \item[(c)]\textit{Extra credit}: is the set of points $x\in \mathbb{R}$ at which $f$ is differentiable a Borel set?
  \end{itemize}
  
\begin{solution}
    \textbf{of (a):} No. Consider the following counterexample: \\
\[
f(x)=
\begin{cases}
x^2\sin\Bigl(\frac{1}{x}\Bigr), & x\neq 0\\
0, & x=0
\end{cases}
\]
We know that
\[
\frac{f(x)-f(0)}{x-0} = \frac{x^2\sin(1/x)}{x} = x\sin(1/x)
\]
Note \(|x\sin(1/x)|\le |x|\), so when \( x\to 0 \) we have:
\[
\lim_{x\to 0} x\sin(1/x) = 0
\]
Thus $f$ is differentiable at $0$ and \( f'(0)=0 \).

\begin{lemma}
 $f:\mathbb{R} \rightarrow \mathbb{ R}$ is strongly differentiable at $x$ $\implies$ it is differentiable at $x$, and $\lambda$ is uniquely equal to the derivative at $x$.
\end{lemma}
\begin{proof}
    \textbf{of lemma 4.1:}\\
   Suppose $f:\mathbb{R} \rightarrow \mathbb{ R}$ is strongly differentiable at $x$, so for any \(\epsilon>0\), there exists \(\delta>0\) s.t. for all $y,z\in B_\delta (x)$, we have: \[
\Bigl|f(y)-f(z)-\lambda(y-z)\Bigr|\le \epsilon\,|y-z|.
\]
Suppose $y\not = z$, then dividing by \(|y-z|\) on both sides, we have
\[
\Bigl|\frac{f(y)-f(x)}{y-x}-\lambda\Bigr|\le \epsilon
\]
Since $\epsilon$ is arbitrary, this proves that 
\[
f'(x) = \lim_{y\to x}\frac{f(y)-f(x)}{y-x}=\lambda  
\]
\end{proof}

Now we go back to the counterexample. Suppose for contradiction that $f$ is strongly differentiable at $0$, then $\lambda = 0$, so for all \(\epsilon>0\), there exist \(\delta>0\) s.t. for all $y,z \in B_\delta(0)$, we have
\[
|f(y)-f(z)| \le \epsilon\,|y-z|
\]
Consider \(\epsilon=\tfrac{1}{4}\). Let $\delta > 0$. Take $n\in\mathbb{N}$ s.t.
\[
\frac{1}{(2n+\tfrac{3}{2})\pi} < \delta
\]and then take
\[
y_n:= \frac{1}{\left(2n+\tfrac{1}{2}\right)\pi},\quad
z_n:= \frac{1}{\left(2n+\tfrac{3}{2}\right)\pi}
\]Note that each $|y_n|,|z_n| < \delta$. And we have \[
   \sin\Bigl[\Bigl(2n+\tfrac{1}{2}\Bigr)\pi\Bigr]=(-1)^n,\quad
   \sin\Bigl[\Bigl(2n+\tfrac{3}{2}\Bigr)\pi\Bigr]= -(-1)^n
   \]Thus \[
   f(y_n)-f(z_n)= (-1)^n\bigl[y_n^2+z_n^2\bigr]
   \]while 
\[
y_n-z_n = \frac{1}{\left(2n+\tfrac{1}{2}\right)\pi} - \frac{1}{\left(2n+\tfrac{3}{2}\right)\pi}
=\frac{1}{\pi\left(2n+\tfrac{1}{2}\right)\left(2n+\tfrac{3}{2}\right)}
\]
Taking limit of this behavior (increasing $n$), we get the sequential limit of \(\frac{|f(y_n)-f(z_n)|}{|y_n-z_n|} \) indexing over $n$ is $\frac{\frac{1}{2\pi^2 n^2}}{\frac{1}{4\pi n^2}} = \frac{2}{\pi}$.
By taking large enough $n$, we can alwasy get \(\frac{|f(y_n)-f(z_n)|}{|y_n-z_n|} \) to be arbitrarily close to \( \frac{2}{\pi}> \frac{1}{4}\).
This shows that $f$ is not strongly differentiable at $0$.
\end{solution}


\begin{proof}
   \textbf{ of (b):}\\
Let $f\colon\mathbb{R}\to\mathbb{R}$ be any a function.Denote \[ E: = \{x\in \mathbb{R}\mid f \text{ is strongly differentiable at } x\}\]
WTS: $E$ is a Borel set.

Set for each $\lambda \in \mathbb{R}, m,n\in\mathbb{N}$: \[ E_{\lambda,m,n}:=\{x\in \mathbb{R} \mid |f(y)-f(z)-\lambda(y-z)|\le\tfrac1n|y-z|\   \;\; \forall y,z \in B_{\frac{1}{m}}(x)\}\]
where $ B_{\frac{1}{m}}(x)$ denote the open ball centered at $x$ with radius $\frac{1}{m}$.

Then by the definition of strongly differentiable, we have: \[
   E=\bigcup_{\lambda\in\mathbb{R}}\bigcap_{n\in\mathbb{N}}\bigcup_{m\in\mathbb{N}} E_{\lambda,m,n}\,
   \]
\noindent \textbf{Claim 3.1: Each $ E_{\lambda,m,n}$ is open.}\\
\textbf{Proof of Claim 3.1:} Let \(x\in E_{\lambda,m,n}\). Then
\[\forall y,z\in B_{1/m}(x),\quad \bigl|f(y)-f(z)-\lambda(y-z)\bigr|\le \frac1n\,|y -z|\]
In particular, the inequality holds for all \(y,z\in B_{1/(2m)}(x)\). Now consider $B_{1/(2m)}(x)$, let $x' \in B_{1/(2m)}(x)$, then for every \(y\in B_{1/(2m)}(x')\), we have \[
|y-x|\le |y-x'|+|x'-x|<\frac{1}{2m}+\frac{1}{2m}=\frac{1}{m} \]
so \(B_{1/(2m)}(x')\subset B_{1/m}(x)\). Hence the inequality holds for all \(y,z\in B_{1/(2m)}(x')\). This confirms that every \(x\in E_{\lambda,m,n}\) has a neighborhood contained in \(E_{\lambda,m,n}\), proving that \(E_{\lambda,m,n}\) is open.\\\\


Now that each $ E_{\lambda,m,n}$ is open, we have $\bigcup_{m\in\mathbb{N}} E_{\lambda,m,n}$ is each for each $\lambda , n$; thus each for each $\lambda$, \(\ G_\lambda := \bigcap_{n\in\mathbb{N}}\bigcup_{m\in\mathbb{N}} E_{\lambda,m,n}\) is a $G_\delta$ set.

\[E = \bigcup_{\lambda\in\mathbb{R}} G_\lambda \]
is a union of $G_\delta$ sets.

(I do not now how to deal with it then, it might be that we somehow reduce it to countable union of $G_\delta$ sets, getting something like $E = \bigcup_{\lambda\in\mathbb{Q}} G_\lambda $ using the density of $\mathbb{Q}$ in $\mathbb{R}$, thus confirming that it is Borel.)
\textcolor{red}{-2. 这里的正解是: 要利用 density of $\mathbb{Q}$ in $\mathbb{R}$ 的话, 只需要考虑交换 set operation 的顺序就好了. 我们会发现其实: \[
  E=\bigcap_{n\in\mathbb{N}} \bigcup_{\lambda\in\mathbb{Q}} \bigcup_{m\in\mathbb{N}} E_{\lambda,m,n}\,
\]就这么简单。。
}
\end{proof}
\begin{proof}
    of extra credit: \textcolor{red}{yes. 这个解法非常麻烦. 需要再多考虑两层. 
令 $E_{\lambda, k,l,m,n}$ 表示 the set of points $x$ s.t. \[
|f(y) - f(z) - \lambda (y-z) | \leq \frac{1}{n} |y-z|
\]
whenever \[
\frac{1}{2^{l+1}} (1 +\frac{1}{2^k}) \leq |y-x| \leq  \frac{1}{2^{l-1}} (1 -\frac{1}{2^k})  \quad \text{and} \quad |z-x| \leq \frac{1}{2^m} (1 -\frac{1}{2^k}) 
\]
Claim: \[
 f  \text{ is differentiable at x} \text{ iff }     x \in E:= \bigcap_{n\in\mathbb{N}} \bigcup_{\lambda \in\mathbb{Q}} \bigcup_{l \in \mathbb{N}} \bigcap_{r \geq l} \bigcup_{m\geq 1} \bigcup_{k \in \mathbb{N}}    E_{\lambda, k,r,m,n}
\]
}
\end{proof}

  
\section*{decreasing MCT: 成立当且仅当 integral 的 limit 是 finite 的}
  Let $(f_n)_1^\infty$ be a \emph{decreasing} sequence of non-negative measurable functions on a measure space. 
  \begin{itemize}
  \item[(a)]  Prove that if $\lim_n\int f_n<\infty$, then $\lim_n\int f_n=\int\lim_nf_n$. 
  \item[(b)]Give an example of a decreasing sequence $(f_n)_n$ of nonnegative measurable functions such that $\lim_n\int f_n\ne\int\lim_nf_n$. 
  \end{itemize}
  \textit{Hint}: use MCT correctly. 

\begin{proof}
\textbf{of (a):}\\
Since \((f_n)\) is a decreasing sequence, i.e. for every \(x\in X\) we have
\[
f_1(x) \ge f_2(x) \ge f_3(x) \ge \cdots 
\]
We can define the function
\[
g_n(x) = f_1(x) - f_n(x)
\] for each $n \in \mathbb{N}$. Then for the seq $(g_n(x))$ we have:
\begin{itemize}
    \item non-negatice: \(g_n(x) \ge 0 \;\; \forall x\) because \(f_1(x) \ge f_n(x)\).
    \item  increasing in $n$:\[
  g_n(x) = f_1(x) - f_n(x) \le f_1(x) - f_m(x) = g_m(x) \;\;\forall  m\geq n, \forall x
  \] since \((f_n)\) is decreasing.\\\\
\end{itemize}

Define $f(x) := \lim_n f_n(x) \in \overline{\mathbb{R}}$ for each $x \in X$.

Since \(f_n(x)\) decreases to \(f(x):=\lim_{n\to\infty} f_n(x)\), we have \[
  \lim_{n\to\infty} g_n(x) = f_1(x) - \lim_{n\to\infty} f_n(x) = f_1(x) - f(x)
  \]
Now we \textbf{apply MCT to the increasing sequence \((g_n)\)}. We have:
\[
\lim_{n\to\infty} \int g_n\,d\mu = \int \Bigl(\lim_{n\to\infty} g_n\Bigr)\,d\mu = \int (f_1 - f)\,d\mu
\]
And since \(\lim_{n\to\infty}\int f_n\,d\mu < \infty\), we have\[
\lim_{n\to\infty} \int g_n\,d\mu = \int f_1 \, d\mu - \int f \;d\mu
\] Also, because of \(\lim_{n\to\infty}\int f_n\,d\mu < \infty\), \textbf{$\int f_n $ is eventually finite}. Say, it is finite after $n\geq N \in \mathbb{N}$. We only need to consider $n\geq N$ when considering the limit behavior.\\
Then for each \(n \geq N\),
\[
\int g_n\,d\mu = \int \bigl(f_1 - f_n\bigr)\,d\mu = \int f_1\,d\mu - \int f_n\,d\mu
\]
\textcolor{red}{-2. 这里注意, 我们既然知道 $f_1$ 的 integral 未必 finite, 就不能这么定义 $g_n$. 正解是取 $N$ s.t. $\int f_N$ finite, 然后定义 $g_n := f_N - f_n$.}
Taking the limit as \(n\to\infty\), have
\[
\lim_{n\to\infty} \int g_n\,d\mu = \lim_{n\to\infty} \left(\int f_1\,d\mu - \int f_n\,d\mu\right) = \int f_1\,d\mu - \lim_{n\to\infty} \int f_n\,d\mu
\]
by linearity of numerical sequence.\\
Thus, combining with the result from MCT we have:
\[
\int f_1\,d\mu - \lim_{n\to\infty} \int f_n\,d\mu =  \int f_1 \,d\mu- \int f \;d\mu
\]
Rearrange to get:
\[
\lim_{n\to\infty} \int f_n\,d\mu = \int f\,d\mu,
\]
which is exactly what we wanted to prove.
\end{proof}

\begin{solution}
    \textbf{of (b):}\\
 Consider defining $(f_n: \mathbb{R} \rightarrow \mathbb{ R})_{n\in\mathbb{N}}$ with
\[
f_n(x) = \chi_{[n,\infty)}(x)
\]
Note that: \begin{itemize}
    \item  \textbf{$f_n$ is a decreasing seq}: For each \(n\) and every \(x\in\mathbb{R}\),\[
   f_{n+1}(x) = \chi_{[n+1,\infty)}(x) \le \chi_{[n,\infty)}(x) = f_n(x)
   \]
since \([n+1,\infty) \subset [n,\infty)\).
\item \textbf{$(f_n)$ the pointwise limit}:\[
   \lim_{n\to\infty} f_n(x) = 0 \quad \forall x\in\mathbb{R}
   \]since for each $x$ there exists an \(N\) (any integer greater than \(x\)) such that for all \(n \ge N\), \(x < n\) and hence \(f_n(x)=0\).
   \item For each $n$, \[
     \int_{\mathbb{R}} f_n\,d\lambda = \int_n^\infty 1\,dx = \infty
     \]
   But on the other hand
     \[
     \int_{\mathbb{R}} \Bigl(\lim_{n\to\infty} f_n\Bigr)\,d\lambda = \int_{\mathbb{R}} 0\,d\lambda = 0
     \]

\end{itemize}
Then we have the decreasing seq of function with
\[
\lim_{n\to\infty} \int f_n\,d\lambda = \infty \quad \text{while} \quad \int \Bigl(\lim_{n\to\infty} f_n\Bigr)\,d\lambda = 0
\]
This shows that in the absence of the finiteness assumption, the limit and integration need not commute.
\end{solution}





\section*{Vitali meet Cantor.} 
  Construct a function $f\colon[0,1]\to[0,1]$ such that:
  \begin{itemize}
  \item[(a)] $f$ fails to be Lebesgue measurable;
  \item[(b)]there exists a compact subset $K\subset(0,1)$ of positive Lebesgue measure such that $f$ is differentiable at every point $x\in K$.
  \end{itemize}
  \textit{Hint}: use the function $g(x)=\inf\{|x-y|\mid y\in K\}$; then square this with the title of the problem.


\begin{solution}
Let $V$ be a Vitali set on $[0,1]$, $C$ be the fat Cantor set on $[0,1]$ by recursively taking away the middle open subinterval of length $\frac{1}{4^{n}}$ on the $n$th recursion. 
    We consider the function: \[ f(x) = \chi_V \cdot d(x,C)^2\]
where \[d(x,C) := \{\inf\{|x-y|\mid y\in C\}\]
By Hw3, we know $V$ is not Lebesgue measurable, and $C$ is compact with positive Lebesgue measure $\frac{1}{2}$.

And since $f^{-1} (\{1\}) = V$, mapping a not measurable set to a measurable set, \textbf{$\chi_V$ is not measurable function. }

And since the distance function $d(x,C)$ is a continuous function of $[0,1]$, it is measurable, by the result proved in class that a continuous funciton on a topological space is measurable.

\begin{lemma}
The product of a measurable $f:\mathbb{R}\rightarrow \mathbb{R}_{>0}$ and a not measurable $g: \mathbb{R}\rightarrow \mathbb{R}$ is not measurable.
\end{lemma}

\noindent \textbf{Proof of Lemma 4.2: } 
$f$ measurable $\implies$ $1/f$ measurable. Suppose for contradiction that $fg$ is measurable, then  $g = \frac{1}{f}(f g) $ is the product of two measurable functions, thus measurable, contradicting the fact that $g$ is not measurable. Thus $fg$ is not measurable.\\\\

\noindent \textbf{Claim 5.1: $f$ is not measurable.}
\noindent \textbf{Proof of claim 5.1:}
Thus on the open set $A = [0,1] \setminus C$, $d(x,C)^2$ is positive, so $\chi_V |_A d(x,C)^2|_A $  is not measurable since it is a product of measurable and not measurable function by lemma 4.2. Thus\textbf{ $f$ is not measurable,} otherwise its restriction on $A$ should also be measurable.\\\\

\noindent \textbf{Claim 5.2: $f$ is differentiable on $C$.}
\textbf{Proof of claim 5.2:} 
Fix \(x\in C\), then $f(x) = 0$. We want to show:\(f'(x)=\lim_{h\to 0}\frac{f(x+h)-f(x)}{h} = \lim_{h\to 0}\frac{f(x+h)}{h}\) exists
Let $h > 0$.
Case 1: \(x+h\notin V\), then \(\chi_V(x+h)=0\), so we have \(f(x+h)=\chi_V(x+h)\,d(x+h,C)^2=0
  \), then 
  \(\frac{f(x+h)}{h}=0\). 
Case 2: \(x+h\in V\), we have: \[
  d(x+h,C)=\inf_{y\in C}|(x+h)-y|\le |(x+h)-x|=|h|
  \] So  \[
  \left|\frac{f(x+h)}{h}\right|=\frac{d(x+h,C)^2}{|h|}\le\frac{|h|^2}{|h|}=|h|
  \]
Therefore for all cases we have:
\[
\left|\frac{f(x+h)-f(x)}{h}\right|=\left|\frac{f(x+h)}{h}\right|\le |h|
\]
This confirms that 
\[
f'(x)=\lim_{h\to 0}\frac{f(x+h)-f(x)}{h}=0
\]

This finishes the proof of required properties of $f$.
\end{solution}



\subsection{harder Vitali meet Cantor (extra credit)}
We change the requirement of (a) to be: "the restriction of $f$ to any open interval $I\subset[0,1]$ fails to be Lebesgue measurable". Then how can we make the construction?
\begin{solution}
    I don't know.\\
    \textcolor{red}{官方答案: 我在前一问给出的  \[ f(x) = \chi_V \cdot d(x,C)^2\] 这个函数, 同样也是满足这一问的答案. (对于 $C$, 不仅可以选择 fat Cantor set, 实际上任何 choice of compact nowhere dense set 都可以.) }
\end{solution}