\chapter{differentiation on real spaces}
\section{differentiation of regular Borel measures on $\mathbb{R}^n$ [Fol 3.4, finished]}

\begin{definition}{regular Borel measure}
    一个 Borel measure $\nu$ on $\mathbb{R}^n$ 被称为 regular 的, if $\nu$ is locally finite (finite on every compact set).
\end{definition}


\begin{theorem}{ regular Borel measure: 蕴含了 regularity}
一个 regular Borel measure $\nu$ on $\mathbb{R}^n$ 一定满足:
\begin{enumerate}
    \item outer regularity: $$\nu (E) = \inf \{ \nu(U) \mid  E \subset U  \text{ open}    \}$$
    \item inner regularity: $$\nu (E) = \inf \{ \nu(U) \mid  E \subset U  \text{ open}    \}$$
\end{enumerate}    
\end{theorem}
\begin{remark}
    这里不证明这两个性质. 因为目前的知识不够.\\
     regular $\implies$ outer regularity: Hard, 其推导需要 Ch7 
regular 和 inner regularity $\implies$ outer regularity: 这个简单, same as proof of Thm 1.18 on Folland.
\end{remark}


\begin{example}
 任何 LS measure on $\mathbb{R}$  (restrict to Borel sets) 都是 regular measure.
    Lebesgue measure $m$ on $\mathbb{R}^n$ (restrict to Borel sets)  是 regular measure.
\end{example}


当然, 这个概念也可以推广至 signed/coplex measure 上.
\begin{definition}{regular signed/complex measure}
    一个 signed/complex measure $\nu$ on $\mathbb{R}^n$ 被称为 regular measure, if $|\nu|$ is regular 的.
\end{definition}




\begin{lemma}
    如果 $f\in L^1_{loc} (\mathbb{R}^n)$, 则 $f\, dm$ 是一个 regular measure.
    如果 $f:\mathbb{R}^n \to \mathbb{R}$ 是 extended-integrable 的, 则 \[
    f \in L^1_{loc} (m) \iff f\, dm \text{ is a regular measure}
    \]
\end{lemma}
\begin{proof}
    Folland p99. 这显然, 因为 $f$ locally integrable 就说明 $f \, dm$ 是 locally finite 的.
\end{proof}



\begin{lemma}
    如果 $\rho, \lambda$ 是 signed/complex measure 并且 $\rho \bot \lambda$, 那么 \[
    \rho, \lambda \text{ are regular measures} \iff \rho + \lambda  \text{ is a regular measure}
    \]
\end{lemma}

\begin{proof}
在 hw 10 中.\\
Note:  STS it for positive measure, 这是因为对于 regular signed / complex measure 而言, \[ 
    \lambda \perp \rho \quad \iff \quad \exists A
    \in  \mathcal{A}\; \text{ s.t. } \; |\lambda|(A^c)=0\text{ and }|\rho|(A)=0  \quad \iff \quad |\lambda| \perp |\rho|
    \]
 并且从而     \[
\nu=\lambda+\rho ,   \lambda \perp \rho \implies |\nu|=|\lambda + \rho| = |\lambda|+|\rho|
\]
这一命题的证明也在 hw 10 中,\\
\end{proof}




\subsection{LDT meets LRNT: 任何 regular Borel measure $\nu$ on $\mathbb{R}^n$ 对于 $m$ 的 RN-derivative $=$ relative density}
\begin{theorem}{LDT meets LRNT: computing RN derivative on $\mathbb{R}^n$}
    Let $\nu$ be a regular Borel measure on $\mathbb{R}^n$, with LRN decomposition $$\nu = \lambda + \rho, \quad d\rho = f \, dm,\quad  \lambda \bot m$$即\[
    d\nu = d\lambda  + f\,dm
    \]
那么: 对于 $m$-a.e. $x\in \mathbb{R}^n$, 都有: \[
    \lim_{r \to 0} \frac{\nu(B(x,r))}{m(B(x,r))}  = f(x)
    \]
\end{theorem}
\begin{remark}
    Also true with $B(x,r)$ replaced with shrinking $E_r$.\\
    这一 Thm 一句话概括即: \textbf{如果 $\nu$ 是一个 regular measure, 那么几乎处处, 我们都可以用这一点上的 density of $\nu$ over $m$ 来获得它对 $m$ 的 RN derivative $f$.}
\end{remark}

\begin{proof}
由 $\nu=\lambda+\rho$ 得到:
$$
\begin{aligned}
\frac{\nu(B(x, r))}{m(B(x, r))}=\frac{\lambda(B(x, r))}{m(B(x, r))}+\frac{\rho(B(x, r))}{m(B(x, r))}
\end{aligned}
$$
\textbf{By LDT, 我们有: }$$
\lim _{r \rightarrow 0} \frac{\rho(B(x, r))}{m(B(x, r))}=f(x), \quad \text { for } m \text {-a.e. } x \text {. }
$$
于是, 原命题即转化为 WTS: $$
\lim _{r \rightarrow 0} \frac{\lambda(B(x, r))}{m(B(x, r))}=0, \quad \text { for } m \text {-a.e. } x \text {. }
$$
(Notice: 这里 $0$ 也就是 $d \lambda / dm$, 因为两个 mutually singular 的 measure,其 RN derivative = $0$ a.e.).\\
By lemma: 因为 $\nu$ regular, 且 $\lambda \perp \rho $ ,可以推出: $\lambda,\rho $ 也是 regular 的.\\
WLOG 我们可以 suppose $\lambda$ 是 positive measure, 因为 $\lambda \perp m \iff |\lambda| \perp m$, 并且$|\lambda (E)| \leq |\lambda|(E)$ for any $E$. 因而 $\lambda$ 是 positive measure 的情况中这个极限为 $0$ 也自然推广到 complex measure 上.\\
注意: 由于 $\lambda \perp m$, 我们可以选取 Borel set $A$ such that \[
\lambda(A) = 0, \quad m(A^c) = 0
\]从而: 只需要证明 $\lim _{r \rightarrow 0} \frac{\lambda(B(x, r))}{m(B(x, r))}=0$ for a.e. $x\in A$ 就可以了, 因为 $A^c$ 本身也是 $m$ 的 null set.\\
我们 set: \[
F_k : = \bigg\{x\in A : \lim _{r \rightarrow 0} \frac{\lambda(B(x, r))}{m(B(x, r))}\geq  \frac{1}{k}  \bigg\}
\]
从而 STS: 对于任意 $k$, $m(F_k) = 0$.\\
我们 Fix 一个 $k$, by $\lambda$ 的 inner regularity, STS: 对于任意的 cpt $K \subset F_k$ compact, 都有 $m(K) = 0$. \\
于是我们 fix 一个 compact set $K\subset F_k$, 并 fix $\epsilon > 0$, STS: $m(K) < \epsilon$.\\
By $\lambda$ 的 outer regularity, 存在 $U_{\epsilon}\supset A$ open 使得 \[
\lambda (U_\epsilon) < \frac{\epsilon}{3^n k }
\]
By $F_k$ 的定义, 对于任意的 $x\in F_k$, 都存在某个 $r_x > 0$ 使得 \[
\nu(B(x,r_x) ) > \frac{1}{k} m(B(x,r_x) )
\]
Since \[
K \subset \bigcup_{x\in K} B(x,r_x)
\]从而 by finite open covering thm, 一定存在某个 finite set $K'$ 使得 \[
K \subset \bigcup_{x\in K'} B(x,r_x)
\]
我们 recall VItali covering lemma: For given collection of balls $\{B_j \subset \mathbb{R}^n\}_{j=1}^k$, 存在 \textbf{disjoint} subcollection $\{B_{j_1},\cdots, B_{j_m}\}$ 使得\[
\bigcup_{j=1}^k B_j  \subset \bigcup_{i=1}^m (3B_{j_i}) 
    \]
代入这里, 得到: 存在 $K'' \subset K'$ s.t. for all $x\in K''$, $B(x,r_x)$ 都是 disjoint 的, with \[
K \subset\bigcup_{x\in K''} 3 B(x,r_x)
\]
于是 \begin{align*}
    m(K) &\leq \sum_{x\in K''} m(3B(x,r_x))\\
    & = 3^n  \sum_{x\in K''} m(B(x,r_x))\\
    &\leq 3^n   k \sum_{x\in K''} \lambda(B(x,r_x))\\
    &\leq 3^n k \lambda(U_\epsilon) \leq  \epsilon
\end{align*}
Since $\epsilon$ 任意, $m(K) = 0$.\\
Since $K$ 任意, $m(F_k) = 0$.\\
Since $k$ 任意, \[
m\bigg( \bigg\{x\in A : \lim _{r \rightarrow 0} \frac{\lambda(B(x, r))}{m(B(x, r))} > 0  \bigg\}\bigg) =  0 
\]
从而得证.
\end{proof}
 \begin{remark}
     这实际上是一件比较自然的事情. 因为 $\lambda\perp m$, 那么存在一个划分: 使得某边 $\lambda$ null, $m$ 具有全测度.   在这一边几乎每一点上, $\lambda$ 相对于 $m$ 的密度理应为 $0$; 而另一边, $m$ 是零测度的; 从而整体, 在至多一个 $m$ 的 null set 外,  $\lambda$ 相对于 $m$ 的密度为 $0$.\\
     在这个证明中, 我们巧妙地同时运用了 inner 和 outer regularity 来简化要证明的结论, 最后 reduce to finite open covering 并自然地使用  VItali covering lemma 来 bound measure. 属于比较好看的证明.
 \end{remark}


\subsection{Differentiation on $\mathbb{R}$}
\subsection{$\{\text{positive regular Borel measures on }\mathbb{R}\} \simeq \{\text{distribution functions }F:\mathbb{R}\to\mathbb{R} \}$}
Recall that: 
\textbf{每个 distribution function (非严格 increasing, right ctn function) 都对应了唯一的一个  regular Borel measures $\mu_F$ on $\mathbb{R}$, 反之亦然.}\\
给定一个 regular Borel measures $\mu_F$,
$$
F_\mu(x ) := \begin{cases}
    \mu((0,x]) \quad  , x \geq 0 \\
     -\mu((x,0]) \; , x < 0
\end{cases}
$$
为它的 unique distribution function. 即 $\mu((a,b]) = F(b) - F(a)$, for all h-intervals.\\
而给定对于 distribution function $F$, 我们 define $\mu_0$ by:
$$
\mu_0(\bigcup_{i=1}^n (a_i, b_i]) = \sum_{i=1}^n (F(b_i) - F(a_i))
$$
然后 \textbf{by Hahn-Kolmogrov, extend to a regular Borel measure $\mu_F$}, 使得 $\mu_F ((a,b]) = F(b) - F(a)$ for any h-interval, i.e. $F$ 是 $\mu_F$ 的 distribution function, 并且 unique in the sense that 任意其他的 such function $G$ 如果也是$\mu_F$ 的 distribition function, 则必然有 $F-G$ 为 const. 
从而 \[
\mu_F \longleftrightarrow F
\]
之间构成了一个 measures 和 functions 的空间的 bijection.

distribution function 和 regular measure 之间的对应关系, 关键用处在于什么呢?  
我们 recall 刚刚才证明的定理, 不过使用一个更 general 的 version (can easily be extended from what we proved):
\begin{theorem}{slightly more general version of LRNT meets LDT}
    Let $\nu$ be a regular Borel measure on $\mathbb{R}^n$, with LRN decomposition $$\nu = \lambda + \rho, \quad d\rho = f \, dm,\quad  \lambda \bot m$$那么:  对于 $m$-a.e. $x\in \mathbb{R}^n$, 取任意 nicely shrinking $\{E_r\}$ to $x$, 都有: \[
    \lim_{r \to 0} \frac{\nu(E_r)}{m(E_r)}  = f(x)
    \]
\end{theorem}
因而如果我们有一个 $\mathbb{R}$ 上的 regualr measure $\mu_G  $, 那么考虑  $E_r : = (x,x+r]$, 我们有:
\[
\frac{\mu_G(E_r)}{m(E_r)}  = \frac{G(x+r)-G(x)}{r}
\]
从而我们发现 for a.e. $x$, 都有:  \[
\lim_{r\to 0}\frac{\mu_G(E_r)}{m(E_r)}  = G'(x)
\]
我们可以得到: 这\textbf{个 regualr measure $\mu_G  $ 相对于 $m$ 的 LRN derivative, 就等于它的 distribution function 的 derivative!}
甚至, 我们可以由此判断:\textbf{ 如果 $G:\mathbb{R}\to\mathbb{R}$ 是一个 distribution function (increasing, right ctn), 那么它的 derivative 一定是 a.e. 存在的!} Since $\mu_G  $ regular $\implies$ $G$ locally intble $\implies$ by LDT, 这个 density limit 是 a.e. 存在的.\\
至此, 我们发现了 Monotone Differentiation Theorem.\\


\subsection{Monotone Differentiation Theorem}
\begin{theorem}{Monotone Differentiation Theorem}
令 $F: \mathbb{R} \to \mathbb{R}$ 为一个 increasing (nondecreasing) function, set: \[
G(x) : = F(x+)
\]即 $F$ 的右极限函数. (note: $G$ 一定是 \textbf{increasing} 且 \textbf{right ctn} 的, 因而是一个 distribution function)\\
则有: 
\begin{itemize}
    \item $D_F := \{x: F \text{ disctn at } x \}$ 是至多 ctbl 的 (从而一定 zero measure)
    \item $F,G$ 都 differentaitble $m$-a.e., 并且 $$F' = G' \text{ a.e.}$$
\end{itemize}
\end{theorem}

\begin{proof}
    \textbf{of (a):} STS that, 对于任意 $m,n \in \mathbb{N}$, \[
    Z_{m,n}: = \bigg \{ x\in [-m,m] : F(x+) - F(x-)  \geq \frac{1}{n} \bigg \}
    \]是一个 finite set. 从而 $D_F = \bigcup_{m,n} Z_{m,n}$ 是 at most ctbl 的.\\
 而 $Z_{m,n}$ 确实是 finite 的, 因为 $F(-m) - F(m)$ 是 bounded 的, 所以 $Z_{m,n}$ 一定是 finite 的. (至多经历 $F(-m) - F(m) / (1/n)$ 个这样的点).\\
\end{proof}
\begin{proof}
    \textbf{of (b):}
首先我们知道, $G$ right ctn + increasing $\implies \mu_G$ 是一个 LS (thus regular when restricted to Borel sets) measure on $\mathbb{R}$.\\
Apply LDT to $\mu_G$, take $E_r : = (x,x+r]$ as the shrinking family to $x$.\\
于是
\[
\frac{\mu_G(E_r)}{m(E_r)}  = \frac{G(x+r)-G(x)}{r}
\]
由 LDT \(\implies\) 上式的 limit exist for a.e. $x$ ( $\mu_G$ w.r.t. $m$ 的 RN derivative), 它就是 $G'$, 且 $G'$ a.e. 存在, 等于其 induce 的 LS measure 的 RN derivaive. \\
现在 remains to show: $F$ 的 derivative 也 a.e. 存在, 并且和 $G$ 的相等. 我们 set: \[
H: = G- F
\]
从而 STS: $H'$ a.e. 存在且为 $0$.\\
首先, $H>0$ 并且 $H \not = 0$ 只有可能在 discontinuous points (which is at most ctbl)上. We set: \[
\mu : = \sum_{x\in D_F} H(x) \delta_x  
\]
从而对于任意区间 $I$, $$
\mu(I)=\sum_{x \in D_F \cap I} H(x)
$$
由于 $F,G$ locally intble,\textbf{ 这个 $\mu$ 是一个 regular Borel measure}.\\
并且, 这个 $\mu$ 的 null set 为 $D_f^c$, 而 $D_f$, as we have proved, is at most ctbl, 因而是 $m$ 的 null set. 从而得到: \[
\mu \perp m
\]于是 \[
\frac{d\mu}{d m} = 0\quad a.e
\]
从而由 LDT 得: \[
\frac{\mu((x-r,x+r))}{2r} \overset{r\to 0}{\to} 0 \quad a.e.
\]
因而对于任意的 $h> 0$,  \[
 \bigg|\frac{H(x+h) - H(x)}{h} \bigg| \leq \frac{H(x+h) + H(x)}{|h|}\leq \frac{\mu((x-2|h|,x+2|h|))}{4|h|} \overset{h\to 0}{\to} 0 
\]
finishing the proof.
\end{proof}
\begin{remark}
\textbf{As conclusion: \(\mathbb{R}\) 上的任意 increasing 函数 $F$, 其 right limit induce 的 LS measure 对于 $m$ 的 RN derivative, 就等于它的 derivative a.e.}
\end{remark}




\section{functions of bounded variation: $F\in BV$ [Fol 3.5]}
上一节课我们证明了 Monotone Differentiation Theorem: 它表明的是, 任何 $\mathbb{R}\to \mathbb{R}$ 的 non-decreasing function 都是 differentiable a.e. 的.\\
我们知道一个函数在一整个区间上 differentiable 其实是一个比价严格的条件, 但是 differentiable a.e. 的条件就略好达到一些.\\
Question: 如果一个函数在 $[a,b] $ 上 differentiable a.e., 那么 in a.e. sense, 可以在 $[a,b] $ 上定义它的 derivative $F'$. 那么, 是否一定有 \[
F(b)  - F(a)  = \int_a^b F'(x) \, dx
\]呢? 答案肯定是不一定的. 以下是三个反例: 1. Heaviside function; 2. Cantor function; 3.$F'(x) = 0$ a.e., but not $0$ on a null set.\\
这也很显然: 因为单点的值是无法控制的. 我们只能控制 in sense of a.e. , 因而有 outlier 的 $a,b$ 是很正常的.\\
我们之后将 revisit 这一问题, 给出这个等式成立的 condition.

接下来我们将


\subsection{total variation function $T_F$ of a function $F$}
\begin{definition}{total variation function}
    给定一个 function $F:\mathbb{R}\to \mathbb{C}$, 我们定义它的 total variation function $T_F$ 为:\begin{align*}
        T_F : \mathbb{R} &\to [0,\infty]\\
        x &\mapsto \sup \{     \sum_{j=1}^n | F(x_j) - F(x_{j+1})| : -\infty <  x_0 < \cdots < x_n  = x    \} 
    \end{align*}
\end{definition}
\begin{remark}
    Total variation 是一个很形象的定义. $T_F(x)$ 表示的是 $F$ 从 $-\infty$ 到当前 $x$ 的这段定义域上, 总的变化量. 它 count into 所有的变化, 包括离散的和连续的, 正方向的和负方向的.\\
    \pic[0.5]{assets/ch3-pics-Screenshot 2025-04-10 at 15.39.23.png}
\end{remark}

\begin{lemma}
 对于任意的 $F:\mathbb{R}\to \mathbb{C}$,  $T_F$ 都是 increasing 的; 并且对于任意 $a<b$, 有: \[
 T_F(b) =  T_F(a) + T_F(a;b)
 \]
 where \[
 T_F (a;b) = \sup\{\sum_{j=1}^n  | F(x_j) - F(x_{j+1})| : b =  x_0 < \cdots < x_n  = a \}
 \]
表示 $F$ 的定义域限制在 $[a,b]$ 上的 total variation. 
\end{lemma}
\begin{proof}
    显然, 由于 total variation 是 increasing 的,  \textbf{我们总是可以 greedyly 选择 partition}. 对于一个 partition, 总是可以插入一个中间点把它分成两半, 而这两半的 sub partition 的 total variation 的和 $\geq$ 原先的 partition 的 total variation.
\end{proof}

\subsection{space of functions of bounded variation: $BV$ 的基本性质 }

\begin{definition}{function of bounded variation}
 如果 $T_F(\infty) < \infty$, 我们称 $F:\mathbb{R}\to \mathbb{C}$ is \textbf{of bounded variation} 的, 写作 $F \in BV$.
\end{definition}

\begin{definition}{function of bounded variation on an interval}
如果 $T_F(a;b) < \infty$,  我们称 $F:\mathbb{R}\to \mathbb{C}$ is \textbf{of bounded variation} on $[a,b]$, 写成 $F \in BV([a,b])$.
\end{definition}

首先显然, $F\in BV$ 可以 reduce to real-valued 的情况来讨论.
\begin{proposition}\[
    F \in BV \iff \Re f \in BV \text{ and } \Im f \in BV
    \]
\end{proposition}


\subsection{$BV$ as a vector space }

\begin{lemma}{$BV$ 是一个 complex vector space}
    如果 $F,G \in BV$, 那么对于任意的 $a,b\in \mathbb{C}$, we have \[ T_{aF + bG}  \leq |a| T_F + |b| T_G   \]
    从而$$aF + bG \in BV$$
\end{lemma}
\begin{proof}
易得. 显然, 函数的 total variation 是线性可加的.
\end{proof}

\subsection{$F\in BV$ 的 $T_F$ 的 limit behavior}

我们知道,$F\in BV$ if $T_F(\infty) < \infty$. 而关于 $ T_F(-\infty)$, 同样有强结论:
\begin{proposition} \[
F \in BV \implies T_F(-\infty)  = 0
\]
\end{proposition}
\begin{proof}
    Let $\epsilon  > 0$.\\
    从而对于任意的 $x \in \mathbb{R}$, since $F\in BV$ 那么 $T_F$ bounded, $T_F(x)$ 是一个 real number.\\
    因而我们可以找到一组 partition points $x_0 < \cdots < x_n$ 使得 \[
    \sum_{1}^n |F(x_j)- F(x_{j-1})| \geq T_F(x) - \epsilon
    \]从而 \[
    T_F(x) - T_F(x_0) \geq T_F(x) - \epsilon
    \]从而 \[
    T_F(y) \leq \epsilon,\quad \forall y\leq x_0
    \]
Since $\epsilon  > 0$ arbitrary, 这证明了 $T_F(-\infty) = 0$
\end{proof}
\begin{remark}
$F$ bounded variation 的必要条件是它在 $x \to \infty$ 时, 截止 $x$ 处的 variation $\to 0$. 
\end{remark}


\begin{lemma}{$F\in BV$ right ctn $\implies T_F$ right ctn}
$F\in BV$ right ctn $\implies T_F$ 也 right ctn
\end{lemma}
\begin{proof}
Let $x\in \mathbb{R}, \epsilon > 0$.\\
Let \[
\alpha : = T_F(x+) - T_F(x)
\]WTS: $\alpha = 0$.\\
By right ctnity of $F$ 和 $T_F$ increasing, 我们可以选择 $\delta > 0$, 同时满足: $|F(x+ h) - F(x)|<\epsilon$, $T_F(x+h) - T_F(x+) < \epsilon$ whenever $0<h<\delta$.\\
Fix 一个满足 $0<h<\delta$ 的 $h$. 其后的证明见 Folland 104.
\end{proof}




\subsection{属于 $BV,BV(I)$ 的函数 }

\begin{lemma}{哪些函数一定 $BV$ or $BV(I)$}
\begin{enumerate}
    \item 如果 $F:\mathbb{R}\to \mathbb{R}$  bounded 且 increasing, 那么  $F \in BV$ 且 $T_F(x) = F(x) - F(-\infty)$.\\
    \item 如果 $F: \mathbb{R} \to \mathbb{R}$  是 Lipschitz countinuous 的, 那么$F \in BV(I)$ for 任意的 cpt interval $I$
    \item 如果 $F: \mathbb{R} \to \mathbb{R}$  是 differentiable 且 $F'$ bounded 的, 那么 $F \in BV(I)$ for 任意的 cpt interval $I$
\end{enumerate}
\end{lemma}
\begin{proof}
(1) trivial.\\
(2) by def: 考虑 Lipschitz const $M$, 则 $T_F(a;b) \leq M (b-a)$.\\
(3): 这是 (2) 的推论, 因为 recall: by MCT 可得: $F: \mathbb{R} \to \mathbb{R}$  是 differentiable 且 $F'$ bounded $\implies F$ Lipstchiz ctn.
\end{proof}

\begin{proposition}
以下是一些经典的函数的 variational behavior: 
\begin{enumerate}
    \item $f(x) = \sin(x)$: 属于 $BV(I)$ for 任意 cpt $I$, 但不属于 $BV$.
    \item $f(x) = x\sin \frac{1}{x}, f(0) = 0$:  \textbf{属于$BV(I)$ iff $0\not\in I$.}
    \item $f(x) = x^2\sin \frac{1}{x^2}, f(0) = 0$:   \textbf{属于$BV(I)$ iff $0\not\in I$.}
\end{enumerate}
\end{proposition}
\begin{proof}
  (1) 显然;
  (2),(3) 见 HW 11. 其实它们基本相同. 
 ($\implies$): if $0 \not\in I$ then $F \in BV(I)$. 是简单的, we differentiate $F(x)=x \sin (1 / x)$ for $x \neq 0$:
$$
F^{\prime}(x)=\frac{d}{d x}\left(x \cdot \sin \left(\frac{1}{x}\right)\right)=\sin \left(\frac{1}{x}\right)+x \cdot \cos \left(\frac{1}{x}\right) \cdot\left(-\frac{1}{x^2}\right)=\sin \left(\frac{1}{x}\right)-\frac{1}{x} \cos \left(\frac{1}{x}\right)
$$
在不含 $0$ 的区间上, 它是 bounded 的. 于是 by lemma 得证.\\
($\impliedby$): if $F \in BV(I)$ then  $0 \not\in I$. This is equiv to: if $0 \in I $ then $F \not \in BV(I)$.\\
Suppose $0 \in I= [a,b] $ then $a \leq  0$ and $b \geq  0 $, one of which is strict. WLOG we suppose $b > 0$. \\
我们的 idea 是 harmonic series. 考虑
$$
y_n:=\frac{1}{n \pi+\pi / 2} \rightarrow 0^{+}
$$
we have:
$$
F\left(y_n\right)=y_n \sin \left(\frac{1}{y_n}\right)=\frac{1}{n \pi+\pi / 2} \cdot \sin (n \pi+\pi / 2)
$$For odd $n$, $F(y_n) = \frac{-1}{n \pi+\pi / 2}$, for even $n$, $F({y_n}) =  \frac{1}{n \pi+\pi / 2}$. Since $b > 0$, for some $N_0$ we have $y_{N_0} < b$. Then we consider the partition: pick $N \in \mathbb{N}$, and use $ x_0 = 0,x_1 = y_{N_0 + N-1},x_2 =y_{N_0 +N-2},\cdots, x_{N} = y_{N_0},x_{N+1} = b$ as the partition points of $[0,b]$.\\
Then we have \[
\sum_{n=1}^{N+1} |F (x_n) - F(x_{n-1}) | \geq  \sum_{n=N_0}^{N_0 -2+ N}  \frac{1}{\pi n+\pi / 2} + \frac{1}{\pi (n+1)+\pi / 2} \geq 2\sum_{n=N_0}^{N_0 -2+ N} \frac{1}{\pi n+\pi / 2}
\]
As $N \to \infty$, this sum $\sum_{n=1}^{N+2} |F (x_j) - F(x_{j-1}) |  \to \infty$, by the harmonic series.
\end{proof}



\subsection{Jordan decomposition for $f\in BV$: $f = \frac{1}{2}(T_F+F) - \frac{1}{2}(T_F-F)$}

\begin{lemma}
   如果 real-valued  $F \in BV$, 那么 $T_F + F, T_F - F$ 都是 increasing 的.
\end{lemma}
\begin{remark}
    $T_F + F$ 即: $F$ increasing 的地方加倍 increasing, $F$ decreasing 的地方 const;\\
     $T_F - F$ 即: $F$ decreasing 的地方反向加倍 increasing, $F$ increasing 的地方 const;
     \pic[0.5]{assets/ch3-pics-Screenshot 2025-04-10 at 15.39.58.png}
\end{remark}
\begin{proof}
    任取 $x<y$.\\
    Let $\epsilon > 0$.\\
    Can find $x_0 <  x_ 1 < \cdots < x_N = x$, s.t. \[
    \sum_{1}^N |F(x_j) - F(x_{j-1}) | \geq T_F(x) - \epsilon
    \] 从而 
    \begin{align*}
T_F(y)  &\geq    \sum_{1}^N |F(x_j) - F(x_{j-1}) |  + |F(y) - F(x)| \\
&\geq T_F(x) - \epsilon + |F(y) - F(x)|
    \end{align*}
由于 $\epsilon > 0$ 任意, 可以得到: \[
 T_F(y) -  T_F(x)\geq  |F(y) - F(x)|
\]
因而: \[
( T_F(y)  - F(y)) -  (T_F(x)  -F(x)) \geq  |F(y) - F(x)| - (F(y) - F(x)) \geq 0
\]
\end{proof} 



\begin{theorem}{Jordan decomposition for $ F \in BV $}
  对于 $F: \mathbb{R}\to \mathbb{R}$ (注意是 real-valued): \[
  F \in BV \iff F \text{ 等于两个 bounded increasing functions 的差}
  \]  Specially, \[
    F \in BV \iff  T_F \pm F \text{ bounded}
  \]
  因而 for $F\in BV$, 我们总是可以把它写作 \[
  F = \frac{1}{2}(T_F+F) - \frac{1}{2}(T_F-F)
  \]
  where we call it as the\textbf{ Jordan decomposition} of $F\in BV$. 其中,  $ \frac{1}{2}(T_F+F)$ 被称为 $F$ 的 \textbf{positive variation}; $ \frac{1}{2}(T_F-F)$ 被称为 $F$ 的 \textbf{negative variation}.
\end{theorem}
\begin{proof}
    显然, $F\in BV \implies F$ bdd, 因为\[
    |F(y) - F(x) | \leq T_F(\infty) - T_F(-\infty)
    \]
    For $F\in BV$, we have $T_F(\infty) < \infty , \; T_F(-\infty)  = 0$.\\
    又 $F\in BV \implies T_F$ bdd by def, 我们得到: \[
        F \in BV \implies  T_F \pm F \text{ bounded}
    \]而反向 trivial (bounded function 的差仍然 bounded).
\end{proof}
\begin{remark}
我们可以通过 $F$ 和 $0$ 的 min, max 取到它的正负部分, 把它按正负部分分解成 \[
F = F^+ - F^-
\]
而这里, Jordan decomposition 则是按照它正负方向上的 variation 来分: \[
  F = \frac{1}{2}(T_F+F) - \frac{1}{2}(T_F-F)
\]
$F$ 的 \textbf{positive variation}, \textbf{negative variation} 和 total variation 一样也是很形象.
\begin{figure}
    \centering
    \includegraphics[width=0.5\linewidth]{assets/ch3-pics-Screenshot 2025-04-19 at 02.30.17.png}
    \caption{positive/negative variation} 
    \label{fig:positive/negative variation}
\end{figure}
我们把 \[
F_+ : = \frac{1}{2}T_F + F,\quad F_- : = \frac{1}{2} T_F - F
\]
(这和正负部分的拆分的记号差别在正负号的上下.)
\end{remark}

\subsection{corollaries of Jordan decomposition}
\begin{corollary}
Let $F \in BV$. By Jordan decomposition,  $F$ 等于两个 bounded increasing functions 的差.从而我们\textbf{ by MDT 得}: 
    \begin{itemize}
        \item $F(x+), F(x-)$ 存在 for all $x$; $F(\pm\infty)$ 也存在.
        \item $D_F = \{x: F \text{ disctn at }x \}$ 是 at most ctbl 的.
        \item 定义 $G (x) : = F(x+)$, 则 $F,G$ 都 a.e. differentiable 且 $F' = G'$ $m$-a.e.
    \end{itemize}
\end{corollary}
这里最重要的是: \[
F \in BV \implies F \text{ a.e. differentiable}
\]






\section{ $ NBV \;\&\; AC $ 空间, 以及其上的 FTC for Lebesgue integral [Fol 3.5, finished] }
\subsection{$NBV$ 及其性质}
\begin{definition}{NBV}
    For $F:\mathbb{R}\to \mathbb{C}$, 我们定义: $F \in NBV$, if $F \in BV$ 且 $F$ right ctn, $F(-\infty) = 0$.
\end{definition}
\begin{remark}
这个要求中 $F(-\infty) = 0$ 这一条并不要紧, 因为我们知道 for $F\in BV$, $F(-\infty) = c$ for some const $c$ 是一定的 (因为 $T_F(-\infty) =0$), 因而 right ctn 的 $F\in BV$ 减去一个常数一定是 $NBV$ 的.
    $F \in NBV$ 的
\end{remark}

\begin{proposition}
    $NBV \subset BV$ 是一个 linear subspace.
\end{proposition}
\begin{remark}
我们容易发现\[
    F \in BV \iff T_F \in BV \iff F_+, F_-  \in BV
    \]
又因为, $T_F (-\infty) = 0$ 对于 $F\in BV$ 总是成立, 且我们知道 $F$ right ctn $\implies T_F $ right ctn; 于是: \[
    F \in BV \text{ and right ctn} \implies T_F \in NBV \implies  F_+, F_- \in NBV
    \]
\end{remark}



我们已经知道:
$$\{\text{positive regular Borel measures on }\mathbb{R}\} \simeq \{\text{distribution functions }F:\mathbb{R}\to\mathbb{R} \}$$
Notice: 这一点 is achieved by 
$$
F_\mu(x ) := \begin{cases}
    \mu((0,x]) \quad  , x \geq 0 \\
     -\mu((x,0]) \; , x < 0
\end{cases}
$$
等同于: $$\mu_F ((a,b]) = F(b) - F(a)$$注意: positive regular Borel measure 和 distribution functions 都有一个共同点: 它们在 bounded set 上是 bounded 的, 但是整体可以 unbounded.\\

而现在我们证明:
\subsection{$\{\text{complex Borel measures on }\mathbb{R}\} \simeq NBV$}
注意: 一个 complex Borel measure 和一个 $F \in NBV$ 都是 finite 的.

\begin{theorem}{$\{\text{complex Borel measures on }\mathbb{R}\} \simeq NBV$}
\begin{enumerate}
    \item 对于 $\mathbb{R}$ 上的 complex measure $\mu$,  defining \[F(x) : = \mu((-\infty,x])\] 则有: \[F \in NBV\]
    \item 对于 $F \in NBV$, 一定存在某个 unique complex measure $\mu_F$ on $\mathbb{R}$, 使得 \[ \mu((-\infty,x]) = F(x) \quad \forall x \]
\end{enumerate}
\end{theorem}
\begin{proof}
    (1): 当 $\mu$ 是 positive 的情况下, 是显然的. complex 的情况就是 re/im 部分分别叠加即可.\\
    (2): 同样, WLOG 我们可以假设 $F$ 是 real-valued 的.\\
    $F\in NBV\implies F = F_+ - F_-$, 这两个都是 bounded increasing functions 且 NBV, 从而存在两个 finite signed measure 满足: \[  \mu_{\pm}((-\infty,x])  = F_{\pm}(x) - F_{\pm}(-\infty)  \] 再定义 \[\mu : = \mu_+ - \mu_-\] 即可
\end{proof}
\begin{remark}
    这里其实和 positive regular measure 关联 distribution function 的 way:
$$
F_\mu(x ) := \begin{cases}
    \mu((0,x]) \quad  , x \geq 0 \\
     -\mu((x,0]) \; , x < 0
\end{cases}
$$
是等价的, 只不过\textbf{差了一个常数 $\mu((-\infty,0])$} 而已. \\
之所以我们这里可以直接定义 \[F(x) : = \mu((-\infty,x])\]是因为, \textbf{对于 complex measure (thus finite) 而言, 这个常数 $\mu((-\infty,0])$ 一定是 finite 的. 但是对于 regular positive regular measure 而言, 这个常数可能是无穷} (且很可能). 因而对于 regular positive regular measure 我们采用这种迂回的定义方式来避开无穷值, 但是对于 complex measure 我们可以直接爽快地定义\[F(x) : = \mu((-\infty,x])\]
\end{remark}


\begin{theorem}{$\mu_F$ 的 total variation measure $=$ $\mu_{T_F}$}
    对于任意的 $F \in NBV$, 我们有: \[|\mu_F| = \mu_{T_F} \] Specially when $F$ 是 real-valued 情况下, 那么 $\mu_F$ 是一个 finite positive measure, 且有 \[\mu_{\pm} = \mu_{F_{\pm}}\]
\end{theorem}



Now: Given $F\in NBV$ with associated c.m. $\mu_F$, 什么时候 $\mu_F \perp m$, 什么时候 $\mu_F \ll m$?

\begin{theorem}{characterization of $\mu_F \perp m$ 和 $\mu_F \ll m$, for $F\in NBV$ }
    对于 $F \in NBV$, 我们已经知道: $F'$ $m$-a.e. 存在, 且 $F'\in L^1(m)$.\\
    Now we claim, 有: \[
    \mu_F \perp m \iff F' = 0 \quad m\text{-a.e.}
    \]以及 \[
        \mu_F \ll m \iff F(x) = \int_{-\infty}^x F'(t)\,d t \quad \forall x
    \]
\end{theorem}
\begin{proof}
Let $x\in \mathbb{R}$.  Applying LDT and LRNT, with $E_r: = (x,x+r]$: \[
    \lim_{r\to 0} \frac{\mu_F(E_r)}{m(E_r)} = \lim_{r\to 0} \frac{F(x+r)- F(x)}{r} = F'
    \]
    因而 $F'$ 就是这个 RN derivative. 对于 \[
    \mu_F = \lambda + \rho
    \]
    where $\lambda \perp m, \rho \ll m$, 我们有: \[F(x) : = \mu_F((-\infty,x]) =\lambda ((-\infty,x]) +\rho ((-\infty,x]) \]
我们知道, $    \mu_F \perp m \iff \rho = 0$, 从而 by LDT meets LRNT, we know that \[
F' = 0 \,\, a.e.
\]
而   $    \mu_F \ll m \iff \lambda = 0$, 从而直接 : \[F(x) : = \rho ((-\infty,x]) = F'dm((-\infty,x]) =   \int_{-\infty}^x F'(t)\,d t \]
\end{proof}



\subsection{$AC$ 及其性质}
\begin{definition}{absolutely continuous function}
我们定义 $F: \mathbb{R}\to \mathbb{C}$ 是 absolutely ctn 的, if 对于任意 $\epsilon > 0$ 都存在 $\delta > 0$ 使得对于任意的 disjoint intervals $(a_1,b_1),\cdots, (a_N,b_N)$, 都有: \[
\sum_1^N |F(b_j) - F(a_j)| <\epsilon \quad  \text{whenever}\quad \sum_1^N |b_j - a_j| <\delta
\]
\end{definition}
\begin{remark}
    absolutely continuous 是比 uniformly continuous 严格更强的条件: 我们只考虑 $N=1$ 而非任意正整数时这个 def 就 reduce 为 uniform ctn.\\
     absolutely continuous  表示了一种更强的控制性: 选取任意一些地方的变化足够小的 $x$, 其引发的 $y$ 的变化一定可控. ($y$ 的变化的可控性完全由 $x$ 的变化量决定, 不由 $x$ 的位置决定)\\
     这一看就和 measure 有关系.
\end{remark}
\begin{definition}{absolutely continuous function on a cpt interval}
我们定义 $F:I \to \mathbb{C}$ 是 absolutely ctn 的, if 对于任意 $\epsilon > 0$ 都存在 $\delta > 0$ 使得对于任意的 disjoint intervals $(a_1,b_1),\cdots, (a_N,b_N) \subset I$, 都有: \[
\sum_1^N |F(b_j) - F(a_j)| <\epsilon \quad  \text{whenever}\quad \sum_1^N |b_j - a_j| <\delta
\]
\end{definition}

\begin{lemma}{$F\in NBV$ abs ctn $\iff$ $\mu_F \ll m$}
    对于 $F\in NBV$, $$F \in AC\iff \mu_F \ll m$$
\end{lemma}
\begin{proof}
    我们 recall, abs ctn 除了 "$m$ 的 nullsets 也一定是 $\mu_F$ 的 null sets" 之外, 还有另一个 characterization: $\mu_F \ll m$ 当且仅当对于任意 $\epsilon > 0$ 都存在 $\delta > 0$ 使得 $ m(E) < \delta \implies|\mu_F(E)| < \epsilon$.\\
    显然, 这个 characterization 和这里的命题有关. 我们发现, $\mu_F  \ll m \implies F \in AC$ 直接 naturally follows from 这个 form. Let $\epsilon  >0$, 存在 $\delta$ 使得 $ m(E) < \delta \implies|\mu_F(E)| < \epsilon$. 那么考虑 $E = \bigsqcup_1^N (a_j,b_j)$ with $m(E) < \delta $, 直接有\[
    |\mu_F(E)|  = \sum_1^N |F(b_j) - F(a_j)| < \epsilon
    \]
从而得证.\\
而反向, 我们考虑 $m(E) = 0$, 并利用 outer regularity 取一个逼近它的 open set (每个是 union of finite disjoint open intervals) seq, 逼近 $E$, with $m(U_1) < \delta$. 由 $F \in AC$ 可以得到 \[
\mu_F(U_j) \leq \mu_F(U_1) < \epsilon
\] for all $j$, 从而 $\mu_F(E) \le \epsilon$. 从而得证, since $\epsilon $ arbitrary.
\end{proof}


\subsection{FTC for Lebesgue integral on $\mathbb{R}$: requires $NBV + AC$}
\begin{corollary}{}
    如果 $f \in L^1(m)$, 那么 \[
    F(x) : = \int_{-\infty}^x f(t) \,d t \in NBV \cap AC,\quad     f = F' \quad a.e.
    \]
Conversely, 如果 $F \in NBV \cap AC$, 那么 \[
    F' \in L^1(m), \quad F(x) = \int_{-\infty}^x F'(t) \, dt
    \]
\end{corollary}
\begin{remark}
forward 方向即: $f\in L^1(m)$ 则它的累积函数 $\int_{-\infty}^x f(t) \,d t $ 具有良好的连续性和变差有界性, 从而满足 FTC: a.e. 有 \[
\frac{d}{dx} \bigg( F(x):=\int_{-\infty}^x f(t) \,d t  \bigg) = f(x)
\]
并且 for $  F(x) : = \int_{-\infty}^x f(t) \,d t $, 这个函数是 $NBV \cap AC$ 的. 这可以推出: \[
F(x) - F(y) = \int_y^x f(t) \, dt
\]

backward 方向即: 如果 $F$ 具有良好的连续性和变差有界性, 那么它的导数满足: \[F(x) = \int_{-\infty}^x F'(t) \, dt    \]
简而言之: \textbf{FTC-I for Lebesgue integral 在整个 $\mathbb{R}$ 上都成立当且仅当 $F \in NBV \cap AC$.}
\end{remark}


\subsection{FTC for Lebesgue integral on a cpt interval: 只需要 $AC$}
比起刚才的 FTC-I, FTC-II 的条件要宽松很多, 只需要 $F$ 在它需要被用到的 compact interval 上 AC 即可以. 这是因为, 我们不需要用到 NBV 只需要 BV, 并且在 cpt interval 上, AC 本身就可以推出 BV.
\begin{lemma}
    如果 $F\in AC([a,b])$, 那么 $F\in BV([a,b])$.\\
    即 \[
    AC([a,b]) \subset BV([a,b])
    \]
\end{lemma}
\begin{proof}
    这是显然的. 我们看到 $F \in AC([a,b])$ 的定义: on $[a,b]$ we have: \[
\sum_1^N |F(b_j) - F(a_j)| <\epsilon \quad  \text{whenever}\quad \sum_1^N |b_j - a_j| <\delta
\]
我们不妨考虑 $\epsilon = 1$, 然后可以划分 $[a,b]$ into 一个个总长度为 $\frac{\delta}{2}$ 的由 disjoint intervals 构成的块 (补空缺没事), 从而 Bound 住这个划分上的 variation by parition \(\sum_1^{N_0} |F(b_j) - F(a_j)| \). 而我们发现: 这个时候我们不论怎么 fine 这个划分, 每个块的总长度总归是不变的, 从而仍然可以使用原先的 bound.\\
我们 recall: for total variation, partition 的选取是 greddy 的. 从而这就足以得证.
\end{proof}
\begin{remark}
这里没有仔细证明, 但是理解这个 idea 即可. 这表现了 locally, $AC$ 是一个比 $BV$ 更强的条件, 因为 total variation 就是 sup of variations over 所有划分, 而 \textbf{$AC$ 的定义正好就是: 无视划分的方法, 只要这个集合的总长度小于 $\delta$, 它上面的 variation by partition 就要小于 $\epsilon$. }
\end{remark}

\begin{theorem}{FTC-II for Lebesgue integral on a cpt interval}
TFAE:
\begin{itemize}
    \item $F\in AC[a,b]$
    \item $F$ 是 diffble a.e. on $[a,b]$ 的, 且 $F'\in L^1([a,b],m)$, 且 \[F(x) - F(a) = \int_a^x F'(t)\, dt\]\textbf{for all $x\in [a,b]$}.
\end{itemize}
\end{theorem}
\begin{proof}
    首先, $F\in AC[a,b]\implies F \in BV[a,b] \implies F$ diffble a.e. on $[a,b]$. \\
    Notice that: 这里我们只考虑 $F|_{[a,b]}$, 于是我们可以将其他部分的值都设为 smooth 的, 并 normalize it: 把 $F(x) := 0 $ for $x< a$, $F(x): = b$ for $x > b$.\\
    又 $F \in AC \implies F$ right ctn for sure, 我们 then have: $F \in NBV$, 从而
\end{proof}



\subsection{characterization for Lipschitz ctn}
\begin{theorem}{characterization for Lipschitz ctn}
    对于 $F:\mathbb{R}\to \mathbb{C}$, we have: \[
    F \text{ Lipschitz ctn with const } M \iff  F \in AC \text{ and } |F'(x)|\le M \text{ a.e.}
    \]
\end{theorem}
\begin{proof}
见 HW 12.
\end{proof}

从而我们得到: ctnity 条件的递推关系: \[
\text{ Lipschitz ctn }\implies   \text{ abs ctn }\implies   \text{ uniformly ctn } \implies  \text{ ctn } 
\]
关于连续函数的可导性:  Lipschitz ctn 可以推得 a.e. diffble $+$ bounded derivative; abs ctn 可以推得 a.e. diffble 且在 cpt interval 上 derivative $L^1$; 而往后的 uniform ctn 则不蕴含可导性条件.


而关于和可导性紧密相关的变差性质: 
abs ctn 在 bounded interval 上是比 BV, NBV 更强的条件, 而在 $\mathbb{R}$ 上则并不是.

在 $\mathbb{R}$ 上, NBV + AC 的函数可运用 FTC. 而在 bounded area 上 AC 的函数就可以运用 FTC.