\chapter{$L_p$ space and inequalities}
\section{Banach Space and $L_p$ space [Fol 5.1; 6.1]}
对应  Folland 5.1(1), 6.1(1).
\subsection{norm and completeness}
Recall:
\begin{definition}{semi-norm, norm}
  一个\textbf{semi norm} 是一个函数 $||\cdot||: V \to [0,\infty)$ starting from a vector space $V$. 其满足 (1): tri eq 和 (2): homogeneity.\\
   如果一个 semi-norm 满足 (3): $||v|| = 0$ iff $v = 0$, 则称它为一个 \textbf{norm}.
\end{definition}

\begin{definition}{Banach space}
一个 normed vector space $(V, ||\cdot||)$ 的 induced metric space 如果是 complete 的, 它就被称为一个 \textbf{Banach space}.\\
\end{definition}
\begin{remark}
    Cauchy 指的是对于任意 $\epsilon$, 都存在 $N$ 使得对于任意 $n,m \geq N$ 都有 \[
    \| v_n - v_m\| < \epsilon
    \]
而 convergent 指的是存在一个极限 $v$, 使得对于任意 $\epsilon$, 都存在 $N$ 使得对于任意 $n\geq N$ 都有 \[
\|v_n - v\| \leq \epsilon
\]
By tri ineq 容易证明: 在 genral normed VS 中, convergent imply Cauchy, 反之未必. convergent 是更强的条件. (interestingly, convergence in measure 却不 imply Cauchy in measure)
\end{remark}

\begin{example}
    $\mathbb{R}^n, \mathbb{C}^n$ with Euclidean norm is a Banach space.\\
    $C^0([0,1])$: space of ctn functions on $[0,1]$ equipped with $\sup$ norm\textbf{ is Banach}. \[
    ||f- g|| := \sup_{x\in[0,1]} |f(x) - g(x)|
    \]
    $C^0_c(\mathbb{R})$: space of ctn functions with cpt supp on $\mathbb{R}$ equipped with $\sup$ norm\textbf{ is not Banach}! 这是因为, 一个有 cpt supp 的 function seq 的极限未必有 cpt supp. 比如 $(\chi_{[-n,n]})_{n\in \mathbb{N}}$.
\end{example}

\begin{lemma}
    A metric space $(X,\rho)$ is \textbf{complete} iff \textbf{every Cauchy seq has a subseq that converges.}
\end{lemma}
\begin{proof}
Trivial. \\
\(\implies\): Clear. \\
\(\impliedby\): subseq conv dist bound + Cauchy dist bound can bound the whole tail with arbitrary $\epsilon$.
\end{proof}这个 statement, 直接把 complete 的定义从每个 Cauchy seq 都收敛, 优化为每个 Cauchy seq 都有一个收敛 subseq. 

\subsection{every Cachy seq conv (complete) $\iff$ every abs conv series convs}
\begin{definition}{series: convergence 和 absolute convergence}
对于一个 normed VS $(V,||\cdot||)$ 中的 seq $(v_n)$, 我们称 \(\sum_{n=1}^\infty v_n\) \textbf{converges}, 如果存在 $v\in V$ s.t. \[
\lim_{N\to\infty} \sum_{n=1}^N v_n  = v
\]
即 \[
\lim_{N\to\infty} \|  v - \sum_{n=1}^N v_n\| = 0
\]
我们称 \(\sum_{n=1}^\infty v_n\) \textbf{absolutely converges}, 如果 \[ \sum_{n=1}^\infty ||v_n|| < \infty\]
即这个 series 对应的 norm series converges to some real number.
\end{definition}


\begin{theorem}{another criterion for Banach space}
\label{Equiv Condition for space being Banach}
    A normed VS $(V,||\cdot||)$ is a Banach space iff every absolutely convergent series converges.
\end{theorem}
\begin{proof}
    ``$\implies$": 如果 $(V,||\cdot||)$ is a Banach space,   Suppose $\sum_{n=1}^\infty ||v_n|| < \infty$, 取部分和序列 \[
    S_N := \sum_{n=1}^N v_n 
    \]有 \[
   \|S_m - S_n\| = \Bigl\|\sum_{k=n+1}^m v_k\Bigr\|
   \le \sum_{k=n+1}^m \|v_k\|\]
 For  large enough $m,n$ 这个 bound 可以无限小, 因而 $(S_N)$ is Cauchy. 
    ``$\impliedby$": 如果 $(V,||\cdot||)$ 中 every absolutely convergent series converges.\\
    Suppose $(v_n)$ is Cauchy. WTS it converges.\\
By Cauchy, 存在 subseq, say labeled $n_1 < n_2 < \cdots $, s.t. $||v_{m} - v_{n}|| < \frac{1}{3^j} $for all $m,n\geq n_j$
    Then \[
    \sum_{j=1}^\infty ||v_{n_{j+1}} - v_{n_j}|| < \infty
    \]
    Let $(y_j)$ be s.t. $y_1 = v_{n_1}$, $y_j = v_{n_{j+1}} - v_{n_j}$, then \[
\sum_{j=1}^\infty \| y_j\| \leq \| y_1 \| + \sum_j {\frac{1}{2^j}}  = \|y_1\| + 1 < \infty
  \]
并且有: \[
    v_{n_j} = \sum_{k=1}^j y_k
    \]由于 $\sum_{j=1}^\infty \| y_j\| < \infty$, by our assumption 得到, 这个极限 $    \lim_{j\to \infty} v_{n_j}  = \sum_{k=1}^\infty y_k $ 是存在的.
\end{proof}
\begin{remark}
    这个证明中也有一个简略但是有用的结论: 任意 normed VS 中, \textbf{一个 series absolutely convergent 可以推出它的部分和 seq 是 Cauchy 的. (反向则未必成立).}\\
    整个 imply 关系的示意图:
\begin{align*}
    \sum_{k=1}^{\infty} \|x_k\| <\infty   \implies S_N  \text{ Cauchy}& \overset{\text{if Banach}}{\implies}  S_N \text{ converges}  \Longleftrightarrow  \sum_{k=1}^{\infty} x_k\text{ converges} \implies (x_k) \to 0
        \\    &\overset{\text{always}}{\impliedby}
    \end{align*}
(这个图直观说明了为什么 Banach 和 "every abs conv seq conv" 是等价的. 因为这只是\textbf{在 Cauchy imply conv 的前后套了两个必然发生的 implication 关系}而已. 但有时候, 这个关系反而更加好证明.)\\
\textbf{(注意, partial sum seq Cauchy 并不 imply 原 series absolutely converge!})
\end{remark}


\subsection{任何 finite dim normed VS 一定 Banach, infinite dim 则不一定 Banach }

\begin{remark}
    Note, 我们知道在 $\mathbb{R}^n$, $\mathbb{C}^n$ 上, abs conv 一定 imply con; 但是在 general (infinite dimension) 的 normed VS 上, \textbf{absolutely converge 并不 imply converge.}\\

1. As is known to all, \(\mathbb{R}^n,\mathbb{C}^n\) 上 Euclidean norm 的 induced metric 就是 Euclidean metric, making it complete metric space, 从而是 Banach space.\\
2. recall in elementary functional analysis: 
\begin{definition}
    我们称两个 norms $\| \cdot\|_a, \| \cdot\|_b$ on a vector space 是 equivalent, 如果存在常数 \(C_1, C_2 > 0\) 使得对于任意 $x$ 都有 \[C_1\|x\|_a \leq \|x\|_b \leq C_2\|x\|_a \]
这一定义即 topologically equivalent. 因为 equivalent norms define \textbf{equivalent metric, 从而 same topology. }
\end{definition}
以及这个经典的定理:
\begin{theorem}
finite dimensional vector space $X$ 上, 所有 norms 都 equivalent.
\end{theorem}
这里先不证明. \\
利用这个定理, 我们发现 \textbf{ \(\mathbb{R}^n,\mathbb{C}^n\) 上采用任何 norm 都是 Banach space.}\\
3. 我们 recall: 任何 finite dim $\mathbb{R}$-vector space 或者 $\mathbb{C}$-vector space 都 isomorphic to some $\mathbb{R}^n$, $\mathbb{C}^n$. 因而
利用这 theorem, 我们得到: \textbf{任何 finite dim normed VS 都是 complete metric space (Banach space), regardless of choice of norm.}\\
然而\textbf{ infinite dim normed VS  则未必一定 Banach.}\\
一个常见的反例: \[
    V = \mathbb{R}[x]
  \]
所有的 polynomials with real coeffs. 考虑这一 norm:  \[
    \|p\|_{\infty} = \sup_{x \in [0,1]} |p(x)|
  \]
\(\mathbb{R}[x]\) 是无限维的, 因为多项式的次数可以任意提高.\\
$(\mathbb{R}[x], \| p\|_{\sup})$ 不是 complete 的, 其 completion 是 Banach 空间 \(C[0,1]\), 所有在 \([0,1]\) 上的连续函数, with the same \(\sup\) norm.\\
\end{remark}
下面我们将介绍一类 infinite dimension 但是 Banach 的 normed VS: $L^p$ spaces.




\subsection{$L^p$ spaces}
\begin{definition}{$L_p$ spaces}
    Consider $p\in(0,\infty)$.\\
    Let $(X,\mathcal{A},\mu)$ 为一个 measure space.\\
    Define for $f: X \to \mathbb{R}$ measurable: \[
    ||f||_p : = \Big(\int |f|^p \; d\mu  \Big)^{\frac{1}{p}} \;\; \in [0,\infty]
    \]
    Define \[
    L^p(\mu) : = \{ f :  ||f||_p  < \infty \} / \sim
    \] where $f \sim g$ if $f=g$ a.e.
\end{definition}
固定一个 measure space $(X,\mathcal{A},\mu)$, 我们将用 $L_p$ 来简易指代 $ L^p(\mu)$.\\
\begin{remark}
    注意, 我们容易发现:if $ 0<p< \infty$ and $f$ measurable, TFAE:
\begin{itemize}
    \item $f \in L^p$
    \item $|f| \in L^p$
    \item $|f|^p \in L^1$
\end{itemize}
\end{remark}

\begin{example}
    $(X,\mathcal{A},\mu) := (\mathbb{R},\mathcal{L}, m)$, \[
    f(x) : = \frac{1}{x^\alpha} \chi_{(0,1)}, \;\; f\in L^p(m) \Longleftrightarrow   \alpha p < 1
    \]  \[
    f(x) : = \frac{1}{x^\alpha} \chi_{(1,\infty)}, \;\; f\in L^p(m) \Longleftrightarrow   \alpha p > 1
    \]
    $(X,\mathcal{A},\mu) := (\mathbb{N},\mathcal{P}(\mathbb{N}), \mu_{counting})$, \[
    L^p( \mu_{counting}) = \{ (a_n)_{n\in\mathbb{N}} : \sum_{n=1}^\infty |a_n|^p < \infty   \}
    \]
\end{example}

\begin{lemma}{$L_p$ space is a vector space}
    $L_p$ space is a $\mathbb{C}$-vector space.
\end{lemma}
\begin{proof}
 Suppose $f, g\in L^p $.\\
由于 \begin{align}
        |f+ g|^p   \leq (|f| + |g| )^p \leq (2 \max\{|f|, |g|\})^p  \leq 2^p (|f|^p + |g|^p)
    \end{align}
于是 by linearity of integral, 得到:
$$f, g\in L^p \implies f+g \in L^p$$
(Note: $p>1$ 时也可以 by $|x|^p$ 这一函数的 convexity 得到这个 bound, 但是这个方法只有效于 $p>1$)
\end{proof}

但是 \textbf{Question 1: }\textbf{Is $L_p$ a normed VS? 即, $\| \cdot\|_p$ 总是一个 valid norm 吗?}
A: \textbf{True for $p\in [1,\infty)$, false for $p\in (0,1)$.} 
Homogeneity 和 $\|f\|_p = 0$ iff $f = 0$ (a.e.) 是显然的, 但是我们发现, tri ineq 没有显然的证明.\\
Next lecture, we will show the Minkowski's ineq, 即 $L^p$ space 上的三角不等式:
\[
||f + g||_p \leq ||f||_p + ||g||_p
\]
但是这个不等式只 hold for $p\in [1,\infty)$, 并且 fail otherwise.\\
(因而对于 $L^p$ space 的研究, 我们将 \textbf{focus on $p \in [1,\infty)$ 的情况.})\\

\textbf{Question 2: Is $L^p$ space, $p\in [1,\infty)$, Banach? Answer: Yes.}\\
我们也将在 next lecture 证明它.




\section{inequilities on $L^p$ spaces [Fol 6.1]}
对应 Folland 6.1(2).\\
我们将证明 Hölder's ineq 以及它的 corollary Minkowski's ineq, 从而证明: $L^p$ 是一个 normed VS, 并且是一个 Banach space (这里 $1\leq p <\infty$, 但是 later we will also prove $L^\infty$ 也是 Banach space).\\
这两个不等式非常重要.

\subsection{Hölder's ineq}

\begin{theorem}{Hölder's ineq}
    Consider conjugate pair: $p,q \in [1,\infty)$ s.t.  \[
    \frac{1}{p} + \frac{1}{q} = 1
    \]
则对于任意两个 measurable function $f,g:X \to \mathbb{C}$,  一定有: \[
 \| fg\|_1 \leq \| f\|_p \cdot \|g\|_q 
 \]特别地, 如果 $f \in L^p(\mu)$, $g\in L^q(\mu)$, 则 $fg \in L^1(\mu)$, 并且 equality holds iff \[
   \|g\|_q^q |f|^p = \|f\|_p^p |g|^q \quad \mu\text{-a.e.}
 \]
\end{theorem}
\begin{remark}
For $(p,q) = (2,2)$, this is \textbf{Cauchy-Swartz ineq}: \[
 \| fg\|_1 \leq \| f\|_2 \cdot \|g\|_2
\]
即: \[
 \bigg(\int  f\,\overline{g}   \bigg) \leq \int |fg| \leq  \sqrt{\bigg(\int |f|^2 \bigg)\bigg(\int |g|^2 \bigg)}
\]
\end{remark}
\begin{proof}
Trivial Case 1: 如果 $\|f\|_p = 0$ (或者$\|g\|_q = 0$  ), then $f$ is zero $\mu$-almost everywhere, and the product $fg$ is zero $\mu$-almost everywhere, 于是两边都是 $0$, ineq trivially true.\\
Trivial Case 2: 如果 $\|f\|_p = \infty$ or $\|g\|_q = \infty$, 则右边 infinite, ineq trivially true.
因而我们只需要考虑 $\|f\|_p$ and $\|g\|_q$ are in $(0, \infty)$ 的情况就好了.\\

Main case: 我们需要一个 Lemma: 
\begin{lemma}{Young's inequality for products}
Whenever $p, q \in (1, \infty)$ with $\frac{1}{p} + \frac{1}{q} = 1$, 都有
\begin{equation}
    ab \leq \frac{a^p}{p} + \frac{b^q}{q} ,\quad \forall a,b\geq 0
\end{equation}
where equality is achieved if and only if $a^p = b^q$.\\
另一个等价形式是: \[
a^{\lambda} b^{1-\lambda} \leq \lambda a + (1-\lambda)b,\quad \forall a,b\geq 0
\]
\end{lemma}
\begin{proof}
    \textbf{of Lemma:}\\
    $b=0$ 则 trivial case. 因而 setting $t : =\frac{a}{b}$, reduced to show: \[
    t^\lambda \leq \lambda t  + (1-\lambda)
    \]
  with eq iff $t=1$. 这是显然的, 因为 by Calculus, $t^\lambda -\lambda t$ 是 strictly increasing for $t<1$, strictly decreasing for $t>1$ 的, max 在  $t=1$, 正好是 $1-\lambda$.\end{proof}
使用 Young's inequality for products 得到:
\begin{equation}
    \frac{|f(x)|}{\|f\|_p} \frac{|g(x)|}{\|g\|_q} \leq \frac{|f(x)|^p}{p \|f\|_p^p} + \frac{|g(x)|^q}{q \|g\|_q^q}, \quad x \in X
\end{equation}

Integrating both sides gives
\begin{equation}
    \frac{\|fg\|_1}{\|f\|_p \|g\|_q} \leq \frac{\|f\|_p^p}{p \|f\|_p^p} + \frac{\|g\|_q^q}{q \|g\|_q^q} = \frac{1}{p} + \frac{1}{q} = 1,
\end{equation}
which proves the claim.\\
Integration 的 equality holds iff point equality holds a.e., 并且, by Young's inequality for products, 上面的 equality holds iff \[   \|g\|_q^q |f|^p = \|f\|_p^p |g|^q \quad \mu\text{-a.e.}\]
\end{proof}
\begin{remark}
1. 显然, 根据我们的证明过程可知: \textbf{Hölder's ineq also holds on any measurable subset $S \subset X$}: \[
\int_S |fg| \leq  \bigg(\int_S |f|^p \bigg)^{\frac{1}{p}}\bigg(\int_S |g|^q \bigg)^{\frac{1}{q}}
\]
2. 这里的满足 $\frac{1}{p} + \frac{1}{q} = 1$ 的 $p,q$ 我们称之为: \textbf{Hölder conjugate}, 并称它们互为对方的 \textbf{conjugate exponent}.\\
3. 左边实际上是两个正值函数的 inner product, 相当于把一个投影到另一个上;  \\
几何直观: Hölder's ineq 在退化为 Cauchy-Swartz 时表示, 两个函数/向量的内积一定小于等于长度积; 而 Hölder's ineq 更广义: 表示它们的内积一定小于它们取任意相互 conjugate 的 norm 长度的积.
并且 sooner 我们会学到: 对作为 Hölder conjugates 的 $p,q$, $L^p$ 和 $L^q$ 互为 dual space, 从而 Hölder ineq 表示的是就是 norm 与其 dual norm 之间的 maximal inner product 控制关系.
\end{remark}


\begin{remark}
    Hölder's ineq 有一个 generalization:
    对于任意 $0<s<\infty$ and $0<p_1,\dots, p_n< \infty$ such that 
\[
  \frac1{p_1}+\frac1{p_2}+\dots+\frac1{p_n}=\frac1{s};
\]
都有
\[
  \| f_1f_2\cdots f_n\|_s\le \|f_1\|_{p_1}\|f_2\|_{p_2}\cdots \|f_n\|_{p_n}.
\]

    This generalization will be proved in hw8.
\end{remark}



\subsection{Minkowski's ineq: tri ineq on $L^p$, 确认 $\|\cdot\|_p$-norm 是 $L^p$ 上的 valid norm }
Minkowski's ineq 即 $L^p$ space 上的 tri ineq.
\begin{corollary}{Mincowski's ineq}
对于任意 $1\leq p < \infty$, 都有: \[
\|f +g \| \leq \|f\|_p  + \| g\|_p
\]
\end{corollary}
\begin{proof}
显然, 对于任意 $x$ 都有: \[
    |f + g|^p \leq \bigg( |f| + |g|\bigg) |f+g|^{p-1} 
    \]
    因而: \begin{align*}
           \int |f + g|^p &\leq \int |f| \cdot |f+g|^{p-1} \; + \; \int |g| \cdot |f+g|^{p-1}
    \end{align*}
    我们定义 $$h(x):= |f(x)+g(x)|^{p-1}$$于是 \begin{align*}
          \int  |f + g|^p &\leq \int |fh| + \int |gh| \\
            &\leq \|f\|_p \| h\|_q + \|g\|_p \| h\|_q \\
            &= \bigg(\|f\|_p + \|g\|_p \bigg)  \bigg(\int |f+g|^{(p-1)q}\bigg )^{1/q}
    \end{align*}
其中 $q$ 是 $p$ 的 Hölder conjugate.  这里的 punchline is actually: 由于 \[
q : = \frac{p}{p-1}
\] actually, \[
(p-1)q = p
\]
因而: 
\begin{align*}
        \int  |f + g|^p  &\leq \bigg(\|f\|_p + \|g\|_p \bigg)  \bigg(\int |f+g|^{(p-1)q}\bigg )^{1/q}\\ &= \bigg(\|f\|_p + \|g\|_p \bigg)  \bigg(\int |f+g|^{p}\bigg )^{1/q}\\
        & = \bigg(\|f\|_p + \|g\|_p \bigg)  \bigg(\int |f+g|^{p}\bigg )^{1-1/p}
\end{align*}
两边同时除以 $\big(\int |f+g|^{p}\big )^{1-1/p}$  得到: \[
\bigg(  \int  |f + g|^p \bigg)^{1/p} =: \| f+g\|_p \leq \|f\|_p + \|g\|_p
\]
从而得证.
\end{proof}
\begin{remark}
    这里的技巧是: 把一个 $p$ 次方的函数拆成一个 $1$ 次方的函数和一个 $p-1$ 次方的函数, 并且使用Hölder, 这样就得到了一个 1 次的函数的 $p$-norm 和另一个 $p-1$ 次的函数的 $q$ norm, 但是注意 $q(p-1) = p$, 因而这个函数就变成了 \[
    \bigg( \int |\phi|^p \bigg)^{1/q}
    \]的形式. 并且注意到: \[
    \frac{ \displaystyle \int |\phi|^p }{    \bigg( \displaystyle\int |\phi|^p  \bigg)^{1/q}} = \bigg( \displaystyle\int |\phi|^p  \bigg)^{1/p} = \|\phi\|_p
    \]
\end{remark}
\begin{remark}
    Minkowski 不等式证明的是 \(1\leq p < \infty\) 时的 $p$-norm 的三角不等式. 但是对于 $0<p<1$, 它并不成立. 因为这个时候 $p-1< 0$, 我们刚才的证明不作效.\\
    直观的证明: 在 $p\geq 1$ 的时候, $|x|^p$ 是一个 strictly convex 的函数; 而在 $0< p<1$ 的时候, $|x|^p$ 则是一个 strictly concave 的函数.\\
因而我们运用 strictly concave 的性质: \[
|a+b|^p > |a|^p + |b|^p
\]
再由积分可得到反例. (比如取 indicator function 进行积分)
\end{remark}

\subsection{properties of $L^p$ spaces ($1\leq p < \infty$)}
\subsection{$L^p$ ($1\leq p < \infty$) is Banach}
\begin{theorem}{$L^p$ space ($1\leq p < \infty$) is Banach}
    $L^p$ ($1\leq p < \infty$) is Banach.
\end{theorem}
\begin{proof}
    By last lec 的定理: 一个 NVS 是 Banach 的等价条件是任意 abs conv series 都 conv. 因而我们证明这一点即可.\\
Suppose $f_n \in L^p$ for each $n$, 并且这个 series abs conv, 即: \[
B := \sum_{k=1}^\infty \| f_k \|_p < \infty
\]
我们 define: \[
g(x) : = \sum_{k=1}^\infty f_k(x),\quad g_n(x) : = \sum_{k=1}^n f_k(x)
\]
我们 WTS: \[
\lim_{n\to\infty} g_n = g
\]
in $p$-norm induced metric sense, 即, for some $f \in L^p$, 有 \[
\lim_{n\to\infty} \big\|  g - g_n \big \|_p = 0
\]
我们 Set: \[
G_n := \sum_{k=1}^n |f_k|,\quad G:= \sum_{k=1}^\infty |f_k|
\]
这个函数以及函数列的定义是为了使用 DCT, 作 donimating function 用. \\
By measurable function 的 limit behavior, 有 \[
G_n,G\in L^+
\]
并且 \[
\|G_n\|_p \leq \sum_{k=1}^n \|f_k\|_p \leq B
\]
由于 $G_n \nearrow G$, by MCT 有 \[
\int G^p = \lim_{n\to \infty} \int G_n^p \leq B^p < \infty
\] 由于 $G\in L^p$, 有 $$G(x) < \infty \quad a.e.$$ 于是: \[
g(x) : = \sum_{k=1}^\infty f_k(x) <\infty \quad a.e.
\]
又 $|g_n|,|g|\leq G, g_n \to g$, 可得到: \[
|g_n - g|^p \leq 2^p G^p \in L^1
\] 因而 by DCT 可以得到: \[
\lim _n\int |g_n - g|^p    = 0
\]
从而\[
  \lim_{n\to\infty} \big\|  g - g_n \big \|_p = \bigg( \lim _n\int |g_n - g|^p   \bigg)^{1/p} = 0
\]
\end{proof}
\begin{remark}
1. 我们说一个 function seq converge to 一个 function 指的是 in the sense of distance, 而这里就是 metric induced by norm, 即\textbf{它们的差的 $L_p$ norm converge to $0$.}\\
2. 注意, 我们 recall: $f_k \to f$ a.e. 并不说明 $f_k\to f$ in $L^1$, 因为每个点 converge 的速度不一样. 当然, 对 $L^p$ 也同理.\\
3. \textbf{虽然 a.e. convergence 不能推出 $L^p$ convergence, 但是配合 DCT, 则可以推出.} \textbf{DCT 是我们证明 $L^p$ convergence 的关键.}\\
4. 要证明 \[
  \lim_{n\to\infty} \big\|  g - g_n \big \|_p = 0
\]完全可以忽略积分外的 $1/p$ 次方. 其实只需要证明 \[
\lim _n\int |g_n - g|^p    = 0 
\] 就可以了. 证明 $L^p$ convergence, 比起 $L^1$ convergence 略困难的地方就是被积函数变得更大了.
\end{remark}


\subsection{Criterion for $L^p$ convergence: 逐点 a.e. conv $+$ $L^p$ 积分值 conv}
我们刚才 mention: DCT 对于 function seq $L^p$ convergence 的证明有很大作用. 这里我们就提供一个 DCT 推出的 $L^p$ convergence 的判断准则: 
\begin{theorem}{Criterion for $L^p$ convergence}
    if $f_n\to f$ a.e. and $\|f_n\|_p\to\|f\|_p$, then $\|f_n-f\|_p\to0$. 
\end{theorem}
即 \[
\text{a.e. conv } + L^p\text{  norm conv} \implies L^p conv 
\]
但是 converse 并不成立. 反例是 typewriter function.
\begin{proof}
    In Hw 8.
\end{proof}



\subsection{dense subsets of $L^p$, and specially $L^p(\mathbb{R},m)$ }
\begin{proposition}
    对于任意 $1\leq p < \infty$, the set of $\{$simple functions$\}$, is dense in $L^p$.\\
    即: \[
    \{f:X \to \mathbb{C}\mid f=\sum_1^n  a_j \chi_{E_j},\mu(E_j)<\infty \}
    \]是 $L^p$ 的 dense subset.
\end{proposition}
\begin{remark}
    我们已经 proved this for $L^1$, 而其实这个 density 推广至 $L^p$ 也成立.
\end{remark}
\begin{proof}
对 $f$ 使用 simple function seq 逼近, 使用 $2^p |f|^p$ 作为 dominating function of $|f_k-f|^p$; 而后使用 DCT 得证.
\end{proof}

\begin{theorem}{$C_c^0(\mathbb{R}^n)$ is dense in $L^p(\mathbb{R},m)$ for $1\leq p < \infty$}
$C_c^0(\mathbb{R}^n)$ is dense in $L^p(\mathbb{R},m)$ for $1\leq p < \infty$
\end{theorem}
\begin{proof}
    exercise. Similar to the proof for $L^1$, 只需要使用加入 $p$ power 的 function 作为 dominating function 即可.
\end{proof}




\section{$L^\infty$ space, and relationship between $L^p$ spaces ($0\leq p \leq \infty$) [Fol 6.1, finished]}
对应 Folland 6.1(3), finishing 6.1.\\
我们已经完成了对 $1\leq p < \infty$ 的 $L^p$ space 的构建. 现在, 我们来构建最后一块拼图: $L^\infty$ space.
\subsection{$L^\infty$ space}
我们考虑这个启发式的例子: \[
X := \{ 1,2,\cdots, n\},\quad\mathcal{A} := \mathcal{P}(X),\quad \mu = \mu_{counting}
\]
于是: \[
L^p (\mu)   = \{(a_1,\cdots, a_n) : \| (a_1,\cdots, a_n) \|_p = \big( \sum |a_i|^p \big)^{1/p} < \infty\}= \mathbb{C}^n
\]
我们发现: \[
\| (a_1,\cdots, a_n) \|_p \to \max_j |a_j| \quad \text{as}\quad p\to \infty
\]
因为 $p$ 取得越大, 最大的 entry 的 contribution 占比就越突出.\\
对于这样的 $L^p$ space, 我们可以定义 $\sup$ norm, 定义为最大的 entry.\\
即便 $X$ 是 countable 的, 这个定义也可以定义为 $\sup_j |a_j|$, make sense.\\
那么如果我们想要给任意的 measure space 定义 sup norm 呢? 我们可以考虑 \[
\| f\|_\infty : = \sup_{x\in X} |f(x)| \; ?
\]
实际上我们有更好的定义方式: 

\begin{definition}{essential supremum}  \[
    \| f\|_\infty : = \inf \{ a\geq 0 : \mu\{x:|f(x)|>a\}  = 0 \}
    \]
    也可以写作: \[
    \text{ess} \sup_{x\in X} |f(x)|
    \]
\end{definition}
\begin{remark}
essential sup 是一个比较容易搞错的定义. \\
一个 function 的 essential supremum 即: 这个 function 几乎处处的 sup.\\
它 $\leq \sup f$ , 因为它允许在零测集上存在一些点的函数值大于它.\\
这是合理的, 因为积分可以不考虑零测集.\\
\end{remark}
\begin{remark}
  对于零测集只有空集的 measure space 上的函数, 比如  对于 $\ell^\infty(\mathbb{N})$ 上的函数, 其 essentail supermum 即 supermum.\\
  对于\[
   \sup_{x\in X} |f(x)| 
  \]我们也有一个称呼, 称其为 \textbf{uniform norm}. 即: \[
\|f\|_u :=   \sup_{x\in X} |f(x)| 
  \]
\end{remark}

\begin{definition}{$L^\infty$ space} \[
    L^\infty(\mu) : = \{ f:X \to \mathbb{C} \text{ measurable} : \| f\|_\infty < \infty \} / \sim
    \]
    where $\sim$ 表示 a.e. 相等的函数的 equiv class.
\end{definition}
\begin{remark}
    注意: \textbf{$f\in L^\infty(\mu)$, 并不等价于 $f$ a.e. bounded!}\\
实则 recall: $f$ a.e. \textbf{bounded  是 $f \in L^p (\mu)$ for any $1\leq p \leq \infty$ 的必要条件}, 否则, 函数积分不可能 $<\infty$, 函数 $p$ 次方的积分更加不可能 $<\infty$.\\
$f\in L^\infty(\mu)$ 是一个很严格的条件, 当然严格强于 $f$ a.e. bounded. \\
比方说: \textbf{$f = \frac{1}{x}$, 只有在 $0$ 这一个点上 $f$ 是 unbounded 的, 但是它的 essential supermum 仍然是 $\infty$}, 因为不可能通过去掉一个 measure $0$ set 来使它 bounded. \\
无法找到一个 $M$, 使得 $f$ 在几乎处处都小于 $M$. 你只能控制, $f$ 在 $(0,1/M)$ 上小于 $M$, 这个集合的测度随 $M$ 增大越来越小, 但是永远都是正测度.
(同样这个函数也不属于任何 $L^p(m)$.)\\
一个函数 essential supermum $<\infty$, 即 $\in L^\infty$, 则必须要它 unbounded 的这个行为是可以忽略不计的, 不能是明显的. 比如它在 $\mathbb{Q}$ 上 unbounded. 如果是在一个点上连续 blow up, 那么它就不可能 $\in L^\infty$. 类似于这里的  $f = \frac{1}{x}$.\\
\end{remark}
\begin{remark}
    我们在本节课还会证明, 如果 measure space $X$ has finite measure, 那么有 \[
    L^\infty(X) \subset \cdots \subset L^p(X)\subset \cdots  \subset L^{q} (X)\subset \cdots \subset L^1(X)
    \]
for 任意的 $p \geq q$.\\ 这表明的是, 在一定要求下, $L^\infty$ 是要求最严格的 space.
\end{remark}


下面是一个比较典型的例子:
\subsection{$\ell^\infty$ space}
\begin{definition}{$\ell^\infty$} \[
    \ell^\infty : = \{  (a_j)_1^\infty : \| (a_j)\|_\infty : = \sup_j |a_j|  <\infty \}
    \]\end{definition}
\begin{example}
    \[
    f = x\chi_{\mathbb{Q}} \in L^\infty (m)
    \]with \[
    \| f\|_\infty = 0
    \]
    因为整个 $\mathbb{Q}$ 都是零测的.
\end{example}

\begin{remark}
$\ell^\infty$ 其实就是: \[
X: = \mathbb{N},\quad \mathcal{A}: = \mathcal{P}(X),\quad \mu = \mu_{counting}
\]的 measure space 上的 $L^\infty(\mu)$. \[
\ell^\infty =L^\infty (\mathbb{N}, \mathcal{P}(\mathbb{N}), \mu_{counting} )
\]
一个 seq 就是一个从 $\mathbb{N}$ to $\mathbb{C}$ 的函数, 把每个 entry map to 一个 complex number.\\
\textbf{而对于 counting measure 作为 measure 的 measure space 上, 唯一的零测集就是空集}, 因为哪怕只取一个元素, 这个子集的测度也是 1.\\
比如, 我们只取三个 entry $1,2,8$, 看 $\{|a_n|\}_1^\infty \setminus \{|a_1|,|a_2|, |a_3|)\}$ 中的 $\sup$ value, 也不符合 essential supremum 的定义.\\
因而我们发现, \textbf{对于 唯一的零测集就是空集 的 measure space, for example, 任何以 counting measure 作为 measure 的 measure space, 其 essential sup norm 就是普通的 sup value norm.}\\
比如 \[
\mathbb{C}^1,\mathbb{C}^2,\mathbb{C}^3,\cdots,\ell^\infty
\]
\end{remark}

\subsection{$L^\infty$ 的基本性质: as a NVS; Hölder's ineq on it; dense subsets }

\begin{lemma}
如果 $f \in L^\infty(\mu) $ 则: 
\begin{itemize}
    \item 一定有 $|f(x)| \leq \|f\|_\infty$ for a.e. $x$.
    \item 存在一个 bounded 函数 $g$, 使得 $f=g$ a.e.
\end{itemize}
\end{lemma}
\begin{proof}
    显然.
\end{proof}
\begin{remark}
    是否有在某个零测集上 unbounded 但是却 $L^\infty$ 的函数? 答案是肯定的:\[
f(x) =
\begin{cases}
\frac{1}{x}, & x \in \mathbb{Q} \cap (0,1] \\
0, & \text{otherwise}
\end{cases}
\]
有 $\|f\|_\infty = 0$.
\end{remark}

\begin{theorem}
    \begin{itemize}
        \item \[ \|fg\|_\infty \leq \|f\|_1 \|g\|_\infty\]
可以把它看作 \textbf{Hölder 的一部分特殊情况}, 因为可以看作 \[ \frac{1}{1} + \frac{1}{\infty} = 1\] 从而补充完整了 Hölder ineq for $1 \leq p,q\leq \infty$
        \item $L^\infty$ 是一个 \textbf{normed vector space}, equipped with $\| \cdot \|_\infty$
        \item simple functions are dense in $L^\infty$
    \end{itemize}
\end{theorem}
\begin{proof}
    容易证明. 
\end{proof}
\begin{remark}
    注意, $L^\infty$ 和 $L^p$ 有一个出入点是: \textbf{$C_c^0(\mathbb{R}^n)$ 并不是 $L^\infty (\mathbb{R}^n, m)$ 上的 dense subspace!}\\
\end{remark}


\subsection{$L^\infty$-convergence 作为 (finite measure space 下) 最强的 $L^p$ convergence: 等价于 uni. conv a.e.}
\begin{theorem}{convergence in $L^\infty$ $\iff$uniform convergence a.e.}
\[  f_n \to f \; \text{ in } L^\infty \iff \text{exists null set } E\subset X\; s.t. f_n\to f \text{ uniformly on } E^c \]
(注意, 这\textbf{不是 conv almost uniformly}, 而是一个比 almost uniformly\textbf{ 更强}的条件:\textbf{ conv uniformly almost everywhere}, 因为 almost uniformly 只要求对于任意的 $\epsilon$, 都存在一个 measure 小于 $\epsilon$ 的 $E$, 使得在 $E^c$ 上 uni conv 即可.)
\end{theorem}
\begin{remark}
    这一条 convergence 十分惊人. 因为\textbf{对于普通的 $L^p$ space, converge in $L^p$ 和 a.e. convergence 并没有任何的互推关系}; 但是对于 $L^\infty$ convergence, 我们却可以把它\textbf{等价于 uniform convergence almost everywhere}, which is 一个\textbf{比 a.u convergence 更强, 比 a.e. convergence 更强的逐点 convergence}. 可以看出 $L^\infty$ convergence 是比任何 $L^p$ convergence 都要强一个层次的收敛性质.\\
    这一点
\end{remark}
\begin{proof}
⇐: Suppose $f_n \to f$ uni. a.e; WTS: $f_n \to f$ in $L^\infty$
 $f_n \to f$ uni. a.e 即: 存在零测集 \(E\subset X\), \(f_n \to f\) on $E^c$.\\
Let $\epsilon > 0$.\\
$f_n \to f$ uni. a.e 表明, 存在 \(N\) 使得 for all \(n \ge N\) 有 \[
\forall x \in E^c,\quad |f_n(x) - f(x)| < \epsilon
\]
by def, exactly is: \[
\|f_n - f\|_{L^\infty} = \operatorname{ess\,sup}_{x \in X} |f_n(x) - f(x)| \le \epsilon
\]
This shows that \(\|f_n - f\|_{L^\infty} \to 0\), 即 \(f_n \to f\) in \(L^\infty\).\\

⇐: Suppose $f_n \to f$ in $L^\infty$; WTS: $f_n \to f$ uni. a.e.Denote: \[
\epsilon_n := \|f_n - f\|_{L^\infty}
\]
By assumption, $\epsilon_n \to 0$. Define for each $n$:
\[
A_n := \{x \in X : |f_n(x) - f(x)| > \epsilon_n\}
\]
By def \(\|f_n - f\|_{L^\infty} = \text{ess sup}_{x} |f_n(x) - f(x)| \le \epsilon_n\), 于是 \( \mu(A_n) = 0\)
那么令:  \[
E := \bigcup_{n=1}^\infty A_n
\]by subadditivity of measure 有 \(\mu(E) = 0\).
于是对于任意 $\epsilon_n$, 都有
\[
 |f_n(x) - f(x)| \le \epsilon_n \to 0,\quad \text{for all } x \in E^c
\]
由于 $\epsilon_n \to 0$, showing that outside $E$, 有  \(\|f_n - f\|_{L^\infty} \to 0\).
\end{proof}
\begin{remark}
这两个 convergence 直觉上是自然相等的.\\
但是这并不能够说明 $   L^\infty(\mu)\text{-convergence}$ 就是强于任何 $L^p(\mu)\text{-convergence} $ 的. 因为即便是 uniform 的 ptwise conv 也无法推出 $L^p$ conv. \\
特殊情况是, 如果整个 base space $X$ 是 finite measure 的, 则可以推出\[
     L^\infty(\mu)\text{-convergence} \implies  L^p(\mu)\text{-convergence} \implies  L^q(\mu)\text{-convergence} \implies\cdots
    \]
    whenever $p > q$. (可证明)\\
    但是对于无限测度空间, 这种推论未必成立.
\end{remark}



\subsection{$L^\infty$ as Banach space}
\begin{theorem}{$L^p$ ($1\leq p \leq \infty$) is Banach}
For any measure space $(X,\mathcal{A},\mu)$, $L^p(\mu)$ is Banach for all $1\leq p \leq \infty$
\end{theorem}
\begin{proof}
我们已经 proved 了 $1\leq p < \infty$ 的 case, 现在 prove $p = \infty$ 的 case.\\
By \ref{Equiv Condition for space being Banach}, 我们知道 STS: every abs conv series conv in $L^\infty$.\\
我们 suppose $f_k \in L^\infty$ 有 \[
\sum_{k=1}^\infty \| f_k \|_\infty < \infty
\]
WTS: \(\sum_{k=1}^\infty f_k \) converges.\\
Set: \[
E_k := \{x: |f_k(x)| > \|f_k||_\infty\}
\]
于是有 \[
\mu(E_k) = 0 \quad \text{for each }k
\]
因而 setting \[
E : = \bigcup_{k=1}^\infty E_k
\]有 \[
\mu(E) = 0
\]
note: \[
x \in E^c \implies \sum_{k=1}^\infty |f_k(x)| \leq \sum_{k=1}^\infty \| f_k \|_\infty < \infty
\]
从而, \[
g: = \sum_{k=1}^\infty  f_k
\]在 $E^c$ 上是 well-defined 的, 且 bounded by $\sum_{k=1}^\infty \| f_k \|_\infty $.\\
对于 $x\in E$, 我们可以随便设置值, 比如 $\pi$, 然后 define $g(x)= \pi$ on $x\in E$. 然后对于 each $n$, 我们 set: \[
g_n(x) : = \begin{cases}
    \sum_{k=1}^n f_k(x),\quad x \in E^c \\
    \frac{1}{\pi},\quad x\in E
\end{cases}
\] 从而
\begin{align*}
    \|g_n - g \|_\infty \leq \sup_{x\in E^c} |g_n (x) -g (x)|  
   &\leq \sup_{x\in E^c} |\sum_{n+1}^\infty f_k(x)| \\
   &\leq \sup_{x\in E^c} \sum_{n+1}^\infty |f_k(x)| \\
    &\leq \sum_{n+1}^\infty  \|f_k\|_\infty \to 0\\
\end{align*}
\end{proof}
\begin{remark}
My reflection: 不 Banach 的 normed vector space 是什么样子的呢? 即, 这个 space 中存在某些 series, 其对应的 norm series absolutely conv 但是它却不 converge to 一个元素呢? \\
我们考虑空间 \( c_{00} \),它是所有 finite supp 的 seq 组成的空间:
\[ c_{00} := \{ x = (x_1, x_2, \dots) \in \mathbb{R}^\mathbb{N} \mid \text{only finite } x_i \neq 0 \}
\]
with \( \ell^1 \) norm: \[
\|x\| = \sum_{i=1}^\infty |x_i|
\]
 \( c_{00} \) 是一个 normed vector space, 但不是 Banach space, 它的完备化是 \( \ell^1 \).
我们考虑 series, with: \[
x_n = e_n / 2^n
\]
其中 \( e_n = (0, \dots, 0, 1, 0, \dots) \), 第 \( n \) 个位置是 1, 其余是 $0$, 是这个 NVS 的 standard basis.\\
显然每个 \( x_n \in c_{00} \),并且: \[
\|x_n\| = \frac{1}{2^n} \quad \Rightarrow \quad \sum_{n=1}^\infty \|x_n\| = \sum_{n=1}^\infty \frac{1}{2^n} = 1
\]这是一个 absolutely convergent series,  但其和
\[
\sum_{n=1}^\infty x_n = \left(\frac{1}{2}, \frac{1}{4}, \frac{1}{8}, \dots\right) \notin c_{00}
\]
$L^p$ space 的 Banach 性表示了其\textbf{极限存在的稳定性}. recall, Banach 即 complete NVS, 而 \textbf{complete 是比 closed 更强的条件}. \\
因而\textbf{任何一个 $L^p$ 函数列, 如果 Cauchy / converge in $L^p$ norm, 那么它的极限一定在 $L^p$ 里.}
\end{remark}



\subsection{relationship between $L^p$ spaces}

\subsection{$L^{m}(\mu) \subset L^{n}(\mu), 0 < n\leq m\leq \infty$, for measure finite space)}
刚才我们已经 state 了, 但还没有证明: 
\begin{theorem}{ inclusion relation between $L^p$ spaces (when base space is finite measure)}
    如果 measure space $X$ has finite measure, 那么有 \[
    L^\infty(X) \subset \cdots \subset L^m(X)\subset \cdots  \subset L^{n} (X)\subset \cdots 
    \]
for 任意的 $m \geq n$.
\end{theorem}
这是我们首次把 $p<1$ 也 include 进我们的讨论.

这个 statement 即: 对于 from finite measure space to $\mathbb{C}$ 的 function $f$, 它的 $\|f\|_m< \infty$ 是比 $\|f\|_n<\infty$ 更强的条件. \\
尤其, 除去 $L^\infty$ 的情况, 它更直接的意思是: 对于 $0 <  n \leq m <\infty$ 而言, $f$ 的绝对值的 $m$ 次方的积分 $<\infty $ 是比 $f$ 的绝对值的 $n$ 次方的积分 $<\infty$ 要更强的条件. 

这其实是一件比较直观的事情. 因为对于 $|f| \geq 1$ 的部分, 
\[
\int_{|f| \geq 1} |f|^{large} \geq \int_{|f| \geq 1}  |f|^{small}
\]
而对于 $|f| <1$ 的部分,\[
\int_{|f| < 1} |f|^{large} \leq \int_{|f|< 1}  |f|^{small}
\]
然而\textbf{由于整个 space 的 measure 是 finite 的, $|f| <1$ 的部分并不影响}. 因为 \[
\int_{|f| < 1} |f|^{large} \leq \int_{|f|< 1}  |f|^{small} \leq \int_{|f|< 1}  1 \leq \mu(X)
\]
因而, 对于 $\mu(X)<\infty$ 的情况, 显然有 $\|f\|_{large}< \infty$ 是比 $\|f\|_{small}<\infty$ 更强的条件.\\

\textbf{(实际上, 如果只有 measure finite 的 $x$ 上 $|f(x)| < 1$, 那么即便 $\mu(X)=\infty$, $\|f\|_{large}< \infty$ 也是比 $\|f\|_{small}<\infty$ 更强的条件; 而如果有 measure infinite 的 $x$ 上 $|f(x)| <1$, 那么有可能 $\|f\|_{large}< \infty$ 是比 $\|f\|_{small}<\infty$ 更弱的条件)}\\
My point: 虽然说 $|f(x)|^{large}$ 比起 $|f(x)|^{small}$ 是更大还是更小取决于 $|f(x)|$ 是否 $\geq 1$ or $<1$, 但是 $\geq 1$ 的值是可以 unbounded 的, 而 $<1$ 的值再怎么通过小次方变得更大, 也超不过 $1$. 因而 $|f(x)| \geq 1$ 的部分通常更能函数积分值的有限性, 除非在一个 measure infinite 的集合上 $|f(x)| <1$.

这里有一个更加严格的证明:
\begin{proof}
首先, 对于 $m = \infty$ 的 case, 如果 $f\in L^{m} = L^{\infty}$, 那么取任意 $1\leq n < \infty$ 都有: \[
\int |f|^n \leq \int \|f\|_\infty^n = \|f\|_\infty ^ n  \mu(X) < \infty
\]
 其次, 对于正常的  $m < \infty$ 的 case, 我们使用 Hölder: 
如果 $f\in L^m$, 那么对于任意 $n<m$, 我们可以构造出 Hölder conjugate $\frac{m}{n}$ 和 $\frac{m}{m-n}$,从而:
\begin{align*}
      \int |f|^n &= \int |f|^n \cdot 1 \\
      &\leq \bigg(  \int (|f|^n)^{\frac{m}{n}}  \bigg)^{\frac{n}{m}}\bigg(  \int 1^{\frac{m}{m-n}}  \bigg)^{\frac{m-n}{m}}\\
      & = \|f\|_m^n \mu(X)^{\frac{m-n}{m}} < \infty 
 \end{align*}
 从而 $$\| f\|_m <\infty \implies \|f \|_n < \infty$$
 这一 proof 利用 Hölder conjuate, 通过构造包含 $\frac{m}{n}$ 的 Hölder conjugate, 把 $ \int |f|^n$ 改成了 $ \|f\|_m$ 的 expression.
\end{proof}
以下是一个经典的例子:
\begin{example}
考虑\textbf{ measure finite 的 measure space $(0,1)$}: 通过经典的 Calculus 我们知道:
\[ f(x) = \frac{1}{x^m} \in L^p(0,1) \quad \text{ for all } p < \frac{1}{m}\]
但是对于任意的 $m$, 都有: \[
 f(x) = \frac{1}{x^m} \not\in L^p(0,1)
\]
而我们再看一个 \textbf{measure infinite 的 measure space $(1,\infty)$ 上的反例}, 采用同一个函数:
$$f(x) = \frac{1}{x^m},\quad x\in (1,\infty)$$
这个时候, $p$ 越大, $\int |f|^p =  \|f \|_p^p$ 反而越小, 通过经典的 Calculus 我们知道:我们知道
而对于 \[
f(x) = \frac{1}{x^m} \in L^p(1,\infty)\quad \text{for all } p > \frac{1}{m}
\]并且 $f \in L^\infty (1,\infty)$, 因为 $\|f\|_\infty  = 1$.\\
这个空间上的这个函数正对应了我们刚才讨论的, 如果有 infinite measure 数量的 $x$ 上 $|f(x)| <1$, 那么很可能 $\|f\|_{large}< \infty$ 是比 $\|f\|_{small}<\infty$ 更弱的条件
\end{example}


\subsection{control arbitrary $\|f \|_m $ 和 $\|f \|_n$ 的大小比例, in measure finite space}
\begin{remark}
刚才我们的推导中, \begin{align*}
      \int |f|^n \leq  \|f\|_m^n \mu(X)^{\frac{m-n}{m}} < \infty 
 \end{align*}
 两边开 $p$ 方, 可以得到一个不等式:  
 \begin{theorem}
     对于 measure finite space $X$, 对于任意的 $0< n\leq m \leq \infty$, 有: \[
     \|f\|_n \leq  \|  f \|_m \,\mu(X)^{\frac{1}{n} - \frac{1}{m}}
     \]\end{theorem}
 这也是一个有用的不等式. 它在 measure finite space 上, 对于任意的可测函数, 控制了两个任意的 function $p$-norm (虽然 for $p<1$ 不能严格地称为 norm) 之间的大小关系。
\end{remark}


\subsection{$(L^n\cap L^r) \subset L^m \subset (L^n + L^r)$, 对任意 $0< n < m < r \leq \infty$} 

\begin{proposition}
对于 measurable $f: X \to \mathbb{C}$, \[
t \mapsto \| f\|_{\frac{1}{t}} 
\]is \textbf{log-convex}.\\
equivalently 即: 对于任意的 $0< n < m < r \leq \infty$, 都有 \[
\|f \|_m \leq \|f \|_n ^\lambda  \cdot \| f\|_r ^{1-\lambda}
\]
where \[
\lambda := \frac{\frac{1}{m} - \frac{1}{r}}{\frac{1}{n} - \frac{1}{r}} \in (0,1),\quad i.e. \bigg(\frac{1}{m}\bigg) = \lambda\bigg( \frac{1}{n}\bigg) + (1-\lambda) \bigg(\frac{1}{r}\bigg)
\]
\end{proposition}
\begin{remark}
    log convex 即: 这个函数的 $\log$ 函数是 convex 的. 即对于任意 $x,y$, 以及 $[x,y]$ 上的任意一点, 即 $\lambda x + (1-\lambda) y $ for some $\lambda \in [0,1]$, 都有: \[
    \log f (\lambda x + (1-\lambda) y) \leq \lambda  \log f(x) + (1-\lambda) \log f(y)
    \] 即: \[
    f (\lambda x + (1-\lambda) y)  \leq f(x)^\lambda f(y)^{1-\lambda}
    \]
例如: $e^x, e^{x^2}, x^x$ 都是 log-convex 的. convex 函数的几何意义是 \textbf{"函数值小于等于两端的线性插值"}, 中点值 $\leq $两端值的\textbf{算术平均}, 而 log-convex 函数的几何意义是: , 中点值 $\leq $两端值的\textbf{几何平均}.\\
这里, 两端点是 $\frac{1}{r} <\frac{1}{n}$, 而中间的取点则是 $\frac{1}{m}$. log convexity 性质表明: \[
\|f \|_m \leq \|f \|_n ^\lambda  \cdot \| f\|_r ^{1-\lambda}
\]
\end{remark}
\begin{proof}
    For $r = \infty$, then $\lambda = \frac{n}{m}$.\\
    Since \[
    |f |^m  =  |f |^n \cdot | f|^{m-n} \leq |f |^n \cdot \| f\|_\infty ^{m-n} \quad a.e.
    \]
    可以得到 \[
    \int |f |^m \leq \bigg(\int |f|^n \bigg) \cdot \|f \|_\infty ^{m-n}  =  \|f ||_n ^n \cdot  \|f \|_\infty ^{m-n}
    \]
    从而 Taking $q$th root 得到结果:\[
    \| f \|_m \leq \|f \|_n ^{n/m} \|f \|_\infty ^{1-n/m}
    \]
    For $r < \infty$: 我们采用 conjugate exponents: \[
    \frac{n}{\lambda m}, \frac{r}{(1-\lambda)m}
    \]
    这是因为:  \[
     \bigg(\frac{1}{m}\bigg) = \lambda\bigg( \frac{1}{n}\bigg) + (1-\lambda) \bigg(\frac{1}{r}\bigg) \implies 1 = \lambda\bigg( \frac{m}{n}\bigg) + (1-\lambda) \bigg(\frac{m}{r}\bigg)
    \]
    从而 Applying Hölder: \begin{align*}
         \int |f|^m &= \int |f|^{\lambda m  } |f|^{(1-\lambda)m}\\
         &\leq \bigg( \int {|f|^n}\bigg)^ {\frac{\lambda m}{n}} \bigg(\int |f|^r \bigg)^{\frac{(1-\lambda)m}{r}} \\
         & = \| f\|_n ^{\lambda m} \cdot \|f \|_r ^{(1-r) m}
    \end{align*}
    Taking $q$ th root 得到结果.
\end{proof}
\begin{remark}
    Hölder's ineq 仍然是这里重要的一步. 我们这里需要利用 convexity 表述中的 "point on a line segment" 条件来构造一个 conjugate.
\end{remark}
\begin{remark}
    此处我们可以由这个 proposition 直接得到一个推论: \

\begin{corollary}
    对于任意的 $0< n < m < r \leq \infty$, 都有  \[
   (L^n \cap L^r) \subset L^m
    \]
\end{corollary}
\end{remark}

\begin{example}
    令 $A$ 为任意集合, $0\leq p < q \leq \infty$, 有: \[
 \|f \|_q \leq \|f \|_p \quad \text{and thus} \quad \ell^p(A) \subset \ell^q(A)
    \]
    这是因为 \[
    \|f \|_\infty ^p  = \sup_\alpha |f(\alpha)|^p \leq \sum_\alpha |f(\alpha)|^p = \|f \|_p^p
    \]
    于是 for $q\not = \infty$ case  \[
    \|f \|_q \leq \|f \|_p^\lambda \|f\|_\infty^{1-\lambda }  \leq \|f\|_p
    \]
(另一 case, trivial.)\\
我们发现 $\ell^p$ 空间, $p$ 越小要求反而越严格. \\
这是因为 $\ell^p$ 空间中一个函数就是一个 seq, 其 $p$-norm 就是各项的 $p$ 次方和, 再开 $p$ 次方根.\\
对于一个 seq, 如果它的累和 series 收敛, 它的各项肯定是 \textbf{eventually 收敛的}, 那么这些除了有限项外的这些项的绝对值都是 $<1$ 的, 那么\textbf{ $p$ 越大,  它们 $p$ 次方和只会越小}. 这正对应了我们之前说的 \textbf{"$|f(x)| < 1$ 的点主导函数" 的情况.}\\\\
\end{example}

相对于这个inclusion 关系, 我们还有另外一个 inclusion 关系: 
\begin{proposition}{每个 $L^m$ 函数都是一个 $L^n$ 函数和一个 $L^r$ 函数的和 ($0< n < m < r \leq \infty$)}
 对于任意的 $0< n < m < r \leq \infty$, 都有  \[
   L^m\subset  (L^n + L^r)
    \]
\end{proposition}
这个 inclusion 关系有一种调和的感觉在里面. 它 roughly mean 给定一个函数, 它可以拆成一个更加容易积的函数和一个更加不容易积的函数, 并且我们很大程度上可以控制这两个函数的可积性.\\
但其实很简单, 就是用我们之前的 $|f(x)|<1$ 和 $\geq 1$ 的点作为区分, 把函数的定义域分成两部分. 如果 $|f|$ 的 m 次方是可积的, 那么更小的 $n$ 次方, 对于 $|f(x)| \geq 1$ 的部分肯定也是可积的; 更大的 $r$ 次方, 对于 $|f(x)|< 1$ 的部分肯定也是可积的;

\begin{proof}
Suppose $f \in L^m$. Let \[
E:= \{ x:|f(x)|>1\} 
\]
let \[
g: = f\chi_E,\quad h := f \chi_{E^c}
\]
于是 $g \in L^n$ for all $0< n \leq m$, $h \in L^r$ for all $r \geq m$ and $r= \infty$.
\end{proof}