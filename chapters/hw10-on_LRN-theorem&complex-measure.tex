\chapter*{Homework 10: on LRN Theorem and complex measure (40/40)}
(Note: For this homework I applied for an one-day extension since I met with some emergent problem with my bank and rent payment.)

\section*{complex measure 的 total variation 的 formulas}
  Let $\nu$ be a complex measure on a measurable space $(X,\mathcal{A})$. Prove that, for any $E\in\mathcal{A}$:   \begin{align*}
    |\nu|(E)
    &=\sup\{\sum_{j=1}^n|\nu(E_j)|\mid n\in\mathbb{N}, E_1\dots E_n\ \text{disjoint},\ E=\bigcup_{j=1}^nE_j\}\\
    &=\sup\{\sum_{j=1}^\infty|\nu(E_j)|\mid E_1, E_2,\dots \text{disjoint},\ E=\bigcup_{j=1}^\infty E_j\}\\
    &=\sup\{\big|\int_E f\,d\nu\big|\mid f\colon X\to\mathbb{C}\ \text{measurable}, |f|\le 1\}.
  \end{align*}
\begin{proof}
Take some positive measure $\mu$ s.t.  $\nu \ll \mu$ (e.g. $\mu: =  |\Re \nu | + | \Im \nu |$), then by RN Thm there exists $\mu$-unique RN derivative $f$, and $|\nu|$ can be defined by \[
d |\nu|  : =  |f | \,d \mu
\]
Now we denote: \begin{align*}
    \mu_1(E) &:=\sup\{\sum_{j=1}^n|\nu(E_j)|\mid n\in\mathbb{N}, E_1\dots E_n\ \text{disjoint},\ E=\bigcup_{j=1}^nE_j\}\\
     \mu_2(E) &:=\sup\{\sum_{j=1}^\infty|\nu(E_j)|\mid E_1, E_2,\dots \text{disjoint},\ E=\bigcup_{j=1}^\infty E_j\}\\
    \mu_3(E) &:=\sup\{\big|\int_E f\,d\nu\big|\mid f\colon X\to\mathbb{C}\ \text{measurable}, |f|\le 1\}.
\end{align*}
We will prove the equality by showing that $\mu_1 \leq \mu_2 \leq |\nu|(E) \leq \mu_3  \leq \mu_1$.\\
\textbf{Claim 1: $\mu_1 \leq \mu_2$.}\\
Proof: This is trivial since for each finite disjoint segmentation $E=\bigsqcup_{j=1}^nE_j$ of $E$ can be made into a countable segmentation of $E$, by taking all $E_N = \varnothing$ for $N \geq n+1$. So every value included in $\{\sum_{j=1}^n|\nu(E_j)|\mid E=\bigsqcup_{j=1}^nE_j\}$ is also in $\{\sum_{j=1}^\infty|\nu(E_j)|\mid E=\bigsqcup_{j=1}^\infty E_j\}$. Thus taking $\sup$, we have the ineq.\\
\textbf{Claim 2: $\mu_2 \leq  | \nu | \leq \mu_3$.}\\
Since $\nu \ll |\nu|$ (Folland prop 3.13), by complex RN Thm we have have  $$f:=\frac{d \nu}{d|\nu|} \in L^1(|\nu|)$$
Notice that\textbf{ $f$ have absolute value $1$, $|\nu|$-a.e.} (Folland prop 3.13)\\
Suppose $E=\sqcup_1^{\infty} E_j$, we have:
$$
\begin{aligned}
\sum_{j=1}^{\infty}\left|\nu\left(E_j\right)\right| & \leq \sum_{j=1}^{\infty}|\nu|\left(E_j\right)  \quad &\text{by property of total variation measure}\\
& =|\nu|(E)=\int_E 1 \,d|\nu| \quad &\text{by ctbl disjoint additivity }\\
& =\int_E|f|^2 d|\nu|  =\int_E \bar{f} f d|\nu|\quad &\text{since $f$ have absolute value $1$  $\nu$-a.e.}\\
& =\int_E \bar{f} \frac{d \nu}{d|\nu|}  d|\nu|
\end{aligned}
$$
To confirm this equal to $\int \bar{f}\, d\nu$, we extend Folland prop 3.9 to the complex case.
\begin{proposition}
For complex measure $\nu$ and $\sigma$-finite positive measure $\mu$ s.t. $\nu \ll \mu$, if $g\in L^1(\nu)$, then \[
g\bigg(\frac{d\nu}{d\mu}\bigg) \in L^1(\mu),\quad \int g \,d\nu =  \int g\bigg(\frac{d\nu}{d\mu}\bigg)  d\mu
\]
\end{proposition}
And the proof just follows from the finite signed-measure case, applied both to im part and re part.
\begin{align*}
\int g \,d\nu  &= \int g \, d(\Re \nu ) + i \int g\, d(\Im \nu)    \\
& = \int g \bigg( \frac{d (\Re\nu)}{d\mu}\bigg)\, d\mu + i \int g\bigg( \frac{d (\Im\nu)}{d\mu}\bigg)\, d\mu\\
& =  \int g \bigg( \Re\frac{d \nu}{d\mu} +  i\Im\frac{d \nu}{d\mu}   \bigg)\, d\mu\\
& = \int g\bigg(\frac{d\nu}{d\mu}\bigg)  d\mu
\end{align*}
Now we back to Claim 2, since $f,\bar{f} \in L^1(\nu) $, we have: 
\begin{align*}
\sum_{j=1}^{\infty}|\nu\left(E_j\right)| & \leq |\nu|(E) \\
& =\int_E \bar{f} \frac{d \nu}{d|\nu|}  d|\nu|\\
& = \int_E  \bar{f} \, d\nu \\
& \leq \left|\int_E \bar{f} d \nu\right|
\end{align*}
Since $|\bar{f}| \leq 1$ (in $\nu$-a.e. sense), this shows that every element in $\{\sum_{j=1}^\infty|\nu(E_j)|\mid E=\bigsqcup_{j=1}^\infty E_j\}$ is less then or equal to $|\nu| (E)|$, and $|\nu| (E)|$ is less then some element in $\{\big|\int_E f\,d\nu\big|\mid \text{measurable }|f|\le 1\}$,  proves that $\mu_2 \leq|\nu| \leq \mu_3$.\\
\textbf{Claim 3: $\mu_3 \leq \mu_1$.}\\
For arbitrary simple function $\phi:=\sum_1^n c_k \chi_{E_k}$ where $\left|c_k\right| \leq 1$ for all $k, E_i$ s are disjoint and $\bigcup_{i=1}^n E_i=E$. We have\begin{align*}
    \left|\int_E \phi d \nu\right| &\leq \sum_{k=1}^n\left|c_k \int_{E_k} \chi_{E_k} d \nu\right|\\
    &=\sum_{k=1}^n\left|c_k\right|\left|\nu\left(E_k\right)\right| \\
    &\leq \sum_{k=1}^n\left|\nu\left(E_k\right)\right| \\
    &\leq \mu_1(E)
\end{align*}
Now we consider the general case: any measurable $f$.\\
Fix arbitrary measurable $f$ s.t. $|f| \leq 1$, since it is measurable, we can choose seq of simple functions  $(\phi_n)_1^\infty$ that approximate $f$ pointwisely from below.\\
\[
\lim_{n\to \infty} \phi_n   = f
\]with \[
  0\leq |\phi_1 |\leq |\phi_2| \leq \cdots \leq | f| 
\]
Then $|f|$ as a dominating function for $( |\phi_n|)_n$, \textbf{by DCT }we obtain: \[
\int_E f \, d (\Re \nu) = \lim_{n\to \infty} \int_E \phi_n \, d (\Re \nu) 
\]
and \[
\int_E f \, d (\Im \nu) =  \lim_{n\to \infty} \int_E \phi_n \, d (\Im \nu) 
\]
Thus
\begin{align*}
    \int_E f \, d\nu &= \int_E  f \, d(\Re \nu) + i\int_E  f \, d(\Im \nu)\\
    &= \lim_{n\to \infty} \bigg( \int_E \phi_n \, d (\Re \nu)  + i  \int_E \phi_n \, d (\Im \nu)  \bigg)\\
    & =  \lim_{n\to \infty} \int _E \phi_n d \nu
\end{align*}
Since for each $\phi_n$, we have $0 \leq |\phi_n(x)|  \leq |f(x)|\leq 1$ for a.e. $x\in E$, we can apply the ineq we obtained that \[  \left|\int_E \phi_n \, d \nu\right|   \leq  \mu_1(E)\] for each $n$. Thus taking limit we get: \[
        \left|\int_E f \, d \nu\right|   \leq  \mu_1(E) 
    \]
Taking supremum over $f$, proves that $\mu_3(E) \leq \mu_1(E)$. \\
Thus since we have shown $\mu_1 \leq \mu_2 \leq|\nu| \leq \mu_3 \leq \mu_1$, every inequality above is an equality, i.e.\[
\mu_1=\mu_2=\mu_3=|\nu|
\]finishing the proof.
\end{proof}






\section*{complex measure 与其 total variation measure 之间的关系: 整体即可决定局部}
  Let $\nu$ be a complex measure on a measurable space $(X,\mathcal{A})$. 
\subsection{$\nu(X)=|\nu|(X) \iff \nu =|\nu| \iff \nu  \text{ positive}$}
    \begin{itemize}
    \item[(i)]$\nu(X)=|\nu|(X)$; 
    \item[(ii)]      $\nu$ is a (finite) positive measure; 
    \item[(iii)]      $\nu=|\nu|$. 
  \end{itemize}
\begin{proof}
 \textbf{(ii) $\implies$ (iii):} If $\nu$ is positive then $\nu^- = 0$, so $\nu =  |\nu| = \nu^+$.\\
\textbf{(iii) $\implies$ (i):} Trivially true by taking $E = X$.\\
\textbf{(i) $\implies$ (ii):} Take some positive measure $\mu$ s.t.  $\nu \ll \mu$ (e.g. $\mu: =  |\Re \nu | + | \Im \nu |$), then by RN Thm there exists $\mu$-unique RN derivative $f$, and $|\nu|$ can be defined by \[
d |\nu|  : =  |f | \,d \mu
\]
Then by def \[ \int f \, d\mu  = \int |f| \, d\mu,\quad i.e.\quad \int \Re f \, d\mu   + i \int \Im f \, d\mu = \int |f| \, d\mu\]
Since the right hand side is real, we have:\[ \int ( |f| - \Re f) \, d\mu  =  0
\]
Note that, $|f| - \Re f$ is always nonnegative, so this implies that $\Re f = |f|\; \mu\text{-a.e.}$ \\
Thus $\Im f = 0\; \mu\text{-a.e.}$, so $f = |f|$ is real and positive $\mu$-a.e.
Thus \[
 \nu(E)  = \int_E f \, d\mu  \in \mathbb{R}_+,\quad \forall E\in\mathcal{A}
\]
finishing the proof that $\nu$ is a positive measure.
\end{proof}


\subsection{$|\nu(X)|=|\nu|(X) \iff \nu=\lambda|\nu|$ for some $|\lambda| =1$ }
Prove that the following two conditions are equivalent: 
\begin{itemize}
    \item[(i)]      $|\nu(X)|=|\nu|(X)$; 
    \item[(ii)]      there exists a complex number $\lambda$ with $|\lambda|=1$ such that $\nu=\lambda|\nu|$. 
\end{itemize}
\begin{proof}
\textbf{(i) $\implies$ (ii):} Since $\nu \ll |\nu|$, by complex RN Thm we have RN derivative  $$h:=\frac{d \nu}{d|\nu|} \in L^1(|\nu|)$$
Notice that\textbf{ $h$ have absolute value $1$, $|\nu|$-a.e.}\\
Then by def of RN derivative we have \[\nu(X)=\int_X h d|\nu| \]
Thus
$$|\nu(X)|=\left|\int_X h d | \nu| \right| \leq \int_X|h| d|\nu|=\int_X 1 d|\nu|=|\nu|(X)$$
Since we have $|\nu(X)|=|\nu|(X)$, it implie that: \[
\left|\int_X h d | \nu| \right| = \int_X|h| d|\nu|
\]
\textbf{Claim: $h$ is constant $|\nu|$-a.e.}\\
We first prove a lemma:
\begin{lemma}
   Let $\mu$ be a finite positive measure.\\
   For measurable function $f: X \to\mathbb{C}$, if $|f| = k$ a.e. for some nonzero constant $k$ and     \[
\bigg| \int f  \, d\mu \bigg|  = \int |f| \, d\mu 
\]
then $f$ must be a.e. constant. 
\end{lemma}
Proof of Lemma: 
Set:$$
c:=\frac{\int f d \mu}{\left|\int f d \mu\right|}
$$Then $|c|=1$, and we consider:
$$
\int f d \mu= c \left|\int f d \mu\right| = c \int|f| d \mu
$$
Define $g(x):=\bar{c} f(x)$, so:
$$
\int g \,d \mu=\bar{c} \int f \,d \mu=\bar{c} c \int|f|\, d \mu=\int|f| \,d \mu
$$
Notice $\int|f| \,d \mu \in \mathbb{R}_+$ and \[
\int g \,d \mu = \int \Re g \, d\mu  + i \int \Im g \, d\mu \in \mathbb{C}
\]Thus $$
\int\Re g  \,d \mu =  \int|g|\, d \mu =  \int|f|\, d \mu \implies \int (\Re g - |g|) \, d\mu = 0
$$
Since by def:$$
0 \leq \Re g \leq|g| 
$$
We must have\[
\Re g =  |g|\quad a.e.
\]
This proves that $g$ is a.e. real. And also since $|g|  = |f| = k$ a.e., \textbf{$g$ is then constant $k$ a.e.}\\
Therefore, \textbf{$f$ is constant $\frac{k}{\bar{c} }$ a.e.}\\\\
Now we go back to the proof of the original statement. By our Lemma we get: \[
h = \frac{\left|\int h\, d \mu\right|}{\overline{\int h\, d \mu}}\quad \text{constant for $|\nu|$-a.e. } x
\]
Therefore, \[
\nu=\frac{\left|\int h\, d \mu\right|}{\overline{\int h\, d \mu}}|\nu|
\]
This finishes the proof of (i) $\implies$ (ii).\\
\textbf{(ii) $\implies$(i): } This direction is trivial. Since $\nu=\lambda|\nu|$, we have \[
|\nu(X)| = |\lambda|  |\nu|(X)  = 1 |\nu|(X)    = |\nu|(X)  
\]
\end{proof}


  \section { complex measures on $(X,\mathcal{A})$ 组成一个 complex Banach space}
  Let $(X,\mathcal{A})$ be a measurable space. Prove that the set $\mathcal{M}$ of complex measures on $(X,\mathcal{A})$ is a complex Banach space, with norm given by $\|\nu\|:=|\nu|(X)$. 
\begin{proof}
 \textbf{   Claim 1: $\mathcal{M}$ is a complex vector space}, with addition operation defined by the addition of two complex measures, and scalar multiplication defined by scaling a complex measure by a complex number.\\
 Proof of Claim 1: 
For $\nu, \mu \in \mathcal{M}$, and $\alpha \in \mathbb{C}$, define:
    \begin{itemize}
        \item $(\nu+\mu)(E):=\nu(E)+\mu(E)$ for all $E \in \mathcal{A}$
        \item $(\alpha \nu)(E):=\alpha \cdot \nu(E)$ for all $E \in \mathcal{A}$.
    \end{itemize}
Then: $(\nu+\mu)(\varnothing)  = 0  + 0  = 0, (\alpha \nu)(\varnothing) = \alpha 0 = 0$.\\
Also, $\nu+\mu$ and $\alpha \nu$ are both countably additive, since sum and scalar multiples preserve this property: for $E = \bigsqcup_{j=1}^\infty E_j$ with each $E_j \in \mathcal{A}$, we have:     \[
(\nu + \mu) (\bigsqcup_{j=1}^\infty E_j) = \nu  (\bigsqcup_{j=1}^\infty E_j) + \mu (\bigsqcup_{j=1}^\infty E_j) = \nu (E ) + \nu(E) = (\nu + \mu) (E)
\]and \[
(\alpha \nu)(\bigsqcup_{j=1}^\infty E_j)  = \alpha \cdot \nu(\bigsqcup_{j=1}^\infty E_j) = \alpha \nu (E)
\]So they are also complex measures, showing that $\mathcal{M}$ is closed under addition and scalar multiplication, thus a complex vector space.\\
\textbf{Claim 2: total variation $\| \nu \| : = | \nu(X)|$ defines a norm on $\mathcal{M}$.}\\
Proof of Claim 2: 
To verify this is a norm, we check the norm requirements:
\begin{itemize}
    \item  \textbf{Nonnegative}: $\|\nu\| \geq 0$, and $\|\nu\|=0 \Longleftrightarrow \nu=0$\\
Proof: $\|\nu\| \geq 0$ follows from that $|\nu|$ is a p.m.\\
Since we know $\nu \ll |\nu | $, if $ |\nu|(X) = 0$ then $X$ is a null set of $|\nu|$, and thus is a null set for $\nu$, so $\nu = 0$; \\
Conversely, if $\nu = 0$ then \[ \|\nu \| := 
|\nu|(X) =     \sup\{\sum_{j=1}^n|\nu(E_j)| :X=\bigsqcup_{j=1}^nE_j\} = \sup \{ 0 \} = 0
\]finishing the proof that $\|\nu\|=0 \Longleftrightarrow \nu=0$
    \item \textbf{Homogeneity}: $\|\alpha \nu\|=|\alpha| \cdot\|\nu\|$\\
Proof: \begin{align*}
 \| \alpha \nu \| := 
|\alpha \nu|(X) &=     \sup\{\sum_{j=1}^n|\alpha \nu(E_j)| :X=\bigsqcup_{j=1}^nE_j\} \\
&= |\alpha|\sup \{\sum_{j=1}^n| \nu(E_j)| :X=\bigsqcup_{j=1}^nE_j\}\\
& =|\alpha| |\nu | (X)  = |\alpha|\| \nu \|
\end{align*}
    \item \textbf{Triangle inequality}: $\|\nu+\mu\| \leq\|\nu\|+\|\mu\|$\\
Proof: 
\begin{align*}
    |\nu + \kappa|(X) &=\sup \left\{ \sum_{i=1}^n |(\nu + \kappa)(E_i)| 
 : X= \bigsqcup_{i=1}^N E_i \right\} \\
    & \leq \sup \left\{ \sum_{i=1}^n (|\nu(E_i)| +|\kappa(E_i)|)  
 : X= \bigsqcup_{i=1}^N E_i \right\} \quad &\text{by tri ineq in } \mathbb{R}           \\
 & =  \sup \left\{ \sum_{i=1}^n |\nu(E_i)| +  \sum_{i=1}^n|\kappa(E_i)|  
 : X= \bigsqcup_{i=1}^N E_i \right\}       \\
 &\leq \sup \left\{ \sum_{i=1}^n |\nu(E_i)| 
 : X= \bigsqcup_{i=1}^N E_i \right\} + \sup \left\{ \sum_{i=1}^n |\kappa(E_i)|
 : X= \bigsqcup_{i=1}^N E_i \right\}\\
 &= |\nu|(X)+|\kappa|(X) 
\end{align*}
\end{itemize}
Here we have finished the proof of $(\mathcal{M}, \| \cdot \|)$ being a normed $\mathbb{C}$-vector space.\\
\textbf{Claim 3: $(\mathcal{M}, \| \cdot \|)$ is complete (thus Banach space)}\\
Proof: 
Let $(\nu_n)$ be a Cauchy sequence in $\mathcal{M}$. We have
\[
|\nu_n(B) - \nu_m(B)| = |(\nu_n - \nu_m)(B)| \leq  |(\nu_n - \nu_m)(X)| =   \|\nu_n - \nu_m\| \quad \text{for all } B \in \mathcal{A}
\]
In particular, $(\nu_n(B))_n$ is a Cauchy sequence for all $B \in \mathcal{A}$. For each $B \in \mathcal{A}$, this is a Cauchy seq in $\mathbb{C}$, thus converges. So we can get: \[
\nu(B) := \lim_n \nu_n(B)
\]
as the pointwise limit (by a point we mean a set).\\
\textbf{Claim 3.1: $\nu \in \mathcal{M}$}.\\
Since for all $n$, $\nu_n(\varnothing) = 0$, we have:
\[
\nu(\varnothing) := \lim_n \nu_n(\varnothing) = 0
\]
For a countable disjoint union of measurable sets \(E = \bigsqcup_{i=1}^\infty E_i\), \[
  \nu(E) =  \lim_n   \nu_n(E) = \lim_n \sum_i \nu_n(E_i)
\]We know by property of total variation measure that for each $n$ we have: \[
\sum_i |\nu_n(E_i)| <  |\nu_n| (X) = \| \nu_n\| <M 
\]for some uniform bound $M$ for each $n$, since $\|\nu_n\|$ is a Cauchy seq in $\mathbb{C}$. Thus we can exchange the order of taking limit and sum. Then we get: \[
  \nu(E) = \lim_n \nu_n(E) = \lim_n \sum_i  \nu_n(E_i) = \sum_i \lim_n \nu_n(E_i) = \sum_i \nu(E_i)
  \]
verifying the countable disjoint additivity. \\
And notice, as we have mentioned, for each measurable set $E\in \mathcal{A}$, since $(\nu_n(E))_n$ is a Cauchy sequence in $\mathbb{C}$\textbf{, it is bounded}, verifying that $\nu$ \textbf{is a valid complex measure.}\\
\textbf{Claim 3.2: $\nu_n \to \nu$ in $\|\cdot\|$.}\\
Fix $\epsilon > 0$.\\
By Cauchy in $\| \cdot \|$, there exists $N \in \mathbb{N}$ s.t. for all $m,n \geq N$
, we have \[
 \|\nu_m - \nu_n \| = |\nu_m - \nu_n| (X) < \epsilon
\]Fix $n \geq N$, and consider the sequence $\nu_m$. Then $\nu_m \rightarrow \nu$ pointwise implies \textbf{$\nu_n-\nu_m \rightarrow \nu_n-\nu$ pointwise. } Thus
$$
\left\|\nu_n-\nu\right\| =  | \nu_n - \nu| (X)  \leq \liminf _{m \rightarrow \infty} |\nu_n - \nu_m| (X)<\epsilon
$$
Since $\epsilon > 0$ is arbitrary, this shows that, $\left\|\nu_n-\nu\right\| \rightarrow 0$ as $n\to \infty$, proving the convergence is in norm.\\
Now we conclude that $(\mathcal{M}, \| \cdot \|)$ is a Banach space.
\end{proof}





 \section*{Positivity}
  Let $\nu_1$, $\nu_2$ be complex measures on a measurable space $(X,\mathcal{A})$ such that $\|\nu_1+\nu_2\|=\|\nu_1\|+\|\nu_2\|$. Is it true that there exists a nonzero constant $a\in\mathbb{C}$ such that $a\nu_1$ and $a\nu_2$ are both positive measures? 
\begin{solution}
    No, not necessarily.
\end{solution}
\begin{proof}
Consider $X : = \{ m,n \}$\\
Define $\nu_1, \nu_2$ by atoms: \[
\nu_1 ( \{ m\})= \nu_2 ( \{ m\}) = 1,\quad \nu_1 ( \{ n\}) =\nu_2  ( \{ n\}) = -1
\]   Then \[
\| \nu_1 + \nu_2\| = \| 2\nu_1 \|  = |2\nu_1| (X)   = 4 = \| \nu_1 \| + \| \nu_2\|
\]
But there is no nonzero constant $a\in\mathbb{C}$ such that $a\nu_1$ and $a\nu_2$ are both positive measures. \\
This is because for any nonzero constant $a$ scaled on $\nu_1$: \textbf{if $a$ real, then it either flip, or preserve the sign of  $\nu_1( \{ m\})$ and $\nu_1 ( \{ n\})$}\textbf{, where there is always one positive number and one negative number between them; if $a$ complex, then make the two numbers complex.}\\
In both case, $\nu_1$ cannot become a positive measure. And since $\nu_2$ is defined the same as $\nu_1$, same for it.
Therefore it can never become positive measure by scaling a nonzero constant.
\end{proof}




 

\section*{Averaging: Conditional Expectation}
  Let $(X,\mathcal{A},\mu)$ be a finite measure space (i.e.\ a measure space such that $\mu(X)<\infty$). Let $\mathcal{B}\subset\mathcal{A}$ be a sub-$\sigma$-algebra, and set $\nu:=\mu|_\mathcal{B}$. Thus $(X,\mathcal{B},\nu)$ is also a finite measure space. 
  \begin{itemize}
  \item[(a)]   Prove that if $f\colon X\to\mathbb{C}$ is $\mathcal{B}$-measurable, then $f$ is $\mathcal{A}$-measurable. Is the converse true? 
  \item[(b)]    Suppose that $f\in L^1(\mu)$. Prove that there exists a $\mathcal{B}$-measurable function $g\in L^1(\nu)$ such that $\int_Ef\,d\mu=\int_Eg\,d\nu$ for all $E\in\mathcal{B}$. Also prove that any two such functions $g$ must agree outside a set of $\nu$-measure zero. 
  \item[(c)]Construct $g$ explicitly in the case when $X=\{1,2,3,4\}$, $\mathcal{A}=\mathcal{P}(X)$, $\mu(\{i\})=1/4$ for $i\in X$, and $\mathcal{B}=\{\emptyset,\{1,2\},\{3,4\},X\}$. 
    Thus, given the four complex numbers $f(i)$, $1\le i\le 4$, you should find the four complex numbers $g(i)$, $1\le i\le 4$. 
  \end{itemize}
 \textit{Hint}: use the Radon--Nikodym Theorem. 
 \textit{Remark}: if $\mu$ is a probability measure, then we can view $g$ as the conditional expectation of (the random variable) $f$ with respect to the $\sigma$-algebra $\mathcal{B}$. 


\begin{proof}
    \textbf{of (a):}
Suppose $f: X \rightarrow \mathbb{C}$ is $\mathcal{B}$-measurable, then for any Borel set $B \subset \mathbb{C}$, $f^{-1}(B) \in \mathcal{B} \subset \mathcal{A}$, so $f$ is $\mathcal{A}$ -measurable.\\
The converse is not true.\\
Consider $X = \{ 0,1,2,3 \}, A:= \mathcal{P}(X), \mathcal{B}:= \{\varnothing,X\}$.\\
Consider $f:x\mapsto x$ from $X$ to $\mathbb{R}$.\\
$f$ is $\mathcal{A}$-measurable since $\mathcal{A}$ is the power set, containing all subsets of $X$.\\
But $f^{-1}(\{0\}) = \{0\} \not \in \mathcal{B}$. Thus $f$ is not $\mathcal{B}$-measurable.
\end{proof}

\begin{proof}
    \textbf{of (b):}
    Let $\nu:=\left.\mu\right|_{\mathcal{B}}$, and define a signed measure on $\mathcal{B}$ by:
$$
\,\lambda(E):=\int_E f d \mu, \quad  E \in \mathcal{B}
$$
Then $\lambda \ll\nu$, since $\nu(E)=\mu(E)=0 \implies \lambda(E)=0$.\\
By Radon-Nikodym Thm, there exists a $\mathcal{B}$-measurable function $g \in L^1(\nu)$ such that
$$
\lambda(E)=\int_E g\, d \nu \quad \text { for all } E \in \mathcal{B}
$$
Then $$
\int_E f\, d \mu=\int_E g \,d \nu, \quad \forall E \in \mathcal{B}
$$
Suppose $g_1, g_2$ are both such functions, then
$$
\int_E\left(g_1-g_2\right) d \nu=0 \quad \forall E \in \mathcal{B}
$$
Define $$G^+ : = \{ g_1 - g_2 > 0\} , G^- : = \{ g_1 - g_2 < 0\} $$ These two sets are in  $\mathcal{B}$ since $g_1,g_2$ are $\mathcal{B}$-measurable.
Then we have: \[
 \int_{ G^+} (g_1-g_2)\, d \nu =  \int_{ G^-} (g_1-g_2) \, d \nu  = 0
\]
Since on $ G^+$ we have $g_1 - g_2>0$, \[
\int_{ G^+} (g_1-g_2)\, d \nu =  0\implies \int_{ G^+} |g_1-g_2|\, d \nu = 0  \implies g_1 = g_2 \,\, \nu  \text{-a.e. on } G^+ \implies \nu(G^+) = 0
\]
Similarly, since on $ G^-$ we have $g_1 - g_2<0$, \[
\int_{ G^-} (g_1-g_2)\, d \nu =  0\implies -\int_{ G^-} |g_1-g_2|\, d \nu = 0  \implies g_1 = g_2 \,\,\nu\text{-a.e. on } G^+ \implies \nu(G^-) = 0
\]
Thus \[
\nu\{g_1 \not = g_2\} =  \nu(G^+) + \nu(G^-)  = 0
\]
This finishes the proof.
\end{proof}

\begin{solution}
    \textbf{of (c):}
    Given:
\begin{itemize}
    \item $X=\{1,2,3,4\}$
    \item $\mathcal{A}=\mathcal{P}(X)$
    \item $\mu(\{i\})=1 / 4$ for each $i$
    \item $\mathcal{B}=\{\emptyset,\{1,2\},\{3,4\}, X\}$
\end{itemize}
Suppose we have: $f: X \rightarrow \mathbb{C}$, so $f(i) \in \mathbb{C}$ for $i=1,2,3,4$. We want to find: $g(i) \in \mathbb{C}, i=1,2,3,4$, such that $g$ is $\mathcal{B}$-measurable and $$
\int_E f \,d \mu=\int_E g \,d \nu \quad \text { for all } E \in \mathcal{B}
$$
Notice that $\mathcal{B}=\{\emptyset,\{1,2\},\{3,4\},X\}$, we must set $g(1) = g(2)$ and $g(3) = g(4)$, this is because, suppose if we set $g(1) \not = g(2)$, then it will happen that \[
1\in g^{-1} (g(1)) \not \ni 2
\]
No set in $\mathcal{B}$ satisfy this condition, thus $g^{-1} (g(1)) \not \in \mathcal{B}$, contradicts that $g$ is $\mathcal{B}$-measurable.\\
Thus we set \[
g(1) =g(2) = a,\quad g(3) = g(4) = b
\]We have: \[
\int_{\{1,2\}} g\, d\nu  = \int_{\{1,2\}} f\, d\nu = f(1)\mu(\{1\}) + f(2)\mu(\{2\}) = \frac{f(1)+ f(2)}{4}
\]and \[
\int_{\{3,4\}} g\, d\nu  = \int_{\{1,2\}} f\, d\nu = f(3)\mu(\{3\}) + f(4)\mu(\{4\}) = \frac{f(3)+ f(4)}{4}
\]
while on the other hand \[
\int_{\{1,2\}} g\, d\nu = \frac{g(1) + g(2)}{4} = \frac{a}{2},\quad \int_{\{3,4\}} g\, d\nu = \frac{g(3) + g(4)}{4} = \frac{b}{2}
\]
Thus $g$ is defined by: \[
g(1) =g(2) = \frac{f(1)+ f(2)}{2},\quad g(3) = g(4) = \frac{f(3)+ f(4)}{2}
\]
Thus what $g$ expressses: is the conditonal expectation of $f$ on $\{1,2\},\{3,4\}$.\\
(Therefore it can be generalized: given any sub $\sigma$-algebra $\mathcal{B} \subset \mathcal{A}$, there exists a $\mu|_{\mathcal{B}}$-unique $\mathcal{B}$ measurable function $g \in L^1(\mu|_{\mathcal{B}})$, that is the conditional expectation
$$
g=\mathbb{E}[f \mid \mathcal{B}]
$$
s.t. for $B \in \mathcal{B}$, $$
\int_B f d \mu=\int_B \mathbb{E}[f \mid \mathcal{B}] \,d \mu
$$
it gives the average of $f$ on sets in $\mathcal{B}$.)
\end{solution}



 

\vspace*{10mm}
\begin{center}
  \textit{Nur f\"ur Verr\"uckte}
\end{center}
(It's \textbf{really} not necessary to attempt these problems. Do not, under any circumstances, hand them in!)
  To any measure space $(X,\mathcal{A})$ we can associate a new measure space $(Y,\mathcal{B})$, where $Y$ is the Banach space of complex measures on $(X,\mathcal{A})$, and $\mathcal{B}$ is the Borel $\sigma$-algebra on $Y$.
  \begin{itemize}
  \item[(a)]Does this operation define a functor from the category of measurable spaces to itself. Is this functor (if well defined) full? Is it faithful? Is it essentially surjective?
  \item[(b)]Does the operation above admit any nontrivial fixed points (up to isomorphism)?
  \end{itemize}